\documentclass{article}

\usepackage[margin=1in]{geometry}
\usepackage{amsmath,amssymb, amsthm}
\usepackage[backend=biber,
	style=alphabetic,
	%	citestyle=authoryear,
	natbib=true,
	url=true, 
	doi=true]{biblatex}

\addbibresource{../refs.bib}

\usepackage{microtype}
\usepackage{tikz}

\usepackage{color}
\definecolor{deepgreen}{rgb}{0,0.5,0}

\usepackage[colorlinks=true, citecolor=deepgreen]{hyperref}

\newcommand\geqc{\succcurlyeq}
\newcommand\leqc{\preccurlyeq}

\setlength{\skip\footins}{1cm}
\setlength{\footnotesep}{0.4cm}


\begin{document}
	\section{What I want to do}
	I'm trying to get a theory of preference dynamics.
	
	Currently preferences are thought of as static objects, fixed as part of the structure and identity of an agent, independent of beliefs, complete, and over some fixed domain. This is clearly not at all how human preferences work, and I posit that it's not the right way to think of preference for other agents either.
	
	A good model of preference dynamics should
	\begin{enumerate}
		\item An answer to the `value loading' problem: show how you can acquire ``reasonable'' preferences by interacting with the world
		\item Some safety guarantees: show that in most circumstances, 
		\item Reduce to static models for some parameter settings
	\end{enumerate}

	In humans, we also have
	\begin{enumerate}
		\item Value Capture
	\end{enumerate}
	
	\section{Examples: Informal}
	
	
	\subsection{Value Capture}
	
	\section{Formalism}
	\section{Worked Examples}
\end{document}