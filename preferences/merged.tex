\documentclass{article}

\usepackage[margin=1in]{geometry}
\usepackage{mathtools} %also loads amsmath
\usepackage{amssymb, bbm}
\usepackage[backend=biber,
	style=alphabetic,
	%	citestyle=authoryear,
	natbib=true,
	url=true, 
	doi=true]{biblatex}

%\usepackage{blkarray} % for matrices with labels
\usepackage{microtype}
\usepackage{relsize}
\usepackage{environ}% http://ctan.org/pkg/environ; for capturing body as a parameter for idxmats
\usepackage{tikz}
	\usetikzlibrary{positioning,fit,calc, decorations, arrows, shapes, shapes.geometric}
	\pgfdeclaredecoration{arrows}{draw}{
		\state{draw}[width=\pgfdecoratedinputsegmentlength]{%
			\path [every arrow subpath/.try] \pgfextra{%
				\pgfpathmoveto{\pgfpointdecoratedinputsegmentfirst}%
				\pgfpathlineto{\pgfpointdecoratedinputsegmentlast}%
			};
	}}
	%%%%%%%%%%%%
	
	\tikzset{dpadded/.style={rounded corners=2, inner sep=0.9em, draw, outer sep=0.4em, fill=gray, fill opacity=0.08, text opacity=1}}
	\tikzset{active/.style={fill=blue, fill opacity=0.1}}
	\tikzset{square/.style={regular polygon,regular polygon sides=4, rounded corners = 0}}
	\tikzset{octagon/.style={regular polygon,regular polygon sides=8, rounded corners = 0}}
	
	
	\tikzset{alternative/.style args={#1|#2|#3}{name=#1, circle, fill, inner sep=1pt,label={[name={lab-#1},gray!30!black]#3:\scriptsize #2}} }
	
	
	\tikzset{bpt/.style args={#1|#2}{alternative={#1|#2|above}} }
	\tikzset{tpt/.style args={#1|#2}{alternative={#1|#2|below}} }
	\tikzset{pt/.style args={#1}{alternative={#1|#1|above}} }
	

	\tikzset{mpt/.style args={#1|#2}{name=#1, circle, fill, inner sep=1pt,label={[name={lab-#1},gray]\scriptsize #2}} }
	\tikzset{pt/.style args={#1}{name=#1, circle, fill, inner sep=1pt,label={[name={lab-#1},gray]\scriptsize #1}} }
	
		
		 %\foreach \x in {#1}{(\x) (lab-\x) } 
		 
	\tikzset{Dom/.style args={#1 (#2) around #3}{dpadded, name=#2, label={[name={lab-#2}] #1}, fit={ #3 } }}
	\tikzset{Dom/.style args={#1 (#2) around #3}{dpadded, name=#2, label={[name={lab-#2}] #1}, fit={ #3 } }}
	\tikzset{bDom/.style args={#1 (#2) around #3}{dpadded, name=#2, label={[name={lab-#2}]below:#1}, fit={ #3 } }}
	\tikzset{arr/.style={draw, ->, thick, shorten <=3pt, shorten >=3pt}}
	\tikzset{archain/.style args={#1}{arr, every arrow subpath/.style={draw,arr, #1}, decoration=arrows, decorate}}
	%\tikzset{every label/.append style={text=red, font=\scriptsize}}
	
%	\newcommand\tikzdom[#1;#2](#3,#4[#5]){
%		\foreach [evaluate=\x as \y using (\x-#2/2)/#5 + #3] \x in {0, 1, ..., #2} {
%			\node[bpt={#1\x | $\n_\x$}] at (\y,#4) {};
%		}
%		\node[Dom={$\sf W$ (W) around \lab{w1}\lab{w3}}] {};
%	}

\usepackage{color}
\definecolor{deepgreen}{rgb}{0,0.5,0}

\usepackage[colorlinks=true, citecolor=deepgreen]{hyperref}


\setlength{\skip\footins}{1cm}
\setlength{\footnotesep}{0.4cm}

\usepackage{parskip}
\usepackage{amsthm, thmtools}
\usepackage{
	nameref,%\nameref
	hyperref,%\autoref
	% n.b. \Autoref is defined by thmtools
	cleveref,% \cref
	% n.b. cleveref after! hyperref
}

\begingroup
\makeatletter
\@for\theoremstyle:=definition,remark,plain\do{%
	\expandafter\g@addto@macro\csname th@\theoremstyle\endcsname{%
		\addtolength\thm@preskip\parskip
	}%
}
\endgroup
\makeatother

\theoremstyle{plain}
\newtheorem{theorem}{Theorem}[section]
\newtheorem{coro}{Corollary}[theorem]
\newtheorem{prop}[theorem]{Proposition}
\newtheorem{lemma}[theorem]{Lemma}
\newtheorem{fact}[theorem]{Fact}
\newtheorem{conj}[theorem]{Conjecture}

\theoremstyle{definition}
\newtheorem{defn}{Definition}[section]
\newtheorem{examplex}{Example}[section]
\newenvironment{example}
	{\pushQED{\qed}\renewcommand{\qedsymbol}{$\triangle$}\examplex}
	{\popQED\endexamplex\vspace{-1em}\rule{1cm}{0.7pt}\vspace{0.5em}}

\theoremstyle{remark}
\newtheorem*{remark}{Remark}

\usepackage{xstring}
\usepackage{enumitem}

%\newcommand\duplicat[1]{\gdef\mylist{}\foreach \x in {#1}{\xdef\mylist{\mylist (\x) (lab-\x) }}\mylist} %% this doesn't work :((
\newcommand\lab[1]{(#1)(lab-#1)}


\newcommand{\todo}[1]{{\color{red}\large\textbf{[todo}: {\normalsize\itshape#1}\textbf{]}}}
\newcommand\geqc{\succcurlyeq}
\newcommand\leqc{\preccurlyeq}
\newcommand\mat[1]{\mathbf #1}
\newcommand{\indi}[1]{\mathbbm{1}_{\left[\vphantom{\big[}#1 \vphantom{\big]}\right]}}
\newcommand\m[1]{\mathbf m_{\mathsf #1}}

\newcommand\recall[1]{ Recall \expandarg\cref{rex:#1}:\vspace{-1em} \begingroup\small\color{gray!80!black}\begin{quotation} \expandafter\csname rex:#1\endcsname* \end{quotation}\endgroup }

%OMG THIS WORKS

\def\wrapwith#1[#2;#3]{
	\expandafter#2{\expandarg\StrBefore{#1}{,}}
	\expandarg\StrBehind{#1}{,}[\tmp] 
	\xdef\tmp{\expandafter\unexpanded\expandafter{\tmp}}
	#3
	\expandarg\IfSubStr{\tmp}{,}{\wrapwith{\tmp}[#2;{#3}]}{ \expandafter#2{\tmp} }
}
\def\hwrapcells#1[#2]{\wrapwith#1[#2;&]}
\def\vwrapcells#1[#2]{\wrapwith#1[#2;\\]}



\newsavebox{\idxmatsavebox}
\def\makeinvisibleidxstyle#1#2{\phantom{\hbox{#1#2}}}
\newenvironment{idxmat}[3][\footnotesize\color{gray}\text]{%
	\def\idxstyle{#1}
	\def\colitems{#3}
	\def\rowitems{#2}
	\begin{lrbox}{\idxmatsavebox}$
	\begin{matrix}  \begin{matrix} \hwrapcells{\colitems}[\idxstyle]  \end{matrix} \\[0.1em]
		\left[ 
		\begin{matrix}
			\hwrapcells{\colitems}[\expandafter\makeinvisibleidxstyle\idxstyle]  \\[-1em]
	}{
		\end{matrix}\right]		&\hspace{-0.5em}\begin{matrix*}[l] \vwrapcells{\rowitems}[\idxstyle] \end{matrix*}
	\end{matrix}
	$\end{lrbox}
	\raisebox{0.75em}{\usebox\idxmatsavebox}
%	\vspace{-0.5em}
}

\newenvironment{sqidxmat}[2][\footnotesize\color{gray}\text]
	{\begingroup\idxmat[#1]{#2}{#2}}
	{\endidxmat\endgroup}
\addbibresource{../refs.bib}
\addbibresource{../maths.bib}

\title{Merged: Dynamic Preferences}
\author{Oliver Richardson  \texttt{oli@cs.cornell.edu}}


\begin{document}
	\maketitle
	
%	\section{}

	\section*{A Quick Overview}
	This project is a model of joint preference and belief revision. The core representation we suggest, the marginal constraint graph, is more flexible than the standard ones, as agents need not always have a perfectly consistent picture of reality, intermediate representations have meanings, and we can capture phenomena such as learning preferences from data and revising them in various places. In special cases, we recover standard results, including expected utility calculations, belief updating by marginalization (or under uncertainty, with Jeffrey's or Pearl's rules).
	
	We can roughly separate the work into two pieces:
	\begin{enumerate}%[nosep]
		\item Representation: the sufficiency of our new representation as a unification of existing results. In particular, we think of an agent's values as distributed across a network of concepts, whose nodes are preferences on small sets of alternatives, and whose edges are beliefs about the impact of one variable on another.
		\item Dynamics: A formulation of how to revise the representation in light of new information or additional computation. We will show how can be dealt with through inconsistency, illustrate some options for regularization, and ultimately factor out these as general meta-preferences. 
	\end{enumerate}
	
	\part{Motivations}
%	This work an be motivated in many different ways; I imagine the most compelling one to depend on who you are and what you already care about. In any individual paper I plan to only include one of these, but until the audience has been narrowed, I want to keep all of them around. %to remind myself which directions I want to push, and to record selling points outside of the best publishing path.

	\begin{center}
	\begin{tikzpicture}
		\node[circle,draw,align=center] (R) at (0,0) {World\\ Representation};
		\node[circle,draw] (B) at (3,1) {Beliefs};
		\node[circle,draw] (V) at (3,-1) {Values};
		\node[circle,draw] (A) at (5.5,0) {Actions};
		
		\draw[arr,-] (B) -- (4,0) -- (V);
		\draw[arr, shorten <=0] (4,0) -- (A);
 	\end{tikzpicture}
	\end{center}

	\section{Good Reasons Not To Maximize Utility}
	\begin{enumerate}
		\item \textbf{No utility function}. Maybe your preferences are not representable in this way --- maybe they violate the vNM axioms, or you just don't have anything that resembles an ordered set. Maybe your intrinsic utility is stochastic.
		
		\item \textbf{No complete prior.} It is possible that you're not a Bayesian, and don't have (access to) a probability distribution on all salient features of your world\footnote{In fact, once we add enough representational complexity to our world, this is impossible to have; see section \ref{sec:impossible-prior}}.  
%		Of course, if you accept Savage's axioms, then you act as though you have both a probability distribution over states and a utility function
		
		\item \textbf{Thinking takes time.} (this constrains the complexity of your utility and probability information, or at least requires them to be in a form that admits fast computations)
		
		\item \textbf{Internal Actions.} In addition to the external actions, there are also invisible mental ones: nobody has complete control over what they do in all respects at all times. Emotionally preparing yourself to talk to your boss, warming up your fingers to play piano, and intentionally committing a phone number to memory are all examples of mental actions, which are rarely modeled. More generally, for any interface with the world that a model prescribes, there will be extra 
		
		 %you may not know exactly what it is you have to do in orde to 
		\item \textbf{Different world representation.} 
%			You have a distorted perception of the world. 
		Models, both in probabalistic reasoning and decision theory, usually start by defining the set of possible states of the system. For a modal logician, this is the underlying set of a Kripke structure; for probability theorists, it's a set of outcomes. Everything works well if you set up a model in which the agents are aware of the structure you've set up, but simply don't know what world they're in --- but nothing precludes them from conceptualizing things in entirely different terms
			
		There are of course ways of putting together
		
		
		\item \textbf{Things Change over Time.} To 
	\end{enumerate}	

	Despite the fact that there has been work backing off of each of these assumptions, utility maximization is still ubiquitous (and taken for granted) in most primary applications, including consumption models and the construction of artificial agents. In section \ref{sec:optimization-bad} I argue that this is the source of some cultural problems. But part of the reason that many of the other techniques has not gained traction is that none is as clean and as useful as expected utility maximization, and to the best of my knowledge people try to isolate and tackle these issues individually. My framework provides a way of representing many of these issues effectively and cleanly, while also reducing to the expected utility computations when none of these issues apply. 


	\section{Desiderata}
	
	
%	Currently preferences are thought of as static objects, fixed as part of the structure and identity of an agent, independent of beliefs, complete, and over some fixed domain. This is clearly not at all how human preferences work, and I posit that it's not the right way to think of preference for synthetic agents either.
	
	Perhaps among other things, any model of preference dynamics that is to be implemented as the heart of a preference system, 
	\begin{enumerate}[noitemsep]
		\item Provide an answer to the `value loading' problem: show how you can learn ``reasonable'' preferences by interacting with the world
		\item Reduce to static models for some parameter settings
		\item Behave reasonably when combined with changes of perspective 
		\item Be resistant to standard challenges to irrationality, such as dutch book arguments
		\item Have weak safety guarantees: an agent should not eagerly adopt preferences which are totally in conflict with its current ones
	\end{enumerate}
	
	%	We might also require that a theory be useful for explaining human behavior. We expect ours to be able to explain:
	Because the view of preferences we adopt here is different from the standard ones in economics (in particular, it lends itself naturally to incorporating boundedness), we have a hope of explaining some behavior which was previously classified irrational, as an optimal in some sense. The following psychological effects lend themselves to explanation:
	\begin{enumerate}[noitemsep]
		\item \textbf{Value Capture.} You really care about $X$ (say, learning math), which is vague and hard to measure, so you come up with some metric $Y$ (say, scores on undergrad math exams). Over time, optimizing for $Y$ will cause you to optimize less effectively for $X$ and assign intrinsic value to $Y$ (getting good exam scores becomes valuable, independent of whether you learn math\footnote{Social signaling plays into this, but this occurs also in cases where people try to hide the signal: constant grade checking, pokemon go addiction, etc.}). Related to Goodhart and Cambell's laws.
		\item \textbf{Framing Effects.} It is well-documented that the style of presentation, even for logically equivalent scenarios, can have a significant impact on a person's choice. But if we formalize these as the same outcome, there's no way for utility-maximizing agent to behave this way.
		\item \textbf{Connoisseur Effects.} Someone who listens to a lot of rap music has more nuanced, complicated preferences on rap music than someone who has not heard as much. Similarly, people work to develop palates for wine, and ``all Indian food tastes the same'' is an insult, indicating a shallow experience with the cuisine. Technically, we want to capture the fact that additional experiences increase preference complexity.
		
		\item \textbf{Adaptive Preferences.} Even things we find objectionable are normalized over time, and people often change to prefer things they're used to, even if they are initially opposed. This is sometimes thought of as a prioritization of safety, and is maybe best thought of as a thought-saving feature: the things you're used to have gotten you this far already. 
		
		\item \textbf{}
		
		\item \textbf{Novelty Effects (anti-adaptive preferences).} 
		
	\end{enumerate}
%	\section{A Temporal Extension}
%	
%	Many issues with decision theory can be attributed to a failure to account for the passage of time. 
%	One popular example of this is the ubiquitous assumption of cognitive unboundedness: in a static world, where everything is frozen except your own thoughts, you can afford to do decision theory exactly by computing expected utility. You have only one decision to make
%	
%	Still, cognitive boundedness is only one example. When we move to a dynamic setting, most classical results no longer apply, though people often like to pretend that they do. 
%	\begin{enumerate}
%		\item There's no reason to 
%	\end{enumerate}
%	
%	If your beliefs
	
	
%	\section{Better Models of Humans}



	\part{Intuitions and Examples}
	So how do we solve these problems?
	
	
	\section{}

	
	\part{Marginal Constraint Graphs}
	
	\begin{defn}\label{def:mcg}
		A marginal constraint graph (MCG) is a tuple 
		\[ \left(\mathcal N : \mathbf{FinSet},~~\mathcal L : 2^{\cal N \times N},~~ \langle\mathcal S, \Sigma\rangle : \mathcal N \to \mathbf{MeasSet}, ~~\mathbf p : \prod_{(A,B): \mathcal L} \left[ \Big. \mathcal S_A \times \Sigma_B \to [0,1]\right] \right) \]
		where $p(L)$ is a Markov kernel, i.e., for every $L[A,B] : \mathcal L$, and $a \in \mathcal S_A$, $\mathbf p_L[a \mid \cdot~]$ is a probability distribution on $(\mathcal S_B, \Sigma_B)$, and $\mathbf p_L[~\cdot \mid B]$ is $\Sigma_A$-measurable for every $B \in \Sigma_B$.
	\end{defn}
%	\part{Intuitions }

	\section{Reduction: Expected Utility}
	\section{Reduction: }

	\part{Values on Marginal Constraint Graphs}
	
	The first strand of this project is a general representation of many ways that have been historically used to represent values: namely, utilities, goals, desires, preferences, choice functions, and natural extensions of these suggested by this representation, some of which are also well-studied. 
	
	\section{Setup}
	
	% There are two key insights / differences:
	% (1) allowing for a variety of scopes: you don't need to supply the data on all possible configurations at once.
	% (2) taking a categorical perspecitve, viewing these data as all special kinds of maps.
	
	The general idea is that to attach some additional preference data to each alternative in a domain
	
	\subsection{Definitions}
	

	
	\section{Reductions}
	
	The standard tool to represent values (and the ones that people are most familiar with) are orders, i.e., flat categories. 
	
	The simplest non-trivial order is:
	\begin{center}
		\raisebox{0.7em}{$\mathbb B = $}~\begin{tikzpicture}
			\node[pt=0] at (0,1){};
			\node[pt=1, right=1.2em of 0] {};
			\coordinate[below=0.3em of 0] (gutter);
			\node[right=0.6em of 0,anchor=center]{$\leqc$};
			\node[draw=gray, rounded corners=2, inner sep=0.5em, fit={\lab{0}\lab{1}(gutter)}] {}; 
		\end{tikzpicture}
	\end{center}
	
	\subsection{Desires}
	
	I want $\varphi$, is 
	
	One standard logical approach would be to give a Kripke model, so that a statment like $s \vDash W_{\alpha} \varphi$, for instance, would be true when agent $\alpha$ wants $\varphi$ in state $s$. In such a model, there is already machinery for talking about the set of all possible states (call it $\cal W$), and so effectively we have specified a function $W_\alpha \varphi: \mathcal W \to \mathbb B$, which assigns 1 to worlds where $\alpha$ wants $\varphi$ and 0 to the others.
	
	This is the behavior we would like to capture, but rather than use the set of possible worlds $\cal W$, we will adopt 
	
	
	\subsection{Utilities}
	
		
	
	A utility function $u : \Omega \to \mathbb R$ is a $\mathbb R$-value on the set of outcomes $\Omega$. 
	
	Effectively, 

	
	\subsection{Preferences}
	We can also consider orders which are not total; a discrete order encodes an inability to compare any of the options, lattices encode a possible inability to compare individual options, but provide a way to formalize the combination the best and worst features of two alternatives.
	
	
	An order is a function $\leq: \Omega \times \Omega \to \mathbbm 2$
	
	\part{Dynamics}
	
	\section{Consistency}
	\subsection{Pairwise Consistency}
	Many of our motivating examples can be thought of as a ``behavioral inconsistency'': you value one thing, but your actions are inconsistent with it. Maybe you value sleep, are aware that to get sleep you need to go to bed early, and yet still do something you know is less valuable instead, such as checking social media. Maybe you 

	Any example where you have to domains like this.
	
	\section{Reduction: Belief Updating}
	
	
	\appendix
	\section{} %todo section titles
	
	\subsection{Can't Have Priors on Everything}\label{sec:impossible-prior}
	
	
%	Just as one can get an order from a utility function, which 
		
	
	
\end{document}