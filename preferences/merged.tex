\documentclass{article}

\usepackage[margin=1in]{geometry}
\usepackage{mathtools} %also loads amsmath
\usepackage{amssymb, bbm}
\usepackage[backend=biber,
	style=alphabetic,
	%	citestyle=authoryear,
	natbib=true,
	url=true, 
	doi=true]{biblatex}

%\usepackage{blkarray} % for matrices with labels
\usepackage{microtype}
\usepackage{relsize}
\usepackage{environ}% http://ctan.org/pkg/environ; for capturing body as a parameter for idxmats
\usepackage{tikz}
	\usetikzlibrary{positioning,fit,calc, decorations, arrows, shapes, shapes.geometric}
	\pgfdeclaredecoration{arrows}{draw}{
		\state{draw}[width=\pgfdecoratedinputsegmentlength]{%
			\path [every arrow subpath/.try] \pgfextra{%
				\pgfpathmoveto{\pgfpointdecoratedinputsegmentfirst}%
				\pgfpathlineto{\pgfpointdecoratedinputsegmentlast}%
			};
	}}
	%%%%%%%%%%%%
	
	\tikzset{dpadded/.style={rounded corners=2, inner sep=0.9em, draw, outer sep=0.4em, fill=gray, fill opacity=0.08, text opacity=1}}
	\tikzset{active/.style={fill=blue, fill opacity=0.1}}
	\tikzset{square/.style={regular polygon,regular polygon sides=4, rounded corners = 0}}
	\tikzset{octagon/.style={regular polygon,regular polygon sides=8, rounded corners = 0}}
	
	
	\tikzset{alternative/.style args={#1|#2|#3}{name=#1, circle, fill, inner sep=1pt,label={[name={lab-#1},gray!30!black]#3:\scriptsize #2}} }
	
	
	\tikzset{bpt/.style args={#1|#2}{alternative={#1|#2|above}} }
	\tikzset{tpt/.style args={#1|#2}{alternative={#1|#2|below}} }
	\tikzset{pt/.style args={#1}{alternative={#1|#1|above}} }
	

	\tikzset{mpt/.style args={#1|#2}{name=#1, circle, fill, inner sep=1pt,label={[name={lab-#1},gray]\scriptsize #2}} }
	\tikzset{pt/.style args={#1}{name=#1, circle, fill, inner sep=1pt,label={[name={lab-#1},gray]\scriptsize #1}} }
	
		
		 %\foreach \x in {#1}{(\x) (lab-\x) } 
		 
	\tikzset{Dom/.style args={#1 (#2) around #3}{dpadded, name=#2, label={[name={lab-#2}] #1}, fit={ #3 } }}
	\tikzset{Dom/.style args={#1 (#2) around #3}{dpadded, name=#2, label={[name={lab-#2}] #1}, fit={ #3 } }}
	\tikzset{bDom/.style args={#1 (#2) around #3}{dpadded, name=#2, label={[name={lab-#2}]below:#1}, fit={ #3 } }}
	\tikzset{arr/.style={draw, ->, thick, shorten <=3pt, shorten >=3pt}}
	\tikzset{archain/.style args={#1}{arr, every arrow subpath/.style={draw,arr, #1}, decoration=arrows, decorate}}
	%\tikzset{every label/.append style={text=red, font=\scriptsize}}
	
%	\newcommand\tikzdom[#1;#2](#3,#4[#5]){
%		\foreach [evaluate=\x as \y using (\x-#2/2)/#5 + #3] \x in {0, 1, ..., #2} {
%			\node[bpt={#1\x | $\n_\x$}] at (\y,#4) {};
%		}
%		\node[Dom={$\sf W$ (W) around \lab{w1}\lab{w3}}] {};
%	}

\usepackage{color}
\definecolor{deepgreen}{rgb}{0,0.5,0}

\usepackage[colorlinks=true, citecolor=deepgreen]{hyperref}


\setlength{\skip\footins}{1cm}
\setlength{\footnotesep}{0.4cm}

\usepackage{parskip}
\usepackage{amsthm, thmtools}
\usepackage{
	nameref,%\nameref
	hyperref,%\autoref
	% n.b. \Autoref is defined by thmtools
	cleveref,% \cref
	% n.b. cleveref after! hyperref
}

\begingroup
\makeatletter
\@for\theoremstyle:=definition,remark,plain\do{%
	\expandafter\g@addto@macro\csname th@\theoremstyle\endcsname{%
		\addtolength\thm@preskip\parskip
	}%
}
\endgroup
\makeatother

\theoremstyle{plain}
\newtheorem{theorem}{Theorem}[section]
\newtheorem{coro}{Corollary}[theorem]
\newtheorem{prop}[theorem]{Proposition}
\newtheorem{lemma}[theorem]{Lemma}
\newtheorem{fact}[theorem]{Fact}
\newtheorem{conj}[theorem]{Conjecture}

\theoremstyle{definition}
\newtheorem{defn}{Definition}[section]
\newtheorem{examplex}{Example}[section]
\newenvironment{example}
	{\pushQED{\qed}\renewcommand{\qedsymbol}{$\triangle$}\examplex}
	{\popQED\endexamplex\vspace{-1em}\rule{1cm}{0.7pt}\vspace{0.5em}}

\theoremstyle{remark}
\newtheorem*{remark}{Remark}

\usepackage{xstring}
\usepackage{enumitem}

%\newcommand\duplicat[1]{\gdef\mylist{}\foreach \x in {#1}{\xdef\mylist{\mylist (\x) (lab-\x) }}\mylist} %% this doesn't work :((
\newcommand\lab[1]{(#1)(lab-#1)}


\newcommand{\todo}[1]{{\color{red}\large\textbf{[todo}: {\normalsize\itshape#1}\textbf{]}}}
\newcommand\geqc{\succcurlyeq}
\newcommand\leqc{\preccurlyeq}
\newcommand\mat[1]{\mathbf #1}
\newcommand{\indi}[1]{\mathbbm{1}_{\left[\vphantom{\big[}#1 \vphantom{\big]}\right]}}
\newcommand\m[1]{\mathbf m_{\mathsf #1}}

\newcommand\recall[1]{ Recall \expandarg\cref{rex:#1}:\vspace{-1em} \begingroup\small\color{gray!80!black}\begin{quotation} \expandafter\csname rex:#1\endcsname* \end{quotation}\endgroup }

%OMG THIS WORKS

\def\wrapwith#1[#2;#3]{
	\expandafter#2{\expandarg\StrBefore{#1}{,}}
	\expandarg\StrBehind{#1}{,}[\tmp] 
	\xdef\tmp{\expandafter\unexpanded\expandafter{\tmp}}
	#3
	\expandarg\IfSubStr{\tmp}{,}{\wrapwith{\tmp}[#2;{#3}]}{ \expandafter#2{\tmp} }
}
\def\hwrapcells#1[#2]{\wrapwith#1[#2;&]}
\def\vwrapcells#1[#2]{\wrapwith#1[#2;\\]}



\newsavebox{\idxmatsavebox}
\def\makeinvisibleidxstyle#1#2{\phantom{\hbox{#1#2}}}
\newenvironment{idxmat}[3][\footnotesize\color{gray}\text]{%
	\def\idxstyle{#1}
	\def\colitems{#3}
	\def\rowitems{#2}
	\begin{lrbox}{\idxmatsavebox}$
	\begin{matrix}  \begin{matrix} \hwrapcells{\colitems}[\idxstyle]  \end{matrix} \\[0.1em]
		\left[ 
		\begin{matrix}
			\hwrapcells{\colitems}[\expandafter\makeinvisibleidxstyle\idxstyle]  \\[-1em]
	}{
		\end{matrix}\right]		&\hspace{-0.5em}\begin{matrix*}[l] \vwrapcells{\rowitems}[\idxstyle] \end{matrix*}
	\end{matrix}
	$\end{lrbox}
	\raisebox{0.75em}{\usebox\idxmatsavebox}
%	\vspace{-0.5em}
}

\newenvironment{sqidxmat}[2][\footnotesize\color{gray}\text]
	{\begingroup\idxmat[#1]{#2}{#2}}
	{\endidxmat\endgroup}
\addbibresource{../refs.bib}
\addbibresource{../maths.bib}

\title{Merged: Dynamic Preferences}
\author{Oliver Richardson  \texttt{oli@cs.cornell.edu}}


\begin{document}
	\maketitle
	
	
	
	
	\section{}
	
%		\part{Motivations and Intuition}
%	This work an be motivated in many different ways; I imagine the most compelling one to depend on who you are and what you already care about. In any individual paper I plan to only include one of these, but until the audience has been narrowed, I want to keep all of them around. %to remind myself which directions I want to push, and to record selling points outside of the best publishing path.

	


	
%	\section{A Temporal Extension}
%	
%	Many issues with decision theory can be attributed to a failure to account for the passage of time. 
%	One popular example of this is the ubiquitous assumption of cognitive unboundedness: in a static world, where everything is frozen except your own thoughts, you can afford to do decision theory exactly by computing expected utility. You have only one decision to make
%	
%	Still, cognitive boundedness is only one example. When we move to a dynamic setting, most classical results no longer apply, though people often like to pretend that they do. 
%	\begin{enumerate}
%		\item There's no reason to 
%	\end{enumerate}
%	
%	If your beliefs
	
	
%	\section{Better Models of Humans}


	
	\part{Belief Representations}
	
	\begin{defn}\label{def:mcg}
		A marginal constraint graph (MCG) is a tuple 
		\[ \left(\mathcal N : \mathbf{FinSet},~~\mathcal L : 2^{\cal N \times N},~~ \langle\mathcal S, \Sigma\rangle : \mathcal N \to \mathbf{MeasSet}, ~~\mathbf p : \prod_{(A,B): \mathcal L} \left[ \Big. \mathcal S_A \times \Sigma_B \to [0,1]\right] \right) \]
		where $p(L)$ is a Markov kernel, i.e., for every $L[A,B] : \mathcal L$, and $a \in \mathcal S_A$, $\mathbf p_L[a \mid \cdot~]$ is a probability distribution on $(\mathcal S_B, \Sigma_B)$, and $\mathbf p_L[~\cdot \mid B]$ is $\Sigma_A$-measurable for every $B \in \Sigma_B$.
	\end{defn}
%	\part{Intuitions }

	\part{Value Representations}
	
	The first strand of this project is a general representation of many ways that have been historically used to represent values: namely, utilities, goals, desires, preferences, choice functions, and natural extensions of these suggested by this representation, some of which are also well-studied. 
	
	\section{Setup}
	
	% There are two key insights / differences:
	% (1) allowing for a variety of scopes: you don't need to supply the data on all possible configurations at once.
	% (2) taking a categorical perspecitve, viewing these data as all special kinds of maps.
	
	The general idea is that to attach some additional preference data to each alternative in a domain
	
	\subsection{Definitions}
	

	
	\section{Reductions}
	
	The standard tool to represent values (and the ones that people are most familiar with) are orders, i.e., flat categories. 
	
	The simplest non-trivial order is:
	\begin{center}
		\raisebox{0.7em}{$\mathbb B = $}~\begin{tikzpicture}
			\node[pt=0] at (0,1){};
			\node[pt=1, right=1.2em of 0] {};
			\coordinate[below=0.3em of 0] (gutter);
			\node[right=0.6em of 0,anchor=center]{$\leqc$};
			\node[draw=gray, rounded corners=2, inner sep=0.5em, fit={\lab{0}\lab{1}(gutter)}] {}; 
		\end{tikzpicture}
	\end{center}
	
	\subsection{Desires}
	
	I want $\varphi$, is 
	
	One standard logical approach would be to give a Kripke model, so that a statment like $s \vDash W_{\alpha} \varphi$, for instance, would be true when agent $\alpha$ wants $\varphi$ in state $s$. In such a model, there is already machinery for talking about the set of all possible states (call it $\cal W$), and so effectively we have specified a function $W_\alpha \varphi: \mathcal W \to \mathbb B$, which assigns 1 to worlds where $\alpha$ wants $\varphi$ and 0 to the others.
	
	This is the behavior we would like to capture, but rather than use the set of possible worlds $\cal W$, we will adopt 
	
	
	\subsection{Utilities}
	
		
	
	A utility function $u : \Omega \to \mathbb R$ is a $\mathbb R$-value on the set of outcomes $\Omega$. 
	
	Effectively, 

	
	\subsection{Preferences}
	We can also consider orders which are not total; a discrete order encodes an inability to compare any of the options, lattices encode a possible inability to compare individual options, but provide a way to formalize the combination the best and worst features of two alternatives.
	
	
	An order is a function $\leq: \Omega \times \Omega \to \mathbbm 2$
	
	\part{Updating}
	
	\section{}
%	Just as one can get an order from a utility function, which 
		
	
	
\end{document}