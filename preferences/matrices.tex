\documentclass{article}

\usepackage[margin=1in]{geometry}
\usepackage{amsmath,amssymb, amsthm}
\usepackage[backend=biber,
style=alphabetic,
%	citestyle=authoryear,
natbib=true,
url=true, 
doi=true]{biblatex}

\addbibresource{../refs.bib}

\usepackage{microtype}
\usepackage{tikz}

\usepackage{color}
\definecolor{deepgreen}{rgb}{0,0.5,0}

\usepackage[colorlinks=true, citecolor=deepgreen]{hyperref}

\newcommand\geqc{\succcurlyeq}
\newcommand\leqc{\preccurlyeq}

\setlength{\skip\footins}{1cm}
\setlength{\footnotesep}{0.4cm}

\theoremstyle{plain}
\newtheorem{theorem}{Theorem}[section]
\newtheorem{coro}{Corollary}[theorem]
\newtheorem{lemma}[theorem]{Lemma}

\theoremstyle{definition}
\newtheorem{defn}{Definition}[section]
\newtheorem{example}{Example}[section]

\theoremstyle{remark}
\newtheorem*{remark}{Remark}

\begin{document}
	\section{Story Examples}
	\subsection{}
	Suppose you could get ice cream from Alice, Bob, both, or not have any ice cream at all. The possible actions are: going to Alice, going to Bob
	
	This creates a game tree
	
	\clearpage
	\section{Preliminaries}
	\subsection{Inclines And Matrix Powers}
	
	
	
	\section{Formalism}
	\subsection{Representation}
	% explanation
	% At each point in time, a representation of the agent's state $(\mathcal D_t, \mathcal B_t)$ consists of a collection of domains $\mathcal D_t$, each of which has some preference data, and a collection of beliefs $\mathcal B_t$ connecting some subset of pairs of domains together.
	
	\begin{defn}
		A \emph{preference domain} $(D, \mathcal V, \mathbf D)$ consists of a set of alternatives $D$, an incline $\mathcal V$, and a $\mathcal V$-valued incline matrix $\mathbf D: D \times D \to \mathcal V$. A preference domain is said to be 
		\begin{itemize}
			\item \textit{transitive} if $\mathbf D^2 \leq D$
			\item \textit{anti-symmetric} if $\mathbf D_{i,j} + \mathbf D_{j,i} = 1$ and $\mathbf D_{i,j} \cdot \mathbf D_{j,i} = 0$
		\end{itemize}
	\end{defn}
		
%	There are multiple ways of nailing down the formalism further by explicating the ordering requirements from the domains, and structure of the beliefs\footnote{For example, we could also have used joint probabilities and undirected models to get a Markov Network, or possibly even used a causal model, where the structural equations are edges, and the nodes are variables. However, this would require a different consistency metric.}; as a first pass, we will assume only that the ordering $\leq_D$ on each domain $D \in \mathcal D_t$ is transitive and reflexive (making it a pre-order), and give the beliefs the structure of a conditional probability distribution:
	
	\[  \mathcal B_t \subseteq \prod_{A, B \in \mathcal D_t} \Big( A \to \Delta B \Big) \] 
	
	i.e., for some subset of pairs of domains $\{ (A_i, B_i) \}_i$, we have a probability distribution $\Delta B_i$ over $B_i$ associated each element of $A_i$. We will write $B [X \to Y]$ or $\Pr(Y | X)$ to denote the conditional distribution $B \in \mathcal B$ over $Y$ whose co-domain is $X$.
	
	
	\begin{remark}
		Fixing a time $t$, if we were to forget the orderings and treat the domains as sets, and the edges formed by $\mathcal B_t$ are acyclic, then $(\mathcal D_t, \mathcal B_t)$ forms a Bayesian Network.
	\end{remark}
	
	
	
	\subsection{Dynamics}
	
	Now, for the dynamic piece. We define the consistency of a link $B : X \to \Delta Y$ as: 	
	\[ \zeta\big(B[X \to Y]\big) =  \sigma \left( \left(1- \frac{|W_X - W_Y|}{W_X + W_Y}\right) \sum_{x,x' \in X}\sum_{y,y' \in Y} X(x,x') Y(y,y') B(x)(y) B(x')(y') \right) \]
	where $\sigma : \mathbb R \to [-1, 1]$ is an activation function such as sigmoid or $\tanh$, $B(y)(x)$ is the probability mass on $y$ in the distribution $B(x)$, and for a domain $D$, $D(d, d')$ is a signed preference indicator, defined as 
	\[ D(d, d') := \begin{cases}
	1 & d \prec_D d' \\
	-1 & d \succ_D d' \\
	0 & \text{otherwise}
	\end{cases} \]
	
	We can also sum across the entire graph to get a measure for the whole model:
	\[ \zeta(\mathcal D, \mathcal B, W) = \sum_{B[X\to Y] \in \mathcal B}~ \zeta(B) \]
	
	
	\begin{remark}
		If $\mathcal D$ consists of only a dingle domain $D$, with the identity distribution $\mathcal B = {B(x,y) = \delta_{x,y}: D \to D}$, then $\mathcal D$ is consistent, i.e., $\zeta(\mathcal D, \mathcal B, W) = 1$.
	\end{remark}
	
	Finally, the update rule is just backpropagation of the consistency through the parameters: for each parameter 
	\[ p \in \bigcup_{B[X\to Y] \in \mathcal B}  \Bigg( \{ X(x,x') | x, x' \in X\} \cup \{ Y(y, y') | y,y' \in Y\} \cup \{ W_X, W_y \} \cup \{B(x)(y) | x \in X, y \in Y\}  \Bigg) \]
	
	\[ \frac{\partial \zeta(B) }{\partial B(x)(y) } \]
	
	\subsection{Explanation of Update Rule}	 
	
	
	\begin{theorem}
		
	\end{theorem}

%	$\mathbf A$ is skew symmetric if and on
%	
%	\begin{align*}
%		\mathbf A + \mathbf A^T = 0
%	\end{align*}
%	
\end{document}