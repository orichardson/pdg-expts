\documentclass{article}

\usepackage[margin=1in]{geometry}
\usepackage{amsmath,amssymb, amsthm}
\usepackage[backend=biber,
style=alphabetic,
%	citestyle=authoryear,
natbib=true,
url=true, 
doi=true]{biblatex}

\addbibresource{../refs.bib}

\usepackage{microtype}
\usepackage{tikz}

\usepackage{color}
\definecolor{deepgreen}{rgb}{0,0.5,0}

\usepackage[colorlinks=true, citecolor=deepgreen]{hyperref}

\newcommand\geqc{\succcurlyeq}
\newcommand\leqc{\preccurlyeq}

\setlength{\skip\footins}{1cm}
\setlength{\footnotesep}{0.4cm}

\theoremstyle{plain}
\newtheorem{theorem}{Theorem}[section]
\newtheorem{corollary}{Corollary}[theorem]
\newtheorem{lemma}[theorem]{Lemma}

\theoremstyle{definition}
\newtheorem{defn}{Definition}[section]
\newtheorem{example}{Example}[section]

\theoremstyle{remark}
\newtheorem*{remark}{Remark}

\begin{document}
	
	\section{Preliminaries}
	\subsection{Inclines And Matrix Powers}
	
	
	
	\section{Formalism}
	 $\mathcal D$. Each element of $\mathcal D$ is a set $D$ equipped with an incline $\mathcal V$ and a function $\mathbf D: D \times D \to \mathcal V$.
	
	
	

%	$\mathbf A$ is skew symmetric if and on
%	
%	\begin{align*}
%		\mathbf A + \mathbf A^T = 0
%	\end{align*}
%	
\end{document}