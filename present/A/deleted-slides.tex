
%%%%%%%%%% Semantics %%%%%%%%%%
\begin{frame}<1-5>[label=semantics] %%% SEMANTICS %%%
	\frametitle{PDG Semantics}
	{	\setbeamersize{description width=1.2cm}
		\setbeamercovered{transparent=30}
		\begin{description}
			\item<uncover@1,2,9,10> [\only<9->{\color{structurecolor}1.~}\alert<2,9,10>{$\SD{\dg M}$}] %\notation{$:\mathcal P{\Delta \V(\dg M)}$}
				The set of joint distributions consistent with $\dg M$;
				\only<2-8>{\visible<2->{\uncover<2>{
					\[ \Big\{ \mu%~{\color{gray!60} \in \Delta[\V(\dg M)]}~
						\in \Delta[\V(\dg M)]:~\text{for all } \ed LXY \in \Ed.~~\mu(Y\!\mid\!X) = \bp(Y\!\mid\! X)  \Big\} \]
					}}}
			\uncover<1,3,9>{\item [\only<9->{\color{structurecolor}2.~}\alert<3,9>{$\bbr{\dg M}_\gamma$}] %\notation{$:{\Delta \V(\dg M)} \to \mathbb R$}
				\alt<1>{A function, scoring distributions by}{
					A loss function (parameterized by $\gamma$), scoring a joint distribution's}
				 compatibility with $\dg M$;

				\only<2-8>{\visible<3->{\uncover<3>{
					\alt<-8>{ \[
						\bbr{\dg M}_\gamma(\mu) :=
							{\color{benchcolor1}\underbrace{
								\Inc_{\dg M}(\mu) }_{\left(\parbox{0.8in}%
									{\centering\tiny quantitative\\[-0.2em]\tiny term}\right)}} +
									{\gamma} \extra<3>[above right=1em and -1 em of call point, 
										inner sep=0.3em, font=\small, name=tp, draw=alertcolor, text=alertcolor!40!black,
										fill=alertcolor!10]{\small tradeoff parameter $\gamma \ge 0$}
							{\color{benchcolor2}\underbrace{\IDef{\dg M}(\mu) }_{\left(\parbox{0.75in}%
									{\centering\tiny qualitative\\[-0.2em]\tiny term}\right)}}
						\]}{\[
							\bbr{\dg M}_\gamma(\mu) := \Inc_{\dg M}(\mu) + \gamma\;	\IDef{\dg M}(\mu)
					\] }}}}}	
				\only<3>{\tikzro{\draw[alertcolor] ([yshift=0.5em,xshift=-0.3em]call point.center) -- ([xshift=1.5em]tp.south west);}}	
			\only<7,8>{
				\vspace{-5em}
				\begin{prop}[{\itshape\normalfont {uniqueness for small $\gamma$}}]
					\begin{enumerate}
					\item If  $0 < \gamma \leq \min_L \beta_L^{\dg M}$, then
					$\bbr{\dg M}_\gamma^*$ is a singleton.
					\item $\displaystyle\lim_{\gamma\to0}\bbr{\dg M}_\gamma^*$ exists and is unique.
					\end{enumerate}
				\end{prop}	}

			\item<uncover@1,4,6-8,10> [\only<9->{\color{structurecolor}3.~}\alert<4,6-8,10>{$\bbr{\dg M}^*\only<4->{_\gamma}$}] %\notation{$:{\Delta \V(\dg M)}$}
				The \only<8>{{\color{benchcolor1}(unique)}} ``best'' joint distribution\only<8>{{\small\color{benchcolor1}~(in the quantitative limit)}};
				\only<2-8>{\visible<4-8>{\uncover<4,7,8>{
					\[ \bbr{\dg M}\only<4->{_\gamma}^* := \only<8>{{\color{benchcolor1}\lim_{\gamma\to 0}}} \argmin_\mu \bbr{\dg M}_\gamma(\mu) \]
					}}}
					
			\item<uncover@1,5> [\only<9->{\color{structurecolor}4.~}\alert<5>{$\aar{\dg M}_\gamma$}]
				% The smallest incompatibility of $\dg M$ with any distribution (inconsistency).
				The inconsistency of $\dg M$ (best possible score).
				\only<2-8>{\visible<5-8>{\uncover<5>{
					\[ \aar{\dg M}_\gamma := \inf_\mu \bbr{\dg M}_\gamma(\mu) \]
				}}}
			
			
				
		\end{description}	}

		% \only<1> {
		%
		% 	}
	

		\only<8->{
			\begin{prop}[{\normalsize{\it the second semantics extends the first\;}}]
					$\SD{\dg M} \!= \big\{ \mu : \bbr{\dg M}_0(\mu) \!=\! 0 \big\}$.
				\end{prop}

			\onslide<9->{
				\begin{prop}[{\normalsize{\it If there there are distributions consistent with $\dg M$, the best distribution is one of them.\;}}]\label{prop:consist}
						$\bbr{\dg M}^* \in \bbr{\dg M}_0^*$, so if $\dg M$ is consistent,
						then $\bbr{\dg M}^* \in \SD{\dg  M}$.
					\end{prop}
				}
		}
	\end{frame}


%%%%%%%%%%%%%%%%%%%%   SCORING FUNCTION  %%%%%%%%%%%%%%%%
\relax % frame setup
	\colorlet{localcolor}{Lavender!80!black}
	\colorlet{globalcolor}{Sepia!80!orange}
	% \def\INCfst{2}
	\def\INCmotfst{2}
		\def\INCmotlen{4}
	
	\edef\INCfst{\the\numexpr \INCmotfst + \INCmotlen \relax}
		\def\INClen{5}
		\edef\INClst{\the\numexpr\INCfst+\INClen-1\relax}
		\def\INCrange{\INCfst-\INClst}
		\def\INC+#1{\atslide{\INCfst+#1}}
	
	% \def\INCexfst{7}
	\edef\INCexfst{\the\numexpr \INClst + 1 \relax}
		\def\INCexlen{0}
		\edef\INCexlst{\the\numexpr\INCexfst+\INCexlen-1\relax}
		\def\INCexrange{\INCexfst-\INCexlst}
		\def\INCex+#1{\atslide{\INCexfst+#1}}
	
	\edef\IDEFmotfst{\the\numexpr \INCexlst + 1 \relax} 
		\def\IDEFmotlen{3}
		\def\IDEFmot+#1{\atslide{\IDEFmotfst+#1}}
	
	% \def\IDEFfst{12}
	\edef\IDEFfst{\the\numexpr \IDEFmotfst + \IDEFmotlen \relax} 
		\def\IDEFlen{3}
		\edef\IDEFlst{\the\numexpr\IDEFfst+\IDEFlen-1\relax}
		\def\IDEFrange{\IDEFfst-\IDEFlst}
		\def\IDEF+#1{\atslide{\IDEFfst+#1}}
	
	% \def\IDEFexfst{15}
	\edef\IDEFexfst{\the\numexpr \IDEFlst + 1 \relax}
		\def\IDEFexlen{5}
		\edef\IDEFexlst{\the\numexpr\IDEFexfst+\IDEFexlen-1\relax}
		\def\IDEFexrange{\IDEFexfst-\IDEFexlst}
		\def\IDEFex+#1{\atslide{\IDEFexfst+#1}}
	
	% \def\INCandIDEF{19}
	\edef\INCandIDEFfst{\the\numexpr\IDEFexlst + 1 \relax}
		\edef\INCandIDEF{\INCandIDEFfst-}
		\def\IIboth+#1{\atslide{\IDEFexlst + 1 + #1}}
		\edef\beforeINCandIDEF{-\the\numexpr\INCandIDEFfst - 1\relax}
		% \show\INCandIDEF
	\def\atslide#1{\the\numexpr#1\relax}
	\def\aitem@#1+#2{\item<\atslide{\csname#1fst\endcsname+#2}-|alert@\atslide{\csname#1fst\endcsname+#2}>}
	\def\halfblocks{\INCexrange,\IDEFexrange,\INCandIDEF}
\begin{frame}[t,label=semantics2] %%%
	\frametitle{The Scoring Function }
	\only<\IIboth+1->{
		\begin{itemize}
			\item<\IIboth+1-|alert@\IIboth+1> A BN strictly enforces the qualitative picture (large $\gamma$)
			\item<\IIboth+2-|alert@\IIboth+2> we are interested in the quantitative limit (small $\gamma$)
		\end{itemize}			
		\bigskip
	}
	
	{\alt<\INCexrange,\IDEFexrange>{\vspace{-0.5em}\small}{\centering}\only<\INCandIDEF>{\centering}
		$\displaystyle%
			\bbr{\dg M}_\gamma(\mu) := \hl[alertcolor!30!benchcolor1!99!structurecolor]%
					<2-\INClst,\INCexrange,\INCandIDEF>{\Inc_{\dg M}(\mu)}
				+ \gamma%
				\extra<\IIboth+0-\IIboth+1>[below right=0.5em and -1 em of call point, 
					inner sep=0.3em, font=\small, name=tp, draw=alertcolor, text=alertcolor!40!black,
					fill=alertcolor!10]{\small tradeoff parameter $\gamma \ge 0$}
				%
				\;\hl[alertcolor!30!benchcolor2!99!structurecolor]<\IDEFmotfst->{\IDef{\dg M}(\mu)}$\par}
			\only<\IIboth+0->{\tikzro{%
					\draw[alertcolor] ([yshift=-0.2em,xshift=-0.3em]call point.center) -- ([xshift=1.0em]tp.north west);}}	

	{\medskip}
	\setlength\pdgdefnwidth{0.95\textwidth}
	\only<\halfblocks>{\setlength\pdgdefnwidth{0.43\textwidth}}
	
	\only<2-\atslide{\INCfst-1}> {
		\smallskip
		{\centering\parbox{0.8\textwidth}{\raggedright
			\textbf{Intuition:}
			% beyond stating whether or not $\mu$ is consistent with $\dg M$, we 
			Measure the extent to which $\mu$ violates the constraints $\{\bp[]\}$ of $\dg M$. 
		}\par}
	
		{\color{structurecolor}\scshape{Motivating Examples}.}
		% \begin{center}
		\hfill
		$\dg M := \quad$
		\begin{tikzpicture}[paperfig]
			\node[dpadded] (1) {$\pdgunit$};
			\node[dpadded, right=1 of 1] (X) {$X$};
			\draw[arr1] (1) to[bend left=30] node[above] {$q$} (X);
			\draw[arr1] (1) to[bend right=30] node[below] {$p$} (X);
		\end{tikzpicture}
		\hfill~
		% \end{center}
		% \vspace{-1em}
		
		\begin{itemize}
			% \aitem@{INCmot}+1 If $p = \begin{idxmat}{$\star$}{$x_1$,$x_2$} .4 & .6 \end{idxmat} = q$, then $\dg M$ is consistent, and compatible with the joint distribution $\mu(X) = p$. 
			\aitem@{INCmot}+1 If $p = q$, then $\dg M$ is clearly consistent, and compatible with the joint distribution $\mu(X) = p = q$, so $\Inc_{\dg M}(p) = 0$. 
			\aitem@{INCmot}+2 If $p = \begin{idxmat}{$\star$}{$x_1$,$x_2$} .4 & .6 \end{idxmat}$ and 
					$q = \begin{idxmat}{$\star$}{$x_1$,$x_2$} .5 & .5 \end{idxmat}$, 
					then $\dg M$ is not consistent, but 
					
					$\mu = \begin{bmatrix}%{$\star$}{$x_1$,$x_2$}
						.45 & .55 \end{bmatrix}$ matches better than 
					$\mu = \begin{bmatrix} .9 & .1 \end{bmatrix}$.
					\medskip

			\aitem@{INCmot}+3 If $p = \begin{bmatrix} .4 & .6 \end{bmatrix}$ and 
					$q = \begin{bmatrix} 0 & 1 \end{bmatrix}$, 
					then $\dg M$ is much more inconsistent than before, even though $\SD{\dg M} = \emptyset$ in both cases.
		\end{itemize}
	}
		
	\begin{columns}[c]
		\only<\INCrange,\INCexrange,\INCandIDEF>{{ %%%%%%%%%% INC %%%%%%%%%%%%%%%
			\begin{column}{\pdgdefnwidth}
			\setbeamercolor{block title}{bg=benchcolor1!80!structurecolor!50}
			\setbeamercolor{block body}{bg=benchcolor1!80!structurecolor!20}
			\setbeamercovered{transparent=35}
			\only<\INCexrange>{\setbeamertemplate{blocks}[rounded][shadow=false]}
			\only<\INCexrange,\INCandIDEF>{\small}
			\begin{defn}[$\Inc$]<\INCrange,\INCandIDEF>\label{def:inc}
			The \emph{incompatibility} of
			\only<\INCrange>{a joint distribution }$\mu$ with $\dg M$\alt<\INCrange>{ is given by}{:}
			\def\fullnotate{\atslide{\INCfst+3}}
			\alt<-\INC+2,\INCex+0->{ \[ % The simpler, less clear notation
					\Inc_{\dg M}(\mu) :=
						\sum_{\mathclap{\ed LXY}}\; \visible<\INC+1->{\alert<\INC+1>{\beta_L}} \;
						\tikzmark{relentD}
						\kldiv{ \mu_{Y|X} }{ \bp }
				\] }{ \[ % The notation used in our paper
					\Inc_{\dg M }( \mu) :=
						\sum_{\mathclap{\ed LXY}} \beta_L \alert<\INC+3>{ \Ex_{x \sim \mu_{_X}} }
						\tikzmark{relentD}
						\kldiv[\Big]{ \mu \alert<\INC+3>{(Y \mid X \!=\! x)} }{\bp \alert<\INC+3>{(x)} }. \] }
			%
			\only<\INC+2-\INC+3>{
				\extra[name=Dexplain, below=1cm of pic cs:relentD, align=right,
							fill=benchcolor1!40!structurecolor!20!white]{%
						$\displaystyle\kldiv\mu\nu = \sum_{\mathclap{w \in \mathop{Supp}(\mu)}} \mu(w) \log \frac{\mu(w)}{\nu(w)}$
						is the relative entropy \\[-1.2em] from $\nu$ to $\mu$.
					}
				\tikzro \draw[thick,structurecolor!70!black,draw opacity=0.5]
					([xshift=0.5em,yshift=-0.2em]pic cs:relentD) -- ([xshift=0.5em]Dexplain.north); }
			% {\setbeamercovered{invisible}
			% 	\onslide<\INClst->{
			% 		The \emph{inconsistency} of $\dg M$ is \only<-\INClst>{the smallest possible incompatibility,}
			% 		\[ \Inc(\dg M) := \!\!\inf_{ \mu \in \Delta\V(\dg M)}\!\! \Inc_{\dg M}(\mu) . \]	 }}
			\end{defn}
			\end{column}
			}}
		\only<\INCexrange>{
			\column{\textwidth-\pdgdefnwidth}
			\vspace{-1em}
			{\centering\Large\color{structurecolor}\textsc{Examples}\par}
			\begin{center}
				\begin{tikzpicture}[throughlab/.style={fill=white, fill opacity=1,text opacity=1,inner sep=1pt}]
					\node[dpadded,thick] (C) at (0,0) {$\mathit C$};
					\node[dpadded,thick] (T) at (2,0){$\mathit T$};
					\draw[arr,draw=colororiginal,text=black] (T) to[bend right] node[throughlab]{$q$} (C);
					\draw[arr, draw=colorsmoking] (T) to node[throughlab]{$p$} (C);
					\end{tikzpicture}
				\end{center}
			
			\setlength{\leftmargini}{1em}
			\begin{itemize}
				\aitem@{INCex}+1 if $p = q$, then $\Inc(\dg M) = 0$. 
				\aitem@{INCex}+2 if {\small $p =\begin{idxmat}%
							[\alt<\INCex+2>{\color{alertcolor!40}}{\color{gray!50}}\smalltext]
						{$t$,$\bar t$}{$c$} 0.4 \\ 0.1 \end{idxmat}$} and
					{\small $q = \begin{idxmat}%
							[\alt<\INCex+2>{\color{alertcolor!40}}{\color{gray!50}}\smalltext]
						{$t$,$\bar t$}{$c$} 0.9 \\ 0.1 \end{idxmat}$}, then \\[0.05em] still $\Inc(\dg M) = 0$ 
						(maybe nobody uses tanning beds).\vspace{0.4em}

				\aitem@{INCex}+3 {\small Not all inconsistencies are equally bad; \\ 
					\!\!$\Inc\!\left\{\!p \shorteq\!\! \begin{bmatrix} .55 \\ .12 \end{bmatrix}\!\!;
					q \shorteq\!\! \begin{bmatrix} .45 \\ .15 \end{bmatrix}\! \right\}\! < \! 
					\Inc\! \left\{ \!p \shorteq\!\!\begin{bmatrix} .3 \\ .4 \end{bmatrix}\!\!;
					q \shorteq\!\! \begin{bmatrix} .9 \\ .1 \end{bmatrix}\! \right\}\!$}
	
				\end{itemize}
		}
		\only<\IDEFmotfst-\atslide{\IDEFfst-1}>{{
			\begin{column}{0.75\textwidth}
			
			\bigskip
			
			\textbf{Intuition:} each edge $\ed LXY$ indicates that $Y$ is determined (perhaps noisily) by $X$ alone. 
			
			\bigskip
			\bigskip
			\bigskip

			\onslide<\IDEFmot+1->{				
				So a $\mu$ with
				\tikzmark{beg-ce}%
					\hl[localcolor]<\IDEFmot+2->{uncertainty in $Y$ after $X$ is known}\tikzmark{end-ce}
				(beyond 
				\tikzmark{beg-e}\hl[globalcolor]<\IDEFmot+2->{pure noise}\tikzmark{end-e})
				is qualitatively worse.  
				
				\only<\IDEFmot+2>{
					\tikzro{ \draw[very thick,localcolor] ([yshift=0.8em]pic cs:beg-ce) -- %
						([yshift=1em]pic cs:beg-ce) -- 
							node[above,inner sep=2pt] {{\small\itshape measured by $\H(Y\mid X)$}} 
						([yshift=1em]pic cs:end-ce) -- ([yshift=0.8em]pic cs:end-ce);}
				
					\tikzro{ \draw[very thick,globalcolor] ([yshift=-0.40em]pic cs:beg-e) --
							([yshift=-0.60em]pic cs:beg-e) -- node[below,inner sep=2pt] {$\H(\mu)$} 
							([yshift=-0.60em]pic cs:end-e) -- ([yshift=-0.40em]pic cs:end-e);}
				}
			}
			
			\end{column}
			}}
		\only<\IDEFfst->{{ %%%%%%%%%% IDEF %%%%%%%%%%%%%
			\begin{column}{\pdgdefnwidth}
			\setbeamercolor{block title}{bg=benchcolor2!80!structurecolor!50}
			\setbeamercolor{block body}{bg=benchcolor2!80!structurecolor!20}
			\setbeamercovered{transparent=35}
			\only<\IDEFexrange>{\setbeamertemplate{blocks}[rounded][shadow=false]\small}
			\only<\IDEFexrange,\INCandIDEF>{\small}
			\begin{defn}[$\IDef{}$]<-\IDEFlst,\INCandIDEF>
				\alt<-\IDEFlst>
					{The \emph{information deficit} of a distribution $\mu$ with respect to $\dg M$ is \vspace{4.2em}}
					{The $\dg M$-\emph{information deficit} of $\mu$: \vspace{2.0em}}
				
				\def\idefdef{\sum_{\mathclap{\ed LXY}} \alt<\IDEFexfst->{\alpha\ssub L\!}{\alpha_L} \H_\mu(Y\!\mid\! X)}
				\[ \!\IDef{\dg M}(\mu) \alt<\IDEFexfst->{\!=\!}{:=}
					\alt<\IDEF+2->{{\color{localcolor} \overbrace{\idefdef}^{\tikzmark{lmark}}}}{ \idefdef }
						-
					\alt<\IDEF+1->{{\color{globalcolor}\underbrace{\H(\mu)}_{\tikzmark{gmark}}}}{ \H(\mu) } \alt<\halfblocks>{}{.}  \]
				\only<\IDEF+2->{% Overlay: (b) # bits to separately determine each target, knowing src...
					\extra[above left=0.6em and 1.2em of {pic cs:lmark},
						onslide=<\halfblocks>{inner sep=0.5em},
						anchor=south,align=center,
						name=idefseparate,
						fill=structurecolor!20!localcolor!40!white,
						text=localcolor!50!black]%
							{\alt<\halfblocks>{}{(b)} \# bits \alt<\halfblocks>{}{required }to separately
							determine\\ each target, knowing the source}
					\tikzro \draw[thick,draw=structurecolor!20!localcolor!70!black,draw opacity=0.6]
						(pic cs:lmark) -- ([xshift=1.2em]idefseparate.south);
					}
				\only<\IDEF+1->{% Overlay: (a) # bits to determine all variables...
					\extra[below right=1em and 2em of {pic cs:gmark},
						onslide=<\halfblocks>{below = 1em of pic cs:gmark, inner sep=0.5em},
						anchor=north east,name=entropy,
						fill=structurecolor!20!globalcolor!40!white,text=globalcolor!50!black]%
							{\alt<\halfblocks>{}{(a)} \# bits \alt<\halfblocks>{}{needed }to determine all \alt<\halfblocks>{vars}{variables}}
					\tikzro \draw[thick,draw=structurecolor!20!globalcolor!70!black,draw opacity=0.6]
						([yshift=0.5em]{pic cs:gmark}) -- ([xshift=-2em]{entropy.north east});
					}
				
				\alt<\halfblocks>{\vspace{1.8em}}{\vspace{3em}}
			\end{defn}
			\end{column}}}
		\only<\IDEFexrange>{ %%%%%%%%%% IDEF EXAMPLES %%%%%%%%%%
			\column{\textwidth-\pdgdefnwidth}
			\vspace{-4em}

			{\centering\Large\color{structurecolor}\textsc{Examples}\par}

			\medskip
			\setlength{\leftmargini}{1.5em}
			\begin{itemize}
				% \setbeamercolor{alerted text}{fg=alertcolor!50!black}
				\aitem@{IDEFex}+0 
					{$\dg M_0=\quad$\color<\IDEFex+0>{alertcolor!60!black}\begin{tikzpicture}[paperfig]
						\node[dpadded] (X) {$X$};
						\node[dpadded,right=0.8 of X] (Y) {$Y$};
					\end{tikzpicture}\hfill}~\\[0.3em]
					{\small\!$\IDef{\dg M_0}\!(\mu)\! =\!  - \H_\mu(X,Y)$}
					{\footnotesize\color{gray}\color<\IDEFex+0>{alertcolor!70!benchcolor2}% 
					\\[-0.3em]~~(optimal $\mu$ maximizes entropy of $X,Y$)}~ \\[0.2em]
				\aitem@{IDEFex}+1
					{$\dg M_1=\quad$\color<\IDEFex+1>{alertcolor!60!black}\begin{tikzpicture}[paperfig]
						\node[dpadded] (X) {$X$};
						\node[dpadded,right=0.8 of X] (Y) {$Y$};
						\draw[arr2] (X) -- (Y);
					\end{tikzpicture}\hfill}~\\[0.3em]
					{\small\!$\IDef{\dg M_1}\!(\mu)\! =\! -\!\H_\mu(X)$}
					{\footnotesize\color{gray}\color<\IDEFex+1>{alertcolor!70!benchcolor2}\\[-0.3em]%
						~~(optimal $\mu$ maximizes entropy of $X$)}~  \\[0.2em]
				\aitem@{IDEFex}+2
					{$\dg M_2=\quad$\color<\IDEFex+2>{alertcolor!60!black}\begin{tikzpicture}[paperfig]
						\node[dpadded] (X) {$X$};
						\node[dpadded,right=0.8 of X] (Y) {$Y$};
						\draw[arr2] (X) to[bend left=15] (Y);
						\draw[arr2] (X) to[bend right=15] (Y);
					\end{tikzpicture}\hfill}~\\[0.3em]
					{\small\!$\IDef{\dg M_2}\!(\mu)\! =\!  - \H_\mu(X) + \H_\mu(Y\mid X)$}
					{\footnotesize\\\color{gray}\color<\IDEFex+2>{alertcolor!70!benchcolor2}%
						~~(optimal $\mu$ maximizes entropy for $X$, and\\[-0.4em]~~~~ makes $Y$ a function of $X$)}  \\[0.2em]
				\aitem@{IDEFex}+3
					{$\dg M_3=\quad$  \color<\IDEFex+3>{alertcolor!60!black}\begin{tikzpicture}[paperfig]
						\node[dpadded] (X) {$X$};
						\node[dpadded,right=0.8 of X] (Y) {$Y$};
						\draw[arr2] (X) to[bend left=10] (Y);
						\draw[arr2] (Y) to[bend left=10] (X);
					\end{tikzpicture}\hfill}~\\[0.3em]
					{\small\!$\IDef{\dg M_3}\!(\mu)\! =\!  - \I_\mu(X;Y)$}
					{\footnotesize\\[-0.3em]\color{gray} \color<\IDEFex+3>{alertcolor!70!benchcolor2}%
						(opt. $\mu$ makes $X,Y$ share information)}
			\end{itemize}
			
			\onslide<\IDEFex+4>
				{\hyperlink{idefextra}{\beamerbutton{Information Diagrams}}
				\hypertarget<\IDEFex+4>{returntosemantics}{}}
		}
		\end{columns}
	\end{frame}

	\againframe<6->{semantics}
