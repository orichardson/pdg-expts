
	\colorlet{hypergraphcolor}{benchcolor1!85!black}
	\colorlet{simplegraphcolor}{structurecolor}
% Hypergraphs and Graphs
\begin{frame} %%%%%         HYPER-GRAPHS AND GRAPHS      %%%%%%
	\frametitle{{\color{hypergraphcolor}Hyper-graphs?} Or merely {\color{simplegraphcolor!85}graphs}?}
	\hfill
	\begin{tikzpicture||precompiled}[center base,draw=hypergraphcolor]{grok-pre}
		\fill[fill opacity=0.07,hypergraphcolor, draw, draw opacity=0.2]
			(-0.6,0.6) rectangle (2.1, -2.1);

		\node[dpadded] (C) at (0,0) {$\mathit C$};
		\node[dpadded] (T) at (1.5,0){$\mathit T$};
		\node[dpadded] (SL) at (.75,-1.5){$\it SL$};

		\draw[arr] (T) to[bend right]  (C);
		\alert<2>{ \mergearr{C}{T}{SL} }
		% \drawbb
		\end{tikzpicture||precompiled}
	\onslide<2->{
		\hfill
		\begin{tikzpicture||precompiled}[center base]{widget}%[center base]
		\fill[fill opacity=0.07,simplegraphcolor, draw, draw opacity=0.2]
			(-2.1,-1.85) rectangle (2.1, 1.6);

			\node[dpadded] (SL) at (0,-1.3) {$\mathit{SL}$};

			\node[dpadded] (C) at (-1.5,1) {$\mathit C$};
			\node[dpadded] (T) at (1.5,1) {$\mathit T$};

			\alert<2>{
				\node[dpadded,light pad] (CT) at (0, 0){$\scriptstyle C \times T$};
				\draw[arr1, ->>] (CT) -- (C);
				\draw[arr1, ->>] (CT) -- (T);
				\draw[arr1] (CT) -- (SL);
				}
			\draw[arr] (T) to [bend right=15] (C);
			% \drawbb
			\end{tikzpicture||precompiled}
		}
		\hfill~
	\onslide<3->{
		\vspace{1em}
		\begin{itemize}
			\item<3-|alert@+> This widget expands state space, but graphs are simpler.
			\item<4-> \alert<4>{There is a natural correspondence}
			\[ \text{\color{hypergraphcolor} joint distributions}\quad\leftrightarrows\quad\parbox{15em}{\color{simplegraphcolor}\centering expanded joint distributions\\ satisfying coherence constraints} \]
		\end{itemize}
		}
	\onslide<5->
		{\small\color{gray} (working directly with hypergraphs is also possible)}
	\end{frame} %----------

% Details of PDG definition componnets & subsets
\begin{frame} %%%%%%%%%%%%% Definition of PDG, details (removed for AAAI) %%%%%%%%%%%
	\colorlet{notationcolor}{benchcolor2!40!alertcolor}
	\setbeamercolor{block body}{bg=structurecolor!5!black}
	\setbeamercolor{block title}{bg=structurecolor!80!black,fg=structurecolor!30!white}
	\setlength{\pdgdefnwidth}{0.7\textwidth}
	
	\vspace{-0.5em}
	\begin{columns}[t]
		\hfill
		\column{\pdgdefnwidth}
		\begin{defn}[PDG]\label{def:model}
			\begin{description}%
				\item[$\N$]<alert@1-5> \notation{$:\Set$}%
					\hfill (node set)
					\begin{description}
						\item[$\V$]<hideme@2,3> \notation{$:\N \to \Set$}<1,4-5>%
							\hfill (node values)%
						\end{description}

				\item[$\Ed$]<hideme@1 |alert@2-5> \notation{$\subseteq \N \times \N \times \mathit{Label}~~~$}<2->%
					\hfill (edge set)\\[0.14em]
					{\color{gray}For $\ed LXY \in \Ed$,}
					%, each with a source $X$ and target $Y$ in $\N$;
					\begin{description}%[<+-| alert@+>]
						\item[$\bp$]<hideme@1-3 | alert@4-5> %:\big(\!(X,Y,\ell)\in\!\Ed \big) \to
							\notation{$:\V(X) \to \Delta\V(Y)$}<4-5>%
							\hfill (edge cpd)
						\item[$\alpha\ssub L$]<hideme@1-2,4-5> \notation{$: [0,1]$}<3>%
							\hfill(qual. edge  confidence)
						\item[$\beta\ssub L$]<hideme@1-5> \notation{$: \mathbb R^+$}<0>%
							\hfill(quant. edge confidence)
						\end{description}
				\end{description}
			\end{defn}
			% \column{\textwidth-\pdgdefnwidth}
			\hfill
		\end{columns}
		
	\vspace{-0.5em}
	\begin{itemize}
		\onslide<1>{\item<1-|alert@+> $(\N, \V) \cong $ \texttt{Set<Variable>}\\[-1em]}
		\item<2-|alert@+> $(\N, \Ed) \cong $ \texttt{MultiGraph}
		\item<3-|alert@+> $(\N, \Ed, \alpha)$ is a qualitative PDG.
		\item<4-> \alert<+>{We call $(\N, \Ed, \V, \mat p)$ an \emph{unweighted} PDG}
			\begin{itemize}
					\item and give it semantics as though $\alpha_L = \beta_L = 1$.
					% \item $(\N, \Ed, \V)$ are (mostly) implicit in $\mat p$,
					% 	and so \\ ``collection of cpds'' $\cong$ \texttt{UnweightedPDG}.
				\end{itemize}
		\end{itemize}
	
	\begin{alertblock}<5>{Aside}
		An unweighted PDG is a functor
		$\langle \mat p, \V\rangle\colon \mathit{Free}(\N, \Ed) \to \mathbf{Mark}$
		\end{alertblock}
	\end{frame}

% Scoring function breakdown
\begin{frame} %%%%%%%      SCORING FUNCTION BREAKDOWN       %%%%%%
	% \begin{prop}\label{prop:nice-score}%
		Letting $x^{\mat w}$ and $y^{\mat w}$ denote the values of
		$X$ and $Y$, respectively, in $\mat w \in \V(\dg M)$,
		we have
		\begin{equation*}\label{eq:semantics-breakdown}
			\begin{split}
				\bbr{\dg M}(\mu) =  \Ex_{\mat w \sim \mu}\! \Bigg\{
				 \sum_{ X \xrightarrow{\!\!L} Y  }
				\bigg[\,
				    {\color{benchcolor1}\overbrace{\color{black}
				      \beta_L \log \frac{1}{\bp(y^{\mat w} |x^{\mat w})}
					}^{\color{benchcolor1}\smash{\mathclap{\text{log likelihood / cross entropy}}}}}~+
					 \qquad\qquad\qquad\\[-0.7em]\qquad\qquad
				    {\color{benchcolor2}\underbrace{\color{black}
				({\alpha_L}\gamma - \beta_L ) \log \frac{1}{\mu(y^{\mat w} |x^{\mat w})}
					}_{\color{benchcolor2}\smash{\mathclap{\text{local regularization ($\beta_L >
					\alpha_L
					\gamma$)}}}}}~\bigg] - \color{structurecolor!70!blue}\underbrace{\color{black}
				\gamma \log \frac{1}{\mu(\mat w)}
					}_{\color{structurecolor!70!blue}\smash{\mathclap{\text{~~~~global
				        regularization}}}}\color{black} \Bigg\} .
				\end{split}
			\end{equation*}
		% \end{prop}
	\end{frame}

% Illustrations of IDef
\begin{frame}\frametitle{Illustrations of $\IDef{}$}
	\def\vsize{0.4}
	\def\bnslide{4}
	\def\spacerlength{0.5em}
	\begin{tikzpicture}[center base]\label{subfig:justX-0}
		\node[dpad0] (X) at (0,1){$X$};
		\draw[fill=green!50!black]  (0,0) circle (\vsize)  ++(-90:.22) node[label=below:\tiny$X$]{};
		\useasboundingbox (current bounding box);
		\node at (-0.5, 0.6){};
		\end{tikzpicture}
	\begin{tabular}{c}
	\begin{tikzpicture}[onslide=<\bnslide>{is bn}]\label{subfig:justX-1}
		\node[dpad0] (1) at (-0.4,.85){$\!\pdgunit\!$};
		\node[dpad0] (X) at (0.4,.85){$X$};
		\draw[arr1] (1)  -- (X);
		\draw[fill=white!70!black]  (0,0) circle (\vsize) ++(-90:.22) node[label=below:\tiny$X$]{};
		\node at (-0.6,0.35){};
		\useasboundingbox (current bounding box);
		\node at (-0.7, 0.35){};
		\end{tikzpicture} \\[0.5em]
	\begin{tikzpicture}\label{subfig:justX-2}
		\node[dpad0] (1) at  (-0.45,.85){$\!\pdgunit\!$};
		\node[dpad0] (X) at  (0.45,.85){$X$};
		\draw[arr1] (1) to[bend left=20] (X);
		\draw[arr1] (1) to[bend right=20] (X);
		\draw[fill=red!50!black] (0,0) circle (\vsize) ++(-90:.22) node[label=below:\tiny$X$]{};
		\useasboundingbox (current bounding box);
		\node at (-0.7, 0.35){};
		\end{tikzpicture}
	\end{tabular}%}
	\hspace{\spacerlength}\pause\vrule\hspace{\spacerlength}
	%% EXAMPLE: X  Y
	% \adjustbox{valign=b}{
	\begin{tabular}{c}
	\begin{tikzpicture}[]  \label{subfig:justXY}
		% \node[dpad0] (1) at (0,2){$\pdgunit$};
		\node[dpad0] (X) at (-0.45,.85){$X$};
		\node[dpad0] (Y) at (0.45,.85){$Y$};
		\path[fill=green!50!black] (-0.2,0) circle (\vsize) ++(-110:.23) node[label=below:\tiny$X$]{};
		\path[fill=green!50!black] (0.2,0) circle (\vsize) ++(-70:.23) node[label=below:\tiny$Y$]{};
		\begin{scope}
			\clip (-0.2,0) circle (\vsize);
			\clip (0.2,0) circle (\vsize);
			\fill[green!50!black] (-1,-1) rectangle (3,3);
			% \draw[ultra thick,white] (-0.2,0) circle (\vsize);
			% \draw[ultra thick,white] (0.2,0) circle (\vsize);
		\end{scope}
		\draw (-0.2,0) circle (\vsize);
		\draw (0.2,0) circle (\vsize);
		\useasboundingbox (current bounding box);
		\node at (-0.8, 0.4){};
		\end{tikzpicture}\\[0.5em]
	%% EXAMPLE: X -> Y
	\begin{tikzpicture}[]\label{subfig:XtoY}
		% \node[dpad0] (1) at (0,2){$\pdgunit$};
		\node[dpad0] (X) at (-0.45,0.85){$X$};
		\node[dpad0] (Y) at (0.45,0.85){$Y$};
		\draw[arr1] (X) to[] (Y);
		% \draw[arr] (1) to[] (Y);
		\path[fill=green!50!black] (-0.2,0) circle (\vsize) ++(-110:.23) node[label=below:\tiny$X$]{};
		\path[fill=white!70!black] (0.2,0) circle (\vsize) ++(-70:.23) node[label=below:\tiny$Y$]{};
		\begin{scope}
			\clip (-0.2,0) circle (\vsize);
			\clip (0.2,0) circle (\vsize);
			\fill[green!50!black] (-1,-1) rectangle (3,3);
			% \draw[ultra thick,white] (-0.2,0) circle (\vsize);
			% \draw[ultra thick,white] (0.2,0) circle (\vsize);
		\end{scope}
		\draw (-0.2,0) circle (\vsize);
		\draw (0.2,0) circle (\vsize);
		\useasboundingbox (current bounding box);
		\node at (-0.8, 0.4){};
		\end{tikzpicture}
	\end{tabular}%}
	% \hspace{\spacerlength}
	\begin{tabular}{c}
	%% EXAMPLE: X <-> Y
	\begin{tikzpicture}[center base]\label{subfig:XY-cycle}
		% \node[dpad0] (1) at (0,2){$\pdgunit$};
		\node[dpad0] (X) at (-0.45,0.85){$X$};
		\node[dpad0] (Y) at (0.45,0.85){$Y$};
		\draw[arr1] (X) to[bend left] (Y);
		\draw[arr1] (Y) to[bend left] (X);
		\draw[fill=white!70!black] (-0.2,0) circle (\vsize) ++(-110:.25) node[label=below:\tiny$X$]{};
		\draw[fill=white!70!black] (0.2,0) circle (\vsize) ++(-70:.25) node[label=below:\tiny$Y$]{};
		\begin{scope}
			\clip (-0.2,0) circle (\vsize);
			\clip (0.2,0) circle (\vsize);
			\fill[green!50!black] (-1,-1) rectangle (3,3);
			% \draw[ultra thick,white] (-0.2,0) circle (\vsize);
			% \draw[ultra thick,white] (0.2,0) circle (\vsize);
		\end{scope}
		\draw (-0.2,0) circle (\vsize);
		\draw (0.2,0) circle (\vsize);
		\useasboundingbox (current bounding box.south west) rectangle (current bounding box.north east);
		\node at (-0.85, 0.4){};
		\end{tikzpicture}\\[2.5em]
	% \hspace{\spacerlength}%% EXAMPLE: 1 -> Y;1->X
	\begin{tikzpicture}[center base, onslide=<\bnslide>{is bn}] \label{subfig:XYindep}
		\node[dpad0] (1) at (0,0.75){$\!\pdgunit\!$};
		\node[dpad0] (X) at (-0.7,0.95){$X$};
		\node[dpad0] (Y) at (0.7,0.95){$Y$};
		\draw[arr0] (1) to[] (X);
		\draw[arr0] (1) to[] (Y);
		\draw[fill=white!70!black] (-0.2,0) circle (\vsize) ++(-110:.23) node[label=below:\tiny$X$]{};
		\draw[fill=white!70!black] (0.2,0) circle (\vsize) ++(-70:.23) node[label=below:\tiny$Y$]{};
		\begin{scope}
			\clip (-0.2,0) circle (\vsize);
			\clip (0.2,0) circle (\vsize);
			\fill[red!50!black] (-1,-1) rectangle (3,3);
			% \draw[ultra thick,white] (-0.2,0) circle (\vsize);
		% \draw[ultra thick,white] (0.2,0) circle (\vsize);
		\end{scope}
		\draw (-0.2,0) circle (\vsize);
		\draw (0.2,0) circle (\vsize);
		\useasboundingbox (current bounding box.south west) rectangle (current bounding box.north east);
		\node at (-0.88, 0.4){};
		\end{tikzpicture}
	\end{tabular}
	\hspace{\spacerlength}
	 %% EXAMPLE: 1 -> X -> Y
	\begin{tikzpicture}[center base, onslide=<\bnslide>{is bn}]\label{subfig:1XY}
		\node[dpad0] (1) at (0.15,2){$\!\pdgunit\!$};
		\node[dpad0] (X) at (-0.45,1.4){$X$};
		\node[dpad0] (Y) at (0.35,1){$Y$};
		\draw[arr0] (1) to[] (X);
		\draw[arr1] (X) to[] (Y);
		\path[fill=white!70!black] (-0.2,0) circle (\vsize) ++(-110:.23) node[label=below:\tiny$X$]{};
		\path[fill=white!70!black] (0.2,0) circle (\vsize) ++(-70:.23) node[label=below:\tiny$Y$]{};
		\begin{scope}
			\clip (-0.2,0) circle (\vsize);
			\clip (0.2,0) circle (\vsize);
			% \fill[red!50!black] (-1,-1) rectangle (3,3);
			% \draw[ultra thick,white] (-0.2,0) circle (\vsize);
			% \draw[ultra thick,white] (0.2,0) circle (\vsize);					\end{scope}
		\end{scope}
		\draw (-0.2,0) circle (\vsize);
		\draw (0.2,0) circle (\vsize);
		\useasboundingbox (current bounding box);
		\node at (-0.7, 0.6){};
		\end{tikzpicture}
	% \hspace{\spacerlength}\hspace{2.5pt}\vrule\hspace{2.5pt}\hspace{\spacerlength}

	\pause
	%% EXAMPLE: 1 -> X -> Y -> Z
	\begin{tikzpicture}[center base, onslide=<\bnslide>{is bn}] \label{subfig:1XYZ}
		% \node[dpad0] (1) at (-0.5,2.3){$\!\pdgunit\!$};
		% \node[dpad0] (X) at (-0.5,1.5){$X$};
		% \node[dpad0] (Y) at (0.35,1.25){$Y$};
		% \node[dpad0] (Z) at (0.25,2.25){$Z$};
		\node[dpad0] (1) at (-1,0.8){$\!\pdgunit\!$};
		\node[dpad0] (X) at (-0.6,1.6){$X$};
		\node[dpad0] (Y) at (0.35,1.6){$Y$};
		\node[dpad0] (Z) at (1.0,0.9){$Z$};
		\draw[arr0] (1) to (X);
		\draw[arr1] (X) to[] (Y);
		\draw[arr1] (Y) to[] (Z);
		\path[fill=white!70!black] (210:0.22) circle (\vsize) ++(-130:.25) node[label=below:\tiny$X$]{};
		\path[fill=white!70!black] (-30:0.22) circle (\vsize) ++(-50:.25) node[label=below:\tiny$Y$]{};
		\path[fill=white!70!black] (90:0.22) circle (\vsize) ++(40:.29) node[label=above:\tiny$Z$]{};
		\begin{scope}
			\clip (90:0.22) circle (\vsize);
			\clip (210:0.22) circle (\vsize);
			\fill[red!50!black] (-1,-1) rectangle (3,3);
			% \draw[ultra thick,white] (210:0.2) circle (\vsize);
			% \draw[ultra thick,white] (90:0.2) circle (\vsize);
			\clip (-30:0.22) circle (\vsize);
			\fill[white!70!black] (-1,-1) rectangle (3,3);
			% \draw[ultra thick,white] (-30:0.2) circle (\vsize);
			% \draw[ultra thick,white] (210:0.2) circle (\vsize);
			% \draw[ultra thick,white] (90:0.2) circle (\vsize);
		\end{scope}
		\begin{scope}
			\draw[] (-30:0.22) circle (\vsize);
			\draw[] (210:0.22) circle (\vsize);
			\draw[] (90:0.22) circle (\vsize);
		\end{scope}
		\useasboundingbox (current bounding box);
		\node at (-0.7, 0.7){};
		\end{tikzpicture}
	%% EXAMPLE: X -> Y -> Z -> X
	\hspace{\spacerlength}
	\begin{tikzpicture}[center base] \label{subfig:XYZ-cycle}
		% \node[dpad0] (1) at (-0.5,2.3){$\pdgunit$};
		\node[dpad0] (X) at (-0.5,1.75){$X$};
		\node[dpad0] (Y) at (0.35,1.25){$Y$};
		\node[dpad0] (Z) at (0.25,2.25){$Z$};
		% \draw[arr0] (1) to (X);
		\draw[arr1] (X) to[bend right=25] (Y);
		\draw[arr1] (Y) to[bend right=25] (Z);
		\draw[arr1] (Z) to[bend right=25] (X);
		%option: -- either X -> Y -> Z -> X, or <-> Y <-> Z <-> X. For the latter, uncomment the 6 lines below and comment out the next 3.
		% \draw[arr1] (Z) to[bend left=5] (Y);
		% \draw[arr1] (Y) to[bend left=5] (X);
		% \draw[arr1] (X) to[bend left=5] (Z);
		% \draw[fill=red!50!black] (210:0.22) circle (\vsize) ++(-130:.27) node[label=below:\tiny$X$]{};
		% \draw[fill=red!50!black] (-30:0.22) circle (\vsize) ++(-50:.27) node[label=below:\tiny$Y$]{};
		% \draw[fill=red!50!black] (90:0.22) circle (\vsize) ++(140:.31) node[label=above:\tiny$Z$]{};

		% grey filling for one covering.
		\draw[fill=white!70!black] (210:0.22) circle (\vsize) ++(-130:.27) node[label=below:\tiny$X$]{};
		\draw[fill=white!70!black] (-30:0.22) circle (\vsize) ++(-50:.27) node[label=below:\tiny$Y$]{};
		\draw[fill=white!70!black] (90:0.22) circle (\vsize) ++(40:.31) node[label=above:\tiny$Z$]{};

		\begin{scope}
			\clip (-30:0.22) circle (\vsize);
			\clip (210:0.22) circle (\vsize);
			% \fill[white!70!black] (-1,-1) rectangle (3,3);
			\clip (90:0.22) circle (\vsize);
			\fill[green!50!black] (-1,-1) rectangle (3,3);
		\end{scope}
		\begin{scope}
			\draw[] (-30:0.22) circle (\vsize);
			\draw[] (210:0.22) circle (\vsize);
			\draw[] (90:0.22) circle (\vsize);
		\end{scope}
		\useasboundingbox (current bounding box);
		\node at (-0.7, 0.7){};
		\end{tikzpicture}
	%% EXAMPLE: X -> Y <- Z
	\hspace{\spacerlength}
	\begin{tikzpicture}[center base] \label{subfig:XZtoY}
		% \node[dpad0] (1) at (-0.5,2.3){$\pdgunit$};
		\node[dpad0] (X) at (-0.45,1.9){$X$};
		\node[dpad0] (Y) at (0.3,1.25){$Y$};
		\node[dpad0] (Z) at (0.4,2.15){$Z$};
		% \draw[arr0] (1) to (X);
		\draw[arr0] (X) to[] (Y);
		\draw[arr1] (Z) to[] (Y);
		\path[fill=green!50!black] (210:0.22) circle (\vsize) ++(-130:.25) node[label=below:\tiny$X$]{};
		\path[fill=red!50!black] (-30:0.22) circle (\vsize) ++(-50:.25) node[label=below:\tiny$Y$]{};
		\path[fill=green!50!black] (90:0.22) circle (\vsize) ++(40:.29) node[label=above:\tiny$Z$]{};
		\begin{scope}
			\clip (-30:0.22) circle (\vsize);
			\clip (90:0.22) circle (\vsize);
			\fill[white!70!black] (-1,-1) rectangle (3,3);
		\end{scope}
		\begin{scope}
			\clip (-30:0.22) circle (\vsize);
			\clip (210:0.22) circle (\vsize);
			\fill[white!70!black] (-1,-1) rectangle (3,3);

			\clip (90:0.22) circle (\vsize);
			\fill[green!50!black] (-1,-1) rectangle (3,3);
			% \draw[ultra thick,white] (210:0.2) circle (\vsize);
			% \draw[ultra thick,white] (90:0.2) circle (\vsize);
			% \draw[ultra thick,white] (-30:0.2) circle (\vsize);
			% \draw[ultra thick,white] (210:0.2) circle (\vsize);
			% \draw[ultra thick,white] (90:0.2) circle (\vsize);
		\end{scope}
		\draw[] (-30:0.22) circle (\vsize);
		\draw[] (210:0.22) circle (\vsize);
		\draw[] (90:0.22) circle (\vsize);
		\useasboundingbox (current bounding box);
		% \node at (-0.7, 0.7){};
		\end{tikzpicture}
	%% EXAMPLE: X <-> Y <-> Z
	\hspace{\spacerlength}
	\begin{tikzpicture}[center base] \label{subfig:XYZ-bichain}
		% \node[dpad0] (1) at (0.1,2.4){$\pdgunit$};
		\node[dpad0] (X) at (-1,1.2){$X$};
		\node[dpad0] (Y) at (0,1.7){$Y$};
		\node[dpad0] (Z) at (1,1.4){$Z$};
		% \draw[arr1] (1) to (X);
		% \draw[arr1] (1) to (Y);
		\draw[arr1] (X) to[bend right=15] (Y);
		\draw[arr1] (Y) to[bend right=15] (X);
		\draw[arr1] (Y) to[bend right=15] (Z);
		\draw[arr1] (Z) to[bend right=15] (Y);
		\path[fill=white!70!black] (210:0.22) circle (\vsize) ++(-130:.25) node[label=below:\tiny$X$]{};
		\path[fill=red!50!black] (-30:0.22) circle (\vsize) ++(-50:.25) node[label=below:\tiny$Y$]{};
		\path[fill=white!70!black] (90:0.22) circle (\vsize) ++(40:.29) node[label=above:\tiny$Z$]{};
		\begin{scope}
			\clip (-30:0.22) circle (\vsize);
			\clip (90:0.22) circle (\vsize);
			\fill[white!70!black] (-1,-1) rectangle (3,3);
		\end{scope}
		\begin{scope}
			\clip (90:0.22) circle (\vsize);
			\clip (210:0.22) circle (\vsize);
			\fill[red!50!black] (-1,-1) rectangle (3,3);
		\end{scope}
		\begin{scope}
			\clip (-30:0.22) circle (\vsize);
			\clip (210:0.22) circle (\vsize);
			\fill[white!70!black] (-1,-1) rectangle (3,3);

			\clip (90:0.22) circle (\vsize);
			\fill[green!50!black] (-1,-1) rectangle (3,3);
			% \draw[ultra thick,white] (210:0.2) circle (\vsize);
			% \draw[ultra thick,white] (90:0.2) circle (\vsize);
			% \draw[ultra thick,white] (-30:0.2) circle (\vsize);
			% \draw[ultra thick,white] (210:0.2) circle (\vsize);
			% \draw[ultra thick,white] (90:0.2) circle (\vsize);
		\end{scope}
		\draw[] (-30:0.22) circle (\vsize);
		\draw[] (210:0.22) circle (\vsize);
		\draw[] (90:0.22) circle (\vsize);
		\useasboundingbox (current bounding box);
		\node at (-0.7, 0.7){};
		\end{tikzpicture}

\end{frame}

% A half-baked table comparing PDGs, BNs
\begin{frame}<0> %%%%%%%%%         PDGs and BNs           %%%%%%%%%%
    \frametitle{What a PDG is}
    % \begin{columns}
    % 	\column{0.5\textwidth}
    % 	{\Large Bayesian Networks (BNs)}
    % 	\begin{itemize}
    % 		\item<1-> Qualitative Structure: a hyper-graph generated by a DAG
    % 		\item<2-> Contains a CPD at each node
    % 	\end{itemize}
    %
    % 	\column{0.5\textwidth}
    % 	{\Large PDGs}
    % 	\begin{itemize}
    % 		\item<1-> Qualitative Structure may be an arbitrary hypergraph
    % 		\item<2-> Contains a CPD for every edge.
    % 	\end{itemize}
    % \end{columns}

    {\renewcommand{\arraystretch}{1.25}
    \begin{tabular}{r|p{0.3\textwidth}p{0.3\textwidth}}
                % this is a "\ctoprule"
                \addlinespace[-\aboverulesep]
                \cmidrule[\heavyrulewidth]{2-3}
            & BNs & PDGs \pause \\ \cmidrule{2-3}
        Qualitative Structure
            & Hyper-Graph generated by a DAG
            & Arbitrary hyper-graphs  \pause \\[1em]
        CPDs attached to
            & Nodes
            & Edges
            \\ \bottomrule
    \end{tabular}}


    \ifbool{precompiledfigs}{}{
    \begin{center}\begin{tikzpicture}
        \node[dpadded] (X) at (-2,2) {$X$};
        \node[dpadded] (Y) at (0,0) {$Y$};
        \node[dpadded] (Z) at (2,2) {$Z$};
        \draw[arr] (X) -> (Y);
        \draw[arr] (Z) -> (Y);
    \end{tikzpicture}\end{center}}
    \end{frame}%------------%
	
% Intro to factor graphs
\begin{frame} \frametitle{Factor Graphs}
		
	\begin{center}
		\begin{tikzpicture}
			\node [fgnode] (A) {$A$};
			\node [fgnode, right=1 of A] (B) {$B$};
			\node [fgnode, above=0.6 of B] (C) {$C$};
			\node [fgnode, right=1 of C] (D) {$D$};
			\node [fgnode, right=1 of B] (E) {$E$};
			
			\node[factor, left=0.5 of A, label={below:$\phi_1$}] (phi1) {};
				\draw[thick] (phi1) -- (A);
			
			\node[factor,label={below:$\phi_2$}] at ($(A)!.5!(B)$) (phi2) {};
				\draw[thick] (phi2) -- (A);
				\draw[thick] (phi2) -- (B);
			
			\node[factor,above=0.5 of phi2,label={above:$\phi_3$}] (phi3){};
				\draw[thick] (phi3) -- (A);
				\draw[thick] (phi3) -- (B);
				\draw[thick] (phi3) -- (C);				

			\node[factor,label={above:$\phi_4$}] at ($(C)!.5!(D)$) (phi4) {};
				\draw[thick] (phi4) -- (C);
				\draw[thick] (phi4) -- (D);
			\end{tikzpicture}
		\end{center}

	\begin{defn}
		 A \emph{factor graph} $\Phi$ is a set of random variables
		        $\mathcal X = \{X_i\}$ and \emph{factors}
		       $\{\phi_J\colon \V(X_J) \to \mathbb R_{\geq0}\}_{J \in
		\mathcal J }$,
		where $X_J \subseteq \mathcal X$; define 
		\[ {\Pr}_{\Phi}(\vec x) = \frac{1}{Z_{\Phi}}
		 	\prod_{J \in \cal J} \phi_J(\vec x_J), \]
	 	where $Z_{\Phi}$ is the normalization constant.
		\end{defn}
	\end{frame}

% PDGs as factor graphs
\begin{frame} \frametitle{PDGs as Factor Graphs}
	\begin{center}
	\begin{tikzpicture||precompiled}[center base]{smoking-PDG}
		\node[dpadded] (1) at (1.65,-1.1) {$\pdgunit$};
		\node[dpadded] (PS) at (1.65,0.4) {$\mathit{PS}$};
		\node[dpadded] (S) at (3.2, 0.8) {$\mathit S$};
		\node[dpadded] (SH) at (3.35, -0.8) {$\mathit{SH}$};
		\node[dpadded] (C) at (4.8,0.4) {$\mathit C$};
		\node[dpadded] (T) at (4.8,-1.1) {$\mathit T$};

		\draw[arr1] (1) -- (PS);
		\draw[arr2] (PS) -- (S);
		\draw[arr2] (PS) -- (SH);
		\mergearr{SH}{S}{C}
		\draw[arr1] (T) -- (C);
		\end{tikzpicture||precompiled}
	~{\Large$\rightsquigarrow$}~
	\begin{tikzpicture}[center base, xscale=1.2]
		\node[factor] (prior) at (1.65,-1) {};
		\node[factor] (center) at (3.75, 0.1){};

		\node[fgnode] (PS) at (1.65,0.5) {$\mathit{PS}$};
		\node[fgnode] (S) at (3.1, 0.8) {$\mathit S$};
		\node[fgnode] (SH) at (3.0, -0.8) {$\mathit{SH}$};
		\node[fgnode] (C) at (4.8,0.5) {$\mathit C$};

		\draw[thick] (prior) -- (PS);
		\draw[thick] (PS) --node[factor](pss){} (S);
		\draw[thick] (PS) --node[factor](pssh){} (SH);
		\draw[thick] (S) -- (center) (center) -- (SH) (C) -- (center);


		\node[fgnode] (T) at (4.8, -1.3) {$T$};
		\draw[thick] (T) -- node[factor]{}  (C);
		\end{tikzpicture}
		\end{center}


	\pause
	The cpds of a PDG are essentially factors. Are the semantics different?

	\pause
	{Not for $\gamma = 1$.}


	\begin{theorem}
		$\bbr{\dg N}_{1}^* = \Pr_{\FGof{\dg N}}\;$ for all unweighted
		PDGs $\dg N$.
	\end{theorem}
	\pause
	{\setbeamercolor{block body}{bg=structurecolor!50!white}
	 \setbeamercolor{block title}{bg=structurecolor!70!black,fg=white}
	\begin{theorem}\label{thm:pdg-is-wfg}
		For all unweighted PDGs $\dg{N}$ and non-negative vectors $\mat v$
		over the edges of $\dg N$, and all $\gamma > 0$, we have that
		$\bbr{(\dg N, \mat v, \gamma \mat v)}_{\gamma}
		= \gamma\,\GFE_{(\Phi_{\dg N}, \mat v)} $; consequently,
		$\bbr{(\dg N,  \mat v,  \gamma\mat v)}_{\gamma}^*
				= \{\Pr_{(\Phi_{\dg N}, \mat v)} \}$.
	\end{theorem}
	 }
	\end{frame}

% Why use PDGs over factor graphs
\begin{frame}\frametitle{Why not use factor graphs, then?}
	\begin{center}
		$\dg M := $
		\begin{tikzpicture}[center base]
			\node[dpadded, inner sep=0.4em] (1){$\pdgunit$};
			\node[dpadded, inner sep=0.4em, below=1 of 1] (X) {$X$};
			\draw[arr] (1) to[bend left=30] node[right]{$p$} (X);
			\draw[arr] (1) to[bend right=30] node[left]{$q$} (X);
			\end{tikzpicture}
		$\qquad$ \vrule $\qquad$
		\begin{tikzpicture}[center base]
			\node[fgnode, below=1 of 1] (X){$X$};

			\node[factor, above left= 0.5 and 0.5 of X] (phi1) {};
			\node[factor, above right=0.5 and 0.5 of X] (phi2) {};

			\draw[thick] (phi1) -- (X);
			\draw[thick] (phi2) -- (X);
		\end{tikzpicture}$\quad=: \Phi$
	\end{center}
	\pause
	\medskip
	\begin{itemize}[<+-|alert@+>]
		\item If $p = q$, then $\bbr{\dg M}^* = p = q$\textellipsis
		\item \textellipsis but $\Pr_{\Phi} \propto p^2$
		\item More generally, (positive) factors individually have \emph{no meaning},
		\item a factor graph can fail to normalize, in which case it has no global semantics either. 
	\end{itemize}
	\end{frame}

% Factor graphs as PDGs
\begin{frame} \frametitle{Factor Graphs as PDGs}
	\begin{center}
		\begin{tikzpicture}[center base, xscale=1.3,
			fgnode/.append style={minimum width=2.4em, inner sep=0.2em}]
			\node[factor] (prior) at (1.65,-1) {};
			\node[factor] (center) at (3.75, 0.1){};

			\node[fgnode] (PS) at (1.65,0.5) {$\mathit{PS}$};
			\node[fgnode] (S) at (3.1, 0.8) {$\mathit S$};
			\node[fgnode] (SH) at (3.0, -0.8) {$\mathit{SH}$};
			\node[fgnode] (C) at (4.8,0.5) {$\mathit C$};

			\draw[thick] (prior) -- (PS);
			\draw[thick] (PS) --node[factor](pss){} (S);
			\draw[thick] (PS) --node[factor](pssh){} (SH);
			\draw[thick] (S) -- (center) (center) -- (SH) (C) -- (center);


			\node[fgnode] (T) at (4.8, -1.3) {$\mathit T$};
			\draw[thick] (T) -- node[factor]{}  (C);
			\end{tikzpicture}
		~{\Large$\rightsquigarrow$}~
		\pause
		\begin{tikzpicture}[center base, xscale=1.5,
	        newnode/.style={rectangle, inner sep=5pt, fill=gray!30, rounded corners=3, thick,draw}]
			\node[newnode] (prior) at (1.65,-1) {};
			\node[newnode] (center) at (4.1, 0.25){};

			\node[dpadded] (PS) at (1.65,0.5) {$\mathit{PS}$};
			\node[dpadded] (S) at (3.3, 0.8) {$\mathit S$};
			\node[dpadded] (SH) at (3.3, -0.6) {$\mathit{SH}$};
			\node[dpadded] (C) at (4.9,0.5) {$\mathit C$};

			\draw[arr, ->>, shorten <=0pt] (prior) -- (PS);
			\draw[arr, <<->>] (PS) --node[newnode](pss){} (S);
			\draw[arr, <<->>] (PS) --node[newnode](pssh){} (SH);
			\draw[arr, <<-, shorten >=0pt] (S) -- (center);
			\draw[arr, <<-, shorten >=0pt] (SH)-- (center);
			\draw[arr, <<-, shorten >=0pt] (C) -- (center);

			\node[dpadded, fill=blue] (1) at (2.7,-1.8) {$\pdgunit$};

			\draw[blue!50, arr] (1) -- (prior);
			\draw[blue!50, arr] (1) to[bend right=30] (center);
			\draw[blue!50, arr] (1) to[bend right = 5] (pss);
			\draw[blue!50, arr] (1) to[bend left = 10] (pssh);


			\node[dpadded] (T) at (4.8, -1.7) {$T$};
			\draw[arr, <<->>] (T) -- node[newnode](tc){}  (C);

			\draw[blue!50, arr] (1) to[bend right = 10] (tc);
			\end{tikzpicture}
		\end{center}
	\pause

	\begin{theorem}
		$\Pr_{\Phi} = \bbr{\UPDGof{\Phi}}_{1}^*\;$ for all factor graphs $\Phi$.
	\end{theorem}
	\pause
	{\setbeamercolor{block body}{bg=structurecolor!50!white}
	 \setbeamercolor{block title}{bg=structurecolor!70!black,fg=white}
	\begin{theorem}
	For all weighted factor graphs $\Psi = (\Phi,\theta)$ and all $\gamma > 0$,
	we have that
	$\GFE_\Psi
	= \nicefrac1{\gamma} \bbr{{\dg M}_{\Psi,\gamma}}_{\gamma}
	+ C$
	for some constant $C$, so
	$\Pr_{\Psi}$ is the unique element of
	$\bbr{{\dg M}_{\Psi,\gamma}}_{\gamma}^*$.
	\end{theorem}
	}
	\end{frame}
