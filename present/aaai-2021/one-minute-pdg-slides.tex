%  Beamer Style
\documentclass[xcolor={dvipsnames,rgb}]{beamer}
% \includeonlyframes{current}

\usepackage{lmodern}
\usepackage[utf8]{inputenc}
\usetheme{Boadilla}% Beamer Theme Customization...
	\setbeamersize{description width=0.57cm}
	\usefonttheme[stillsansserifsmall]{serif}
		% \usefonttheme{structuresmallcapsserif}
	\usefonttheme[onlylarge]{structuresmallcapsserif}
		% \usefonttheme[onlymath]{serif}
		% \usefonttheme[onlysmall]{structurebold}
	\setbeamerfont{item}{series=\bfseries}
	\setbeamerfont{block title}{series=\bfseries}
	% \setbeamerfont{title}{family=\rmfamily}

	\relax%%% color definitions %%...
		\colorlet{structurecolor}{RoyalPurple!50!black}
		\colorlet{alertcolor}{YellowOrange}
			% \colorlet{alertcolor}{structurecolor>wheel,1,3}
		\colorlet{benchcolor1}{Emerald!85!black}
			% \colorlet{benchcolor1}{structurecolor>wheel,2,3}
		\colorlet{benchcolor2}{YellowOrange!25!magenta}
	\usecolortheme[named=structurecolor]{structure}
		% \usecolortheme{beaver}
		% \setbeamercolor*{palette primary}{bg=color1, fg = green}
		% \setbeamercolor*{palette secondary}{bg=color2, fg = green}
		% \setbeamercolor*{palette tertiary}{bg=color3, fg = green}
		% \setbeamercolor*{palette quaternary}{bg=color4, fg = green}
		% \makeatletter
		% \definecolor{beamer@blendedblue}{rgb}{0.2,0.2,0.7}
		% \colorlet{beamer@blendedblue}{color2}
		% \makeatother
	\setbeamercolor{description item}{bg={structurecolor!20!white}}
	\setbeamercolor{alerted text}{fg=alertcolor}

	\newbool{precompiledfigs}% ...
		\setbool{precompiledfigs}{false}
		% the etoolbox way, which works with beamer.
	% \setbeamercovered{dynamic}

\relax %%%%%%%%  Beamer and slide-specific macros  %%%%%%%%%%%%%%%%%%%
	\newcommand<>{\hl}[2][alertcolor]{\begingroup%
		\setbeamercolor{alerted text}{fg=#1}\alert#3{#2}\endgroup}
	\colorlet{notationcolor}{structurecolor!40}
	\def\notation#1{\!\hl[notationcolor]{#1$\quad$}}
	\def\notation#1{\!\hl[notationcolor]{#1$\quad$}}
	% \newcommand<>{\alertwith}[2]{\begingroup\only#3{\setbeamercolor{alerted text}{fg=#1}}#2\endgroup} % DOESN'T WORK THIS WAY
	\newenvironment{localfocusenv}{\only{\setbeamercolor{local structure}{fg=alertcolor}}}{}
	\newenvironment<>{hidemeenv}{%
		\only#1{\setbeamercolor{alerted text}{fg=black!60}}%
		\begingroup\begin{alertenv}#1%
		}{\end{alertenv}\endgroup}
	\newenvironment<>{tikzpicture||precompiled}[2][]{
			\ifbool{precompiledfigs}{\includegraphics[width=0.8\linewidth]{figure-pdfs/#2}
				}\begingroup\only#3\begingroup\begin{tikzpicture}[#1]
		}{\end{tikzpicture}\endgroup\endgroup}
	\newcommand<>{\extra}[2][]{%
		\only#3{%
			% \tikzmark{call point};%
			\tikzro \node[inner sep=0pt,outer sep=0pt] (call point) {};%
			\begin{tikzpicture}[overlay,remember picture]
				\node[anchor=north west, inner sep=0.8em,
				 			fill=alertcolor!30!structurecolor!30!white,
							draw=structurecolor!70!black, draw opacity=0.5,
							below=1em of call point, #1]{#2};
			\end{tikzpicture}%
		}}
	\newcommand{\tikzro}[1][]{\tikz[remember picture, overlay,#1]}
	\def\Set{\mathbf{Set}}
	\makeatletter
	\newcommand{\shorteq}{%
	  \settowidth{\@tempdima}{-}% Width of hyphen
	  \resizebox{\@tempdima}{\height}{=}%
	}
	\makeatother
	% \newcommand{\tikzmark}[1][last mark]{\tikzro \node (#1){};}

%%%%%%%            Relevant part of PDG Preamble        %%%%%%%%%%%%%%%%
\usepackage{tikz}
	\usetikzlibrary{positioning,calc, arrows, shapes}

	\tikzset{AmpRep/.style={ampersand replacement=\&}}
	\tikzset{center base/.style={baseline={([yshift=-.8ex]current bounding box.center)}}}
	\tikzset{paperfig/.style={center base,scale=0.9, every node/.style={transform shape}}}

	\tikzset{dpadded/.style={rounded corners=2, inner sep=0.6em, draw, outer sep=0.2em, fill={black!50}, fill opacity=0.08, text opacity=1}}
	\tikzset{light pad/.style={outer sep=0.2em, inner sep=0.5em, draw=gray!50}}
	\tikzset{arr/.style={draw, ->, thick, shorten <=3pt, shorten >=3pt}}
	\tikzset{arr0/.style={draw, ->, thick, shorten <=0pt, shorten >=0pt}}
	\tikzset{arr1/.style={draw, ->, thick, shorten <=1pt, shorten >=1pt}}
	\tikzset{arr2/.style={draw, ->, thick, shorten <=2pt, shorten >=2pt}}

	\newcommand{\drawbb}%
		{\draw (current bounding box.south west) rectangle (current bounding box.north east);}
	\ifbool{precompiledfigs}{}{
		\usetikzlibrary{fit, decorations,shapes.geometric}
		\usetikzlibrary{tikzmark}
		\usetikzlibrary{backgrounds}
		\pgfdeclarelayer{foreground}
		\pgfsetlayers{background,main,foreground}

		\pgfdeclaredecoration{arrows}{draw}{
			\state{draw}[width=\pgfdecoratedinputsegmentlength]{%
				\path [every arrow subpath/.try] \pgfextra{%
					\pgfpathmoveto{\pgfpointdecoratedinputsegmentfirst}%
					\pgfpathlineto{\pgfpointdecoratedinputsegmentlast}%
				};
		}}

		\tikzset{dpad0/.style={outer sep=0.05em, inner sep=0.3em, draw=gray!75, rounded corners=4, fill=black!08, fill opacity=1}}
		\tikzset{dpad/.style args={#1}{every matrix/.append style={nodes={dpadded, #1}}}}
		\tikzset{is bn/.style={background rectangle/.style={fill=blue!35,opacity=0.3, rounded corners=5},show background rectangle}}
		% \usetikzlibrary{backgrounds}
		% \usetikzlibrary{patterns}
		\usetikzlibrary{cd}

		\tikzset{fgnode/.style={dpadded,inner sep=0.2em, circle,minimum width=2.3em},
				 factor/.style={light pad, fill=black, outer sep=0pt,draw=none}}


		\newcommand\cmergearr[4]{
			\draw[arr,-] (#1) -- (#4) -- (#2);
			\draw[arr, shorten <=0] (#4) -- (#3);
			}
		\newcommand\mergearr[3]{
			\coordinate (center-#1#2#3) at (barycentric cs:#1=1,#2=1,#3=1.2);
			\cmergearr{#1}{#2}{#3}{center-#1#2#3}
			}
		\newcommand\cunmergearr[4]{
			\draw[arr,-, , shorten >=0] (#1) -- (#4);
			\draw[arr, shorten <=0] (#4) -- (#2);
			\draw[arr, shorten <=0] (#4) -- (#3);
			}
		\newcommand\unmergearr[3]{
			\coordinate (center-#1#2#3) at (barycentric cs:#1=1.2,#2=1,#3=1);
			\cunmergearr{#1}{#2}{#3}{center-#1#2#3}
			}


		\usetikzlibrary{matrix}
		\tikzset{toprule/.style={%
		        execute at end cell={%
		            \draw [line cap=rect,#1]
		            (\tikzmatrixname-\the\pgfmatrixcurrentrow-\the\pgfmatrixcurrentcolumn.north west) -- (\tikzmatrixname-\the\pgfmatrixcurrentrow-\the\pgfmatrixcurrentcolumn.north east);%
		        }
		    },
		    bottomrule/.style={%
		        execute at end cell={%
		            \draw [line cap=rect,#1] (\tikzmatrixname-\the\pgfmatrixcurrentrow-\the\pgfmatrixcurrentcolumn.south west) -- (\tikzmatrixname-\the\pgfmatrixcurrentrow-\the\pgfmatrixcurrentcolumn.south east);%
		        }
		    },
		    leftrule/.style={%
		        execute at end cell={%
		            \draw [line cap=rect,#1] (\tikzmatrixname-\the\pgfmatrixcurrentrow-\the\pgfmatrixcurrentcolumn.north west) -- (\tikzmatrixname-\the\pgfmatrixcurrentrow-\the\pgfmatrixcurrentcolumn.south west);%
		        }
		    },
		    rightrule/.style={%
		        execute at end cell={%
		            \draw [line cap=rect,#1] (\tikzmatrixname-\the\pgfmatrixcurrentrow-\the\pgfmatrixcurrentcolumn.north east) -- (\tikzmatrixname-\the\pgfmatrixcurrentrow-\the\pgfmatrixcurrentcolumn.south east);%
		        }
		    },
		    table with head/.style={
			    matrix of nodes,
			    row sep=-\pgflinewidth,
			    column sep=-\pgflinewidth,
			    nodes={rectangle,minimum width=2.5em, outer sep=0pt},
			    row 1/.style={toprule=thick, bottomrule},
	  	    }
			}
		\usepackage{environ}
\usepackage{xstring}

% Wow this works I'm brilliant
\def\wrapwith#1[#2;#3]{
	\expandarg\IfSubStr{#1}{,}{
		\expandafter#2{\expandarg\StrBefore{#1}{,}}
		\expandarg\StrBehind{#1}{,}[\tmp]
		\xdef\tmp{\expandafter\unexpanded\expandafter{\tmp}}
		#3
		\wrapwith{\tmp}[#2;{#3}]
	}{ \expandafter#2{#1} }
}
\def\hwrapcells#1[#2]{\wrapwith#1[#2;&]}
\def\vwrapcells#1[#2]{\wrapwith#1[#2;\\]}
\NewEnviron{mymathenv}{$\BODY$}

\newcommand{\smalltext}[1]{\text{\footnotesize#1}}
\newsavebox{\idxmatsavebox}
\def\makeinvisibleidxstyle#1#2{\phantom{\hbox{#1#2}}}
\newenvironment{idxmatphant}[4][\color{gray}\smalltext]{%
	\def\idxstyle{#1}
	\def\colitems{#3}
	\def\rowitems{#2}
	\def\phantitems{#4}
	\begin{lrbox}{\idxmatsavebox}$%$\begin{mymathenv}
	\begin{matrix}  \begin{matrix} \hwrapcells{\colitems}[\idxstyle]  \end{matrix}
		% &\vphantom{\idxstyle\colitems}
		\\[-0.05em]
		\left[
		\begin{matrix}
			\hwrapcells{\phantitems}[\expandafter\makeinvisibleidxstyle\idxstyle]  \\[-1.2em]
	}{
		\end{matrix}\right]		&\hspace{-0.8em}\begin{matrix*}[l] \vwrapcells{\rowitems}[\idxstyle] \end{matrix*}\hspace{0.1em}%
	\end{matrix}%
	$%\end{mymathenv}
	\end{lrbox}%
	\raisebox{0.75em}{\usebox\idxmatsavebox}
%	\vspace{-0.5em}
}

\newenvironment{idxmat}[3][\color{gray}\smalltext]
	{\begingroup\idxmatphant[#1]{#2}{#3}{#3}}
	{\endidxmatphant\endgroup}

\newenvironment{sqidxmat}[2][\color{gray}\smalltext]
	{\begingroup\idxmat[#1]{#2}{#2}}
	{\endidxmat\endgroup}


%%%%%%%%%%%%
% better alignment for cases
\makeatletter
\renewenvironment{cases}[1][l]{\matrix@check\cases\env@cases{#1}}{\endarray\right.}
\def\env@cases#1{%
	\let\@ifnextchar\new@ifnextchar
	\left\lbrace\def\arraystretch{1.2}%
	\array{@{}#1@{\quad}l@{}}}
\makeatother

		\tikzset{onslide/.code args={<#1>#2}{%
		  \only<#1>{\pgfkeysalso{#2}} % \pgfkeysalso doesn't change the path
			}}
		}

\usepackage{booktabs,microtype}
\usepackage{mathtools, amsfonts, nicefrac, amssymb, bbm} % mathtools loads amsmath
\usepackage{amsthm,thmtools}

	% \theoremstyle{plain}
	% \let\theorem\relax
	% \newtheorem{theorem}{Theorem}%[section]
	% \newtheorem{coro}{Corollary}[theorem]
	\newtheorem{prop}[theorem]{Proposition}
	% \newtheorem{lemma}[theorem]{Lemma}
	% \newtheorem{fact}[theorem]{Fact}

	\theoremstyle{definition}
	\declaretheorem[name=Definition%,qed=$\square$,numberwithin=section
		]{defn}
	% \declaretheorem[name=Construction,qed=$\square$,sibling=defn]{constr}
	% \declaretheorem[qed=$\square$]{example}
	\theoremstyle{remark}
	\newtheorem*{remark}{Remark}
\relax % Macros (\relax is for folding)
	\let\Horig\H
	\let\H\relax
	\DeclareMathOperator{\H}{\mathrm{H}} %
	\DeclareMathOperator{\I}{\mathrm{I}} %
	\DeclareMathOperator*{\Ex}{\mathbb{E}} %
	\DeclareMathOperator*{\argmin}{arg\;min}
	\newcommand{\CI}{\mathrel{\perp\mspace{-10mu}\perp}} %
	\newcommand\mat[1]{\mathbf{#1}}
	\newcommand\Pa{\mathbf{Pa}}

	\DeclarePairedDelimiterX{\infdivx}[2]{(}{)}{#1\;\delimsize\|\;#2}
	\newcommand{\thickD}{I\mkern-8muD}
	\newcommand{\kldiv}{\thickD\infdivx}

	\newcommand{\tto}{\rightarrow\mathrel{\mspace{-15mu}}\rightarrow}

	\newcommand{\ssub}[1]{_{\!_{#1}\!}}
	\newcommand{\bp}[1][L]{\mat{p}\ssub{#1}}
	\newcommand{\V}{\mathcal V}
	\newcommand{\N}{\mathcal N}
	\newcommand{\Ed}{\mathcal E}

	\DeclareMathAlphabet{\mathdcal}{U}{dutchcal}{m}{n}
	\DeclareMathAlphabet{\mathbdcal}{U}{dutchcal}{b}{n}

	\newcommand{\dg}[1]{\mathbdcal{#1}}
	\newcommand{\pdgunit}{\mathrlap{\mathit 1} \mspace{2.3mu}\mathit 1}

	\newcommand{\IDef}[1]{\mathit{IDef}_{\!#1}}
	\newcommand\Inc{\mathit{Inc}}
	\newcommand{\PDGof}[1]{{\dg M}_{#1}}
	\newcommand{\UPDGof}[1]{{\dg N}_{#1}}
	\newcommand{\WFGof}[1]{\Psi_{{#1}}}
	\newcommand{\FGof}[1]{\Phi_{{#1}}}
	\newcommand{\Gr}{\mathcal G}
	\newcommand\GFE{\mathit{G\mkern-4mu F\mkern-4.5mu E}}

	\newcommand{\ed}[3]{%
		\mathchoice%
		{#2\overset{\smash{\mskip-5mu\raisebox{-3pt}{${#1}$}}}{\xrightarrow{\hphantom{\scriptstyle {#1}}}} #3} %display style
		{#2\overset{\smash{\mskip-5mu\raisebox{-3pt}{$\scriptstyle {#1}$}}}{\xrightarrow{\hphantom{\scriptstyle {#1}}}} #3}% text style
		{#2\overset{\smash{\mskip-5mu\raisebox{-3pt}{$\scriptscriptstyle {#1}$}}}{\xrightarrow{\hphantom{\scriptscriptstyle {#1}}}} #3} %script style
		{#2\overset{\smash{\mskip-5mu\raisebox{-3pt}{$\scriptscriptstyle {#1}$}}}{\xrightarrow{\hphantom{\scriptscriptstyle {#1}}}} #3}} %scriptscriptstyle

	\DeclarePairedDelimiterX{\SD}[1]{\{}{\}}{\,\llap{\delimsize\{}#1\rlap{\delimsize\}}\,}
	%better version.
	\DeclarePairedDelimiterX{\bbr}[1]{[}{]}
		{\mspace{3mu}\mathllap{\delimsize[}#1\mathrlap{\delimsize]}\mspace{3mu}}
	\DeclarePairedDelimiterX{\aar}[1]{\langle}{\rangle}
		{\mspace{3mu}\mathllap{\delimsize\langle}#1\mathrlap{\delimsize\rangle}\mspace{3mu}}
	\DeclarePairedDelimiterXPP{\aard}[1]{}{\langle}{\rangle}{_{\!_\downarrow}}
		{\mspace{-3.5mu}\delimsize\langle#1\delimsize\rangle\mspace{-3.5mu}}

\colorlet{color1}{blue!50!black}
\colorlet{color2}{Emerald!85!black}
\tikzset{hybrid/.style={postaction={draw,color1,dash pattern= on 5pt off 8pt,dash phase=6.5pt,thick},
		draw=color2,dash pattern= on 5pt off 8pt,thick}}

\begin{document}
	\begin{frame}
		\frametitle{Probabilistic Dependency Graphs (PDGs)}
		a new class of probabilistic graphical model, which\textellipsis\pause
		\vspace{-1em}
		\begin{columns}[t]
			\column{0.7\textwidth}
			\begin{itemize}[<+-|alert@+>]
				\item capture inconsistency, including conflicting \tikzmark{conflict-anchor}
					information from multiple sources.
				\item
					are especially modular; to combine info from two sources, simply take a PDG union.
					This merges their information without interference.

					\only<.>{
						\begin{center}
							\color{black}
							\begin{tikzpicture}[paperfig]
								\begin{scope}[dpadded/.append style={inner sep=0.5em,outer sep=0.1em}, draw=color1, thick]
									\node[dpadded] (A) at  (0,1.2) {$A$};
									\node[dpadded] (B) at (0.4, 0) {$B$};
									\node[dpadded] (C)  at (2.1, 0){$C$};

									\draw[arr1] (A) to (B);
									\draw[arr1] (B) to (C);
								\end{scope}
								\end{tikzpicture}
							~{\Large$+$}~
							\begin{tikzpicture}[paperfig]
								\begin{scope}[dpadded/.append style={inner sep=0.5em,outer sep=0.1em},draw=color2,thick]
									\node[dpadded] (B) at (0.4, 0) {$B$};
									\node[dpadded] (C)  at (2.1, 0){$C$};
									\node[dpadded] (D) at (1.5,1.2){$D$};
									\draw[arr1] (B) to (D);
									\mergearr{D}{C}{B}
								\end{scope}
								\end{tikzpicture}
								~{\Large$=$}~
								\begin{tikzpicture}[overlay, scale=0.9, every node/.style={transform shape}]
									\begin{scope}[dpadded/.append style={inner sep=0.5em,outer sep=0.1em},yshift=0.8ex,xshift=1em]
										\node[dpadded,draw=color1, thick] (A) at  (0,0.6) {$A$};
										\node[dpadded,hybrid] (B) at (0.4, -0.6) {$B$};
										\node[dpadded,hybrid] (C)  at (2.1, -0.6){$C$};
										\node[dpadded,hybrid] (D) at (1.5, 0.6){$D$};
									\end{scope}
									\begin{scope}[draw=color1]
										\draw[arr1] (A) to (B);
										\draw[arr1] (B) to (C);
									\end{scope}
									\begin{scope}[draw=color2]
										\draw[arr1] (B) to (D);
										\mergearr{D}{C}{B}
									\end{scope}
									\end{tikzpicture}
							\end{center}
							\vspace{-4.8em}
					}
					
				\item have (several) natural semantics; one of them is a joint distribution.
					Using this distribution, PDGs can capture BNs and factor graphs.
					
					\only<.>{\color{black}
						\begin{center}
						\begin{tikzpicture}[paperfig]
							\node[dpadded] (1) at (0,0) {$\pdgunit$};
							\node[dpadded] (PS) at (1.65,0) {$\mathit{PS}$};
							\node[dpadded]
								(S) at (3.2, 0.8) {$\mathit S$};
							\node[dpadded]
								(SH) at (3.35, -0.8) {$\mathit{SH}$};
							\node[dpadded]
								(C) at (4.8,0) {$\mathit C$};

							\draw[arr1] (1) -- (PS);
							\draw[arr2] (PS) -- (S);
							\draw[arr2] (PS) -- (SH);
							\mergearr{SH}{S}{C}

							\end{tikzpicture}
							\end{center}
						\vspace{-6em}
					}
					% In the latter case, a simple parameter shift in the corresponding PDG eliminates
					% arguably problematic behavior of a factor graph.
				\item cleanly separate quantitative info (the cpds) 
					from qualitative info (the edges), with variable confidence
					in both. 
					%This is captured by  the terms terms $\Inc$ and $\IDef{}$ in their semantics.
				\end{itemize}
			
			\column{0.3\textwidth}
				
				\onslide<2->{
					\begin{tikzpicture}[paperfig,scale=0.9]
						\node[dpadded] (1) {$\pdgunit$};
						\node[dpadded, right=1 of 1] (X) {$X$};
						\draw[arr1] (1) to[bend left=30] node[fill=white, inner sep=2pt] {$q$} (X);
						\draw[arr1] (1) to[bend right=30] node[fill=white, inner sep=2pt] {$p$} (X);
						\end{tikzpicture}
				}
				\bigskip
				
				\onslide<4->{\hfill%
					\begin{tikzpicture}[paperfig]
						\begin{scope}[dpadded/.append style={inner sep=0.5em,outer sep=0.1em}]
							\node[dpadded,draw=color1, thick] (A) at  (0,1.2) {$A$};
							\node[dpadded,hybrid] (B) at (0.4, 0) {$B$};
							\node[dpadded,hybrid] (C)  at (2.1, 0){$C$};
							\node[dpadded,hybrid] (D) at (1.5,1.2){$D$};
						\end{scope}
						\begin{scope}[draw=color1]
							\draw[arr1] (A) to (B);
							\draw[arr1] (B) to (C);
						\end{scope}
						\begin{scope}[draw=color2]
							\draw[arr1] (B) to (D);
							\mergearr{D}{C}{B}
						\end{scope}
						\end{tikzpicture}
				}
				
				\bigskip
								
				\onslide<5->{
					\begin{tikzpicture}[paperfig,dpadded/.append style={inner sep=0.6em,outer sep=0.2em}, scale=0.85]
						\node[dpadded] (1) at (1.65,1) {$\pdgunit$};
						\node[dpadded] (PS) at (1.65,-0.4) {$\mathit{PS}$};
						\node[dpadded] (S) at (3.2, 0.8) {$S$};
						\node[dpadded] (SH) at (3.35, -0.8) {$\mathit{SH}$};
						\node[dpadded] (C) at (4.8,0) {$C$};

						\draw[arr1] (1) -- (PS);
						\draw[arr2] (PS) -- (S);
						\draw[arr2] (PS) -- (SH);
						\mergearr{SH}{S}{C}
					\end{tikzpicture}
				}
			 
			\end{columns}
			\onslide<5->{\color{black}
			\smallskip
			\[
					\bbr{\dg M}_\gamma(\mu) :=
						{\color{benchcolor1}\underbrace{
							\Inc_{\dg M}(\mu) }_{\left(%
								{\footnotesize quantitative}\right)}} +
								{\gamma} 
								% \extra[above right=1em and -1 em of call point, 
								% 	inner sep=0.3em, font=\small, name=tp, draw=alertcolor, text=alertcolor!40!black,
								% 	fill=alertcolor!10]{\small tradeoff parameter $\gamma \ge 0$}
						{\color{benchcolor2}\underbrace{\IDef{\dg M}(\mu) }_{\left(%
								{\footnotesize qualitative}\right)}}
						\]
			}

		\end{frame}
\end{document}
