%  Beamer Style
\documentclass[xcolor={dvipsnames,rgb}]{beamer}
% \includeonlyframes{current}

\usepackage{lmodern}
\usepackage[utf8]{inputenc}
\usetheme{Boadilla}% Beamer Theme Customization...
	\setbeamersize{description width=0.57cm}
	\usefonttheme[stillsansserifsmall]{serif}
		% \usefonttheme{structuresmallcapsserif}
	\usefonttheme[onlylarge]{structuresmallcapsserif}
		% \usefonttheme[onlymath]{serif}
		% \usefonttheme[onlysmall]{structurebold}
	\setbeamerfont{item}{series=\bfseries}
	\setbeamerfont{block title}{series=\bfseries}
	% \setbeamerfont{title}{family=\rmfamily}

	\relax%%% color definitions %%...
		\colorlet{structurecolor}{RoyalPurple!50!black}
		\colorlet{alertcolor}{YellowOrange}
			% \colorlet{alertcolor}{structurecolor>wheel,1,3}
		\colorlet{benchcolor1}{Emerald!85!black}
			% \colorlet{benchcolor1}{structurecolor>wheel,2,3}
		\colorlet{benchcolor2}{YellowOrange!25!magenta}
	\usecolortheme[named=structurecolor]{structure}
		% \usecolortheme{beaver}
		% \setbeamercolor*{palette primary}{bg=color1, fg = green}
		% \setbeamercolor*{palette secondary}{bg=color2, fg = green}
		% \setbeamercolor*{palette tertiary}{bg=color3, fg = green}
		% \setbeamercolor*{palette quaternary}{bg=color4, fg = green}
		% \makeatletter
		% \definecolor{beamer@blendedblue}{rgb}{0.2,0.2,0.7}
		% \colorlet{beamer@blendedblue}{color2}
		% \makeatother
	\setbeamercolor{description item}{bg={structurecolor!20!white}}
	\setbeamercolor{alerted text}{fg=alertcolor}

	\newbool{precompiledfigs}% ...
		% \setbool{precompiledfigs}{true}
		\setbool{precompiledfigs}{false}
		% the etoolbox way, which works with beamer.
	% \setbeamercovered{dynamic}

\relax %%%%%%%%  Beamer and slide-specific macros  %%%%%%%%%%%%%%%%%%%
	\newcommand<>{\hl}[2][alertcolor]{\begingroup%
		\setbeamercolor{alerted text}{fg=#1}\alert#3{#2}\endgroup}
	\colorlet{notationcolor}{structurecolor!40}
	\def\notation#1{\!\hl[notationcolor]{#1$\quad$}}
	\def\notation#1{\!\hl[notationcolor]{#1$\quad$}}
	% \newcommand<>{\alertwith}[2]{\begingroup\only#3{\setbeamercolor{alerted text}{fg=#1}}#2\endgroup} % DOESN'T WORK THIS WAY
	\newenvironment{localfocusenv}{\only{\setbeamercolor{local structure}{fg=alertcolor}}}{}
	\newenvironment<>{hidemeenv}{%
		\only#1{\setbeamercolor{alerted text}{fg=black!60}}%
		\begingroup\begin{alertenv}#1%
		}{\end{alertenv}\endgroup}
	\newenvironment<>{tikzpicture||precompiled}[2][]{
			\ifbool{precompiledfigs}{\includegraphics[width=0.8\linewidth]{figure-pdfs/#2}
				}\begingroup\only#3\begingroup\begin{tikzpicture}[#1]
		}{\end{tikzpicture}\endgroup\endgroup}
	\newcommand<>{\extra}[2][]{%
		\only#3{%
			% \tikzmark{call point};%
			\tikzro \node[inner sep=0pt,outer sep=0pt] (call point) {};%
			\begin{tikzpicture}[overlay,remember picture]
				\node[anchor=north west, inner sep=0.8em,
				 			fill=alertcolor!30!structurecolor!30!white,
							draw=structurecolor!70!black, draw opacity=0.5,
							below=1em of call point, #1]{#2};
			\end{tikzpicture}%
		}}
	\newcommand{\tikzro}[1][]{\tikz[remember picture, overlay,#1]}
	\def\Set{\mathbf{Set}}
	\makeatletter
	\newcommand{\shorteq}{%
	  \settowidth{\@tempdima}{-}% Width of hyphen
	  \resizebox{\@tempdima}{\height}{=}%
	}
	\makeatother
	% \newcommand{\tikzmark}[1][last mark]{\tikzro \node (#1){};}

%%%%%%%            Relevant part of PDG Preamble        %%%%%%%%%%%%%%%%
\usepackage{tikz}
	\usetikzlibrary{positioning,calc, arrows, shapes}

	\tikzset{AmpRep/.style={ampersand replacement=\&}}
	\tikzset{center base/.style={baseline={([yshift=-.8ex]current bounding box.center)}}}
	\tikzset{paperfig/.style={center base,scale=0.9, every node/.style={transform shape}}}

	\tikzset{dpadded/.style={rounded corners=2, inner sep=0.6em, draw, outer sep=0.2em, fill={black!50}, fill opacity=0.08, text opacity=1}}
	\tikzset{light pad/.style={outer sep=0.2em, inner sep=0.5em, draw=gray!50}}
	\tikzset{arr/.style={draw, ->, thick, shorten <=3pt, shorten >=3pt}}
	\tikzset{arr0/.style={draw, ->, thick, shorten <=0pt, shorten >=0pt}}
	\tikzset{arr1/.style={draw, ->, thick, shorten <=1pt, shorten >=1pt}}
	\tikzset{arr2/.style={draw, ->, thick, shorten <=2pt, shorten >=2pt}}

	\newcommand{\drawbb}%
		{\draw (current bounding box.south west) rectangle (current bounding box.north east);}
	\ifbool{precompiledfigs}{}{
		\usetikzlibrary{fit, decorations,shapes.geometric}
		\usetikzlibrary{tikzmark}
		\usetikzlibrary{backgrounds}
		\pgfdeclarelayer{foreground}
		\pgfsetlayers{background,main,foreground}

		\pgfdeclaredecoration{arrows}{draw}{
			\state{draw}[width=\pgfdecoratedinputsegmentlength]{%
				\path [every arrow subpath/.try] \pgfextra{%
					\pgfpathmoveto{\pgfpointdecoratedinputsegmentfirst}%
					\pgfpathlineto{\pgfpointdecoratedinputsegmentlast}%
				};
		}}

		\tikzset{dpad0/.style={outer sep=0.05em, inner sep=0.3em, draw=gray!75, rounded corners=4, fill=black!08, fill opacity=1}}
		\tikzset{dpad/.style args={#1}{every matrix/.append style={nodes={dpadded, #1}}}}
		\tikzset{is bn/.style={background rectangle/.style={fill=blue!35,opacity=0.3, rounded corners=5},show background rectangle}}
		% \usetikzlibrary{backgrounds}
		% \usetikzlibrary{patterns}
		\usetikzlibrary{cd}

		\tikzset{fgnode/.style={dpadded,inner sep=0.2em, circle,minimum width=2.3em},
				 factor/.style={light pad, fill=black, outer sep=0pt,draw=none}}


		\newcommand\cmergearr[4]{
			\draw[arr,-] (#1) -- (#4) -- (#2);
			\draw[arr, shorten <=0] (#4) -- (#3);
			}
		\newcommand\mergearr[3]{
			\coordinate (center-#1#2#3) at (barycentric cs:#1=1,#2=1,#3=1.2);
			\cmergearr{#1}{#2}{#3}{center-#1#2#3}
			}
		\newcommand\cunmergearr[4]{
			\draw[arr,-, , shorten >=0] (#1) -- (#4);
			\draw[arr, shorten <=0] (#4) -- (#2);
			\draw[arr, shorten <=0] (#4) -- (#3);
			}
		\newcommand\unmergearr[3]{
			\coordinate (center-#1#2#3) at (barycentric cs:#1=1.2,#2=1,#3=1);
			\cunmergearr{#1}{#2}{#3}{center-#1#2#3}
			}


		\usetikzlibrary{matrix}
		\tikzset{toprule/.style={%
		        execute at end cell={%
		            \draw [line cap=rect,#1]
		            (\tikzmatrixname-\the\pgfmatrixcurrentrow-\the\pgfmatrixcurrentcolumn.north west) -- (\tikzmatrixname-\the\pgfmatrixcurrentrow-\the\pgfmatrixcurrentcolumn.north east);%
		        }
		    },
		    bottomrule/.style={%
		        execute at end cell={%
		            \draw [line cap=rect,#1] (\tikzmatrixname-\the\pgfmatrixcurrentrow-\the\pgfmatrixcurrentcolumn.south west) -- (\tikzmatrixname-\the\pgfmatrixcurrentrow-\the\pgfmatrixcurrentcolumn.south east);%
		        }
		    },
		    leftrule/.style={%
		        execute at end cell={%
		            \draw [line cap=rect,#1] (\tikzmatrixname-\the\pgfmatrixcurrentrow-\the\pgfmatrixcurrentcolumn.north west) -- (\tikzmatrixname-\the\pgfmatrixcurrentrow-\the\pgfmatrixcurrentcolumn.south west);%
		        }
		    },
		    rightrule/.style={%
		        execute at end cell={%
		            \draw [line cap=rect,#1] (\tikzmatrixname-\the\pgfmatrixcurrentrow-\the\pgfmatrixcurrentcolumn.north east) -- (\tikzmatrixname-\the\pgfmatrixcurrentrow-\the\pgfmatrixcurrentcolumn.south east);%
		        }
		    },
		    table with head/.style={
			    matrix of nodes,
			    row sep=-\pgflinewidth,
			    column sep=-\pgflinewidth,
			    nodes={rectangle,minimum width=2.5em, outer sep=0pt},
			    row 1/.style={toprule=thick, bottomrule},
	  	    }
			}
		\usepackage{environ}
\usepackage{xstring}

% Wow this works I'm brilliant
\def\wrapwith#1[#2;#3]{
	\expandarg\IfSubStr{#1}{,}{
		\expandafter#2{\expandarg\StrBefore{#1}{,}}
		\expandarg\StrBehind{#1}{,}[\tmp]
		\xdef\tmp{\expandafter\unexpanded\expandafter{\tmp}}
		#3
		\wrapwith{\tmp}[#2;{#3}]
	}{ \expandafter#2{#1} }
}
\def\hwrapcells#1[#2]{\wrapwith#1[#2;&]}
\def\vwrapcells#1[#2]{\wrapwith#1[#2;\\]}
\NewEnviron{mymathenv}{$\BODY$}

\newcommand{\smalltext}[1]{\text{\footnotesize#1}}
\newsavebox{\idxmatsavebox}
\def\makeinvisibleidxstyle#1#2{\phantom{\hbox{#1#2}}}
\newenvironment{idxmatphant}[4][\color{gray}\smalltext]{%
	\def\idxstyle{#1}
	\def\colitems{#3}
	\def\rowitems{#2}
	\def\phantitems{#4}
	\begin{lrbox}{\idxmatsavebox}$%$\begin{mymathenv}
	\begin{matrix}  \begin{matrix} \hwrapcells{\colitems}[\idxstyle]  \end{matrix}
		% &\vphantom{\idxstyle\colitems}
		\\[-0.05em]
		\left[
		\begin{matrix}
			\hwrapcells{\phantitems}[\expandafter\makeinvisibleidxstyle\idxstyle]  \\[-1.2em]
	}{
		\end{matrix}\right]		&\hspace{-0.8em}\begin{matrix*}[l] \vwrapcells{\rowitems}[\idxstyle] \end{matrix*}\hspace{0.1em}%
	\end{matrix}%
	$%\end{mymathenv}
	\end{lrbox}%
	\raisebox{0.75em}{\usebox\idxmatsavebox}
%	\vspace{-0.5em}
}

\newenvironment{idxmat}[3][\color{gray}\smalltext]
	{\begingroup\idxmatphant[#1]{#2}{#3}{#3}}
	{\endidxmatphant\endgroup}

\newenvironment{sqidxmat}[2][\color{gray}\smalltext]
	{\begingroup\idxmat[#1]{#2}{#2}}
	{\endidxmat\endgroup}


%%%%%%%%%%%%
% better alignment for cases
\makeatletter
\renewenvironment{cases}[1][l]{\matrix@check\cases\env@cases{#1}}{\endarray\right.}
\def\env@cases#1{%
	\let\@ifnextchar\new@ifnextchar
	\left\lbrace\def\arraystretch{1.2}%
	\array{@{}#1@{\quad}l@{}}}
\makeatother

		\tikzset{onslide/.code args={<#1>#2}{%
		  \only<#1>{\pgfkeysalso{#2}} % \pgfkeysalso doesn't change the path
			}}
		}

\usepackage{booktabs,microtype}
\usepackage{mathtools, amsfonts, nicefrac, amssymb, bbm} % mathtools loads amsmath
\usepackage{amsthm,thmtools}

	% \theoremstyle{plain}
	% \let\theorem\relax
	% \newtheorem{theorem}{Theorem}%[section]
	% \newtheorem{coro}{Corollary}[theorem]
	\newtheorem{prop}[theorem]{Proposition}
	% \newtheorem{lemma}[theorem]{Lemma}
	% \newtheorem{fact}[theorem]{Fact}

	\theoremstyle{definition}
	\declaretheorem[name=Definition%,qed=$\square$,numberwithin=section
		]{defn}
	% \declaretheorem[name=Construction,qed=$\square$,sibling=defn]{constr}
	% \declaretheorem[qed=$\square$]{example}
	\theoremstyle{remark}
	\newtheorem*{remark}{Remark}
\relax % Macros (\relax is for folding)
	\let\Horig\H
	\let\H\relax
	\DeclareMathOperator{\H}{\mathrm{H}} %
	\DeclareMathOperator{\I}{\mathrm{I}} %
	\DeclareMathOperator*{\Ex}{\mathbb{E}} %
	\DeclareMathOperator*{\argmin}{arg\;min}
	\newcommand{\CI}{\mathrel{\perp\mspace{-10mu}\perp}} %
	\newcommand\mat[1]{\mathbf{#1}}
	\newcommand\Pa{\mathbf{Pa}}

	\DeclarePairedDelimiterX{\infdivx}[2]{(}{)}{#1\;\delimsize\|\;#2}
	\newcommand{\thickD}{I\mkern-8muD}
	\newcommand{\kldiv}{\thickD\infdivx}

	\newcommand{\tto}{\rightarrow\mathrel{\mspace{-15mu}}\rightarrow}

	\newcommand{\ssub}[1]{_{\!_{#1}\!}}
	\newcommand{\bp}[1][L]{\mat{p}\ssub{#1}}
	\newcommand{\V}{\mathcal V}
	\newcommand{\N}{\mathcal N}
	\newcommand{\Ed}{\mathcal E}

	\DeclareMathAlphabet{\mathdcal}{U}{dutchcal}{m}{n}
	\DeclareMathAlphabet{\mathbdcal}{U}{dutchcal}{b}{n}

	\newcommand{\dg}[1]{\mathbdcal{#1}}
	\newcommand{\pdgunit}{\mathrlap{\mathit 1} \mspace{2.3mu}\mathit 1}

	\newcommand{\IDef}[1]{\mathit{IDef}_{\!#1}}
	\newcommand\Inc{\mathit{Inc}}
	\newcommand{\PDGof}[1]{{\dg M}_{#1}}
	\newcommand{\UPDGof}[1]{{\dg N}_{#1}}
	\newcommand{\WFGof}[1]{\Psi_{{#1}}}
	\newcommand{\FGof}[1]{\Phi_{{#1}}}
	\newcommand{\Gr}{\mathcal G}
	\newcommand\GFE{\mathit{G\mkern-4mu F\mkern-4.5mu E}}

	\newcommand{\ed}[3]{%
		\mathchoice%
		{#2\overset{\smash{\mskip-5mu\raisebox{-3pt}{${#1}$}}}{\xrightarrow{\hphantom{\scriptstyle {#1}}}} #3} %display style
		{#2\overset{\smash{\mskip-5mu\raisebox{-3pt}{$\scriptstyle {#1}$}}}{\xrightarrow{\hphantom{\scriptstyle {#1}}}} #3}% text style
		{#2\overset{\smash{\mskip-5mu\raisebox{-3pt}{$\scriptscriptstyle {#1}$}}}{\xrightarrow{\hphantom{\scriptscriptstyle {#1}}}} #3} %script style
		{#2\overset{\smash{\mskip-5mu\raisebox{-3pt}{$\scriptscriptstyle {#1}$}}}{\xrightarrow{\hphantom{\scriptscriptstyle {#1}}}} #3}} %scriptscriptstyle

	\DeclarePairedDelimiterX{\SD}[1]{\{}{\}}{\,\llap{\delimsize\{}#1\rlap{\delimsize\}}\,}
	%better version.
	\DeclarePairedDelimiterX{\bbr}[1]{[}{]}
		{\mspace{3mu}\mathllap{\delimsize[}#1\mathrlap{\delimsize]}\mspace{3mu}}
	\DeclarePairedDelimiterX{\aar}[1]{\langle}{\rangle}
		{\mspace{3mu}\mathllap{\delimsize\langle}#1\mathrlap{\delimsize\rangle}\mspace{3mu}}
	\DeclarePairedDelimiterXPP{\aard}[1]{}{\langle}{\rangle}{_{\!_\downarrow}}
		{\mspace{-3.5mu}\delimsize\langle#1\delimsize\rangle\mspace{-3.5mu}}

%Information to be included in the title page:
\title{Probabilsitic Dependency Graphs}
	\author[O.~Richardson, J.~Halpern]{Oliver~E.~Richardson \and Joseph~Y.~Halpern}
	\institute[Cornell]{Cornell University\\Department of Computer Science}
	\date{AAAI, Febuary 2021}

\begin{document}
\frame{\titlepage}


\begin{frame}\frametitle{Yet Another Probabilistic Graphical Model}
	We introduce \textit{probabilistic dependency graphs} (PDGs), a new class of graphical models
	for representing uncertainty. 
	
	\pause
	\bigskip
	{\centering\Large Why do we need another one?\par}
	
	\bigskip
	\pause
	\begin{itemize}[<+-|alert@+>]
		\item To resolve inconsistency, we must first model it.
		\item In doing so, we get much more \ldots
	\end{itemize}
\end{frame}



	\begin{frame} %%%%%%%%%%%        REVIEW OF BNS      %%%%%%%%%%%%%%%%
		\frametitle{Two aspects of Bayesian Networks (BNs)}
		\begin{description}
			\item<+->[{\color{benchcolor2!70!black}Qualitative} BN,~~$\Gr$]~\\
				an independence relation on variables
				\begin{itemize}\small
					\item {\color{gray} $X \CI_{\Gr} Y \mid \Pa(X)$, for all non-descendents $Y$ 	of $X$}
				\end{itemize}\medskip
			\item<+->[({\color{benchcolor1!80!black}Quantitative}) BN,~~$\mathcal B = (\Gr, \mat p)$]~\\
					a qualitative BN ($\Gr$) and a cpd
					$p_{\!_X}(X \mid \Pa(X))$ for each variable $X$.
					\vspace{-1.2em}
					%
					\begin{itemize} \small
						\item {\color{gray}Defines a joint distribution $\Pr_{\mathcal B}$ \hl[benchcolor2]<+>{with the independencies $\CI_\Gr$}.}
					\end{itemize}
			\end{description}
		\vspace{2em}
		\begin{center}
			\begin{tikzpicture}[paperfig]
				\begin{scope}[every node/.style={dpadded, fill opacity=1,fill=black!08, circle, inner sep=2pt, minimum size=2em, draw=gray}]
					\node (PS) at (0,0) {$\mathit{PS}$};
					\node (SH) at (1.5, 0.6) {$\mathit{SH}$};
					\node (S) at (1.5, -0.6) {$\mathit{S}$};
					\node (C) at (3, 0) {$\mathit{C}$};
					\end{scope}
				\draw[->] (PS) to (S);
				\draw[->] (PS) to (SH);
				\draw[->] (SH) to (C);
				\draw[->] (S) to (C);
				\end{tikzpicture}
			\end{center}
		\end{frame}

\section{Modeling Examples}
	\subsection{Inconsistencies}
		\newlength{\bncolwidth}
		\begin{frame} %%%%%%%%%       GUN FLOOMP: BN vs PDG       %%%%%%%%%%
			%% 1 -- Just BN
			%% 2 -- add PDG.
			%% 3 -- List appears, arbitrary new info.
			%% 4 -- list highlighted, p appears both diagrams (dashed), green box.
			%% 5 -- inconsistent PDG?
			%% 6-8 -- variants of BN w / different info. Final point visible.
			\frametitle{Modeling Example: Floomps and Guns}
			% \framesubtitle{The Legality of Floomps and Guns}
			\only<1-3>{
				{\centering	Grok thinks it likely (.95) that guns are illegal,\\
				but that floomps (local slang) are legal (.90).\par}
			}
			\pause
			\colorlet{heldout}{benchcolor1!80!black}

			% \only<2->{ % Headers "BN" and "PDG"
				\vspace{1.2em}
				\begin{columns}[c]
					\Large\color{structurecolor}
					\begin{column}{.45\textwidth}\centering% Add BN to diagram
						\only<4,5>{\color{structurecolor!20}}%
						\textbf{BN}\\[-1em]
						\rule{2.2cm}{0.95pt}\end{column}
					\only<3->{ % add PDG to diagram
						% \vrule
						\begin{column}{.5\textwidth}\centering%
							\only<6->{\color{structurecolor!20}}%
							\textbf{PDG}\\[-1em]
							\rule{2.7cm}{0.95pt}\end{column}
						}
					\end{columns}
				\vspace{0.0em}
				% }

	
			\begin{columns}[c] %% (both diagrams) ...
				\setlength{\bncolwidth}{0.45\textwidth}
				\only<6->{\setlength{\bncolwidth}{0.51\textwidth}}
				\begin{column}{\bncolwidth} % BN Diagram
					% \begin{block}{BN}
						\centering
						% \hspace{-2.5em}
						\begin{tikzpicture||precompiled}{fg-BN}[AmpRep, scale=0.9]
							\def\figtabledist{0.50}
							\def\fignodedist{0.8}
							\def\figtableheight{0.32}

							%% Time to unify the notation for cpds.
							% \matrix [table with head, column 1/.style={leftrule}, anchor=south east,
							% 	 column 2/.style={rightrule}, row 2/.style={bottomrule}] at (-\figtabledist,\figtableheight) {
							% 	\vphantom{$\overline fg$} $f$ \& \vphantom{$\overline fg$}$\overline f$\\
							% 	.9 \& .1\\
							% };
							% \matrix [table with head, column 1/.style={leftrule}, anchor=south west,
							% 	 column 2/.style={rightrule}, row 2/.style={bottomrule}] at (\figtabledist,\figtableheight) {
							% 	 \vphantom{$\overline fg$}$g$ \& \vphantom{$\overline fg$}$\overline g$\\
							% 	 .05 \& .95\\
							% };
							\node[% F's cpd ...
									anchor=south east]
							 	at (-\figtabledist,\figtableheight) {%
								\hspace{-1.2em}%
								\only<1-6,8>{\begin{idxmat} [\color{gray!60}\smalltext]
									{\!\!}{$f$,$\overline f$}
										.90 & .10 \\
									\end{idxmat}}%
								\only<7>{ \begin{idxmat}[\color{gray!60}\smalltext]
									{$g$,$\overline g$}{$f$,$\overline f$}
										.92 & .08 \\
										.08 & .92 \\
									\end{idxmat}\hspace{-1em}\;}%
								};
							\node[% G's cpd ...
									anchor=south west]
								at (\figtabledist,\figtableheight) {%
								\hspace{-1.2em}%
								\only<1-7>{\begin{idxmat}[\color{gray!60}\smalltext]
									{\!\!}{$g$,$\overline g$}
									.05 & .95 \\
									\end{idxmat}}%
								\only<8>{\begin{idxmat}[\color{gray!60}\smalltext]
									{$f$,$\overline f$}{$g$,$\overline g$}
										.92 & .08 \\
										.08 & .92 \\
									\end{idxmat}}%
								% \hspace{-0.5em}~%
								};
							\node[% F (floomp) ...
								dpadded, inner sep=0.5em, circle, fill=black!08, fill opacity=1]
								(floomp) at (-\fignodedist,0) {$F$};
							\node[% G (gun) ...
								dpadded, inner sep=0.5em, circle, fill=black!08, fill opacity=1]
								(gun) at (\fignodedist,0) {$G$};
							% \node (leftbounds) at (-3,0){};
							% \node (rightbounds) at (3,0){};

							\only<7,8>{ \draw[thick, ->, onslide=<8>{<-}, benchcolor2] 
								(gun) -- node[below,align=center,font=\tiny]{must\\choose\\direction} (floomp); }

							\only<6->{\node[align=left,% show which cpds are incorporated...
									rounded corners=5, inner sep=0.5em, 
									fill=structurecolor!50!benchcolor1!10,
									draw=structurecolor!50!benchcolor1!50] (incorporated) 
							 		at (0,-1.5)
								{ \textit{Incorporated}:\\\Large
									\hspace{1em}
									\hl[benchcolor1]<6,8>{$\mu\ssub F$}
									\hspace{1em}
									\hl[benchcolor1]<6,7>{$\mu\ssub G$}
									\hspace{1em}
									\hl[benchcolor1]<8>{$p$}
									\hspace{1em}
									\hl[benchcolor1]<7>{$p'$}
								};\node[below=1em of incorporated]{};}

							\only<4-6>{		\node[text=alertcolor] at (0,+0.8){$p$??};		}
							\only<4,5>{ \fill[white,opacity=0.8] (-3, -0.8) rectangle (3,1.8); }
							\useasboundingbox (-2.7, -0.8) rectangle (2.7,1.8);
							\end{tikzpicture||precompiled}
					% \end{block}
					\vspace{-0.8em}
					\end{column}
				\only<3,4>{{\color{gray}\vrule}}
				\begin{onlyenv}<3->\begin{column}{0.95\textwidth-\bncolwidth}
					\only<6->{{~\hspace{-2em}~}}
					% \centering
					\begin{tikzpicture||precompiled}{fg-PDG}[]
						\def\fignodedist{1.5}
						\node[% for  "1"  /  "true"  ...
							dpadded, fill=gray!20, draw=gray!70, inner sep=0.35em, outer sep=0.37em]
							(true)  at (0,1.3) {$\pdgunit$};
						\node[dpadded] (floomp) at (-\fignodedist, 0) {$F$};
						\node[dpadded] (gun) at (\fignodedist, 0) {$G$};

						\begin{pgfonlayer}{foreground}
							\draw[arr1,{onslide=<6,8>{benchcolor1}}]
								(true.-165) -- %to[bend right=0]
									(floomp.60)
									node[pos=.5,{onslide=<6->{above left}}] (A)
										{\only<6->{$\mu\ssub F$}}
									;
							\draw[arr1,{onslide=<6,7>{benchcolor1}}]
								(true.-15) -- %to[bend left=0]
									(gun.120)
									node[pos=.5,{onslide=<6->{above right}}] (B)
									 	{\only<6->{$\mu\ssub G$}}
									;
							\end{pgfonlayer}

						\node[, % CPT for F...
							above left=1.5em and 2.5em of A.center, anchor=center] {%
							\color<5>{alertcolor}
							\alt<6->{}{
								\hspace{-0.8em}%
								\begin{idxmat}
									[\color{gray!60}\color<5>{alertcolor!50}\smalltext]
									%[\color{black}\smalltext]
									{$\star$}{$f$, $\overline f$}
									.90 & .10 \\
								\end{idxmat}
								\hspace{-1em}~}
							};
						\node[, % CPT for G...
							above right=1.5em and 2.3em of B.center, anchor=center] {
							\color<5>{alertcolor}
							\alt<6->{}{
								\hspace{-0.8em}
								\begin{idxmat}
									[\color{gray!60}\color<5>{alertcolor!50}\smalltext]
									%[\color{black}\smalltext]
									{$\star$}{$g$, $\overline g$}
									.05 & .95 \\
								\end{idxmat}
								\hspace{-1em}~}
							};
						\only<4->{ % Show p & arrows, after initial diagrams...
							\begin{pgfonlayer}{foreground}
								\draw[arr,
										onslide=<4>{heldout, dashed},
								 		{onslide=<8>{benchcolor1}},
										onslide=<5>{text=alertcolor} ]
								 	(floomp.-33) to[bend right=6] node[pos=0.65, fill=white, inner sep=2pt] (C) {$\smash{p}\vphantom{v}$} (gun.210);
								\draw[arr,
										onslide=<4>{heldout, dashed},
										{onslide=<7>{benchcolor1}},
										onslide=<5>{text=alertcolor} ]
								 	(gun.190) to[bend left=5] node[pos=0.668, fill=white, inner sep=2pt] {$\smash{p'}\vphantom{v}$} (floomp.-10);
								\end{pgfonlayer}
							}
						\only<6->{ \fill[white,opacity=0.8] (-2.6, -1) rectangle (2.7,2); }%
						\useasboundingbox (-2.7, -0.8) rectangle (2.7,2.0);
						\end{tikzpicture||precompiled}
					\vspace{-1em}
					\end{column}\end{onlyenv}
				\end{columns}
				\only<-5>{\vspace{1em}}
			\begin{itemize} %% bullet points
				\item<3-| alert@3> The cpds of a PDG are attached to edges, not nodes.
				\item<4-| alert@4> PDGs can incorporate arbitrary new probabilistic information.
				\only<4>{ % New Information! The green block with cpts
					\vspace{0.7em} \color{black}
					\begin{exampleblock}{}
						% You come to believe that Floomps and Guns share legal status (92\%).
						{\small Grok learns that Floomps and Guns have the same legal status (92\%)}
						\vspace{-0.6em}
						$$ {\color{heldout}p(G \!\mid\! F)} =
							\begin{idxmat}[\color{gray!50}\smalltext]
									{$f$,$\overline f$}{$g$, $\overline g$}
								.92 & .08 \\ .08 & .92 \\
							\end{idxmat}
							~=~ {\left(\;{\color{heldout} p'(F \!\mid\! G)}\;\right)^{\textsf T}}$$
						\end{exampleblock}}
				\item<5-| alert@5> PDGs can be inconsistent\only<6>{,}
					\begin{itemize}
						\item<6- | alert@6-> \textellipsis but BNs must resolve inconsistency first, \\
							{\small\color{gray} which may \hl<7-8>[benchcolor2]{break symmetry}
								and irrecoverably lose information.} %joe1
							\note{and so it may be better to wait to resolve it.}
						\end{itemize}
				\end{itemize}
			\end{frame} %------------%

	
	\subsection{Capturing Bayesian Networks}
		\begin{frame} %%%%%%%%%        SMOKING: BN vs PDG         %%%%%%%%%%
			\frametitle{Bayesian Networks as PDGs}
			% \framesubtitle{Smoking and Cancer}
			\colorlet{heldout}{benchcolor1}
			\begin{center}
				\hfill
				\begin{tikzpicture||precompiled}[paperfig]{smoking-BN}
					\begin{scope} % BN Nodes...
						[every node/.style={dpadded, fill opacity=1,fill=black!08, circle, inner sep=2pt, minimum size=2em, draw=gray}]
						\node[onslide=<6->{opacity=0.2, text opacity=0.2}] (PS) at (0,0) {$\mathit{PS}$};
						\node (SH) at (1.5, 0.6) {$\mathit{SH}$};
						\node (S) at (1.5, -0.6) {$\mathit{S}$};
						\node (C) at (3, 0) {$\mathit{C}$};
						\end{scope}
					\draw[->,onslide=<6->{opacity=0.2}] (PS) to (S);
					\draw[->,onslide=<6->{opacity=0.2}] (PS) to (SH);
					\draw[->] (SH) to (C);
					\draw[->] (S) to (C);

					\only<6->{% Describe why restriction of BN is not a BN
						\node[below left=0.65 and 0.4 of PS,anchor=north west] (condBNtext)
						{\parbox{2.0in}{\small\raggedright\color{benchcolor1!60!alertcolor}
							Must now give distributions on $\mathit{SH}$ and $\mathit{S}$, or distinguish them as ``observed'' (a \emph{conditional} BN). %
							}};%
						}
					\end{tikzpicture||precompiled}%
				\hfill\pause\only<-6>{{\color<6->{gray!50}\vrule}}\hfill%
				\onslide<-6>{
				\begin{tikzpicture||precompiled}[paperfig]{smoking-PDG}
					\onslide<5->{ % Matt for restriction
						\fill[fill opacity=0.1, blue!80!black, draw, draw opacity=0.5] (2.73,1.35) rectangle (6.8, -1.35);}

					\node[dpadded] (1) at (0,0) {$\pdgunit$};
					\node[dpadded] (PS) at (1.65,0) {$\mathit{PS}$};
					\node[dpadded, % (S) ...
					 	onslide=<5->{fill=black!.16, fill opacity=0.9}]
						(S) at (3.2, 0.8) {$\mathit S$};
					\node[dpadded, % (SH)...
					 	onslide=<5->{fill=black!.16, fill opacity=0.9}]
						(SH) at (3.35, -0.8) {$\mathit{SH}$};
					\node[dpadded, % (C) ...
					 	onslide=<5->{fill=black!.16, fill opacity=0.9}]
						(C) at (4.8,0) {$\mathit C$};

					\mergearr{SH}{S}{C} % duplicated below; this one is to get coordinates right
					\onslide<3>{ % Draw cpts attached to each edge...
						% \node;
						\begin{scope}[thick,benchcolor1,dashed]
							\draw[] ($(1)!0.4!(PS)$) -- ++(0,0.85)
								node[above] {$p(\mathit{PS})$};
							\draw[] ($(PS)!0.5!(S)$) -- ++(-0.4,1.1)
								node[above] {$p(\mathit{S}\!\mid\!\mathit{PS})$};
							\draw[] ($(PS)!0.48!(SH)$) -- ++(-0.6,-0.75)
								node[below left] {$p(\mathit{SH}\!\mid\!\mathit{PS})$};
							\draw[] ($(center-SHSC)!0.11!(S)$) -- ++(0.6,0.75)
								node[above right] {$p(\mathit{C}\!\mid\!\mathit{S}, \mathit{SH})$};
							\end{scope}
						}

					\draw[arr1] (1) -- (PS);
					\draw[arr2] (PS) -- (S);
					\draw[arr2] (PS) -- (SH);
					\mergearr{SH}{S}{C}

					\onslide<4->{ % Add Tanning Beds & arrow...
						\node[dpadded,
							 	onslide=<4>{fill=benchcolor1!36, fill opacity=0.75,
								 	draw=benchcolor1!20!black, dashed},
								onslide=<5->{fill=black!.16, fill opacity=0.9}]
							(T) at (6.25,0) {$T$};
						\draw[arr1,onslide=<4>{dashed,draw=benchcolor1!60!black}] (T) -- (C);
						}
					\onslide<5>{ % Draw matt & label for restriction...
						\draw[very thick, |-|, color=blue!50!black,text=black] (2.7, 1.35) --coordinate(Q) (6.83,1.35);%
						\fill[white] (2.6, 1.36) rectangle (7.0,1.55);
						\node[above=0.05em of Q]{\small Restricted PDG};
						}
					\onslide<6-7>{ % Dim this picture for BN description...
						\fill[white, opacity=0.8] (current bounding box.south west) rectangle (current bounding box.north east); }
					\end{tikzpicture||precompiled}
				}
				\hfill
			\end{center}
			\vspace{1em}
			\only<7->{%
				\begin{tikzpicture}[overlay,yshift=3.5em,xshift=2.3in]
					%        \pgftransformshift{\pgfpointanchor{current page}{center}}
					\begin{scope}
							[every node/.style={dpadded, fill opacity=1,fill=benchcolor2!10, circle, inner sep=2pt, minimum size=2em, draw=gray!50!benchcolor2}]

						\node[] (A) at (1,0) {$A$};
						\node[right=0.5 of A, fill opacity=0.4, dashed] (B) {$B$};
						\node[right=0.5 of B] (C) {$C$};
					\end{scope}

					\draw[arr] (A) -- (B);
					\draw[arr] (B) -- (C);
					\node[anchor=south west] at (-0.5,0.4) {\parbox{2.5in}{\small\raggedright\color{benchcolor2}%
							In a qualitative BN: \emph{removing} data results in \emph{new} knowledge: $A \CI C$. \note{this is a larger set of distributions.} }};
				\end{tikzpicture}}%
			\pause
			In contrast with BNs:
			\begin{itemize}%[<+-| alert@+>]
				\item<+-| alert@+> edge composition has \emph{quantitative} meaning, since edges have cpds; 
				\item<+-| alert@+> a variable can be the target of more than one cpd;
				\item<+-| alert@+> arbitrary restrictions of PDGs are still PDGs.\\
					{\small\begin{itemize}
						\item<+-| alert@+-+(1)> The analogue is false for BNs!
					\end{itemize}}
			\end{itemize}
			\end{frame}%------------%

	\subsection{Combining and Restricting PDGs}
		\colorlet{colorsmoking}{blue!50!black}
		\colorlet{colororiginal}{benchcolor1!85!black}
		\begin{frame} %%%%%%%%%%         UNION OF PDGs         %%%%%%%%%%%%
			\frametitle{Combining PDGs}
			% \framesubtitle{Smoking and Cancer}
			\colorlet{heldout}{benchcolor1}
				\tikzset{hybrid/.style={postaction={draw,colorsmoking,dash pattern= on 5pt off 8pt,dash phase=6.5pt,thick},
					draw=colororiginal,dash pattern= on 5pt off 8pt,thick}}
			\begin{center}
					\begin{tikzpicture||precompiled}
							[center base, thick, draw=colororiginal, text=black]{grok-pre}

						\node[dpadded] (C) at (0,0) {$\mathit C$};
						\node[dpadded] (T) at (1.5,0){$\mathit T$};
						\node[dpadded] (SL) at (.75,-1.5){$\mathit{SL}$};

						\draw[arr] (T) to[bend right] node[above]{$q$} (C);
						\mergearr{C}{T}{SL}
						\end{tikzpicture||precompiled}
				\only<1-3>{
						\vspace{0.7em}
						\begin{exampleblock}{Grok wants to be supreme leader ($\mathit{SL}$).}

						\begin{itemize}[<+->]
							\item She notices that those who use tanning beds have more power,\\[-0.4em]
							\item \textellipsis but mom says $q(C \mid T) = \begin{idxmat}
								{$t$,$\overline t$}{$c$,$\overline c$}
									.15 & .85 \\ .02 & .98
								\end{idxmat}$.\\[0.9em]
							\item Grok worries getting cancer from a tanning bed will make $\mathit{SL}$ impossible.
						\end{itemize}
						\end{exampleblock}
					}
				\pause
				{\Large$\boldsymbol+$}
				% \hfill{\Large$+$}\hfill
				\begin{tikzpicture||precompiled}[paperfig, text=black]{}
					\fill[fill opacity=0.1, blue!80!black, draw, draw opacity=0.5]
					 	(-2.07,1.35) rectangle (2.07, -1.35);

					% \begin{scope}
						\node[dpadded, fill=black!.16, fill opacity=0.9] (C) at (0,0) {$\mathit C$};
						\node[dpadded, fill=black!.16, fill opacity=0.9] (T) at (1.6,0){$\mathit T$};
						\node[dpadded, fill=black!.16, fill opacity=0.9] (S) at (-1.4, 0.8) {$\mathit S$};
						\node[dpadded, fill=black!.16, fill opacity=0.9] (SH) at (-1.45, -0.8) {$\mathit{SH}$};
						% \end{scope}

					\draw[arr1] (T) to node[above]{$p$} (C);
					\mergearr{S}{SH}{C}
					\end{tikzpicture||precompiled}
				% \hfill%
				\pause
				{\Large$\boldsymbol=$~}
				% \hfill%
				\begin{tikzpicture||precompiled}[paperfig]{grok-post}
					\begin{scope}[postaction={draw,colorsmoking,dash pattern= on 3pt off 5pt,dash phase=4pt,thick}]

						\node[dpadded,hybrid] (C) at (0,0) {$\mathit C$};
						\node[dpadded,hybrid] (T) at (2,0){$\mathit T$};
						\end{scope}

					\begin{scope}[thick, draw=colororiginal, text=black]
						\node[dpadded] (SL) at (1,-1.5){$\it SL$};
						\draw[arr] (T) to[bend right] node[above]{$q$} (C);
						\mergearr{C}{T}{SL}
						\end{scope}

					\begin{scope}[thick, draw=colorsmoking, text=black]
						\node[dpadded] (S) at (-1.4, 0.8) {$S$};
						\node[dpadded] (SH) at (-1.45, -0.8) {$\mathit{SH}$};
						\draw[arr] (T) to node[fill=white, fill opacity=1,text opacity=1,inner sep=1pt]{$p$} (C);
						\mergearr{S}{SH}{C}
						\end{scope}
					\end{tikzpicture||precompiled}
				\end{center}
			\vspace{1em}
			\begin{itemize}[<+(1)-|alert@+(1)>]
				\item Arbitrary PDGs may be combined without loss of information
				\item They may have parallel edges (e.g., $p,q$), which directly conflict.
				\end{itemize}

			\end{frame}%------------%


\section{Formalism and Technical Results}
	\newlength{\pdgdefnwidth}
	\begin{frame} %%%%%%%%%%%       Definition of a PDG          %%%%%%%%
		% \frametitle{Formal Definition}
		\colorlet{notationcolor}{structurecolor!40}
		\setlength{\pdgdefnwidth}{\textwidth}
		\begin{columns}[t]
		\column{\pdgdefnwidth}
		\begin{defn}[Probabilistic Dependency Graph]\label{def:model}
			A PDG is a tuple $\dg M =
			(\N,\Ed,\V,\mat p, \alpha, \beta)$,\pause\ where
			\begin{description}%
				\item[$\N$]<+-> %\notation{$:\Set$}%
					is a finite set of nodes (variables)
					\begin{description}
						\item[$\V$]<+-> %\notation{$:\N \to \Set$}%
							gives a set $\V(X)$ of possible values for each $X$;%
							\extra<3>[anchor=north east, align=right]{
								$\displaystyle \V(\dg M) := \prod_{X \in \N} \V(X)\qquad$ is the set of
								 	possible \\[-0.75em] joint variable settings.		}
						\end{description}

				\item[$\Ed$]<+-> %\notation{$\subseteq \N \times \N \times \mathit{Label}~~~$}%
					is a set of labeled edges $\{ \ed LXY \}$,\\
					\pause[\thebeamerpauses]
					and associated to each $\ed LXY$, there is:

					\pause
					\begin{description}%[<+-| alert@+>]
						\item[$\bp$]<+-> %:\big(\!(X,Y,\ell)\in\!\Ed \big) \to
							%\notation{$:\V(X) \to \Delta\V(Y)$}%
							a cpd $\bp(Y \mid X)$;

						\item[$\alpha\ssub L$]<+(1)-> \notation{$\in [0,\infty)$}%
							a confidence in the functional dependence $X \to Y$% $Y$ is a (noisy) function of $X$;
						\item[$\beta\ssub L$]<+(-1)-> \notation{$\in (0,\infty)$}%
							a confidence in the reliability of	$\bp$.
						\end{description}
				\end{description}
			\end{defn}
			\column{\textwidth-\pdgdefnwidth}
			\end{columns}

		\end{frame}

	\begin{frame}<1-4>[label=semantics] %%%%%%%%%          SEMANTICS           %%%%%%%%%%%%
		\frametitle{PDG Semantics}
		{	\setbeamersize{description width=1.2cm}
			\setbeamercovered{transparent=30}
			\begin{description}
				\item<uncover@1,2,8,9> [\only<8->{\color{structurecolor}1.~}\alert<2,8,9>{$\SD{\dg M}$}] %\notation{$:\mathcal P{\Delta \V(\dg M)}$}
					The set of joint distributions consistent with $\dg M$;
					\only<2-7>{\visible<2->{\uncover<2>{
						\[ \Big\{ \mu%~{\color{gray!60} \in \Delta[\V(\dg M)]}~
						 	\in \Delta[\V(\dg M)]:~\text{for all } \ed LXY \in \Ed.~~\mu(Y\!\mid\!X) = \bp(Y\!\mid\! X)  \Big\} \]
						}}}
				\uncover<1,3,8>{\item [\only<8->{\color{structurecolor}2.~}\alert<3,8>{$\bbr{\dg M}_\gamma$}] %\notation{$:{\Delta \V(\dg M)} \to \mathbb R$}
					\alt<1>{A function, scoring distributions by}{
						A loss function (parameterized by $\gamma$), scoring a joint distribution's}
					 compatibility with $\dg M$;

					\only<2-7>{\visible<3->{\uncover<3>{
						\alt<-7>{ \[
							\bbr{\dg M}_\gamma(\mu) :=
								{\color{benchcolor1}\underbrace{
									\Inc_{\dg M}(\mu) }_{\left(\parbox{0.8in}%
										{\centering quantitative\\[-0.2em]term}\right)}} +
										{\gamma} \extra<3>[above right=1em and -1 em of call point, 
											inner sep=0.3em, font=\small, name=tp, draw=alertcolor, text=alertcolor!40!black,
											fill=alertcolor!10]{\small tradeoff parameter $\gamma \ge 0$}
								{\color{benchcolor2}\underbrace{\IDef{\dg M}(\mu) }_{\left(\parbox{0.75in}%
										{\centering	qualitative\\[-0.2em]term}\right)}}
							\]}{\[
								\bbr{\dg M}_\gamma(\mu) := \Inc_{\dg M}(\mu) + \gamma\;	\IDef{\dg M}(\mu)
						\] }}}}}	
					\only<3>{\tikzro{\draw[alertcolor] ([yshift=0.5em,xshift=-0.3em]call point.center) -- ([xshift=1.5em]tp.south west);}}	
				\only<6,7>{
					\vspace{-5em}
					\begin{prop}[{\itshape\normalfont {uniqueness for small $\gamma$}}]
						\begin{enumerate}
						\item If  $0 < \gamma \leq \min_L \beta_L^{\dg M}$, then
						$\bbr{\dg M}_\gamma^*$ is a singleton.
						\item $\displaystyle\lim_{\gamma\to0}\bbr{\dg M}_\gamma^*$ exists and is unique.
						\end{enumerate}
						\end{prop}	}

				\item<uncover@1,4-7,9> [\only<8->{\color{structurecolor}3.~}\alert<4-7,9>{$\bbr{\dg M}^*\only<4-6>{_\gamma}$}] %\notation{$:{\Delta \V(\dg M)}$}
					The \only<7>{{\color{benchcolor1}(unique)}} ``best'' joint distribution\only<7>{{\small\color{benchcolor1}~(in the quantitative limit)}}.
					\only<2-7>{\visible<4-7>{
						\[ \bbr{\dg M}\only<4-6>{_\gamma}^* := \only<7>{{\color{benchcolor1}\lim_{\gamma\to 0}}} \argmin_\mu \bbr{\dg M}_\gamma(\mu) \]
						}}
			\end{description}	}

			% \only<1> {
			%
			% 	}
		

			\only<8->{
				\begin{prop}[{\normalsize{\it the second semantics extends the first\;}}]
						$\SD{\dg M} \!= \big\{ \mu : \bbr{\dg M}_0(\mu) \!=\! 0 \big\}$.
					\end{prop}

				\onslide<9->{
					\begin{prop}[{\normalsize{\it If there there are distributions consistent with $\dg M$, the best distribution is one of them.\;}}]\label{prop:consist}
							$\bbr{\dg M}^* \in \bbr{\dg M}_0^*$, so if $\dg M$ is consistent,
							then $\bbr{\dg M}^* \in \SD{\dg  M}$.
						\end{prop}
					}
			}
		\end{frame}

	\relax % frame setup
		\colorlet{localcolor}{Lavender!80!black}
		\colorlet{globalcolor}{Sepia!80!orange}
		% \def\INCfst{2}
		\def\INCmotfst{2}
			\def\INCmotlen{4}
		
		\edef\INCfst{\the\numexpr \INCmotfst + \INCmotlen \relax}
			\def\INClen{5}
			\edef\INClst{\the\numexpr\INCfst+\INClen-1\relax}
			\def\INCrange{\INCfst-\INClst}
			\def\INC+#1{\atslide{\INCfst+#1}}
		
		% \def\INCexfst{7}
		\edef\INCexfst{\the\numexpr \INClst + 1 \relax}
			\def\INCexlen{0}
			\edef\INCexlst{\the\numexpr\INCexfst+\INCexlen-1\relax}
			\def\INCexrange{\INCexfst-\INCexlst}
			\def\INCex+#1{\atslide{\INCexfst+#1}}
		
		\edef\IDEFmotfst{\the\numexpr \INCexlst + 1 \relax} 
			\def\IDEFmotlen{3}
			\def\IDEFmot+#1{\atslide{\IDEFmotfst+#1}}
		
		% \def\IDEFfst{12}
		\edef\IDEFfst{\the\numexpr \IDEFmotfst + \IDEFmotlen \relax} 
			\def\IDEFlen{3}
			\edef\IDEFlst{\the\numexpr\IDEFfst+\IDEFlen-1\relax}
			\def\IDEFrange{\IDEFfst-\IDEFlst}
			\def\IDEF+#1{\atslide{\IDEFfst+#1}}
		
		% \def\IDEFexfst{15}
		\edef\IDEFexfst{\the\numexpr \IDEFlst + 1 \relax}
			\def\IDEFexlen{4}
			\edef\IDEFexlst{\the\numexpr\IDEFexfst+\IDEFexlen-1\relax}
			\def\IDEFexrange{\IDEFexfst-\IDEFexlst}
			\def\IDEFex+#1{\atslide{\IDEFexfst+#1}}
		
		% \def\INCandIDEF{19}
		\edef\INCandIDEFfst{\the\numexpr\IDEFexlst + 1 \relax}
			\edef\INCandIDEF{\INCandIDEFfst-}
			\def\IIboth+#1{\atslide{\IDEFexlst + 1 + #1}}
			\edef\beforeINCandIDEF{-\the\numexpr\INCandIDEFfst - 1\relax}
			% \show\INCandIDEF
		\def\atslide#1{\the\numexpr#1\relax}
		\def\aitem@#1+#2{\item<\atslide{\csname#1fst\endcsname+#2}-|alert@\atslide{\csname#1fst\endcsname+#2}>}
		\def\halfblocks{\INCexrange,\IDEFexrange,\INCandIDEF}
	\begin{frame}[t] %%%
		\frametitle{The Scoring Function }
		\only<\IIboth+1->{
			\begin{itemize}
				\item<\IIboth+1-|alert@\IIboth+1> A BN strictly enforces the qualitative picture (large $\gamma$)
				\item<\IIboth+2-|alert@\IIboth+2> we are interested in the quantitative limit (small $\gamma$)
			\end{itemize}			
			\bigskip
		}
		
		{\alt<\INCexrange,\IDEFexrange>{\vspace{-0.5em}\small}{\centering}\only<\INCandIDEF>{\centering}
			$\displaystyle%
				\bbr{\dg M}_\gamma(\mu) := \hl[alertcolor!30!benchcolor1!99!structurecolor]%
						<2-\INClst,\INCexrange,\INCandIDEF>{\Inc_{\dg M}(\mu)}
					+ \gamma%
					\extra<\IIboth+0->[below right=0.5em and -1 em of call point, 
						inner sep=0.3em, font=\small, name=tp, draw=alertcolor, text=alertcolor!40!black,
						fill=alertcolor!10]{\small tradeoff parameter $\gamma \ge 0$}
					%
					\;\hl[alertcolor!30!benchcolor2!99!structurecolor]<\IDEFmotfst->{\IDef{\dg M}(\mu)}$\par}
				\only<\IIboth+0->{\tikzro{%
						\draw[alertcolor] ([yshift=-0.2em,xshift=-0.3em]call point.center) -- ([xshift=1.0em]tp.north west);}}	
	
		{\medskip}
		\setlength\pdgdefnwidth{0.95\textwidth}
		\only<\halfblocks>{\setlength\pdgdefnwidth{0.43\textwidth}}
		
		\only<2-\atslide{\INCfst-1}> {
			\smallskip
			{\centering\parbox{0.8\textwidth}{\raggedright
				\textbf{Intuition:} beyond stating whether or not $\mu$ is consistent with $\dg M$, we score $\mu$'s compatibility with $\dg M$.}\par}
		
			{\color{structurecolor}\scshape{Motivating Examples}.}
			% \begin{center}
			\hfill
			$\dg M := \quad$
			\begin{tikzpicture}[paperfig]
				\node[dpadded] (1) {$\pdgunit$};
				\node[dpadded, right=1 of 1] (X) {$X$};
				\draw[arr1] (1) to[bend left=30] node[above] {$q$} (X);
				\draw[arr1] (1) to[bend right=30] node[below] {$p$} (X);
			\end{tikzpicture}
			\hfill~
			% \end{center}
			\vspace{-1em}
			
			\begin{itemize}
				\aitem@{INCmot}+1 If $p = \begin{idxmat}{$\star$}{$x_1$,$x_2$} .4 & .6 \end{idxmat} = q$, then $\dg M$ is consistent, and compatible with the joint distribution $\mu(X) = p$. 
				\aitem@{INCmot}+2 If $p = \begin{idxmat}{$\star$}{$x_1$,$x_2$} .4 & .6 \end{idxmat}$ and 
						$q = \begin{idxmat}{$\star$}{$x_1$,$x_2$} .5 & .5 \end{idxmat}$, 
						then $\dg M$ is not consistent, but 
						
						$\mu = \begin{bmatrix}%{$\star$}{$x_1$,$x_2$}
						 	.45 & .55 \end{bmatrix}$ matches better than 
						$\mu = \begin{bmatrix} .9 & .1 \end{bmatrix}$.
						\medskip

				\aitem@{INCmot}+3 If $p = \begin{bmatrix} .4 & .6 \end{bmatrix}$ and 
						$q = \begin{bmatrix} 0 & 1 \end{bmatrix}$, 
						then $\dg M$ is much more inconsistent than before, even though $\SD{\dg M} = \emptyset$ in both cases.
			\end{itemize}
		}
			
		\begin{columns}[c]
			\only<\INCrange,\INCexrange,\INCandIDEF>{{ %%%%%%%%%% INC %%%%%%%%%%%%%%%
				\begin{column}{\pdgdefnwidth}
				\setbeamercolor{block title}{bg=benchcolor1!80!structurecolor!50}
				\setbeamercolor{block body}{bg=benchcolor1!80!structurecolor!20}
				\setbeamercovered{transparent=35}
				\only<\INCexrange>{\setbeamertemplate{blocks}[rounded][shadow=false]}
				\only<\INCexrange,\INCandIDEF>{\small}
				\begin{defn}[$\Inc$]<\INCrange,\INCandIDEF>\label{def:inc}
				The \emph{incompatibility} of
				\only<\INCrange>{a joint distribution }$\mu$ with $\dg M$\alt<\INCrange>{ is given by}{:}
				\def\fullnotate{\atslide{\INCfst+3}}
				\alt<-\INC+2,\INCex+0->{ \[ % The simpler, less clear notation
						\Inc_{\dg M}(\mu) :=
							\sum_{\mathclap{\ed LXY}}\; \visible<\INC+1->{\alert<\INC+1>{\beta_L}} \;
							\tikzmark{relentD}
							\kldiv{ \mu_{Y|X} }{ \bp }
					\] }{ \[ % The notation used in our paper
						\Inc_{\dg M }( \mu) :=
							\sum_{\mathclap{\ed LXY}} \beta_L \alert<\INC+3>{ \Ex_{x \sim \mu_{_X}} }
							\tikzmark{relentD}
							\kldiv[\Big]{ \mu \alert<\INC+3>{(Y \mid X \!=\! x)} }{\bp \alert<\INC+3>{(x)} }. \] }
				%
				\only<\INC+2-\INC+3>{
					\extra[name=Dexplain, below=1cm of pic cs:relentD, align=right,
								fill=benchcolor1!40!structurecolor!20!white]{%
							$\displaystyle\kldiv\mu\nu = \sum_{\mathclap{w \in \mathop{Supp}(\mu)}} \mu(w) \log \frac{\mu(w)}{\nu(w)}$
							is the relative entropy \\[-1.2em] from $\nu$ to $\mu$.
						}
					\tikzro \draw[thick,structurecolor!70!black,draw opacity=0.5]
					 	([xshift=0.5em,yshift=-0.2em]pic cs:relentD) -- ([xshift=0.5em]Dexplain.north); }
			
				{\setbeamercovered{invisible}
					\onslide<\INClst->{
						The \emph{inconsistency} of $\dg M$ is \only<-\INClst>{the smallest possible incompatibility,}
						\[ \Inc(\dg M) := \!\!\inf_{ \mu \in \Delta\V(\dg M)}\!\! \Inc_{\dg M}(\mu) . \]	 }}
				\end{defn}
				\end{column}
				}}
			\only<\INCexrange>{
				\column{\textwidth-\pdgdefnwidth}
				\vspace{-1em}
				{\centering\Large\color{structurecolor}\textsc{Examples}\par}
				\begin{center}
					\begin{tikzpicture}[throughlab/.style={fill=white, fill opacity=1,text opacity=1,inner sep=1pt}]
						\node[dpadded,thick] (C) at (0,0) {$\mathit C$};
						\node[dpadded,thick] (T) at (2,0){$\mathit T$};
						\draw[arr,draw=colororiginal,text=black] (T) to[bend right] node[throughlab]{$q$} (C);
						\draw[arr, draw=colorsmoking] (T) to node[throughlab]{$p$} (C);
						\end{tikzpicture}
					\end{center}
				
				\setlength{\leftmargini}{1em}
				\begin{itemize}
					\aitem@{INCex}+1 if $p = q$, then $\Inc(\dg M) = 0$. 
					\aitem@{INCex}+2 if {\small $p =\begin{idxmat}%
								[\alt<\INCex+2>{\color{alertcolor!40}}{\color{gray!50}}\smalltext]
							{$t$,$\bar t$}{$c$} 0.4 \\ 0.1 \end{idxmat}$} and
						{\small $q = \begin{idxmat}%
								[\alt<\INCex+2>{\color{alertcolor!40}}{\color{gray!50}}\smalltext]
							{$t$,$\bar t$}{$c$} 0.9 \\ 0.1 \end{idxmat}$}, then \\[0.05em] still $\Inc(\dg M) = 0$ 
							(maybe nobody uses tanning beds).\vspace{0.4em}
	
					\aitem@{INCex}+3 {\small Not all inconsistencies are equally bad; \\ 
						\!\!$\Inc\!\left\{\!p \shorteq\!\! \begin{bmatrix} .55 \\ .12 \end{bmatrix}\!\!;
		 				q \shorteq\!\! \begin{bmatrix} .45 \\ .15 \end{bmatrix}\! \right\}\! < \! 
						\Inc\! \left\{ \!p \shorteq\!\!\begin{bmatrix} .3 \\ .4 \end{bmatrix}\!\!;
						q \shorteq\!\! \begin{bmatrix} .9 \\ .1 \end{bmatrix}\! \right\}\!$}
		
					\end{itemize}
			}
			\only<\IDEFmotfst-\atslide{\IDEFfst-1}>{{
				\begin{column}{0.75\textwidth}
				
				\bigskip
				
				\textbf{Intuition:} each edge $\ed LXY$ indicates that $Y$ is determined (perhaps noisily) by $X$ alone. 
				
				\bigskip
				\bigskip
				\bigskip

				\onslide<\IDEFmot+1->{				
					So a $\mu$ with
					\tikzmark{beg-ce}%
						\hl[localcolor]<\IDEFmot+2->{uncertainty in $Y$ after $X$ is known}\tikzmark{end-ce}
					(beyond 
					\tikzmark{beg-e}\hl[globalcolor]<\IDEFmot+2->{pure noise}\tikzmark{end-e})
					is qualitatively worse.  
					
					\only<\IDEFmot+2>{
						\tikzro{ \draw[very thick,localcolor] ([yshift=0.8em]pic cs:beg-ce) -- %
							([yshift=1em]pic cs:beg-ce) -- 
								node[above,inner sep=2pt] {{\small\itshape measured by $\H(Y\mid X)$}} 
							([yshift=1em]pic cs:end-ce) -- ([yshift=0.8em]pic cs:end-ce);}
					
						\tikzro{ \draw[very thick,globalcolor] ([yshift=-0.40em]pic cs:beg-e) --
								([yshift=-0.60em]pic cs:beg-e) -- node[below,inner sep=2pt] {$\H(\mu)$} 
								([yshift=-0.60em]pic cs:end-e) -- ([yshift=-0.40em]pic cs:end-e);}
					}
				}
				
				\end{column}
				}}
			\only<\IDEFfst->{{ %%%%%%%%%% IDEF %%%%%%%%%%%%%
				\begin{column}{\pdgdefnwidth}
				\setbeamercolor{block title}{bg=benchcolor2!80!structurecolor!50}
				\setbeamercolor{block body}{bg=benchcolor2!80!structurecolor!20}
				\setbeamercovered{transparent=35}
				\only<\IDEFexrange>{\setbeamertemplate{blocks}[rounded][shadow=false]\small}
				\only<\IDEFexrange,\INCandIDEF>{\small}
				\begin{defn}[$\IDef{}$]<-\IDEFlst,\INCandIDEF>
					\alt<-\IDEFlst>
						{The \emph{information deficit} of a distribution $\mu$ with respect to $\dg M$ is \vspace{4.2em}}
						{The $\dg M$-\emph{information deficit} of $\mu$: \vspace{2.0em}}
					
					\def\idefdef{\sum_{\mathclap{\ed LXY}} \alt<\IDEFexfst->{\alpha\ssub L\!}{\alpha_L} \H_\mu(Y\!\mid\! X)}
					\[ \!\IDef{\dg M}(\mu) \alt<\IDEFexfst->{\!=\!}{:=}
						\alt<\IDEF+2->{{\color{localcolor} \overbrace{\idefdef}^{\tikzmark{lmark}}}}{ \idefdef }
							-
						\alt<\IDEF+1->{{\color{globalcolor}\underbrace{\H(\mu)}_{\tikzmark{gmark}}}}{ \H(\mu) } \alt<\halfblocks>{}{.}  \]
					\only<\IDEF+2->{% Overlay: (b) # bits to separately determine each target, knowing src...
						\extra[above left=0.6em and 1.2em of {pic cs:lmark},
							onslide=<\halfblocks>{inner sep=0.5em},
							anchor=south,align=center,
							name=idefseparate,
							fill=structurecolor!20!localcolor!40!white,
							text=localcolor!50!black]%
								{\alt<\halfblocks>{}{(b)} \# bits \alt<\halfblocks>{}{required }to separately
								determine\\ each target, knowing the source}
						\tikzro \draw[thick,draw=structurecolor!20!localcolor!70!black,draw opacity=0.6]
						 	(pic cs:lmark) -- ([xshift=1.2em]idefseparate.south);
						}
					\only<\IDEF+1->{% Overlay: (a) # bits to determine all variables...
						\extra[below right=1em and 2em of {pic cs:gmark},
							onslide=<\halfblocks>{below = 1em of pic cs:gmark, inner sep=0.5em},
							anchor=north east,name=entropy,
							fill=structurecolor!20!globalcolor!40!white,text=globalcolor!50!black]%
								{\alt<\halfblocks>{}{(a)} \# bits \alt<\halfblocks>{}{needed }to determine all \alt<\halfblocks>{vars}{variables}}
						\tikzro \draw[thick,draw=structurecolor!20!globalcolor!70!black,draw opacity=0.6]
						 	([yshift=0.5em]{pic cs:gmark}) -- ([xshift=-2em]{entropy.north east});
						}
					
					\alt<\halfblocks>{\vspace{1.8em}}{\vspace{3em}}
				\end{defn}
				\end{column}}}
			\only<\IDEFexrange>{ %%%%%%%%%% IDEF EXAMPLES %%%%%%%%%%
				\column{\textwidth-\pdgdefnwidth}
				\vspace{-2em}

				{\centering\Large\color{structurecolor}\textsc{Examples}\par}

				\medskip
				\setlength{\leftmargini}{1.5em}
				\begin{itemize}
					\aitem@{IDEFex}+0 
						{$\dg M_0=\quad$\color<\IDEFex+0>{alertcolor!60!black}\begin{tikzpicture}[paperfig]
							\node[dpadded] (X) {$X$};
							\node[dpadded,right=0.8 of X] (Y) {$Y$};
						\end{tikzpicture}\hfill}~\\[0.3em]
						{\small\!$\IDef{\dg M_0}\!(\mu)\! =\!  - \H_\mu(X,Y)$}
						{\footnotesize\color<\IDEFex+0>{alertcolor!70!benchcolor2}% 
						\\[-0.3em](optimal $\mu$ maximizes entropy of $X,Y$)}~ \\[0.2em]
					\aitem@{IDEFex}+1
						{$\dg M_1=\quad$\color<\IDEFex+1>{alertcolor!60!black}\begin{tikzpicture}[paperfig]
							\node[dpadded] (X) {$X$};
							\node[dpadded,right=0.8 of X] (Y) {$Y$};
							\draw[arr2] (X) -- (Y);
						\end{tikzpicture}\hfill}~\\[0.3em]
						{\small\!$\IDef{\dg M_1}\!(\mu)\! =\! -\!\H_\mu(X)$}
						{\footnotesize\color<\IDEFex+1>{alertcolor!70!benchcolor2}\\[-0.3em]%
							(optimal $\mu$ maximizes entropy of $X$)}~  \\[0.2em]
					\aitem@{IDEFex}+2
						{$\dg M_2=\quad$\color<\IDEFex+2>{alertcolor!60!black}\begin{tikzpicture}[paperfig]
							\node[dpadded] (X) {$X$};
							\node[dpadded,right=0.8 of X] (Y) {$Y$};
							\draw[arr2] (X) to[bend left=15] (Y);
							\draw[arr2] (X) to[bend right=15] (Y);
						\end{tikzpicture}\hfill}~\\[0.3em]
						{\small\!$\IDef{\dg M_2}\!(\mu)\! =\!  - \H_\mu(X) + \H_\mu(Y\mid X)$}
						{\footnotesize\\\color<\IDEFex+2>{alertcolor!70!benchcolor2}%
							(optimal $\mu$ maximizes entropy for $X$, and\\[-0.4em] makes $Y$ a function of $X$)}  \\[0.2em]
					\aitem@{IDEFex}+3
						{$\dg M_3=\quad$  \color<\IDEFex+3>{alertcolor!60!black}\begin{tikzpicture}[paperfig]
							\node[dpadded] (X) {$X$};
							\node[dpadded,right=0.8 of X] (Y) {$Y$};
							\draw[arr2] (X) to[bend left=10] (Y);
							\draw[arr2] (Y) to[bend left=10] (X);
						\end{tikzpicture}\hfill}~\\[0.3em]
						{\small\!$\IDef{\dg M_3}\!(\mu)\! =\!  - \I_\mu(X;Y)$}
						{\footnotesize\\[-0.3em]\color<\IDEFex+3>{alertcolor!70!benchcolor2}%
							(opt. $\mu$ makes $X,Y$ functions of each other)}
				\end{itemize}
			}
			\end{columns}
			

		\end{frame}

	\againframe<5->{semantics}
\subsection{BNs and PDGs}
	% \begin{frame}\frametitle{Construction}
	% \end{frame}
	
	\begin{frame}%%%%%%%%%          BN THEOREM
		\frametitle{Capturing Bayesian Networks}
		Let $\PDGof{\mathcal B, \beta}$ be the PDG
		corresponding to the BN $\mathcal B$, with weights $\beta$.
		\begin{theorem}[{{\it BNs are PDGs}}] \label{thm:bns-are-pdgs}
			% If $\mathcal B$ is a Bayesian network and $\Pr_{\mathcal B}$ is the distribution it specifies, then for all $\gamma > 0$ and all vectors $\beta$,
			If $\mathcal B$ is a BN and $\Pr_{\mathcal B}$ is the distribution it specifies, then for all $\gamma > 0$ and all vectors $\beta$,
			$$
				\bbr{\PDGof{\mathcal B, \beta}}_\gamma^* = \{ \Pr\nolimits_{\mathcal B}\},
				\quad\text{and thus}\quad
				\bbr{\PDGof{\mathcal B, \beta}}^* = \Pr\nolimits_{\mathcal B}.
			$$
			\end{theorem}
			
		\begin{center}
		% \begin{minipage}{3.3cm}
		% 	\small
		% 	\setlength{\leftmargini}{0.2em}
		% 	\begin{itemize}
		% 		\item distributions consistent with $\dg M_{\mathcal B}$
		% 		\item these minimize $\Inc$
		% 	\end{itemize}
		% \end{minipage}
		\begin{tikzpicture}[remember picture,center base]
			\fill[benchcolor1,opacity=0.5] (0.1,0) -- (-1,1) -- (-2,0.7) -- (-2.5,-0.1) -- (-1.5, -0.5) -- cycle;
			\node[align=center] (sd) at (-1.3,0.2) {$\SD{\dg M}$};
			\fill[benchcolor2,opacity=0.5] (-0.1,0) -- (0.5,0.2)  -- (1, 1) -- (2, -0.5) -- cycle;
			\node[align=center] (ind) at (1,0.2) {$\CI_{\mathcal B}$};
			
			\node[left=0.9 of sd, inner sep=0.5pt, align=left, font=\small,
					text=benchcolor1!40!black] {
				space of distributions\\ consistent with $\dg M_{\mathcal B}$\\
				(which minimize $\Inc$)
			};
			\node[right=0.5 of ind, inner sep=0.5pt, align=right, font=\small,
				text=benchcolor2!40!black] {
				space of distributions\\ with independencies of $\mathcal B$\\
				(which can be shown \\ to minimize $\IDef{}$)
			};

			\node[fill=black, circle,inner sep=0.1em] (dot) at (0,0){};
			\node[below=0.5em of dot] (dotl) {$\Pr_{\mathcal B}$};
			\draw (dotl) -- (dot);
		\end{tikzpicture}
		% \begin{minipage}{3.2cm}
		% 	\setlength{\leftmargini}{0.5em}
		% 	\begin{itemize}
		% 		\item distributions with independencies of ${\cal B}$
		% 		\item can show that these minimize $\IDef{}$
		% 	\end{itemize}
		% \end{minipage}
		\end{center}
		\end{frame}

\begin{frame}
	\frametitle{PDGs and Factor Graphs}
		
		
	\begin{theorem}[{\it PDGs capture factor graphs}]
		We can naturally translate factor graphs and their exponential families% 
			\note{(the natural notion of confidence in a factor graph)}%
			, into PDGs, in a way which preserves their semantics. 
	\end{theorem}
	
	\bigskip
	Roughly speaking, 
	\begin{itemize}
		\item a factor graph is a PDG in which qualitative and quantitative parameters are balanced $(\beta = \alpha\gamma)$.
		
		\item They have undesirable properties that do not occur in the quantitative limit.
	\end{itemize}
	% a factor graph is a PDG in which qualitative and quantitative parameters are balanced $(\beta = \alpha\gamma)$. They have undesirable properties that do not occur in the quantitative limit.  
	
	
	See the paper for details!
	
	% \begin{center}
	% 		\begin{tikzpicture}[center base, xscale=1.3,
	% 			fgnode/.append style={minimum width=2.4em, inner sep=0.2em}]
	% 			\node[factor] (prior) at (1.65,-1) {};
	% 			\node[factor] (center) at (3.75, 0.1){};
	% 
	% 			\node[fgnode] (PS) at (1.65,0.5) {$\mathit{PS}$};
	% 			\node[fgnode] (S) at (3.1, 0.8) {$\mathit S$};
	% 			\node[fgnode] (SH) at (3.0, -0.8) {$\mathit{SH}$};
	% 			\node[fgnode] (C) at (4.8,0.5) {$\mathit C$};
	% 
	% 			\draw[thick] (prior) -- (PS);
	% 			\draw[thick] (PS) --node[factor](pss){} (S);
	% 			\draw[thick] (PS) --node[factor](pssh){} (SH);
	% 			\draw[thick] (S) -- (center) (center) -- (SH) (C) -- (center);
	% 
	% 
	% 			\node[fgnode] (T) at (4.8, -1.3) {$\mathit T$};
	% 			\draw[thick] (T) -- node[factor]{}  (C);
	% 			\end{tikzpicture}
	% 		~{\Large$\rightsquigarrow$}~
	% 		\pause
	% 		\begin{tikzpicture}[center base, xscale=1.5,
	% 	        newnode/.style={rectangle, inner sep=5pt, fill=gray!30, rounded corners=3, thick,draw}]
	% 			\node[newnode] (prior) at (1.65,-1) {};
	% 			\node[newnode] (center) at (4.1, 0.25){};
	% 
	% 			\node[dpadded] (PS) at (1.65,0.5) {$\mathit{PS}$};
	% 			\node[dpadded] (S) at (3.3, 0.8) {$\mathit S$};
	% 			\node[dpadded] (SH) at (3.3, -0.6) {$\mathit{SH}$};
	% 			\node[dpadded] (C) at (4.9,0.5) {$\mathit C$};
	% 
	% 			\draw[arr, ->>, shorten <=0pt] (prior) -- (PS);
	% 			\draw[arr, <<->>] (PS) --node[newnode](pss){} (S);
	% 			\draw[arr, <<->>] (PS) --node[newnode](pssh){} (SH);
	% 			\draw[arr, <<-, shorten >=0pt] (S) -- (center);
	% 			\draw[arr, <<-, shorten >=0pt] (SH)-- (center);
	% 			\draw[arr, <<-, shorten >=0pt] (C) -- (center);
	% 
	% 			\node[dpadded, fill=blue] (1) at (2.7,-1.8) {$\pdgunit$};
	% 
	% 			\draw[blue!50, arr] (1) -- (prior);
	% 			\draw[blue!50, arr] (1) to[bend right=30] (center);
	% 			\draw[blue!50, arr] (1) to[bend right = 5] (pss);
	% 			\draw[blue!50, arr] (1) to[bend left = 10] (pssh);
	% 
	% 
	% 			\node[dpadded] (T) at (4.8, -1.7) {$T$};
	% 			\draw[arr, <<->>] (T) -- node[newnode](tc){}  (C);
	% 
	% 			\draw[blue!50, arr] (1) to[bend right = 10] (tc);
	% 			\end{tikzpicture}
	% 		\end{center}			
	% 
	% 	Conversely, PDG semantics behave like a factor graph at $\gamma=1$. But PDGs have nice properties that factor
	% 	graphs do not! See the full paper.
	\end{frame}
		
	\begin{frame}\frametitle{Inference and Inconsistency: a Glimpse.}
		\textbf{Conditioning as inconsistency resolution.}\\ 
		% \parbox{2.6in}{\raggedright
			To condition on $Y\!=\!y$, in $\dg M$, 
			simply add the edge $\ed {\delta_y}{\pdgunit} Y$  to get ${\dg M}_{Y\!=y}$.
			Then $\bbr{{\dg M}_{Y\!=y}}^* = \bbr{\dg M}^* \mid (Y\!=\!y)$.
			% This generalizes to Jeffrey's rule.
		\bigskip\pause
		
		\textbf{Querying $\Pr(Y\mid X)$ in a PDG $\dg M$.}
		\begin{itemize}[<+-|alert@+>]
			\item We can add $\ed pXY$ to $\dg M$ with a cpt $p$, to get ${\dg M}^{+p}$.
			\item The choice of cpd $p$ that minimizes the inconsistency of $\dg M^{+p}$ (which is strongly convex and smooth in $p$) is $\bbr{\dg M}^*(Y\!\mid \!X)$,
			\item so oracle access to inconsistency yields fast inference by gradient descent.
		\end{itemize}	
		% Roughly, the quality of an answer $p$ is the inconsistency of $\dg M^{+p}$.  The function $p \mapsto \Inc({\dg M^{+p}})$ is minimized by $p = \bbr{\dg M}^*(Y\!\mid\! X)$, and is smooth and convex in $p$; with oracle access to $\Inc(-)$ (or an approximation), we can compute optima by gradient methods.
		
		\pause[\thebeamerpauses]
		
		
		\parbox{0.5\textwidth}{This is closely related to  \\ standard variational techniques!}
		\centering
		\begin{tikzpicture}[center base]
			\node[dpadded] (1) {$\pdgunit$};
			\node[dpadded,right=1.0 of 1] (Z) {$Z$};
			\node[dpadded,right=1.5 of Z] (X) {$X$};
			\draw[arr1] (1) to node[above,name=pri] {$p(Z)$} (Z);
			\draw[arr1] (Z) to[bend left=15] node[above,name=dec] {$p(X \!\mid\! Z)$} (X);
			
			\path[gray!80] (dec) -- ++(0,0.2) node[above,inner sep=1pt]{\small decoder};
			\path[gray!80] (pri) -- ++(0,0.2) node[above,inner sep=1pt]{\small prior};
			\onslide<4->{
				\draw[arr1,onslide=<+(1)->{alertcolor}] (X) to[bend left=15] node[below,name=enc,inner sep=1pt] {$q(Z \!\mid\! X)$} (Z);	
				\path[gray!80] (enc) -- ++(0,-0.2) node[below,inner sep=1pt]{\small encoder};}
		\end{tikzpicture}\par
		
		% }
		% \begin{tikzpicture}[center base]	
		% 	% \useasboundingbox (-3,-1) rectangle (3.5,4);
		% 	\node[dpadded] (1) at (0,3) {$\pdgunit$};
		% 	\node[dpadded] (W) at (0,0) {$W$};
		% 	\node[dpadded] (B) at (-2,1) {$B$};
		% 	\node[dpadded] (E) at (2.5, 0){$E$};
		% 	\coordinate (Q) at (6,0); % to even out controls
		% 
		% 	\draw[arr] (1) to node[fill=white]{$p$} (W);
		% 	\draw[arr] (1) to node[fill=white]{$\pi$} (B);
		% 
		% 	\draw[arr, gray, ->>] (W) to[bend left=10] (B);
		% 	\draw[arr, dashed] (B) to[bend right=30] (W);	
		% 
		% 	\draw[arr, ->>] (W) to (E);
		% 
		% 	\draw[arr,blue!50] (1) .. controls (-5.5,1.5) and (-2,-2) .. node[fill=white]{$p'(E)$} (E);
		% 	\draw[arr,orange!70] (1) .. controls (0.5,1) and (1,0.5) .. node[fill=white]{$p(E)$} (E);
		% \end{tikzpicture}
		% 

		\end{frame}

	\begin{frame}
		\frametitle{Summary}
		PDGs\textellipsis 
		\begin{itemize}[<+-|alert@+>]
			\item capture inconsistency, including conflicting information
			from multiple sources with varying reliability.
			\item
				are especially modular; to combine info from two sources, simply take a PDG union.
				This incorporates new data (edge cpds) and concepts (nodes) without affecting previous information.
			\item cleanly separate quantitative info (the cpds) 
				from qualitative info (the edges), with variable confidence
				in both (the weights $\beta$ and $\alpha$).
				This is captured by terms $\Inc$ and $\IDef{}$ in our scoring function.
			\item have (several) natural semantics; one of them allows us to
				pick out a unique distribution.  Using this distrbution, PDGs
				can capture BNs and factor graphs.
				% In the latter case, a simple parameter shift in the corresponding PDG eliminates
				% arguably problematic behavior of a factor graph.
			\end{itemize}
		
		\pause[\thebeamerpauses]
		\medskip
		\textit{But there is much more to be done!}
		\end{frame}
\end{document}
