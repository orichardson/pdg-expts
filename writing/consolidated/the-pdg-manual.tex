% !TeX document-id = {9ed1d4bf-547b-4570-8d8e-5f4056797f0e}
%no tex program
% !TeX TXS-program:bibliography = txs:///bibtex
\documentclass{article}

\usepackage{subfiles}
\usepackage[utf8]{inputenc}
% \usepackage[english]{babel}
\usepackage{import}

%% BIBLIO
\usepackage{natbib}


%% STANDARD PACKAGES
\usepackage{mathtools}		%also loads amsmath
\usepackage{amssymb, bbm}
\usepackage{booktabs}
\usepackage{nicefrac}
\usepackage{algpseudocode}
\usepackage[dvipsnames, cmyk]{xcolor} % must be loaded before tikz
\usepackage{tikz} % setup comes later
\usepackage{enumitem}
\usepackage{array}
\newcolumntype{L}{>{$}l<{$}} % switch mathmode, a description column
% \usepackage{trimclip}
\usepackage{xstring}
\usepackage{environ}% http://ctan.org/pkg/environ;
	% required for use of label_matrix.tex.
\usepackage{nameref,hyperref}%\Autoref is defined by thmtools

\usepackage[noabbrev,nameinlink,capitalize]{cleveref}% n.b. cleveref after! hyperref
\crefname{example}{Example}{Examples}
\crefname{defn}{Definition}{Definitions}
\crefname{prop}{Proposition}{Propositions}
\crefname{constr}{Construction}{Constructions}
\crefname{conj}{Conjecture}{Conjectures}
\crefname{fact}{Fact}{Facts}


% visual appearance
\colorlet{color1}{Emerald}
\colorlet{color2}{color1>wheel,1,3}
\colorlet{color3}{color1>wheel,2,3}
\definecolor{deepgreen}{rgb}{0,0.5,0}
\colorlet{pinkish}{color3!25!magenta}
\usepackage{relsize}
% \hypersetup{colorlinks=true, linkcolor=blue, urlcolor=magenta, citecolor=deepgreen}
\hypersetup{colorlinks=true, linkcolor=color1, urlcolor=magenta, citecolor=color2}
\usepackage{parskip}
\usepackage{microtype}      % microtypography
\usepackage[format=plain,labelfont={sl},textfont={it,small},margin=1cm]{caption}
\usepackage{subcaption}
%\usepackage{subfig}
%\captionsetup[subfigure]{subrefformat=simple,labelformat=simple}
%\renewcommand\thesubfigure{(\alph{subfigure})}


% PREPARING EXTRA SYMBOLS
\usepackage{stmaryrd} % disable me for Joe
\SetSymbolFont{stmry}{bold}{U}{stmry}{m}{n}
\makeatletter
% load lBrace, rBrace from stix... but don't load stix
\@ifpackageloaded{stix}{}{
	\DeclareFontEncoding{LS2}{}{\noaccents@} \DeclareFontSubstitution{LS2}{stix}{m}{n}
	\DeclareSymbolFont{stix@largesymbols}{LS2}{stixex}{m}{n}
	\SetSymbolFont{stix@largesymbols}{bold}{LS2}{stixex}{b}{n}
	\DeclareMathDelimiter{\lBrace}{\mathopen} {stix@largesymbols}{"E8}{stix@largesymbols}{"0E}
	\DeclareMathDelimiter{\rBrace}{\mathclose}{stix@largesymbols}{"E9}{stix@largesymbols}{"0F}
}
% who doesn't love dutchcal?
\DeclareMathAlphabet{\mathdcal}{U}{dutchcal}{m}{n}
\DeclareMathAlphabet{\mathbdcal}{U}{dutchcal}{b}{n}
\DeclareRobustCommand{\shortto}{\mathrel{\mathpalette\short@to\relax}}
\newcommand{\short@to}[2]{%
	\mkern2mu
	\clipbox{{.5\width} 0 0 0}{$\m@th#1\vphantom{+}{\shortrightarrow}$}%
}
% better-spaced cases environment
\makeatletter
\renewenvironment{cases}[1][l]{\matrix@check\cases\env@cases{#1}}{\endarray\right.}
\def\env@cases#1{%
	\let\@ifnextchar\new@ifnextchar
	\left\lbrace\def\arraystretch{1.2}%
	\array{@{}#1@{\quad}l@{}}}
\makeatother
\newcommand\numberthis{\addtocounter{equation}{1}\tag{\theequation}}
\makeatother



% GENERAL SYMBOLS
\let\Horig\H
\let\H\relax
\def\nf{\nicefrac}
\DeclareMathOperator{\H}{\mathrm{H}} % Entropy
\DeclareMathOperator{\I}{\mathrm{I}} % Information
\DeclareMathOperator*{\Ex}{\mathbb{E}} % Expectation
\DeclareMathOperator*{\argmin}{arg\;min}
\newcommand\At[1]{\mathit{A\mkern-2mu t}(#1)} % boolean algebra atoms
\newcommand\lang[1]{\mathcal L^{\mathit{#1}}} % languages
\newcommand\geqc{\succcurlyeq}
\newcommand\leqc{\preccurlyeq}
\newcommand\mat[1]{\mathbf #1}
\newcommand{\tto}{\rightarrow\mathrel{\mspace{-15mu}}\rightarrow}
\newcommand{\indi}[1]{\mathbbm{1}_{\left[\vphantom{\big[}#1 \vphantom{\big]}\right]}}
\newcommand{\CI}{\mathrel{\perp\mspace{-10mu}\perp}}

\DeclarePairedDelimiterX{\infdivx}[2]{(}{)}{#1\;\delimsize\|\;#2}
\newcommand{\thickD}{I\mkern-8muD}
\newcommand{\kldiv}{\thickD\infdivx} % D_\mathrm{KL}
\newcommand{\Div}[1][]{\thickD_{#1}\infdivx} % D_\mathrm{KL}

\DeclarePairedDelimiter{\cbbr}{\lBrace}{\rBrace}
\DeclarePairedDelimiter{\bbr}{\llbracket}{\rrbracket}
\DeclarePairedDelimiter{\ppr}{\llparenthesis}{\rrparenthesis}
\DeclarePairedDelimiterX{\aar}[1]{\langle}{\rangle}
	{\mspace{3mu}\mathllap{\delimsize\langle}#1\mathrlap{\delimsize\rangle}\mspace{3mu}}
\DeclarePairedDelimiterXPP{\aard}[1]{}{\langle}{\rangle}{_{\!_\downarrow}}
	{\mspace{-3.5mu}\delimsize\langle#1\delimsize\rangle\mspace{-3.5mu}}

% STANDARD SETS / SPACES / CATEGORIES
% \newcommand\Mon{\mathbb{M}\mskip-1.5mu\mathbf{o\mskip-1mun}}
% \newcommand\Set{\ensuremath{\mathbb{S}\mathbf{et}}}
% \newcommand\FinSet{\ensuremath{\mathbb{F}\mathbf{in}\mathbf{S}\mathbf{et}}}

\newcommand\Set{\mathbb{S}\mathrm{et}}
\newcommand\FinSet{\mathbb{F}\mathrm{in}\mathrm{S}\mathrm{et}}
\newcommand\Meas{\mathbb{M}\mathrm{eas}}
\newcommand\two{\mathbbm 2}
\newcommand\PDG{\mathbb{P}\mathrm{DG}}


% PDG SYMBOLS
\newcommand{\N}{\mathcal N}
\newcommand{\Ed}{\mathcal E}
\newcommand{\Ar}{\mathcal A}
\newcommand{\V}{\mathcal V}
\newcommand{\bp}[1][L]{\mathbf{p}_{\!_{#1}\!}}

\newcommand{\Gr}{\mathcal G}
\DeclareMathOperator\src{\mathbf{src}}
\DeclareMathOperator\tgt{\mathbf{tgt}}
\newcommand{\qq}{\triangleright}

\newcommand{\dg}[1]{\mathbdcal{#1}}
\newcommand{\var}[1]{\mathsf{#1}}
\newcommand{\bbeta}{\boldsymbol\beta}
\newcommand{\balpha}{\boldsymbol\alpha}

% \newcommand\Src[1]{X_{{#1}}}
% \newcommand\Tgt[1]{Y_{{#1}}}
% \newcommand{\Src}{\mathrm{Src}}
% \newcommand{\Tgt}{\mathrm{Tgt}}
% \newcommand\Src[1]{S\mskip-2mu\mathit{r\mskip-3muc}_{{#1}}}
% \newcommand\Tgt[1]{T\mskip-5mu\mathit{g\mskip-1mut}_{{#1}}}
% \newcommand\Src[1]{\mathsf{S}\mskip-2mu\vphantom{|}_{{#1}}}
% \newcommand\Tgt[1]{\mathsf{T}\mskip-3mu\vphantom{|}_{{#1}}}
\newcommand\Src[1]{S\mskip-2mu\vphantom{|}_{{#1}}}
\newcommand\Tgt[1]{T\mskip-3mu\vphantom{|}_{{#1}}}

\newcommand{\none}{\varobslash}
\newcommand{\ed}[3]{%
	\mathchoice%
	{#2\overset{\smash{\mskip-5mu\raisebox{-3pt}{${#1}$}}}{\xrightarrow{\hphantom{\scriptstyle {#1}}}} #3} %display style
	{#2\overset{\smash{\mskip-5mu\raisebox{-3pt}{$\scriptstyle {#1}$}}}{\xrightarrow{\hphantom{\scriptstyle {#1}}}} #3}% text style
	{#2\overset{\smash{\mskip-5mu\raisebox{-3pt}{$\scriptscriptstyle {#1}$}}}{\xrightarrow{\hphantom{\scriptscriptstyle {#1}}}} #3} %script style
	{#2\overset{\smash{\mskip-5mu\raisebox{-3pt}{$\scriptscriptstyle {#1}$}}}{\xrightarrow{\hphantom{\scriptscriptstyle {#1}}}} #3}} %scriptscriptstyle


\newcommand{\alle}[1][L]{_{\ed {#1}XY}}

\DeclareMathOperator\dcap{\mathop{\dot\cap}}
\DeclareMathOperator{\bundle}{\sqcup}

\newcommand\Inc{\mathit{Inc}}
\newcommand{\IDef}[1]{\mathit{IDef}_{#1}}

% conversions
%\newcommand{\PDGof}[1]{{\dg M}_{#1}}
\makeatletter
\newcommand{\PDGof}[1]{{\mathbdcal{p\kern-0.05em d\kern-0.125em g}} \ifx#1\@empty\else(#1)\fi}
\newcommand{\PDHof}[1]{{\mathbdcal{p\kern-0.05em d\kern-0.125em h}} \ifx#1\@empty\else(#1)\fi}

% \newcommand{\UPDGof}[1]{{\dg N}_{#1}}
\newcommand{\UPDGof}[1]{{\mathbdcal{u\kern-0.05em p\kern-0.05em d\kern-0.125em g}} \ifx#1\@empty\else(#1)\fi}
\makeatother

% \DeclarePairedDelimiter{\SD}{\llbracket}{\rrbracket_{\text{sd}}
% \DeclarePairedDelimiterXPP{\SD}[1]{}{\llbracket}{\rrbracket}{_{\text{sd}}}{#1}
% \DeclarePairedDelimiterX{\SD}[1]{\{}{\}}{\delimsize\{#1\delimsize\}}
\DeclarePairedDelimiterX{\SD}[1]{\{}{\}}{\,\llap{\delimsize\{}#1\rlap{\delimsize\}}\,}
% \def\SD{\cbbr}


%% OTHER PGM SYMBOLS
\newcommand\GFE{\mathit{G\mkern-4mu F\mkern-4.5mu E}}
\newcommand\Pa{\mathbf{Pa}}

% conversions
\newcommand{\WFGof}[1]{\Psi_{{#1}}}
\newcommand{\FGof}[1]{\Phi_{{#1}}}


%% DATABASE SYMBOLS
\newcommand{\D}{\mathbdcal D} % for a database
\newcommand{\Attrs}{\mathdcal A}
\newcommand{\Idx}{\mathcal J}
\newcommand{\Doms}{{\mathcal D}}
% \newcommand{\Rels}{{\mathcal R}}
\newcommand{\Rels}{{\mathbf R}}
\newcommand{\Cols}{\mathcal C}%{\sigma}
\newcommand{\Sch}{\mathdcal S}
\newcommand{\Keys}{\mathcal K}
\newcommand{\DBProb}{\mathdcal P}
\newcommand{\arity}{\mathit{ar}}

\newcommand\Varis{\V\kern-.65pt\mathit{ars}}%
\newcommand\Bmid{\mathrel{\Big|}}%
\newcommand\vals{\mathbf{vals}}
\newcommand\nj{\bowtie}

% before theorems I need this:
\def\structurelevels{part,section,subsection,subsubsection,paragraph}%
\newcommand\nextstructurelevel{%
	\ifnum\value{section}=0 section\else%
	  \ifnum\value{subsection}=0 subsection\else%
	    \ifnum\value{subsubsection}=0 subsubsection\else%
		paragraph\fi\fi\fi}

%% The loop didn't work :( burrying it for now...
% \expandafter\newcommand\nextstructurelevel{%
% 	{\foreach\strl in \structurelevels{%
% 		\ifnum\expandafter\value{\strl}=0%
% 			\strl\breakforeach\fi%
% 		}%
% 	}}
% \newcommand\nextstructurelevel{subsubsection}

%% THEOREMS
\usepackage{amsthm}
\usepackage{thmtools} % asmsymb must be loaded also.
\usepackage[framemethod=TikZ]{mdframed}

	\makeatletter\begingroup

	\@for\theoremstyle:=definition,remark,plain\do{%
			\expandafter\g@addto@macro\csname th@\theoremstyle\endcsname{%
				\addtolength\thm@preskip\parskip}
			}\endgroup
	% add theorems, examples,etc to table of contents
	\def\thmTOCadd{
		\edef\asdffdsa{\nextstructurelevel}
		\ifx\thmt@optarg\@empty
			\addcontentsline{toc}{\asdffdsa}{%
			\texorpdfstring{
			 	\makebox[3cm][r]{\color{gray}{\it\thmt@thmname}~%
				\csname the\thmt@envname\endcsname}}%
			{\thmt@thmname\csname the\thmt@envname\endcsname: \thmt@optarg}}
		\else\addcontentsline{toc}{\asdffdsa}{%
			\texorpdfstring{
			 	\makebox[3cm][r]{\color{gray}{\it\thmt@thmname}~%
				\csname the\thmt@envname\endcsname:}\hspace{1em} {\color{black!50!gray}\small\thmt@optarg}}%
			{\thmt@thmname\csname the\thmt@envname\endcsname: \thmt@optarg} %
		}\fi}
	% macro for clearing theorems (if preamble adapted)
	% my answer presented here: https://tex.stackexchange.com/a/296184/191268
	\def\cleartheorem#1{%
		\expandafter\let\csname#1\endcsname\relax
		\expandafter\let\csname c@#1\endcsname\relax
	}
	\def\clearthms#1{ \@for\tname:=#1\do{\cleartheorem\tname} }
	\makeatother

	\theoremstyle{plain}
	\declaretheorem[within=section,name=Theorem,postheadhook={\thmTOCadd}]{theorem}
	\newtheorem{coro}{Corollary}[theorem]
	\declaretheorem[sibling=theorem,name=Proposition, postheadhook={\thmTOCadd}]{prop}
	\newtheorem{lemma}[theorem]{Lemma}
	\newtheorem{fact}[theorem]{Fact}
	\newtheorem{conj}[theorem]{Conjecture}

	\theoremstyle{definition}
	\newtheorem*{defn*}{Definition}
	\declaretheorem[name=Definition,style=definition,qed=$\square$,within=section, postheadhook={\thmTOCadd}]{defn}
	\declaretheorem[name=Construction,qed=$\square$,sibling=defn, postheadhook={\thmTOCadd}]{constr}
	% \declaretheorem[qed=$\square$]{example}
	\declaretheorem[name=Example,qed=$\square$,postheadhook={\thmTOCadd}]{example}

	\theoremstyle{remark}
	\newtheorem*{remark}{Remark}

	\usepackage{xpatch}
	\makeatletter
	% \xpatchcmd{\thmt@restatable}% Edit \thmt@restatable
	%    {\csname #2\@xa\endcsname\ifx\@nx#1\@nx\else[{#1}]\fi}% Replace this code
	%    % {\ifthmt@thisistheone\csname #2\@xa\endcsname\typeout{oiii[#1;#2\@xa;#3;\csname thmt@stored@#3\endcsname]}\ifx\@nx#1\@nx\else[#1]\fi\else\csname #2\@xa\endcsname\fi}% with this code
	%    {\ifthmt@thisistheone\csname #2\@xa\endcsname\ifx\@nx#1\@nx\else[{#1}]\fi
	%    \else\fi}
	%    {}{\typeout{THMT PATCH FAILURE; recall may not work}} % execute code for success/failure instances
	% \xpatchcmd{\thmt@restatable}% Edit \thmt@restatable
	%    {\csname end#2\endcsname}
	%    {\ifthmt@thisistheone\csname end#2\endcsname\else\fi}
	%    {}{\typeout{THMT PATCH FAILURE; recall may not work}}
	%% version of \recall used in PDG.tex
	% \newcommand{\recall}[1]{\medskip\par\noindent{\bf \expandarg\Cref{thmt@@#1}.}
	%  	\begingroup\em \noindent \expandafter\csname#1\endcsname* \endgroup\par\smallskip}
	%% old version of \recall with quotations
	% \newcommand\recall[1]{\expandarg\cref{#1}:\vspace{-1em} \begingroup\small\color{gray!80!black}\begin{quotation} \expandafter\csname #1\endcsname* \end{quotation}\endgroup }
	%% a hybrid:
	\newcommand{\recall}[1]{\medskip\par
	\begingroup\small\color{gray!20!black}\begin{quotation}
		\noindent{\bf \expandarg\Cref{thmt@@#1}:}
	 	\begingroup\em \noindent \expandafter\csname#1\endcsname* \endgroup
	\end{quotation}\endgroup
	\par\smallskip}
	\makeatother

	\newcommand{\begthm}[3][]{\begin{#2}[{name=#1},restate=#3,label=#3]}

%% ANNOTATION
\newcommand{\todo}[1]{{\color{red}\ \!\Large\smash{\textbf{[}}{\normalsize\textsc{todo:} #1}\ \!\smash{\textbf{]}}}}
\newcommand{\note}[1]{{\color{blue}\ \!\Large\smash{\textbf{[}}{\normalsize\textsc{note:} #1}\ \!\smash{\textbf{]}}}}
\newcommand{\moveme}[1]{{\color{purple}\ \!\Large\smash{\textbf{[}}{\normalsize\textsc{moveme:} #1}\ \!\smash{\textbf{]}}}}

\mdfdefinestyle{mybox}{skipabove=1em,skipbelow=1em,roundcorner=2pt}
\newmdenv[style=mybox,fontcolor=black!25, backgroundcolor=black!3,linecolor=black!20]{inactive}
\newmdenv[style=mybox,backgroundcolor=color3!5,linecolor=color3!20]{highlight-changes}
\newmdenv[style=mybox,linecolor=color2!50,topline=false,bottomline=false,rightline=false,innerleftmargin=1em,linewidth=5pt]{leftbar}
\newmdenv[roundcorner=5pt, subtitlebelowline=true,subtitleaboveline=true, subtitlebackgroundcolor=color1!70!white,
 	backgroundcolor=color3!20!white, frametitle={Annotating},frametitlerule=true, frametitlebackgroundcolor=color1!70!white, skipabove=1em,skipbelow=1em,]{annotating}
\newmdenv[roundcorner=5pt, backgroundcolor=pinkish!20!white, frametitle={$\langle$under construction $\rangle$},frametitlerule=false,
innertopmargin=2pt, frametitlebelowskip=3pt, frametitleaboveskip=2pt, frametitlebackgroundcolor=pinkish!70!white, skipabove=1em,skipbelow=1em, frametitlefont={\normalfont\itshape},leftmargin=-10pt, rightmargin=-10pt]
		{wip}
% \usetikzlibrary{external}
% \tikzexternalize[prefix=tikz/]  % activate!
% \usepackage{etoolbox}
% \AtBeginEnvironment{tikzcd}{\tikzexternaldisable} %... except careful of tikzcd...
% \AtEndEnvironment{tikzcd}{\tikzexternalenable}


%%%%%%%%%%%%%%%% TIKZ SETUP %%%%%%%%%%%%%%%%%%%%%
	\usetikzlibrary{positioning,fit,calc, decorations, arrows, shapes, shapes.geometric}
	\usetikzlibrary{patterns,backgrounds}
	\usetikzlibrary{cd}

	\pgfdeclaredecoration{arrows}{draw}{
		\state{draw}[width=\pgfdecoratedinputsegmentlength]{%
			\path [every arrow subpath/.try] \pgfextra{%
				\pgfpathmoveto{\pgfpointdecoratedinputsegmentfirst}%
				\pgfpathlineto{\pgfpointdecoratedinputsegmentlast}%
			};
	}}
	%%%%%%%%%%%%
	\tikzset{AmpRep/.style={ampersand replacement=\&}}
	\tikzset{center base/.style={baseline={([yshift=-.8ex]current bounding box.center)}}}
	\tikzset{paperfig/.style={center base,scale=1.0, every node/.style={transform shape}}}

	\tikzset{is bn/.style={background rectangle/.style={fill=blue!35,opacity=0.3, rounded corners=5},show background rectangle}}
	% Node Stylings
	\tikzset{dpadded/.style={rounded corners=2, inner sep=0.7em, draw, outer sep=0.3em, fill={black!50}, fill opacity=0.08, text opacity=1}}
	% \tikzset{active/.style={fill=blue, fill opacity=0.1}}
	% \tikzset{square/.style={regular polygon,regular polygon sides=4, rounded corners = 0}}
	% \tikzset{octagon/.style={regular polygon,regular polygon sides=8, rounded corners = 0}}
	\tikzset{dpad0/.style={outer sep=0.05em, inner sep=0.3em, draw=gray!75, rounded corners=4, fill=black!08, fill opacity=1}}
	\tikzset{dpad1/.style={outer sep=0.1em, inner sep=0.4em, draw=gray!75, rounded corners=3, fill=black!08, fill opacity=1}}
	\tikzset{dpad/.style args={#1}{every matrix/.append style={nodes={dpadded, #1}}}}
	\tikzset{light pad/.style={outer sep=0.2em, inner sep=0.5em, draw=gray!50}}

	\tikzset{arr/.style={draw, ->, thick, shorten <=3pt, shorten >=3pt}}
	\tikzset{arr0/.style={draw, ->, thick, shorten <=0pt, shorten >=0pt}}
	\tikzset{arr1/.style={draw, ->, thick, shorten <=1pt, shorten >=1pt}}
	\tikzset{arr2/.style={draw, ->, thick, shorten <=2pt, shorten >=2pt}}
	\tikzset{archain/.style args={#1}{arr, every arrow subpath/.style={draw,arr, #1}, decoration=arrows, decorate}}


	\tikzset{fgnode/.style={dpadded,inner sep=0.6em, circle},
	factor/.style={light pad, fill=black}}

	% For illustrating what's inside of nodes:
	\tikzset{alternative/.style args={#1|#2|#3}{name=#1, circle, fill, inner sep=1pt,label={[name={lab-#1},gray!30!black]#3:\scriptsize #2}} }

	\tikzset{bpt/.style args={#1|#2}{alternative={#1|#2|above}} }
	\tikzset{tpt/.style args={#1|#2}{alternative={#1|#2|below}} }
	\tikzset{lpt/.style args={#1|#2}{alternative={#1|#2|left}} }
	\tikzset{rpt/.style args={#1|#2}{alternative={#1|#2|right}} }
	\tikzset{pt/.style args={#1}{alternative={#1|#1|above}} }
	\tikzset{mpt/.style args={#1|#2}{name=#1, circle, fill, inner sep=1pt,label={[name={lab-#1},gray]\scriptsize #2}} }

	\tikzset{Dom/.style args={#1 (#2) around #3}{dpadded, name=#2, label={[name={lab-#2},align=center] #1}, fit={ #3 } }}
	\tikzset{bDom/.style args={#1 (#2) around #3}{dpadded, name=#2, label={[name={lab-#2},align=center]below:#1}, fit={ #3 } }}


	\newcommand\cmergearr[4]{
		\draw[arr,-] (#1) -- (#4) -- (#2);
		\draw[arr, shorten <=0] (#4) -- (#3);
	}
	\newcommand\mergearr[3]{
		\coordinate (center-#1#2#3) at (barycentric cs:#1=1,#2=1,#3=1.2);
		\cmergearr{#1}{#2}{#3}{center-#1#2#3}
	}
	\newcommand\cunmergearr[4]{
		\draw[arr,-, , shorten >=0] (#1) -- (#4);
		\draw[arr, shorten <=0] (#4) -- (#2);
		\draw[arr, shorten <=0] (#4) -- (#3);
	}
	\newcommand\unmergearr[3]{
		\coordinate (center-#1#2#3) at (barycentric cs:#1=1.2,#2=1,#3=1);
		\cunmergearr{#1}{#2}{#3}{center-#1#2#3}
	}


	\usetikzlibrary{matrix}
	\tikzset{toprule/.style={%
	        execute at end cell={%
	            \draw [line cap=rect,#1]
	            (\tikzmatrixname-\the\pgfmatrixcurrentrow-\the\pgfmatrixcurrentcolumn.north west) -- (\tikzmatrixname-\the\pgfmatrixcurrentrow-\the\pgfmatrixcurrentcolumn.north east);%
	        }
	    },
	    bottomrule/.style={%
	        execute at end cell={%
	            \draw [line cap=rect,#1] (\tikzmatrixname-\the\pgfmatrixcurrentrow-\the\pgfmatrixcurrentcolumn.south west) -- (\tikzmatrixname-\the\pgfmatrixcurrentrow-\the\pgfmatrixcurrentcolumn.south east);%
	        }
	    },
	    leftrule/.style={%
	        execute at end cell={%
	            \draw [line cap=rect,#1] (\tikzmatrixname-\the\pgfmatrixcurrentrow-\the\pgfmatrixcurrentcolumn.north west) -- (\tikzmatrixname-\the\pgfmatrixcurrentrow-\the\pgfmatrixcurrentcolumn.south west);%
	        }
	    },
	    rightrule/.style={%
	        execute at end cell={%
	            \draw [line cap=rect,#1] (\tikzmatrixname-\the\pgfmatrixcurrentrow-\the\pgfmatrixcurrentcolumn.north east) -- (\tikzmatrixname-\the\pgfmatrixcurrentrow-\the\pgfmatrixcurrentcolumn.south east);%
	        }
	    },
	    table with head/.style={
		    matrix of nodes,
		    row sep=-\pgflinewidth,
		    column sep=-\pgflinewidth,
		    nodes={rectangle,minimum width=2.5em, outer sep=0pt},
		    row 1/.style={toprule=thick, bottomrule},
  	    }
	}



% \NewEnviron{ctikzpicture}{\begin{center}\expandafter\begin{tikzpicture}\BODY\end{tikzpicture}\end{center}}
% \usetikzlibrary{shapes.geometric}
\usetikzlibrary{backgrounds}
\tikzset{dpad0/.style={outer sep=0.05em, inner sep=0.3em, draw=gray!75, rounded corners=4, fill=black!08, fill opacity=1}}
\tikzset{arr0/.style={draw, ->, thick, shorten <=0pt, shorten >=0pt}}
\tikzset{arr1/.style={draw, ->, thick, shorten <=1pt, shorten >=1pt}}
\tikzset{arr2/.style={draw, ->, thick, shorten <=2pt, shorten >=2pt}}
\tikzset{is bn/.style={background rectangle/.style={fill=blue!35,opacity=0.3, rounded corners=5},show background rectangle}}
\newcommand\lab[1]{(#1)(lab-#1)}


%%%%%%%%%%%%%% label_matrix.tex %%%%%%%%%%%%%%%%
% \usepackage{environ}
\usepackage{xstring}

% Wow this works I'm brilliant
\def\wrapwith#1[#2;#3]{
	\expandarg\IfSubStr{#1}{,}{
		\expandafter#2{\expandarg\StrBefore{#1}{,}}
		\expandarg\StrBehind{#1}{,}[\tmp]
		\xdef\tmp{\expandafter\unexpanded\expandafter{\tmp}}
		#3
		\wrapwith{\tmp}[#2;{#3}]
	}{ \expandafter#2{#1} }
}
\def\hwrapcells#1[#2]{\wrapwith#1[#2;&]}
\def\vwrapcells#1[#2]{\wrapwith#1[#2;\\]}
\NewEnviron{mymathenv}{$\BODY$}

\newcommand{\smalltext}[1]{\text{\footnotesize#1}}
\newsavebox{\idxmatsavebox}
\def\makeinvisibleidxstyle#1#2{\phantom{\hbox{#1#2}}}
\newenvironment{idxmatphant}[4][\color{gray}\smalltext]{%
	\def\idxstyle{#1}
	\def\colitems{#3}
	\def\rowitems{#2}
	\def\phantitems{#4}
	\begin{lrbox}{\idxmatsavebox}$%$\begin{mymathenv}
	\begin{matrix}  \begin{matrix} \hwrapcells{\colitems}[\idxstyle]  \end{matrix}
		% &\vphantom{\idxstyle\colitems}
		\\[-0.05em]
		\left[
		\begin{matrix}
			\hwrapcells{\phantitems}[\expandafter\makeinvisibleidxstyle\idxstyle]  \\[-1.2em]
	}{
		\end{matrix}\right]		&\hspace{-0.8em}\begin{matrix*}[l] \vwrapcells{\rowitems}[\idxstyle] \end{matrix*}\hspace{0.1em}%
	\end{matrix}%
	$%\end{mymathenv}
	\end{lrbox}%
	\raisebox{0.75em}{\usebox\idxmatsavebox}
%	\vspace{-0.5em}
}

\newenvironment{idxmat}[3][\color{gray}\smalltext]
	{\begingroup\idxmatphant[#1]{#2}{#3}{#3}}
	{\endidxmatphant\endgroup}

\newenvironment{sqidxmat}[2][\color{gray}\smalltext]
	{\begingroup\idxmat[#1]{#2}{#2}}
	{\endidxmat\endgroup}


%%%%%%%%%%%%
% better alignment for cases
\makeatletter
\renewenvironment{cases}[1][l]{\matrix@check\cases\env@cases{#1}}{\endarray\right.}
\def\env@cases#1{%
	\let\@ifnextchar\new@ifnextchar
	\left\lbrace\def\arraystretch{1.2}%
	\array{@{}#1@{\quad}l@{}}}
\makeatother

% should result in the following (which is not maintained)

%OMG THIS WORKS
\def\wrapwith#1[#2;#3]{
	\expandarg\IfSubStr{#1}{,}{
		\expandafter#2{\expandarg\StrBefore{#1}{,}}
		\expandarg\StrBehind{#1}{,}[\tmp]
		\xdef\tmp{\expandafter\unexpanded\expandafter{\tmp}}
		#3
		\wrapwith{\tmp}[#2;{#3}]
	}{ \expandafter#2{#1} }
}
\def\hwrapcells#1[#2]{\wrapwith#1[#2;&]}
\def\vwrapcells#1[#2]{\wrapwith#1[#2;\\]}

\newcommand{\smalltext}[1]{\text{\footnotesize#1}}
\newsavebox{\idxmatsavebox}
\def\makeinvisibleidxstyle#1#2{\phantom{\hbox{#1#2}}}
\newenvironment{idxmatphant}[4][\color{gray}\smalltext]{%
	\def\idxstyle{#1}
	\def\colitems{#3}
	\def\rowitems{#2}
	\def\phantitems{#4}
	\begin{lrbox}{\idxmatsavebox}$
	\begin{matrix}  \begin{matrix} \hwrapcells{\colitems}[\idxstyle]  \end{matrix} \\[0.1em]
		\left[
		\begin{matrix}
			\hwrapcells{\phantitems}[\expandafter\makeinvisibleidxstyle\idxstyle]  \\[-1em]
	}{
		\end{matrix}\right]		&\hspace{-0.5em}\begin{matrix*}[l] \vwrapcells{\rowitems}[\idxstyle] \end{matrix*}
	\end{matrix}
	$\end{lrbox}
	\raisebox{0.75em}{\usebox\idxmatsavebox}
%	\vspace{-0.5em}
}

\newenvironment{idxmat}[3][\color{gray}\smalltext]
	{\begingroup\idxmatphant[#1]{#2}{#3}{#3}}
	{\endidxmatphant\endgroup}

\newenvironment{sqidxmat}[2][\color{gray}\smalltext]
	{\begingroup\idxmat[#1]{#2}{#2}}
	{\endidxmat\endgroup}

\usepackage[margin=1in]{geometry}

\newcommand{\commentout}[1]{\ignorespaces}
\newif\ifprecompiledfigs
\precompiledfigsfalse
% \precompiledfigstrue

\newif\ifexternalizefigures
\externalizefiguresfalse

\ifexternalizefigures
	\usetikzlibrary{external}
	\tikzexternalize[prefix=tikz/]  % activate!
	\usepackage{etoolbox}
	 \AtBeginEnvironment{tikzcd}{\tikzexternaldisable} %... except careful of tikzcd...
	 \AtEndEnvironment{tikzcd}{\tikzexternalenable}
\fi

%\twocolumn
\title{The PDG Manual}
\author{Oliver Richardson  \texttt{oli@cs.cornell.edu}}

\begin{document}

	\maketitle
	\tableofcontents
	%\listoffigures
	% \listoftheorems
	\clearpage
	%some day...
	% \twocolumn
	\subfile{prelim.tex}
	\part{The PDG Representation}
	\subfile{intro.tex}
	\subfile{formalism.tex}
	\subfile{idef.tex}

	\begin{wip} \subfile{trace.tex} \end{wip}

%	\begin{align*}
%		 \tau(\mu)(x,y,\mat z) &:= \mu(x,y,\mat z) \frac{\bp(y \mid x)}{\mu(y\mid x)}
%			 &  \tau(\mu)(x,y,\mat z) &:= \mu(x,y,\mat z) \frac{\bp(y \mid x)}{\mu(y\mid x, \mathbf z)} \\
% 			&= \mu(x)\; \bp(y \mid x)\; \mu(\mat z \mid x,y)
%			 &   &= \mu(x,\mat z)\; \bp(y \mid x)
%	\end{align*}

	\part{Capturing Other Modeling Formalisms}
	PDGs are extremely flexible, and nicely capture

    \section{Raw Probability Distributions}
	Probability distributions themselves are a particular kind of PDG;
	a  triple $(\Omega, \mathcal F, \Pr)$ is naturally identified with the diagram
	\begin{center}
		\begin{tikzpicture}
			\node[dpadded] (1) at (0,0) {$\var 1$};
			\node[dpadded] (W) at (3,0) {$\Omega$};

			\draw[arr] (1) to node[fill=white]{$\Pr$} (W);
		\end{tikzpicture}
	\end{center}

	% \begin{example}\label{ex:worldsonly}
	% \end{example}

    Let $\N$ be a set of variables whose values are given by $\V$. When we use the characterization of PDGs based on directed hyper-graphs, a joint distribution $\mu \in \Delta[\V(\N)]$ is naturally identified with a particular unweighted PDG. Specifically, the data of $\mu$ is given by the PDG $(\N, \{ E_0 \}, \V, \mat p)$ containing a single hyper-edge $E_0$ whose source is empty and whose target is all of $\N$, associated with the cpd $\bp[E](\mat x) := \mu(\mat x)$.

    \begin{example}
        For a 3-variable
    \end{example}



	\subfile{pgms.tex}
	\subfile{db.tex}

	\section{Automata}

	\part{Applications of PDGs}
	\part{The Categorical View}

	\part{Misc}
	\section{Philosophy}
	\section{Scratch}

	\begin{inactive}
		\subsection{}
		The data of a PDG, alternately put, is the set of nodes + an $\alpha$ matrix for each pair of them, the set of cpts, a $\beta$ for each cpt, and

		\begin{prop}
			% what I want to say: IDef entails the independencies of D, in that
			% it causes the region of the information profile
			% associated with any independence of the DN, to be red.
			For any sets of variables $\mat X, \mat Y, \mat Z$, for which $\mat  X \CI_{\mathcal D} \mat Y \mid \mat Z$, we have
			\[ \frac{\partial \IDef{\cal D}}{\partial \I(\mat X; \mat Y \mid \mat Z)}(\mu) < 0 \]
			where $\I(\mat X; \mat Y \mid \mat Z)(\mu)$ is the conditional information between $\mat X$ and $\mat Y$ given $\mat Z$, a non-negative quantity which is zero iff $\mat X \CI_{\mu} \mat Y \mid \mat Z$.
		\end{prop}
	\end{inactive}

	\begin{annotating}[frametitle={Matroids}]
		\subsection{Matroids}
		Does the set of hyper-edges of a PDG form a matroid?
		In the case of joint distributions (hyper-edges have only heads and not tails), then clearly it
		is downward closed, as we can find the marginal on any subspace.


		If the PDG is consistent
	\end{annotating}



	A probabilistic prgram $\tau_{\dg M} : \Delta\V(\dg M) \to \Delta \V(\dg M)$
	\begin{algorithmic}
		\State $i = 3$
		\For{$t = 1, 2, 3, \ldots$}
		    \State Choose \texttt{qual} with probability $\nicefrac{\gamma}{1+\gamma}$ and \texttt{quant} otherwise (probability $\nf1{1+\gamma}$).

			\If{\texttt{quant}}
				\State {Let} $\hat \beta$ be the normalized vector of $\beta$s, such that $\sum_L\hat\beta_L = 1$.
				\State \textbf{Draw}  $L \sim \hat\beta$;
				\State {Let} $X:= \src L;\quad Y := \tgt L;\quad Z:= \N \setminus\{X,Y\}$;
				\State \textbf{Update} $\mu^{t+1} \gets \mu^t(X) \bp(Y \mid X) \mu^t(Z \mid X,Y)$
			\ElsIf{\texttt{qual}}
				\State \textbf{Update} $\mu^{t+1} \gets $
			\EndIf

		\EndFor
	\end{algorithmic}

{
    % \small
    \bibliographystyle{aaai21}
    \bibliography{../allrefs,../z,../joe,../db}
}


	\part*{Appendix}
	\addcontentsline{toc}{part}{Appendix}
	\appendix
	\section{Background Material}
    \subsection{Information Theory}

    \begin{defn}\label{def:entropy}
        The entropy of a random variable $X : \Omega \to \V(X)$ is with respect to a probability distribution $\mu : \Delta \Omega$ given by
        \[ \H_\mu(X) = \sum_{x \in \V(X)} \mu_X(x) \log \frac{1}{\mu_X(x)} ,\]
        where $\mu_X$ is the marginal of $\mu$ on $X$.
    \end{defn}

    \begin{fact}
        For all random variables $X,Y$ over the space of outcomes $\Omega$, if there is a function $f$ such that $Y(\omega) = f(X(\omega))$ for all $\omega$ with $\mu(\omega) > 0$, then $\H_\mu(X) \leq \H_\mu(Y)$.
    \end{fact}
    One consequence is that entropy is independent of the particular representation.
    \begin{prop}[invariance with respect to change of variables]
        If $X : \Omega \to \V(X)$ and $Y : \Omega \to \V(Y)$ are a pair of random variables over $\Omega$ and there exist functions $f : \V(X) \to \V(Y)$ and $g : \V(Y) \to \V(X)$ such that $f(X(\omega)) = Y(\omega)$ and $g(Y(\omega)) = X(\omega)$ for all $\omega \in \Omega$, then $\H_\mu(X) = \H_\mu(Y)$ for all $\mu$.
%       are a pair of functions that commute with the variables (that is, $f(X(\omega)) = Y(\omega)$ and $g(Y(\omega)) = X(\omega)$ for all $\omega \in \Omega$), then $\H_\mu(X) = \H_\mu(Y)$ for all $\mu$.
    \end{prop}

    The setting of the above
    \begin{center}
        \begin{tikzcd}[column sep=1em]
            &\Omega\ar[dl, "X"']\ar[dr, "Y"]\\
            \V(X) \ar[rr, "f"] && \V(Y)
        \end{tikzcd}
    \end{center}

    \subsection{Boolean Algebra}
%   $\mu$ is a measure over $\V(\N) = \prod_{N \in \N}\V(N)$ and if $\N$ and each $\V(N)$ is finite, then every subset of $\V(\N)$ is measurable.

    \begin{defn}[Boolean algebra, atom, natural order, and the free Boolean algebra generated by a set]
        A \emph{Boolean algebra} $B = (S, \land,\lor,\lnot,0,1)$ is a carrier set $S$, together with interpretations of the binary boolean operations $\land $ and $\lor$ as functions $S\times S \to S$, the unary operation $\lnot$ as a function $S \to S$, and distinguished elements $0, 1 \in S$, such that for all $a, b, c \in S$,
        \begin{enumerate}[itemsep=0pt, parsep=1pt,label={BA\arabic*.}]
            \item $\land, \lor$ are associative and commutative,
            \item $a \lor 0 = s$ and $a \land 1 = a$ for all $a \in S$,
            \item $a \lor(b \land c) = (a \lor b) \land (a \lor c)$ and $a \land(b \lor c) = (a \land b) \lor (a \land c)$,  and finally
            \item $a \lor \lnot a = 1$ and $a \land \lnot a = 0$.
        \end{enumerate}
        A Boolean algebra $B$ defines a partial order called the \emph{natural order} (which is a partial order) by declaring $a \leq b$ iff $a \lor b = b$, and declaring that $a < b$ iff $a \leq b$ and $a \neq b$.
        The \emph{atoms} of $B$, denoted $\At B$ are those non-zero elements of $a \in S$ such that there does not exist a nonzero element $x \in S, x \ne 0$ such that $x < a$. Equivalently the atoms of $B$ are those elements $a\in S$ which can only expressed as a disjunction $a = x \lor y$ if either $x = a$ or $y=a$.
        %       \[ \mathit{At}(B) := \{ \} \]
        If $G$ is a set, the \emph{free boolean algebra generated by $G$} is the unique smallest Boolean algebra containing $G$ that does not satisfy any additional equations, beyond {BA1-4}.
    \end{defn}
    \begin{example}
        If $G = \{a, b\}$, the free boolean algebra $BG$ generated by $G$ consists of the sixteen elements

        \medskip
        \begin{minipage}{0.3\textwidth}
            \begin{center}%{R}{3cm}
                %           \let\varnames{X,Y,Z}
                \begin{tikzpicture}
                    \begin{scope}[scale=0.4]
                        \begin{scope}[blend group=hard light, opacity=0.5]
                            \draw[fill=color1!50!white]   ( 0:1.2) circle (2);
                            \draw[fill=color3!50!white] (-180:1.2) circle (2);
                        \end{scope}

                        \draw(0:1.2) circle (2);
                        \draw(-180:1.2) circle (2);

                        \node[yshift=1cm] at (0:2) {$b$};
                        \node[yshift=1cm] at (-180:2) {$a$};
                        \node at (-5,0){$\scriptstyle  \lnot a \land \lnot b$};
                        \node at (0,0){$\scriptstyle a \land b$};
                        \node at (-180:2){$\scriptstyle a \land \lnot b$};
                        \node at (0:2){$\scriptstyle  \lnot a \land b$};
                    \end{scope}
                \end{tikzpicture}
                \refstepcounter{figure}\label{fig:ven2BA}
                %           \caption[a]{B}
            \end{center}
        \end{minipage}\begin{minipage}{0.65\textwidth}
            \begin{equation} \left\{\;
                \begin{aligned}
                    a \land b,\; a \land \lnot b,\; \lnot a \land b,\; \lnot a \land \lnot b,\; \\
                    %           \smash{\overbracket{ a \land b,\; a \land \lnot b,\; \lnot a \lansd b,\; \lnot a \land \lnot b,\;}^{\text{the atoms of $B$}}} \\
                    (a\land b) \lor(\lnot a \land \lnot b),\; (a\land \lnot b) \lor (\lnot a \land b),\; 0,\; 1,\;\\
                    a \lor b,\; a \lor \lnot b,\;  \lnot a \lor b,\; \lnot a \lor \lnot b,\; \\
                    a,\; \lnot a,\; b,\; \lnot b,\;
                \end{aligned}\;
                \right\} \label{eq:exba2} \end{equation}
        \end{minipage}
        \par\smallskip\noindent
        corresponding to the $2^{2^2} = 16$ distinct boolean expressions that can be constructed with the two primitve symbols $\{a, b\}$. The atoms of $BG$ are those elements that appear on the first line of \eqref{eq:exba2}, and correspond to the four ``atomic'' regions of the Venn diagram to their left.
    \end{example}

    \subsection{Hyper-Graphs and Information}
    We originally formalized the structure of PDGs with regular edges, which have a single source and target. However, $\IDef{}$ is most naturally understood in a setting where PDGs are modeled as hyper-graphs; we now provide an characterization in these terms.%
        \footnote{For a translation into the original formulation consult \cref{apx:hyper-vs-graph}.}
    \begin{defn}[hyper graph] \label{defn:hypergraph}
        A \emph{directed multi-hyper-graph}, (which we abbreviate \emph{hyper-graph}), is a set $\N$ of variables, and a set $\Ed = \{ \mat X \to \mat Y \}$ of hyper edges. Each edge $E \in \Ed$ has a subset of the variables $\src(E) \subseteq \N$ which we call the \emph{source} of $E$, and a second subset of variables $\tgt(E) \subseteq \N$ that we call the \emph{target} of $E$. We will often specify an edge $E$ along with its source $\mat X = \src(E)$ and target $\mat Y = \tgt(E)$ by writing $\ed E{\mat X}{\mat Y}$.
    \end{defn}
%   Although this is not always made explicit, any computation involving entropy depends on the values
%   \begin{defn}[variable hypergraph]
%       A \emph{variable hypergraph} is a tuple $(\N, \Ed, \V)$ where $(\N, \Ed)$ is a (directed multi-)hyper graph, whose vertices $\N$ correspond to variables with values $\V$. Concretely, $\V(N)$ is the set of possible values that a variable $N \in \N$ can take.
%   \end{defn}

    \begin{defn}[PDH IDef] \label{defn:idef}
        If $\Gr = (\N, \Ed)$ is a variable hypergraph, and $\mu {\Delta [ \prod_{N\in\N}\V(N)]}$ is a joint probability distribution over variables $\mathcal X \supseteq \N$, then the $\Gr$-information deficiency of $\mu$ is given by
        \begin{equation}
            \IDef{\Gr}(\mu) := \bigg[~\sum_{\ed E{\mat X}{\mat Y}} \H_\mu(\mat Y\mid \mat X)\bigg] - \H_\mu(\N).
% same but with src/tg instead of arrow notation
%           \IDef{\Gr}(\mu) := \bigg[~\sum_{\ed E{\mat X}{\mat Y}} \H_\mu(\mat{tgt} E\mid \mat{src} E)\bigg] - \H_\mu(\N).
            \label{eq:idef}
        \end{equation}
%       where $\H(\mat Y \mid \mat X)$ is the conditional entropy of $\mat Y$ given $\mat X$ with respect to $\mu$
%       (see \cref{apx:info} for more details)
%       , and $\H_\mu(\N)$, often written simply $\H(\mu)$, is the total entropy of $\mu$ across all variables.
    \end{defn}

    % Define the signed measure.

\end{document}
