\documentclass[the-pdg-manual.tex]{subfiles}
\begin{document}
	\section{Probabilistic Automata}
    
    \begin{defn}
        A \emph{probabilistic automaton} (PA) is tuple $(Q, \Sigma, \tau, q_0, b)$, 
        where
        \begin{itemize}[nosep]
            \item $Q $ is a finite set of states;
            \item $\Sigma $ is a finite input alphabet;
            \item $\tau : Q \times \Sigma \to \Delta Q$ is a probabilistic transition function;
            \item $q_0 : \Delta Q$ is the initial distribution over states; 
            \item $b : Q \to 2$ is a set of accepting states. 
        \end{itemize}
    \end{defn}
    
    
    Given a probabilistic automaton $\mathit{PA} = (Q, \Sigma, \tau, q_0, b)$, here are two ways of regarding it as a PDG.
    \begin{equation}
        % \PDGof{\mathit{PA}}_1 = 
        {\dg M}_{1}^{\mathit{PA}} = 
        \begin{tikzpicture}[center base]
            \node[dpadded] at (0,0) (Q) {$Q$};
            \node[dpadded] at (2.2,0) (Q') {$Q'$};
            \node[dpadded] at (1,1.4) (S) {$\Sigma$};            
            \draw[arr, <-] (Q) to node[above]{$q_0$} +(-1.5,0);
            
            \mergearr{Q}{S}{Q'}
            \node[below=1ex of center-QSQ']{$\tau$};
        \end{tikzpicture}
    \end{equation}
    
    
    Second,
    \begin{align*}
        % \PDGof{\mathit{PA}}_2 = 
        {\dg M}_{2}^{\mathit{PA}} =
        \bigg( 
            \N &:= \{ Q \};  \qquad \V Q = Q; \\[-2ex]
            \Ar &:= \{ Q \xrightarrow{\sigma} Q  \}_{\sigma \in \Sigma}; 
                \qquad p_\sigma(q | q') := \tau(q' | q, \sigma)
        \bigg)
    \end{align*}
    
    % \textbf{Automaton Semantics for PDGs.}
    % Conversely, a PDG can be viewed as a probabilistic automaton. 
    
    
    \def\???{{\color{red}\textbf{???}}}
    \begin{constr}
        Given a PDG $\dg M = (\N, \Ar, \V, \mathcal P, \balpha, \bbeta)$, 
        we can form an automaton
        \begin{align*}
            \mathit{PA}_{\dg M} :=
            \smash{\bigg(}&
            % Q := \sum_{a : \Ar} \V(\Src a, \Tgt a);\quad
            Q := \V\N;\quad
                \Sigma := \Ar;  \quad
                \tau_{\sigma}(\mat w' | \mat w) = 
                    % \mathbbm1[\mat w \!\sim_\sigma\!\mat w]\,
                    \mathbbm1[\mat w \sim_{\N-\Tgt \sigma}\!\mat w]\,
                    p_\sigma(\Tgt \sigma \mat w' | \Src \sigma \mat w);
                \\&\quad
                q_0 = \mathrm{Unif}(\V\N); \quad
                b(1|\mat w) \propto \exp(\???) &
            \smash{\bigg)}
        \end{align*}
    \end{constr}
    
    This is different from the confidence automaton 
    \begin{constr}
        \begin{align*}
            \mathit{CA}_{\dg M} :=
            \smash{\bigg(}&
            % Q := \sum_{a : \Ar} \V(\Src a, \Tgt a);\quad
            Q := \Delta\V\N;\quad
                \Sigma := \Ar;  \quad
                \tau_{\sigma}(\mu) = \int_{0}^{1} \hat\nabla \bbr{\dg M}(\mu) 
                \\&\quad
                q_0 = \mathrm{Unif}(\V\N); \quad
                b(\top|\mu) \propto \exp(- \log ) &
            \smash{\bigg)}
        \end{align*}
    \end{constr}
    
    The automaton semantics of a PDG are given by 
    
    \begin{claim}
        \begin{align*}
            \bbr{ {\dg M}_{2}^{\mathit{PA}} }_{\thickD + \gamma \H} &: \Delta Q \to \Delta Q \\
                = \mu &\mapsto 
        \end{align*}
    \end{claim}
\end{document}
