
\documentclass[the-pdg-manual.tex]{subfiles}
\begin{document}
\section{Probabilistic Graphical Models}

\label{sec:other-graphical-models} 
%oli21: no need to mention DNs here, right?
%joe19: right
%We now relate 
We start by relating
PDGs to two of the most popular graphical models: BNs and factor
graphs. PDGs are strictly more general than BNs, and can emulate factor graphs
for a particular value of $\gamma$. 
%oli8: Unecessary, I'll get to it. Would require updating anyway. Deleted.
    % More concretely, we will see
    %     that we can get the standard free energy of factor graphs, and
    %     more generally, of the full exponential family that it
    %     corresponds to, by setting each $\alpha$ to zero, and removing
    %     an implicit  `local regularization' term in $\mathcal U$. 
%	; for others, consult \Cref{fig:model-transformations} and its explanation in \Cref{sec:many-relations-graphical-models}.
%joe10: can cut this to save space if needed to save space
%oli12: done
%joe11: reinstated, since we have the space, but I don't mind cutting
%it again.  But it actually doesn't seem to save space in practice
%joe17: OK; I think we have room.  But we shoudl definitely reinstate
%the section numbers I'm pretty sure they're allowed
%\vfull{
\subsection{Bayesian Networks} 
%}%\end{vfull}
\label{sec:bn-convert}
    A (quantitative) Bayesian Network $(G, f)$ consists of two parts: its qualitative graphical structure $G$, indicating a set of
    variables and
    conditional independencies, and its quantitative data $f$, an assignment of 
    a cpd $p_i(X_i \mid \Pa(X_i))$ to each variable $X_i$.
    %
    %joe4: this may be true, but why bother saying it?
    %oli5: I guess I really have a terrible model of what you view as
    %worth saying. This might not be the most efficient use of space, but
    %I think provides useful historical background, explains why the
    %problem hasn't been solved yet, provides a great deal more intuititon
    %about how this solution works than the proof. It also tells a story.  
    %joe5: My model is ``have a clear conception of the story and
    %ruthlessly restructure things so as to bring it out''
            %oli5: I've changed it to vfull, as I understand we're short on space,
    %but I remain confused about why you don't view it as worth
    %saying---especially in contrast to the verbose expansions of
    %sentences you employ when you rewrite my texts, and reitterations of
    %previous points with "as we've said". 
    %joe5: HOw does it fit the story?  We are not telling a story about
    %BNs, but about PDGs.  Even in the full paper, it doesn't belong.
            %oli5: I also think the narrative and reasons for dropping the independences are important for discussing BNs, which have historically had that focus.
    % I've therefore reinstated this paragraph, and promoted the rest of the comment to the full version.
        The first is usually seen as more fundamental
    %oli8: Updated to reflect new understanding of \alpha, though could use further editing
    %	; one can think of the corresponding PDG as keeping only the second. 
    %joe7
    %	, but equation \eqref{eq:uniqdist}, specifically the limit as
    %        $\gamma \to 0$, can be thought of as elevating the
    %        quantitative data above the independence assumptions.  
            but the third semantics ($\bbr{\dg M}^*$) can be
            can be understood as viewing the quantitative information as
    %joe7*: But this begs the question.  Why are we doing this?
    %oli9: Because with BN's it's impossible to break the independece assumptions.  Worse, there's no way to sepcify constaints ---even constraints consistent with the independence assumptions--- unless they lie on one of the edges of the graph. 
    % In a BN, independence is primary. But I think it's really easy to argue that those independencies aught to take a back seat to the data. This way you can do both at once.
            more important that the qualitative independence assumptions.  
    %oli5: I can do without this sentence though:
    %	Fortunately, there is an intuitive way to recover the
    %independencies by optimizing for a natural information-theoretic
    %quantity: the extra information (\Cref{sec:extra}). 
    %joe5*: Kept the sentence above. Cut the rest.  I don't even know what
    %contravariant means in this context.
    %oli6: I find this a useful explanation of why keeping track of 
    % independences messes up modularity. I explain what I mean by contravarient
    % immediately below. It can be cut for space but I'm marking it for
    % the full paper 
    %joe6: No!  This is a distraction.  We are not writing a paper on BNs,
    %or what is the right way to interpret things to get modularity. Focus
    %on the story!
    This is the more desirable option if one cares about
    modularity, because independencies and cpds specify a distribution in
    contravariant ways: a subset of
    the graph, and hence of the cpds, results in a \emph{super-set} of
    the independencies, and vice versa. It is in part for this reason
    that a BN does not say monotonically less when edges are deleted, or
    more when edges are added. 
   

    %joe7: I don't understand the net paragraph, so I'm just cutting it.

    To do without the independence assumptions, one might hope
            that maximizing entropy would recover the conditional
            independencies, as maximizing entropy tends to make things as
            independent as possible given the constraints --- but
            maximizing entropy alone is not enough
            (\Cref{ex:counterexample}).

        In response, some \cite{williamson2001foundations}\cite{holmes2001independence} have added alternate constraints of a causal flavor, which are perhaps smaller and more palatable than the full set of conditional independencies.  Williamson, for instance, introduces what he calls the \emph{principle of causal irrelevance}--- that extending a BN with variables $\{C_i\}$ with children $\{D_j\}$ where no $C_i$ depends on a $D_j$, restricts to the same distribution as the original.  However, these constraints are also overkill: by merely maximizing entropy one can already get the BN distribution for rooted trees, disconnected graphs, and even graphs that have nodes with multiple incoming edges, so long as every row in each such target node's cpd has the same entropy---none of which are reflected as a weakening of assumptions in a Williamson's principle of causal irrelevance.
        
    
    %oli5: I've rewritten this more dramatically and pulled it out of the comment, as a transition
    %joe5
    %The key insight%
    %joe8: cut this too.  it's no longer consistent with what we do (and I
    %never undrstaood it anyway).
    The key insight is that we can recover the BN distribution if we control for
    %joe5: sorry; I don't understand this.  What does it mean to control
    %for the counterfactual nature of the cpd?  For that matter, what's
    %counterfactult about it?
    %oli6: This is the motivation for the extra information. We
    %acknowledge that a cpd 
    % results in a distribution at its target, whose entropy depends on
    % the distribution at its 
    % source. Therefore, the cpd results in a different constraint,
    % depending on what the distirbution 
    % at the source is (the cpd counterfactually contains information
    % about the distribution at $Y$, even if the distribution at X were to
    % be completely different). In minimizing the information we know
    % about the distribution, we have to control for the fact that cpds
    % have this property, making them very unlike the standard constraints
    % that are used (e.g., for exponential families). The resulting
            % correction gives us the extra information.
    %joe6*: If you want to keep this, you need to slow down.  Look at a BN
    %of the form X -> Y and point out that the cpd for X gives us the
    %actual distribution on X, but the cpd lets us detemine the marginal
    %probabilty of Y for all distributions on X.  In that sense, its
    %giving us counterfactual information.  As I said in an earlier joe6*
    %comment, you probably should say this earlier.
            %Then exlain (slowly) how your definition does account for it.
    %This will be a mysterious definition to many readers, so you have to
    %motivate it much better.         
    the counterfactual nature of the cpd as a constraint, as we
    do in \Cref{sec:scoring-semantics}, allowing us to recover the
    independencies without assuming them.
    %}        
    Nevertheless, as we shall show, our third semantics still allows us to
    recover the independencies.

\Cref{constr:hyperedge-reducton} can be generalized to convert arbitrary Bayesian Networks into PDGs.
%oli19: don't need to wrap in defn
% \begin{defn}[BN to PDG]
%oli21:
% Given a BN $\mathcal B$, and a positive number $\beta_X$ for
%         each variable $X$ of $\cal B$,
%joe19
%Given a BN $\mathcal B$, and a positive confidence $\beta_X$ for
Given a BN $\mathcal B$ and a positive confidence $\beta_X$ for
the cpd of each variable $X$ of $\cal B$,
let $\PDGof{\mathcal B, \beta}$
be the PDG comprising the cpds of $\cal B$
in this way. %we defer the straightforward formal details to the appendix. 

    
    % \begin{theorem}[restate=thmbnsRpdgs]\label{thm:bns-are-pdgs}
    % \begin{restatable}{theorem}{bnsRpdgs}\label{thm:bns-are-pdgs}
\begthm{theorem}{thm:bns-are-pdgs}
      If $\cal B$ is a Bayesian network
%oli15:
    % and specifies the distribution $\Pr_{\cal B}$, then 
%joe14: for consistency
%          and $\Pr_{\cal B}$ is the distribution it specifies, then
          and $\Pr_{\cal B}$ is the distribution it specifies, then
          %oli11: insert \betas, and reword because one semantics distribution is provably unique.
    % for all $\gamma > 0$,
    % we have $\bbr{\PDGof{{\mathcal B}}}_\gamma^* =
    % \bbr{\PDGof{{\mathcal B}}}^*$.  Moreover, the unique probability distribution in
    % $\bbr{\PDGof{{\mathcal B}}}^*$ is the distribution specified by
    %             ${\mathcal B}$.
%joe14: not all vectors \beta
%oli16*: Currently, \beta > 0 by definition; there's no need for this
% if we keep our current definition. Therefore, reverted for now.
%joe15*: You missed my point.  Do you want to rquire that all entries
%are strictly positive? If so, you have to say it.  This does not
%follow from \beta -> 0. 
%oli17: Oh, I see what you're saying. I just assuemed "all vectors beta" was
% valid shorthand for "all valid vectors of positive numbers, like in our
% definition"---but you're probably right to state this
% explicitly. I'm reinstating 
% your version.
%it does in one of our definitions
          % Benefits of mandating \beta > 0:
%  - it means a PDG always represents a unique distribtion as \gamma -> 0
%  - we don't have to keep mentioning this condition (and we're short on space).
% Benefits of allowing \beta=0
%  - Can articulate an edge qualitatively without supplying a cpt (ideally
%       we would articulate how this works better before doing this. Ideally,
%       we could get to the point where people buy the qualitative picture in 
%       well enough that they understand the diagrams and feel like they know
%       what a qualitative edge does)
%  - [works better with \alpha]: \alpha = 0 makes a lot of sense, and so 
%       the symmetry probably worth it when we include \alpha.
        % for all $\gamma > 0$ and all vectors $\beta$,
%oli16: reinstated the above and commented out the below:
        for all $\gamma > 0$ and all vectors $\beta$ such
        that $\beta_L > 0$ for all edges $L$,
%joe15: I will not change this, but if you don't change it back to my
%version, then you have to weaken the requirement  \beta_X > 0 in
%Definition 4.1.
        %joe14
    %    $\bbr{\PDGof{\mathcal B, \beta}}_\gamma^* = \{ \Pr_{\cal B}\}$.
        $\bbr{\PDGof{\mathcal B, \beta}}_\gamma^* = \{ \Pr_{\cal B}\}$, 
    %oli15: added
    %joe14: It's not ``in particular'', since it's not a special case
    %of the above, although it does follow from the above.
    %oli16: does it not make sense to you to say "these two functions
    %    are the same, so in particular, their minima are the same?" 
%joe15: what you wrote below is certainly a logical consequence of
%what you wrote above, but it's not a special case.   I would not say
%(in English) ``in particular, their minima are the same''.   I would
%say ``and so their minima are the same''.  We're arguing about
%English here, not mathematics.
    %oli16: whether or not this is an obvious special case might depend on your 
    % representation of the function (it's natural to represent a convex
    % function as a taylor expansion around its critical points, for instance).
        %    In particular, $\bbr{\PDGof{\mathcal B, \beta}}^* = \Pr_{\cal B}$.
and thus $\bbr{\PDGof{\mathcal B, \beta}}^* = \Pr_{\cal B}$.    
\end{theorem}
%oli24: added discussion 
\Cref{thm:bns-are-pdgs} is quite robust to parameter choices: it holds for every
weight vector $\beta$ and all $\gamma > 0$. However, it does lean heavily on
our assumption that $\alpha = \mathbf 1$, making it our only result
that does not have a natural analog for general $\alpha$.

This is true for PDGs which are structurally just subsets of BNs, where every node has at most one incoming edge (making the BN acyclic, when viewed as a directed hyper-graph). In such a structure, every cpd can be simultaneously attained perfectly regardless of how little you are attached to them ($\beta$) and the strength of the bias towards uncertainty $(\gamma)$.
However, not all PDGs have the particularly nice structure,
    and these parameters are important when there can be conflict between beliefs. 
     
% In proving \cref{thm:bns-are-pdgs}, 
% we show that $\IDef{\PDGof{\mathcal B}}$ measures the extent to which the independencies
% are violated, which requires $\alpha = \mathbf 1$.

%oli11: For the full paper, once we add restriction & combination, add the following result for conditional BNs.
%joe10*: If we include this, we'll need a *much* better story.  The
%goal is not to overwhelm the reader with theorems, but to tell a
%story.  My guess is that if this belongs anywhere, it belongs in a
%section on  modularity, where we have a more general discussion of
%modularity   This will be an example there.
%joe17*: See my comment above
%oli20: oops, agreed.
\begin{inactive}
    \begin{prop}
    If $\mathcal B_1, \mathcal B_2, \ldots$ are a conditional Bayesian
    networks containing whose sets of variables they condition on are
    pairwise disjoint, then they can be combined into one conditional BN
    $\cal B$, and  PDG union 
        %	\[ \dg M} := \bigoplus_i \mathcal
    \[ {\dg M} := \bigoplus_i \PDGof{\mathcal B_i} \]  
            satisfies
            $\bbr{\dg M} = \bbr{\PDGof{\cal B}}$.
    \end{prop}
\end{inactive}

% d-separation? I don't have a lot to say but it the specialness of the ``colider'' or head-to-head nodes in determining connectedness  is related to the difference in interpretations I think.

\subsection{Factor Graphs} 
\label{sec:factor-graphs}
%oli8: moved all of the original material to the appendix, this section is new.
%joe7
%	Factor graphs \cite{koller2009probabilistic}, make some
%        similar promises to PDGs. They generalize BNs, the barrier to
%        adding observations is extremely low, and their failure to
%        normalize in general may be viewed as a kind of inconsistency
%        in a very similar fashion \cite{wainwright2008graphical}.
%joe18: we may want to cite the original paper on factor graphs here,
%along with (probably more accessible) KF09; I on'd tfeel strongly
%about this though.
%oli21: added refernece; is easy to remove. I've also subsituted
% the wainwright reference because it's much more focused on factor graphs
% and the authors more strongly take the "factor graphs generalize BNs" position.
% Factor graphs \cite{KF09},
Factor graphs 
%oli22: The original paper is actually really persuasive and arguably
%a better introduction
%joe20: you need to send me an updated bib file
%\cite{wainwright2008graphical,kschischang2001sumproduct},
\cite{kschischang2001sumproduct},
%	like PDGs, generalize BNs and have a low barrier to adding observations.
%joe19: we never discuss adding observations to a factor graph
%	like PDGs, generalize BNs.
%oli22: fair. On the other hand, they are clearly less strict and
% anyone who knows about them will identify that they solve the "we can't 
% legally add this information" problem.
%oli22: also, I want to soften this b/c  while they can represent the 
% distribution of any quantitative BN, they don't capture the 
% indepencencies of a qualitative BN  (though directed factor graphs do) and
% do not contain the counterfactual information that a BN does. 
% This may seem like a technicality, but it is actually a core part of the
% story: the factor graph representation does not exactly capture the BN; its
% sensitivity to future additions &  depenednce on \gamma,\beta are features
% that the BN does not have.
%Rewording:
%joe20: why is it just ``claim''.  They clearly do, in a precise
%sense.  In what sense are they more modular?  I've never seen the
%modularity of factor graphs discussed; rather, the claim is that
%having the factors makes this more efficent computationally.  Why
%make statements that may be controversial or unclar, that we don't
%make se of anywhere?  This is not a paper on factor graphs.
%like PDGs, claim to generalize BNs, and are also much more modular.
like PDGs, generalize BNs.
%oli22: I still would defend this, but let's not go there right now. Most
%people would agree with
%*%
%joe20: maybe, but what about those that don't?  why introduce this
%when we don't need it.
% moreover, their failure to
%         normalize in general may be viewed as a way of representing
%         some inconsistency.
%oli22: introduce one more acronym
% In this section, we consider the relationship between factor graphs and PDGs.
In this section, we consider the relationship between factor graphs (FGs) and PDGs.
\begin{defn}
%oli22: if "set" is preferable to "collection" for indexed sets when
% we define PDGs, it's certainly prefereable here also. 
 % A \emph{factor graph} $\Phi$ is a collection of random variables
 A \emph{factor graph} $\Phi$ is a set of random variables
        $\mathcal X = \{X_i\}$ and \emph{factors}
       $\{\phi_J\colon \V(X_J) \to \mathbb R_{\geq0}\}_{J \in
%joe19
%\mathcal J }$ %where each $J \in \mathcal J$ is associated
\mathcal J }$,
%joe19
%where each $X_J \subseteq \mathcal X$.
%more precisely, each factor $\phi_J$ is associated with a subset
where $X_J \subseteq \mathcal X$.  
More precisely, each factor $\phi_J$ is associated with a subset
$X_J\subseteq \mathcal{X}$ of variables, and maps
joint settings of $X_J$ to non-negative real numbers.
%
%oli23*:moving this material inside the definition
$\Phi$ specifies a distribution
\[ {\Pr}_{\Phi}(\vec x) = \frac{1}{Z_{\Phi}}
    \prod_{J \in \cal J} \phi_J(\vec x_J), \]
%joe21
%where $\vec{x}$ is a joint setting on all of the variables,
where $\vec{x}$ is a joint setting of all of the variables,
 $\vec{x}_J$ is the restriction of $\vec{x}$ to only the
 variables $X_J$, and $Z_{\Phi}$ is the constant required to
 normalize the distribution.  
\end{defn}

%oli23*: moved all of this material below
% We take a \emph{weighted factor graph} $\Psi$ to be a pair $(\Phi,\theta)$ consisting of a factor graph $\Phi$  together with a vector of non-negative weights  $\{ \theta_J \}_{J \in \mathcal J}$.
% $\Psi$ specifies a distribution 
%oli23: refactor, using \Phi here, no weights, \Psi later.
% \[ {\Pr}_{\Psi}(\vec x) = \frac{1}{Z_{\Psi}}
% \[ {\Pr}_{(\Phi,\theta)}(\vec x) = \frac{1}{Z_{\Phi,\theta}}
% 		\prod_{J \in \cal J} \phi_J(\vec x_J)^{\theta_J} \]


%oli23: eliminating this is a benefit of the refactor
    % A factor graph $\Phi$ defines a distribution $\Pr_\Phi$ and scoring function 
    % $\GFE_{\Phi}$ by implicitly taking every $\theta_J = 1$.

%joe21: this isn't a good story
%The cpds of a PDG straightforwardly constitute the data of a factor graph.
%oli24: I think it's way more persuasive if we just interpret the data
% that's already there in its natural way, rather than actively "associating".
% Besides, the symmetry is not there. I'm doing one translation to 
% show that PDG semantics are not just multiply-the-cpts likea  factor graph
% would do, and the other translation to show we can emulate them; I want one
% to succeed and the other to fail. The one I want to fail definiitely needs
% to be set up as naturally as possible.
    % We can associate with each PDG a unique factor graph and vice versa.
    % The map from PDGs to factor graphs takes the cpds of the PDG to be the
    % facotors of the factor graph.  
%joe22: I changed it because I don't understand the phrase ``data of a
%factor graph''.  Databases have data; factor graphs don't.  I think
%you're using idiosyncratic terminology.  This isn't a great disaster
%-- I think the intent is clear -- but I don't like the terminology,
%although I didn't change it
%oli25: I started using the term "data" because you wrote something that way,
% but It's certainly not my favorite term and I'm open to swapping it out.
% More broadly though, I hope you can see why the "we can associate one to 
% another" story I don't find satisfying. I'm changing the phrase
% "data of a factor graph".
%oli25
%The cpds of a PDG straightforwardly constitute the  data of a factor graph,
%joe23: this version is OK, as far as I''m concerned 
The cpds of a PDG naturally constitute a collection of factors,
so it natural to wonder how the semantics of a PDG compare to 
simply treating the cpds as factors. To answer this question, we start by making
the translation precise.
%joe21
%\begin{defn}[PDG to factor graph]\label{def:PDG2fg}
\begin{defn}[unweighted PDG to factor graph]\label{def:PDG2fg}
%oli22: first do this for an unweighted PDG.
% If $\dg M$ is a PDG, define   
If $\dg N = (\Gr, \mat p)$ is an unweighted PDG, define   
%oli21: I'm envisioning some disagreement about this notation; putting
%this in a macro to make this smoother
%joe19: I think the notation is OK, but as I said above, we might as
%well use \Psi everywhere.
%oli22: unweighted first clarifies this.
% the associated factor graph $(\Phi, \theta)_{\dg M}$ on the
% the associated WFG $\WFGof{\dg M} = (\Phi,\theta)$ on the 
the associated FG $\FGof{\dg N}$ on the 
%oli22: Note: here's another place where we refer to variables by (N,V). I 
% like making it clear that a variable is determined by both \N and \V, but
% again am open to alternate notation, so long as we keep both \N,\V. Factoring
% this also out into a macro.
    % variables $(\N, \V)$ by
variables $(\N,\V)$ by
%oli22: "factors given by the edges" is not quite accurate; it's the index that
% is given by the edges $\Ed$, and the factors are given by $\bp$. 
    % taking the factors to be given by the edges in $\Ed^{\dg M}$, 
%joe21: I couldn't parse this
%taking $\mathcal J := \Ed^{\dg M}$ to be the set of edges,
taking $\mathcal J$ to be the set to be the set of edges, 
%oli22: also changing "X" to "Z" because of possible name conflict; in our
% presentation X_{--} is already a specific variable. Also slowing down and
% describing the translation more carefully.
% and for an edge $L$ from $X$ to $Y$, taking $\phi_L(x,y)$ to be $(\bp^{\dg M}(x))(y)$,
and for an edge $L$ from $Z$ to $Y$, taking $X_{L} = \{Z,Y\}$, and $\phi_L(z,y)$ to be $(\bp^{\dg M}(z))(y)$.
%oli22: now we do the weighted case by re-using the above. 
    % and taking the weight $\theta_L = \beta_L$.
%joe20
%We extend transformation to one that takes a (weighted) PDG $\dg M =
%oli23*: moving the extension to below.
% We extend this transformation to one that takes a (weighted) PDG $\dg M =
% (\dg N, \beta)$  
% to a WFG $\WFGof{\dg M} := (\FGof{\dg N}, \beta)$ by setting $\theta_L = \beta_L$.
\end{defn}


%oli19: new text, some storytelling
%joe19: on't call it a a trick
%Using essentially the same trick as \cref{constr:hyperedge-reducton},
%oli22: I prefer idea singular.
% Using essentially the same ideas as in \cref{constr:hyperedge-reducton},
%oli24: A transition phrase.
It turns out we can also do the reverse. 
Using essentially the same idea as in \cref{constr:hyperedge-reducton},
we can encode a factor graph as an assertion about the unconditional
probability distribution over the variables associated to each
factor.  

%joe21
%\begin{defn}[factor graph to PDG] \label{def:fg2PDG}
\begin{defn}[factor graph to unweighted PDG] \label{def:fg2PDG}
%oli20: shuffle + add \theta (2 lines)
% If $\Phi=(\{\phi_J\}_{J \in \cal J})$ is a factor graph, then $\PDGof{\Phi}$ is
%oli21: this is fine but using WFG terminology instead:
    % For a factor graph $\Phi=(\{\phi_J\}_{J \in \cal J})$ and 
    % non-negative vector $\theta$ over $\cal J$,  let $\PDGof{\Phi,\theta}$ be
%oli22: unweighted case first.
% For a WFG $\Psi = (\Phi,\theta)$, let $\PDGof{\Psi}$ be
For a FG $\Phi$, let $\UPDGof{\Phi}$ be
%oli22: was extremely tricky to read and edit in its fragle
%run-on-sentence form.  
% split it into the bullets as we discussed.
% the PDG whose variables are the variables in $\Phi$ together with $\sf 1$ and a
% variable $X_J := \prod_{j \in J} X_j$ for every factor $J \in \mathcal J$%
% , whose edges consist of projections $X_J \tto X_j$ for each $X_j \in X_J$ and
% unconditional joint distributions ${\mathsf 1} \to X_J$ with
% associated cpd $\bp[J]$ equal to the joint distribution on $X_J$ obtained by
% %normalizing $\phi_J$; 
the unweighted PDG consisting of
\begin{itemize}
    \item the variables in $\Phi$ together
   with $\var 1$ and a variable $X_{\!J} := \prod_{j \in J} X_j$ for every factor $J \in \mathcal J$%
   , and
   \item edges ${\var 1} \!\to\! X_{\!J}$ for each $J$ and $X_{\!J} \!\tto\! X_j$ for each $X_j \in \mat X_J$,
\end{itemize}
where the edges $ X_{\!J} \!\tto\! X_j$ are associated with the appropriate projections, and each ${\var 1} \!\to\! X_{\!J}$ is associated with the unconditional joint distribution on $X_J$ obtained by normalizing $\phi_J$.
%joe19*: Are you claiming that you get the same
%result with \alpha_L = 1 and \alpha_L = \theta?  That's strange (and
%inconsistent with what you wrote later
%oli22: They don't give the same result and I didn't intend to suggest
% that they did. Hopefully this presentation is a lot clearer.
%joe19: no \alphas here.
%% finally, let \valpha{$\alpha_L = $}$\beta_L:= \theta_L$.
%oli22: why remove the colon? Others (and programming languages)
% have told me to go out of my way to distinguish between construction and
% assertion.  Also, I thought you wanted me to put an \alpha 
% in the translation to a factor graph? 
% finally, let $\beta_L = \theta_L$. 
The process is illustrated in \cref{fig:fg2PDG}.
%oli22: now the weighted case.
%oli23: moved below.
\end{defn}



%joe18
%\begin{figure*}
\begin{figure*}[htb]
    \centering
    \hfill
    \ifprecompiledfigs
\raisebox{-0.5\height}{\includegraphics{figure-pdfs/smoking-FG}}
% \raisebox{-0.5\height}{\includegraphics{smoking-FG}}
    \else
    \begin{tikzpicture}[center base, xscale=1.4,
        fgnode/.append style={minimum width=2.7em, inner sep=0.3em}]
        \node[factor] (prior) at (1.65,-1) {};
        \node[factor] (center) at (3.75, 0.1){};
        
        \node[fgnode] (PS) at (1.65,0.5) {$\mathit{PS}$};
        \node[fgnode] (S) at (3.1, 0.8) {$S$};
        \node[fgnode] (SH) at (3.0, -0.8) {$\mathit{SH}$};
        \node[fgnode] (C) at (4.8,0.5) {$C$};
        
        \draw[thick] (prior) -- (PS);
        \draw[thick] (PS) --node[factor](pss){} (S);
        \draw[thick] (PS) --node[factor](pssh){} (SH);
        \draw[thick] (S) -- (center) (center) -- (SH) (C) -- (center);

%		\node[dpadded, fill=blue] (1) at (2.5,-2) {1};
%					
%		\draw[blue!50, arr] (1) -- (prior);
%		\draw[blue!50, arr] (1) -- (center);
%		\draw[blue!50, arr] (1) -- (pss);
%		\draw[blue!50, arr] (1) -- (pssh);
        
        %oli24:
        % \node[fgnode, fill opacity=0.02,dashed] (T) at (4.8, -1.3) {$T$};
        \node[fgnode] (T) at (4.8, -1.3) {$T$};
        \draw[thick] (T) -- node[factor]{}  (C);	
        % \node[factor, draw=black, pattern=north east hatch] at (Q){};
    \end{tikzpicture}
    \fi
    %oli22:improving spacing.
        % ~\vrule~
    \hfill\vrule\hfill
        % \end{subfigure}
        % \begin{subfigure}{0.5\linewidth}\centering
    \ifprecompiledfigs
\raisebox{-0.5\height}{\includegraphics{figure-pdfs/smoking-convert}}
% \raisebox{-0.5\height}{\includegraphics{smoking-convert}}
    \else
    \begin{tikzpicture}[center base, xscale=1.6,
        newnode/.style={rectangle, inner sep=5pt, fill=gray!30, rounded corners=3, thick,draw}]
        \node[newnode] (prior) at (1.65,-1) {};
        \node[newnode] (center) at (4.1, 0.25){};
        
        \node[dpadded] (PS) at (1.65,0.5) {$\mathit{PS}$};
        \node[dpadded] (S) at (3.3, 0.8) {$S$};
        \node[dpadded] (SH) at (3.3, -0.6) {$\mathit{SH}$};
        \node[dpadded] (C) at (4.9,0.5) {$C$};
        
        \draw[arr, ->>, shorten <=0pt] (prior) -- (PS);
        \draw[arr, <<->>] (PS) --node[newnode](pss){} (S);
        \draw[arr, <<->>] (PS) --node[newnode](pssh){} (SH);
        \draw[arr, <<-, shorten >=0pt] (S) -- (center); 
        \draw[arr, <<-, shorten >=0pt] (SH)-- (center); 
        \draw[arr, <<-, shorten >=0pt] (C) -- (center);
        
        \node[dpadded, fill=blue] (1) at (2.7,-1.8) {1};
        
        \draw[blue!50, arr] (1) -- (prior);
        \draw[blue!50, arr] (1) to[bend right=30] (center);
        \draw[blue!50, arr] (1) to[bend right = 5] (pss);
        \draw[blue!50, arr] (1) to[bend left = 10] (pssh);

        
        \node[dpadded] (T) at (4.8, -1.7) {$T$};
        \draw[arr, <<->>] (T) -- node[newnode](tc){}  (C);	

        \draw[blue!50, arr] (1) to[bend right = 10] (tc);
    \end{tikzpicture}
    \fi
        % \end{subfigure}
    \hfill~
    \caption{
%oli20: oops garbled. Fixing.
% The conversion from a PDG to a factor graph to factor
% graph, and vice versa, as defined in \Cref{def:fg2PDG}. The
%joe18: still garbled
%Conversion of the PDG in \cref{ex:smoking} a PDG to a factor graph
Conversion of the PDG in \cref{ex:smoking} to a factor graph
according to \cref{def:PDG2fg} (left), and from that factor graph back
to a PDG by \cref{def:fg2PDG} (right). 
%joe17
%blue edges carry the (renormalized) cpds corresponding to the
%joe18: what does ``renormalized'' mean here?  Why did the cpds have
%to be renoormalized
%oli21: because they're now being regarded as unconditional distributions. To
%illustrate: both a cpt X -> Y and an unconditional distribution 1 -> XY are
%matrices. In the first case, each row sums to 1, whereas in the second, the
%whole matrix sums to 1. We're renormalizing the cpd so that they are
%unconditional distributions. I think your original edit introduced
%this ambiguity.
%joe19*: I'm lost.  Why are they now being regarded as unconditional
%distributions?  I'm OK with the current caption.
%oli21: let me try again to write this clearly.
    % In the latter, blue edges are associated with the cpds corresponding to the
    % original factors, each leading to a new node $X_J$ (displayed as a
    % smaller darker rectangle) whose values are joint settings of the
    % variables connected to the factor $J$. 
%
%joe19
%In the latter, for each $J$ we intorduce a new node $X_J$ (displayed as a
%smaller darker rectangle) whose values are joint settings of the
In the latter, for each $J$ we introduce a new variable $X_J$ (displayed as a
smaller darker rectangle), whose values are joint settings of the
variables connected it, and also an edge $1 \to X_J$ 
%joe19
%(blue)
(shown in blue),
%oli21: AAAI only begrudgingly accepts color I will make sure all the 
% figures, etc. look good in black and white later.
%FIXME
%joe19
%to which we associate with the unconditional
to which we associate the unconditional 
distribution given by normalizing $\phi_J$.
} 
    \label{fig:fg2PDG}
\end{figure*}


%joe1: rewrote
%Surprisingly, despite garbling the structure (see
%\Cref{fig:fg2PDG,fig:fg-intro-examples}), when we fix $\gamma=1$, the
%two operations preserve most of their semantics.
PDGs are directed graphs, while factors graphs are undirected. The
map from PDGs to factor graphs thus loses some important structure.
As shown in
%joe17: I get problems when I latex this.  It says ``As shown in
%Figures 8 and 9 in Figure 9''. Moreover, the actual figure is Figure 4.
%oli20: another problem with comments... should be fixed now. 
\Cref{fig:fg2PDG,fig:fg-intro-examples},
%joe11
%the mappings can change the graphical structure signfiicantly.
%oli24: actually it's just the second one.
% the mappings can change the graphical structure significantly.
this mapping can change the graphical structure significantly.
%oli12: no \alphas, so we have something slightly different is true.
% Nevertheless, if we take $\gamma=1$,
%oli22: As you say in a %joe18* below, let's just state the theorems,
% and then describe useful corolaries, instead of building up to them with
% the theorem statements below, which are evidently quite confusing. 
    % Nevertheless, in the case where every weight is the same,
Nevertheless,
%oli23: not quite true; removing together with the theorem below.
% if we start with an unweighted factor graph, then
% applying the two conversions take us back to the same factor graph, so
% each is the inverse of the other.  Moreover, 
%oli23: we can substantially strengthen this claim now.
% in the case where all the
% weights are the same, then
%oli24: added emphasis & cqualifier
% both conversions preserve the semantics.
%joe22
%\emph{both} conversions preserve the $\gamma=1$ semantics.
%oli25: this is related to the only other major comment in this round of
% edits, so I'll defer discussion until then, but briefly: I think by saying
% 'in the special case' you give the result is mroe restricted than it is;
% this feels off to me for the same reason that 
% "you can see the red color of a cherry in the special case in which you are
% looking at it" seems wrong. A PDG has semantics for ALL gammma, and this is
% a particular one. Rewording differently.
% both conversions preserve the semantics in the special case that $\gamma=1$. 
%joe23: Well, I think it is a special case, but Im OK with your
%wording.  Just correcting a typo; you should probably do a spellcheck
%both conversions preserve the semantics correpsponding to $\gamma=1$.
both conversions preserve the semantics corresponding to $\gamma=1$.
%joe21
% both conversions preserve the semantics, if we associate the
% unweighted PDG $\dg N$ with the (weighted) PDG $(\dg N,{\bf 1})$
% (i.e., we take $\beta$ to be the constant function {\bf 1}).
%oli24: I find this comes off as way less ad-hoc if we don't
% make this definition last-minute. I'm introducing this convention at the
% beginning, along with our convention about \alpha.


    \begthm{prop}{prop:fg-pdg-lossless}
        $\Phi \circ \PDGof{} = \mathrm{Id}_{\text{FG}}$. That
        is, for all factor graphs $\Phi$, we have
        $\Pr_{\Phi(\PDGof{F})} = \Pr_F$.
%joe20*:
% Given an unweighted FG $\Phi$, $\Phi_{\UPDGof{\Phi}} = \Phi$ 
\end{prop}


%oli23*: here are the unweighted analogs.
\begthm{theorem}{thm:fg-is-pdg}
%oli24: I placed the comment in the beginning; this is really not as ad-hoc
% as this notation makes it seem.
%joe22* You have to remind the reader of this convention!  It's not OK
%to use it pages after you've defined 3 pages back (which the reader
%probably missed) it without a reminder.  Be nice to the reader!
%joe23*: I'm willing to live with your notation, but you *must* remind
%the rader of it.  (I would actulaly put the definition here, rather
%than 3 pages back, since I don't belive you use it except for here.
%This is an instance of a general principal: define a notion when you
%need it.  I see no class of reader for which there's a benefit it in
%defining it on page 3 and using it for the first time on p. 6.
%oli25*: I believe that which choice is better depends on what kind of
%reader.  There are at least two classes of reader for which I think
%my version is substantially better: (1) those who are skimming and
%not tracing definitions carefully (because 
% it's simpler and less confusing than the version with the 1, which requires
% a little more notation tracing about what the 1 is (is it \alpha?\beta? What 
% does that do exactly again?), (2) those who are familiar with factor
% graphs and  
% instintcively take unweighted = weights all 1.
% I believe also that there are readers who would appreciate the ${\bf 1}$, but
% I also have a strong aethetic preference to leaving it out. I think putting the
% \bf 1 inside makes it look more complicated than it is, and that it depends on
% more parameters than it does. The detail is technically correct and admittedly
% makes it a more careful presentation in some ways, but you have vehemently 
% objected to other ways in which I have wanted to do things more carefully
% (e.g., adding my type annotations to definitions).  Again, I think making
% a big deal of this is putting emphasis in the wrong place.
$\Pr_{\Phi} = \bbr{\UPDGof{\Phi}}_{1}^*\;$ for all factor graphs $\Phi$.
%joe21
% $\Pr_{\Phi} = \bbr{(\UPDGof{\Phi},{\bf 1})}_{1}^*\;$ for all factor graphs $\Phi$.
\end{theorem}
\begthm{theorem}{thm:pdg-is-fg}
%oli24: and again
$\bbr{\dg N}_{1}^* = \Pr_{\FGof{\dg N}}\;$ for all unweighted
%joe21*
% $\bbr{(\dg N, {\bf 1})}_{1}^* = \Pr_{\FGof{\dg N}}\;$ for all unweighted
    PDGs $\dg N$.  
\end{theorem}
%oli23*: added important discussion.
The correspondence hinges on the fact that we take $\gamma=1$, so that $\Inc$ and
$\IDef{}$ are weighted equally.
%oli24: important discussion.
%joe22*: I find this discussion problematic, since we haven't
%discussed where teh choice of \gamma is coming from. It's not at all clear
%to me that this limitation is unproblematic.  We've solved a problem
%by giving the modeler the freedom to choose k without explaining why
%a particular choice is OK.  I think that the paper is worse off by
%this addition; you're just reminding the reader of our unmotivated
%choices (which I viewed as technical means to get a result, rather
%than natural modeling choices).  
%oli25*: You may view it as a technical means to get the paper to work out,
% but I definitely do not view it that way.
%joe23*: I understand that, but we don't exlain your point of view in
%the paper.   We have given no basis on which to claim something is ``not
%problematic''.   Saying it is will raise flags, so is a net negative.
%Similarly for the claim that we can't get a factor graph to
%``replace'' a PDG.  If a user is perfectly happy with a
%representation using a factor graph, why should he/she want to
%replace it by a PDG?  There is perhaps a point to be made here, but
%it needs to be written in a way that seems reasonable to a reader who
%hasn't read your mind and does not necessarily agree with our viewpoint.
%Moreover, the way the
% paper is currently 
% written, I think that \gamma is motivated fairly well (it's a
% trade-off between 
% a quantiative (by which I mean, "dependent on the cpds") and qualitative (by 
% which I mean, "dependent only on the graph structure") term, and readers
% with some imagination will have some ideas for how to use this. I
% certainly do. 
% In any case, here we only say that it indexes the family of a semantics, whch
% is just another technical fact, and I thiknk drawing attention to it strengthens
% our point a lot.
Because the user of a PDG gets to choose $\gamma$, the fact that the 
translation from WFGs to PDGs holds only for $\gamma=k$ is not problematic; 
the user can simply set $\gamma$ so as to view the original factor graph.
However, in translating back, we lose this ability, making it very difficult for
a factor graph replace a PDG.



%joe21
%What about weighted PDGs, of the form $(\Gr, \mat p, \beta)$?
%Factor graphs, too, have a standard notion of weightedness, 
%but so long as we stick with our convention of setting every $\alpha_L = 1$, 
%we cannot say much about them.
What about weighted PDGs $(\Gr, \mat p, \beta)$ where $\beta \ne {\bf 1}$?
There is also a standard notion of weighted factor graph,
but as long as we stick with our convention of taking  $\alpha = {\bf 1}$, 
we cannot relate them to weighted PDGs.  
%oli24: squeezing some more space
% As we show in the next section,
As we are about to see,
once we drop this convention, we can do much more.

% \subsection{Bayesian Networks}
% \subsection{Factor Graphs}
\subsection{Factored Exponential Families}


A \emph{weighted factor graph (WFG)} $\Psi$ is a pair
$(\Phi,\theta)$ consisting of a factor graph $\Phi$ 
together with a vector of non-negative weights
$\{ \theta_J \}_{J \in \mathcal J}$.
$\Psi$ specifies a canonical scoring function 
\begin{equation}
\GFE_{\Psi}(\mu)
%   \GFE_{(\Phi,\theta)}(\mu)
     := \!\Ex_{\vec x\sim\mu}\left[  \sum_{J \in
           \cal J} \theta_J \log\frac1{\phi_J(\vec
               x_J)}\right] - \H(\mu)  , 
               \label{eqn:free-energy}
\end{equation}
%joe20
%which $\Pr_{(\Phi,\theta)}$ minimizes, called the \emph{variational
%oli23:
% which $\Pr_{\Psi}$ minimizes,
called the \emph{variational
Gibbs free energy} \cite{mezard2009information}. 
%oli23:
$\GFE_{\Psi}$ is uniquely minimized by the distribution
${\Pr}_{\Psi}(\vec x) = \frac{1}{Z_{\Psi}}
    \prod_{J \in \cal J} \phi_J(\vec x_J)$, 
which matches the unweighted case when every $\theta_J = 1$.
The mapping $\theta \mapsto \Pr_{(\Phi,\theta)}$ is known as 
%joe20
%$\Phi$'s exponential family and is a central tool in the analysis
$\Phi$'s \emph{exponential family} and is a central tool in the analysis  
and development of many algorithms for graphical models \cite{wainwright2008graphical}.

PDGs can in fact capture the full exponential family of a factor graph, but only
%oli24
by allowing values of $\alpha$ other than ${\bf 1}$. In this case, the
only definition  
% by allowing values of $\alpha$ other than 1. In this case, the only definition 
that requires alteration is $\IDef{}$, which now depends on the \emph{weighted multigraph}
$(\Gr^{\dg M}, \alpha^{\dg M})$, and is given by
\begin{equation}
%oli24: Let's just define it for M;
    % \IDef{G}(\mu) := \sum_{\ed LXY \in \Ed} \alpha_L \H_\mu(Y\mid X) - \H(\mu). 
    \IDef{\dg M}(\mu) := \sum_{\ed LXY \in \Ed} \alpha_L \H_\mu(Y\mid X) - \H(\mu). 
    \label{eqn:alt-extra2}
\end{equation}
%joe21: rewrote.  The reader won't know what (b) is
%In ths case, each specification in~\ref{item:localinfo} may be
%weighted by differently, so that some edges are more qualitatively
%certain than others, and correspondingly it may be more or less
%important to describe them properly.
Thus, the conditional entropy $\H_\mu(Y\mid X)$ associated with the
edge $\ed LXY$ is multiplied by the weight $\alpha_L$ of that edge.

%oli23: taken from your text.
%oli24: Softening slightly
% The key benefit of using $\alpha$ is that we can
One key benefit of using $\alpha$ is that we can
capture arbitrary WFGs, not just ones with a constant weight
vector.    All we have to do is to ensure that in our translation from
factor graphs to PDGs, the ratio $\alpha_L/\beta_L$ is a
constant.  (Of course, if we allow arbitrary weights, we cannot hope
to do this if $\alpha_L = 1$ for all edges $L$.)  
%oli23: new
%joe21
%We therefore define a family of translations.
We therefore define a family of translations, parameterized by the
ratio of $\alpha_L$ to $\beta_L$.
\begin{defn}[WFG to PDG]\label{def:wfg2pdg}
Given a WFG
%joe20*: going back to the special case of \alpha=1, AS WE HAD AGREED.   What
%do we do for edges not in \J.   I also don't think that this is quite
%right, sonce you haven't defined \beta for edges not in \J.  I now do
%so, although you should check 
%$\Psi=(\Phi, \theta)$ to a PDG $\PDGof{\Psi} = (\UPDGof{\Phi},\theta, \theta)$ 
$\Psi=(\Phi, \theta)$,
and postive number $k$, 
we define the corresponding PDG $\PDGof{\Psi,k} = (\UPDGof{\Phi},\alpha_{\theta}, \beta_{\theta})$ 
%by taking both $\alpha$ and $\beta$ to be $\theta$.
by taking $\beta_J = k \theta_J$ and $\alpha_J = \theta_J$ for the edge $1  \rightarrow X_J$, and
%oli24:
taking $\beta_L = k$ and $\alpha_L = 1$ for the projections $X_J \!\tto\! X_j$.
% taking $\alpha_L = 1$, $\beta_L = k$ for the projections $X_J \!\tto\! X_j$.
\end{defn}

%joe21*: Cut  I don't like the notion of capturing data.  More
%importantlly,
%We now turn to extend \cref{def:PDG2fg}.  
%Since PDGs have two sets of weights, and WFGs only have one,
%we will not be able to capture all the data.
We now extend Definitions~\ref{def:PDG2fg} and \ref{def:fg2PDG} to
(weighted) PDGs and WFGs.  
%oli24:
%In going translating from PDGs to WFGs, 
In translating a PDG to a WFG, 
%oli24: the mismatch isn't just a problem for this direction
% note that we have somewhat of a mismatch: PDGs have two sets of weights, and WFGs
there will necessarily be some loss of information: PDGs have two sets, while WFGs have 
%oli24:
% only have one, So in our translation, we ingore $\alpha$, and consider
only have one. Here we throw out $\alpha$ and keep $\beta$, 
%oli24: added
%joe22*: I have no clue what this addition means.  I also think it's a
%mistake to highlight another ad hoc choice.  I strongly prefer my wording.
%oli25*: I find it very strange what ad-hoc choices you want to hide
% and which ones you want to expose.
%joe23*: we can debate which ad hoc choices to hide (my own feeling
%is: as many as possible).   What is not open
%to debate is the fact that I didn't understand what you wrote.  To
%the extent that I'm somewhat representative (whch I believe I am),
%this must be rewritten.  
%There are many places where I
% feel there is  
% genuinely something there, and when you don't understand my descriptions, we 
% settle on a description of it which (in my view) far undersells our 
% contribution. Here, this is truly an ad-hoc choice, and I have no problem 
% admitting it (we're claiming there's not even a good choice to be had), 
% and it seems you're trying to just state some facts so it seems our particular
% choice is better than some other choices. It's just a choice, and I think 
% that letting the reader know is useful.
though in its role here as a left inverse of \cref{def:wfg2pdg},
chosing either would suffice. 

%there are more PDGs than WFGs, and so there is not a perfect way to
%translate back. 
%We have chosen here to preserve $\beta$.

\begin{defn}[PDG to WFG]
Given a (weighted) PDG $\dg M =
(\dg N, \beta)$, we take its corresponding WFG to be $\WFGof{\dg M} :=
%joe21
%(\FGof{\dg N}, \beta)$ by setting $\theta_L := \beta_L$.
(\FGof{\dg N}, \beta)$; that is, $\theta_L := \beta_L$ for all edges $L$.
\end{defn}



%joe19*: 
%oli22: blank %joe19*.

%oli22: I'm actually also cutting the current presentation of these theorems
% and re-stating them below, 
% using un-weighted objects + weights, which I think is far less confusing.
% allows the most general theorems to be put right away without special
% conversions and specific discussion about what to do with \alpha.
\begin{inactive}
\begthm{theorem}{thm:pdg-is-fg2}
%joe18
%If $\dg M$ is a PDG, and $\gamma$ is a number such that
If $\dg M$ is a PDG such that for some $\gamma >0$, we have that
    $\beta_L = \alpha_L \gamma$ for all  
    %oli12 line shave
    %edges
    $L$, then
    $\bbr{\dg M}_{\gamma} = \gamma\,\GFE_{\Phi_{\dg M}} $ and
    $\bbr{\dg M}_{\gamma}^* = \{\Pr_{\Phi_{\dg M}} \}$.
        % $\kldiv{\mu}{\Pr_{\Phi(\dg M)}} = \bbr{\dg M}_{1}(\mu)$
        % In particular, $\Pr_{\Phi(\dg M)} = \bbr{\dg M}_*^{\gamma := 1}$
\end{theorem}

%joe19

% \begin{theorem}[restate=thmfgispdg]\label{thm:fg-is-pdg}

%joe18* unless I'm missing something, you need to redo Definition 4.3
%to explain how to add the \alphas.  Will you add them in such a way
%that the ratio of \beta_L to \alpha_L is constant?  If so, we can say
%that because the construction made it a constant, the following result holds.
%If the updated Definition 4.3 allows some flexibility in
%choosing \alpha and \beta, then this would reqire more work.
%oli21: Oops, I forgot to add those two characters; we set \alpha_L = \beta_L. 
% In any case, I think the true things you say above can be sold better:
%joe19: As I said, I don't believe that you want \alpha_L = \beta_L
%oli22**: Since in general we cannot know gamma (because the translation meerely
% supplies the data of the PDG, including alpha and beta, but a user can
% laterquery the semantics for multiple different values of gamma),
% we can't do any better than to set them to be proportional. 1 is a nice
% nice multiplicative constant generally, and also is speial because that's the 
% one we need to capture unweighted factor graphs with unweighted PDGs.

%joe19*: AARGH!  I was hoping that you would discuss the more general
%translation here, where \alpha could be arbitrar.  I don't believe
%that the reslt will hold if \alpha_L = \beta_L.  As you ptoint out,
%we need \beta_L = \alpha_L \gamma not \alpha_L = \beta_L.  So
%although you spent a lot of time doing what I explicitly asked you to
%do, you didn't address my main concerns.  You absolutely did not
%explain how to do  the translation when you have an abtrary weight vector.
%specify how you do the translatin here from factor graphs to PDGs.
%I put in what I think you should have put in.
%oli22: As stated in emails, I believe your conversion is correct but 
% slightly deceptive. I have therefore recycled 
The key benefit in using $\alpha$ is that we can
capture arbitrary WFGs, not just ones with a constant weight
vector.    All we have to do is to ensure that in our translation from
factor graphs to PDGs, the ratio ratio $\alpha_L/\beta_L$ is a
constant.  (Of course, if we allow arbitrary weights, we cannot hope
to do this if $\alpha_L = 1$ for all edges $L$.)

Specifically, given
$0 < \gamma \le 1$ and a WFG $\Psi = (\Phi,\theta)$, we take the PDG
${\dg M}_{\Psi,\gamma}$ to be defined just like the PDG ${\dg M}_\Psi$ of
Definition~\ref{def:fg2PDG}, except that instead of having $\alpha_L =
1$, we have $\alpha_L = \gamma\beta_L$. 


    \begthm{theorem}{thm:fg-is-pdg2}
    %oli20: we got to remove the condition!
%joe19
%For every factor graph $\Phi$, and EVERY vector $\theta$ over $\cal J$
For all WFGs $\Psi = (\Phi,\theta)$ and all $\gamma$ with $0
< \gamma \le 1$
    %, and EVERY $\gamma >0$
    we have that
    % for any joint distribution $\mu$ on $\V(\mathcal X)$, we
            % have $\kldiv{\mu}{\Pr_\Phi} = \bbr{\PDGof{(\Phi)}}_{1}(\mu)$ 
    %joe8: again
    %oli10: adding back in
    %joe9
            %	$\gamma \GFE_\Phi = \bbr{\PDGof{(\Phi)}}_{\gamma} + k$
            %        where $k$ is a constant, and in particular, 
%joe19*
%$\GFE_\Phi = \nicefrac1{\gamma} \bbr{\PDGof{\Phi,\theta}}_{\gamma} + C$  
$\GFE_\Psi = \nicefrac1{\gamma} \bbr{\PDGof{\Psi,\gamma}}_{\gamma} + C$  
for some constant $C$, so
%joe19*
%$\Pr_{\Phi, \theta}$ is the unique element of
%$\bbr{\PDGof{\Phi,\theta}}_{\gamma}^*$.
$\Pr_{\Psi}$ is the unique element of
$\bbr{\PDGof{\Psi,\gamma}}_{\gamma}^*$.  
       % Moreover, $\Pr_{\Phi, \theta} = \bbr{\PDGof{\Phi,\theta}}^*$.
    \end{theorem}
\end{inactive}

%oli22**: big insertion: both of the above theorems + recycled material
% from your discussion. 
%joe20: note that I changed the label
%\begthm{theorem}{thm:pdg-is-fg}We can can now generalize our earlier results 


%oli23: canabalized for the above.
% %joe20*: resinstated
% The key benefit of using $\alpha$ is that we can
% capture arbitrary WFGs, not just ones with a constant weight
% vector.    All we have to do is to ensure that in our translation from
% factor graphs to PDGs, the ratio $\alpha_L/\beta_L$ is a
% constant.  (Of course, if we allow arbitrary weights, we cannot hope
% to do this if $\alpha_L = 1$ for all edges $L$.)  Specifically, given
% $0 < \gamma \le 1$ and a WFG $\Psi = (\Phi,\theta)$, we take the PDG
% ${\dg M}_{\Psi,\gamma}$ to be defined just like the PDG ${\dg M}_\Psi$ of
% Definition~\ref{def:fg2PDG}, except that instead of having $\alpha_L =
% %joe20
% %1$, we have $\alpha_L = \gamma\beta_L$.
% 1$, we have $\alpha_L = \beta_L/\gamma$. 

%oli24: double now. Also this gives me a line.
% We now show that we can now capture the entire exponential family of a factor graph,
We now show that we can capture the entire exponential family of a factor graph,
%oli24: let's boast a little about this; we haven't mentioned it yet
and even its associated free energy, 
%joe21
%but only for the value of $\gamma$ equal to the constant $k$ used in
but only for $\gamma$ equal to the constant $k$ used in
the translation.  


\begin{theorem}\label{thm:wfg-is-pdg}
For all WFGs $\Psi = (\Phi,\theta)$ and all $\gamma > 0$,
we have that
$\GFE_\Psi
%joe20*: using notation defined above
%= \nicefrac1{\gamma} \bbr{(\UPDGof{\Phi}, \theta, \nicefrac{1}{\!\gamma\,}\theta)}_{\gamma}
= \nicefrac1{\gamma} \bbr{{\dg M}_{\Psi,\gamma}}_{\gamma} 
+ C$   
for some constant $C$, so
$\Pr_{\Psi}$ is the unique element of
%joe20*: switching alpha and beta again, and using \beta_\theta
%instead of \theta
%$\bbr{(\UPDGof{\Phi}, \theta, \nicefrac{1}{\!\gamma\,}\theta)}_{\gamma}^*$.  
$\bbr{{\dg M}_{\Psi,\gamma}}_{\gamma}^*$.
\end{theorem}

%joe21
%In particular,for $k\!=\!1$, so that $\theta$ is used for both
In particular, for $k\!=\!1$, so that $\theta$ is used for both the functions
$\alpha$ and $\beta$ of the resulting PDG,
\cref{thm:wfg-is-pdg} strictly generalizes \cref{thm:fg-is-pdg}.
\begin{coro}
    For all weighted factor graphs $(\Phi, \theta)$,
    we have that
    $\Pr_{(\Phi,\theta)} = \bbr{(\UPDGof{\Phi}, \theta,\theta)}_1^*$
\end{coro}

%joe21
%Conversely, so long as the ratio of $\alpha_L$ to $\beta_L$ is constant, the
Conversely, as long as the ratio of $\alpha_L$ to $\beta_L$ is constant, the
reverse translation also preserves semantics.
%oli23: insertaion
% Conversely, we can generalize 
% to the case where the ratio of
% $\alpha_L$ to $\beta_L$ is a constant.
%joe20
%}%oli22 \end{commentout}
\begthm{theorem}{thm:pdg-is-wfg}
For all unweighted PDGs $\dg{N}$ and non-negative vectors $\mat v$
over $\Ed^{\dg N}$, and all $\gamma > 0$, we have that 
%joe20*: you're writing \beta,\alpha; switching it to \alpha \beta,
%and making \beta = v, not \alpha = v.  Does the equality still hold?
%$\bbr{(\dg N, \gamma  \mat v, \mat v)}_{\gamma}
$\bbr{(\dg N, \mat v/\gamma, \mat v)}_{\gamma} 
        = \gamma\,\GFE_{(\Phi_{\dg N}, \mat v)} $ and consequently
%joe20*
%$\bbr{(\dg N, \gamma \mat v, \mat v)}_{\gamma}^*
$\bbr{(\dg N, \mat v/\gamma, \mat v)}_{\gamma}^*
        = \{\Pr_{(\Phi_{\dg N}, \mat v)} \}$. 
\end{theorem}
% With these translations, factor graphs therefore define a particular subclass
% of PDGs in which the weights $\alpha$ and $\beta$ are proportional.

%oli23: removed
% \noindent Note that if $\beta_L = \gamma$ for all edges $L$, then
% 		$\alpha$ is the 
% constant function 1 in Theorem~\ref{thm:pdg-is-fg1}, so
% Theorem~\ref{thm:pdg-is-fg1} is a generalization of
% Theorem~\ref{thm:pdg-is-fg}. 
%oli23: nevermind 
% We interpret the fact that the correspondence only holds when the constant
% $k$ used in the translation equals $\gamma$, as a reflection of the fact
% that there is no way to articulate a 

%oli22: I've rephrased the statement of the theorem in a suggestive 
% (but kind of clunky) way below.
%I think this is kind of cool, and a neat story. Do you buy it?
%joe20: It's not so much that I don't buy it, but that I can't make
%sense of it.  Where did the product of cpts come from?  Surely you
%must have a factor graf in the picture here.  If you can explain it
%to me in a way that I can understand it, we may want to reinstate
%this. 
%misleading 
%Thus, PDGs in which the quantitative and qualitative certainties are
%fused, when evaluated in the semantics corresponding to the particular
%trade-off coresponding to their cofficient of proportionality,
%precisely generate the exponential family of the associated factor
%graph.
%In particular, our semantics regard an unweighted PDG as a product of
%its cpts, when $\gamma = 1$.
%\begin{coro}\label{coro:justafg} 
%% For all $\dg N \!=\! (\Gr,\mat p)$,
%% If $\dg M$ is an unweighted PDG, then 
%	$\displaystyle\bbr{(\Gr, \mat p)}_1^* 
%		= \frac{1}{Z_{\dg M}}\prod_{\ed LXY \in \Ed^{\dg M}} \bp^{\dg M}(Y \mid X)
%	\propto \!\!\prod_{\ed LXY \mathrlap{\in \Ed^{\dg M}}} \bp(Y \mid X)$.
%\end{coro}
%joe20
\begin{inactive}
We have seen that only a subset of PDGs can be faithfully 
represented as WFGs; we now show the other side of the correspondence: any
factor graph be captured by more than one PDG (though again, only for a fixed $\gamma$).
\end{inactive}

%\begthm{theorem}{thm:fg-is-pdg}





%oli22: end big insertion.

%joe18: cut from here.  Once we fix the second theorem, the relevant
%discussion should go before the theorem
%	\cref{thm:pdg-is-fg} still has a retriction, but if we can
%        choose the values of $\alpha$ (or alternatively $\beta$), we
%        can get the result to apply for arbitrary values of $\gamma$
%        and $\beta$ (resp. $\alpha$). \Cref{thm:fg-is-pdg2} is
%        substantially stronger than its counterpart: it shows that
%        PDGs in which the quantitative and qualitative certainties are
%        fused ($\alpha_L = \beta_L$) completely adopt the semantics of
%        factor graphs and their exponential families, for $\gamma =
%        1$.\footnotemark 
%joe18*: This is the point; you have to explain how the tranlsation
%works in the presence of \alpha!
%\footnotetext{The result can be achieved for arbitrary
%	$\gamma \neq 1$, if we are willing to set $\alpha$ based on
%	$\gamma$ in our translation of factor graphs to PDGs.}}

%oli19:
% Justification for both theorems is provided by rewriting
%joe18*: We need to add a proof for this result in teh appendix.
%The truth of \Cref{thm:fg-is-pdg,thm:pdg-is-fg} may be seen
%intuitively by rewriting 
The key step in proving \Cref{thm:wfg-is-pdg,thm:pdg-is-wfg}
(and in the proofs of a number of other results) involves 
rewriting  
$\bbr{\dg M}_\gamma$ as follows: 
\begin{prop}%[restate=prop:nice-score,label=prop:nice-score]% \label{}
% \begin{restatable}{prop}{propnicescore}\label{prop:nice-score}
 Letting $x^{\mat w}$ and $y^{\mat w}$ denote the values of
  $X$ and $Y$, respectively, in $\mat w \in \V(\dg M)$, 
we have 
\begin{equation}\label{eq:semantics-breakdown}
\begin{split}
\bbr{\dg M}(\mu) =  \Ex_{\mat w \sim \mu}\! \Bigg\{
% \bbr{\dg M}(\mu) =  \!\!\!\sum_{\mat w \in \V(\dg M)} \!\!\! \mu(\mat w) \Bigg\{
 \sum_{ X \xrightarrow{\!\!L} Y  }
\bigg[\,
    \color{gray}\overbrace{\color{black}
      \!\beta_L \log \frac{1}{\bp(y^{\mat w} |x^{\mat w})}
    }^{\color{gray}\smash{\mathclap{\text{log likelihood / cross entropy}}}} +
     % \\[-0.5em]
    \color{gray}\underbrace{\color{black} 
({\alpha_L}\gamma - \beta_L ) \log \frac{1}{\mu(y^{\mat w} |x^{\mat w})} 
    }_{\color{gray}\smash{\mathclap{\text{local regularization (if $\beta_L > \gamma$)}}}}\bigg] - \underbrace{\color{black}
\gamma \log \frac{1}{\mu(\mat w)}
    }_{\color{gray}\smash{\mathclap{\text{global
        regularization}}}}\color{black} \Bigg\} .
\end{split}
\end{equation}
\end{prop}
%joe17*: sorry; this isn't claer to me at all.  
%assume that $\beta_L = \theta_L$.  But we did that, so that's not so
%bad (although you should say it). Second, we get, the last term
%becomes \gamma H(\mu), so it only matches the free energy if \gamma =
%-1.  Finally, why should p_L(y^W | x^w) = \phi_L(x_J) (even if we
%assume that \phi = p_L.
%  Part of the problem is that you wronte in
%the w  I think what you mean is the E(\gamme \log(1/\mu)
%= \sum_\oemga \gamma \mu(w)/log(\mu) = = H(\mu).  But then ou're
%still out by a factor of -\gamma.
%oli20: A factor of + \gamma, but point taken. I guess it's standard
%to effectively 
% fix \gamma=1 for factor graph because the ratio of the \theta's to the entropy term is
% the only thing that matters, so one can fix that scale to 1.
%joe18: it may be standard, but not everyone knows it (I didn't).
%  I've rewritten it so it's more correct. 
%	The first and last terms of \eqref{eq:semantics-breakdown} are precisely
%	$\GFE_\Phi$ for $\phi = \bp$.  
%joe18
%For any fixed $\gamma$, the first and last terms
For a fixed $\gamma$, the first and last terms
of \eqref{eq:semantics-breakdown} are equal to a scaled
%joe18: in what sense is it equivalent?  
%(which is essentially equivalent)
%joe18*: I'm confused.  This statement seems technically incorrect to
%me, on  two counts.  First, you would have to
%multiply the first term by \gamma (there is no \gamma factor
%currently in the first term); second, for the last term, the factor
%is -\gamma, not \gamma.  
%joe19*: Oliver, you did not address the joe18* immediately above,
%which points out that, unless I'm missing someting, your claim below
%is incorrect.    Please address this.  
%oli22: [as resolved in email]: the sign on the entropy is correct, and
% this conversion is not "the official one", a confusion which I've attempted
% to further eliminate with my edit below 
version of the free energy, $\gamma\GFE_\Phi$, 
%oli22:
% for $\phi_J = \bp$ and $\theta_J= \nicefrac{\beta_L}{\gamma}$
if we set $\phi_J := \bp$ and $\theta_J := \nicefrac{\beta_L}{\gamma}$.  
%oli19
% Thus, if we assume that $\phi = \bp$ and in addition assume that each
%oli20: edit for flow
% Furthermore, if each $\beta_L = \gamma$, the local regularization term disappears, so we get 
%joe18
%If in addition, each $\beta_L = {$\alpha_L$}\gamma$,
If, in addition, $\beta_L = {\alpha_L}\gamma$ for all
edges $L$, then
the local regularization term disappears, giving us
%oli20:
% an exact correspondence with $\GFE_\Phi$%
the desired correspondence. 

%oli20: paragraph break + signposting
%joe18*: cut this; see below for why
\begin{inactive}
We now explain how the middle term may be viewed as a ``local'' regularization,
and why including such a term is necessary for the semantics to 
%joe18*: I find this statement very confusing.  Why shouldn't we take
%them seriously?  And they *are* assertions about probability, no
%matter how we  take them.  
%oli21*: I agree completely! But that is not the way the semantics work, without
% a local regularization term. I'm trying to motivate the need for the locaity.
%joe18*[continued]: Most importantly of all, I see no way in
%which this example tells me anything about how I shoudl view p.  It
%shows potential probablems with the factor graph (which is what I
%believe it was intended to do).  Some signposting would be useful, but I find
%this completely unhelpful.
%oli21*: I'm not sure exactly what you mean about "how you veiew p", but I assume
% you mean something like, whether to take the cpds seriously or think of them
% as energies. The example clearly shows that without a local regularization
% term, the semantics effectively treats them like energies, and the interpretation
% as a probability distribution is not preserved without the local term. As we
% point out, this local term  
%joe18*[continued]: We had some text here which you seem to
%have cut, rather than commenting it out.  I though the text was fine,
%and I'm reinstating it.  (Also, please don't just cut text.)
%oli21: This text was not removed; it's was just at the end of the document 
% serving a slightly different post-example analysis purpose. 
% To be honest I'm perplexed that so many of these stories, whose delivery I take
% quite seriously and that I edit several times before turning over to you, are
% still entirely lost in transit....
%take our cpds $\mat p$ seroiusly, as assertions about probability%
view our cpds $\mat p$ as assertions about probability%
% (instead of viwing them as each contributing to a badness, as factor
% graphs do) 
. 
%oli20: pulled example from the end of the document.
\end{inactive}

%joe18*: reinstating some material from before, with slight rewriting.
\Cref{eq:semantics-breakdown} also makes it clear that 
taking $\beta_L = {\alpha_L} \gamma$ for all edges $L$ is
essentially necessary to get \Cref{thm:pdg-is-fg,thm:fg-is-pdg}.
%oli21*: ! this is the opposite polarity of what we need to say.
%. The equality HOLDING is what gives us strange behavior.  Rewriting.
%	If this equality does not hold, then we can get some 
%{\cref{prop:consist} does not hold unless .}
%joe19: OK
%oli22: oops I hadn't finished the rewrite in this document, only in 
% DN-and-WFG. I'm updating to what I had there (2 lines).
% Because this equality must hold, we can get
%joe20
%Of course, fixed $\gamma$ precludes taking $\lim_{\gamma\to0}$, so
Of course, fixed $\gamma$ precludes taking the limit as $\gamma$ goes
to 0, so 
\cref{prop:consist} does apply. This is reflected in 
%
some strange
behavior in factor graphs trying to capture the same phenomena as
PDGs, as the following example shows.

\begin{example}\label{ex:overdet}
Consider the PDG $\dg M$ containing just $X$ and $1$, and two edges
$p, q: 1 \to X$.
%joe8:
(Recall that such a PDG can arise if we get different information about the
probability of $X$ from two different 
%oli19:added, otherwise this is a reasonable inference
% but not independent
%joe17: I don't understand.  Such a PDG can arise whether or not the
%sources are independent.  There is no inference going on here!  If
%you want to say something about non-independence, it should go
%somewhere else.
%oli20: I'm not trying to say anything about independence, but we want the
%example to be a clear illustration of what PDGs buy you. If I thought the
%sources were independent, I would be justified in combining 0.7 and 0.7 to get
%0.85. I find the example to be more powerful if we explicitly state that we
%don't know them to be independent, so we cannot make this inference. Thoughts?
%joe18: This is not the place to discuss these issues.  We're just
%trying to make a point about factor graphs having strange behavior.
%The issue of combining .7 and .7, and possibly getting .85 needs
%*much* more discussion; this is not the place to do it.
sources; this is a situation we
certainly want to be able to capture!)
Consider the simplest situation, where $p$ and $q$ are both associated
with the same distribution on $X$%
%oli20: added next line
%joe18:
%	, as the same certainty $\beta_p = \beta_q = 1$.
; further suppose that the agent is certain about the distribution, so
$\beta_p = \beta_q = 1$.
%joe18*: what abou \alpha?  If we introduce it, we have to say
%something about it here.
For definiteness, suppose that
$\V(X) = \{x_1,x_2\}$, and
that the distribution associated with both edges is $\mu_{.7}$, which ascribes
%joe9: slowing down.
%probability $.7$ to $x_1$, then  $\bbr{\dg M} = \{\mu_{.7}\}$,
%while it can be shown that 
probability $.7$ to $x_1$. Then, as we would hope  $\bbr{\dg M}^* =
\{\mu_{.7}\}$; after all, both sources agree on the information.
However, it can be shown that 
%oli20: some M's escaped. Added msising subscript.
% $\Pr_{\Phi{{\dg M')}}} = \mu_{.85}$, so  $\bbr{\dg M'} = \{\mu_{.85}\}$.
%joe18: there's a typo here: I wrote what I thought you intended
%$\Pr_{\Phi{{\dg M)}}} = \mu_{.85}$, so  $\bbr{\dg M}_1^* = \{\mu_{.85}\}$.
%oli21: 
% $\Pr_{\Phi_{{\dg M)},1)}} = \mu_{.85}$, so  $\bbr{\dg M}_1^* = \{\mu_{.85}\}$.
%joe19*: this is now inconsistent with the notation I introduced.  If
%you're happy with what I did, it should be corrected.
$\Pr_{\WFGof{\dg M}} = \mu_{.85}$, so  $\bbr{\dg M}_1^* = \{\mu_{.85}\}$.
%joe18*: this arguably also shows the problem with the \bbr{\dg M}_k^*
%semantics when k \ne 0.  We should say something about that.
%oli21: A more accurate takeaway is that if \gamma is strictly
%positive, then a messed 
% up qualitative picture will impact the semantics.
\end{example}

%joe11*: is this what you mean?  Why is there only one such world?
%optimal distribution would put all mass on the single best world; to
%oli13: It's not guaranteed to be unique, but all mass is concentrated
% on those worlds(s) which maximize the function. Added plural.
%oli19: rewrote
% If we had included only the likelihood term (the first one), 
% an optimal distribution would put all mass on the worlds of maximum likelihood. 
%joe17*: I don't undrstand why you made these changes.  How do you know
%that there's a unique world of maximum likelihood?
%oli20: it's not litterally true, but overwhelmingly likely and I don't want to
% complicate the story unnecessarily. See the below.
%the best distribution would put all mass on the highet likelihood world. 
%oli20: commenting out, see discussion + replacement below.
% the best distribution would put all mass on the worlds with maximum likelihood.
%joe17: I really don't like ``devoid of unceratinty''!  Moreover, if
%there isn't a unique world of maximum likelihood, why is there no
%uncertainty?  
%oli20*: while there may be multiple, any "uncertainty" is extremely fragile. 
% With any real data, there is a unique best world with probability 1.If we add
% epsilon noise to every cpt entry, there is a unique best world with probability 1.
% In the same vein: the only way it is possible for an interior point to have any probability at all, is if the product of the factors is is the constant function. 
%oli20: A simple example: a binary variable X, with an edge from 1 and an unconditional distribution % p(X=1) = 0.5 + \epsilon.  But X=1 is the only maximum likelihood world, so the optimal distribution is % the one that places all mass on X=1.  Taking this distribution seems hugely overconfident, and is totally in conflict with the uncertainty in the cpt. This is essentially what is hapepning in a factor graph. 
%oli20: trying again.
%joe18: this is not helpful  
%The log likelihood term
%%, which is as the cross entropy, 
%is the only term that makes use of the cpds.
%joe18*: in what sense is it the most important?
%so in some sense it is the most important.  
%oli21: (because it's the only term that depends on the cpds)
%joe18*: Sorry, I can't make any sense of the next sentence.  In what
%sense are the distributions miscalibrated?
%On its own, however, the distributions it selects are badly
%miscalibrated, placing all mass on worlds with the very highest
%likelihood.
%joe18*: the best distance by what metric?
%oli21: (metric=scoring function with only log likelihood term)
%For instance, the best distribution in \cref{ex:overdet}
%by this metric is $\mu_{1.0}$, even if only link $p$ were given ---
%which is in direct conflict with $p$ itself.  
%Such a distribution is devoid of uncertainty, which usually is in direct
%conflict with the data of the cpds $\mat p$.
%joe17: What issue is it resolving?  How is it resolving it?
%oli20: hopefully clearer now; I'm uncommenting and expanding.
%joe18*: I'm afraid it's not at all clear, and I still have no idea
%what issue is being resolved.
%A factor graph essentially resolves this issue by
%also including the final term,
%oli20: added
%joe18*: Sorry, I can't make any sense of this.  What does it mean to
%``properly balance'' something.  How can ou tell if it's ``properly
%balanced''.  
%oli21*: By "balanced", I am referring technically to the situation where they
% have the same coefficient. But saying this doesn't describe the effect:
% When the coefficient on a regularization term, log[ 1 / \mu(x) ], equals the
% coefficient on the energy term, log [ 1 / p(x)], for some fixed p, the
% distribution \mu that minimizes the total is p (because their sum is then the KL
% divergence from \mu to p).  If the regularization term
% had a smaller coefficient, the optimal \mu would be distorted to be more
% deterministic than p, and if it had a smaller one, it would be more spread
% out. I claim that if p is meant as a description of the actual probability and
% not just some measure of relative goodness, then balancing the two terms is
% important.  
%
%which imposes a cost for uncertainy. For the PDG consisting of only
%$p : 1 \to X$ from \cref{ex:overdet}, this results in $\bbr{1 \to
%p}^* = p$. As we have seen, adding $q$ . It seems that as long as our
%regularization term remains agnostic to the number of links in the
%graph, and which nodes they attach to, it cannot properly balance the
%likelihood so that the cpt is satisfied.    

%joe18*: The next sentence should go earlier, after we've talked about
%that first and last terms of (5) captring the free energy.
%oli21: I keep trying to tell a story about what the equation does by building
% up the terms, but I'm not getting through. I think without communicating the
% instability + overconfidence of the likelihood term, and its standard 
% resolution which involves the free energy, it's not really worth saying this anymore.
%The scoring function that we use for PDGs can thus be viewed as
%joe19: moved up from below, with minor changes to make it flow better.
Although both $\theta$ and $\beta$ are measures of confidence, 
%joe17
%the way $\theta$ that a factor graph varies with $\theta$
%joe19
%the way that a factor graph varies with $\theta$
%is quite different from the way a PDG
the way that the Gibbs free energy varies with $\theta$ 
is quite different from the way that the score of a PDG
varies with $\beta$. 
The scoring function that we use for PDGs can be viewed as
extending ${\GFE}_{\Phi,\theta}$ by including
the local regularization term.
As $\gamma$ approaches zero,
%joe18: it's not the strength, but the importance
%oli21: Huh, I've always heard the adjective "strength" used to
%describe the magnitude 
% of the coefficient of a regularization term.  
%the strenth of global regularization
%drops and the strength of local regularization increases. 
the importance of the global regularization terms decreases relative
%joe19
%to that of the local regularization term.
to that of the local regularization term, so the PDG scoring function
becomes quite different from Gibbs free energy.


%joe11*: why do we want to express uncertainty?   What is it
%uncertainty about.  I'm trusting you that ``regularization'' will be
%meaningful in this context to others.  It's certainly not meaningful
%to me. You have the space to add a sentence of clarification.
%oli13: "Uncertainty" was maybe not the best word choice. Let's try this:
% to capture $\bbr{\dg M}_\gamma(\mu)$, we need to include 
%joe13*: this doesn't make sense to me.  Where did overfitting come
%from?  We're not doing machine learning?  Since you haven't told me
%the objective function, so to speak, I find this worse than useless.
%It makes me think of things that ought to be irrelevant. At a
%minimum, to include them, you must explain the goal (i.e., give some
%intution!).  
%to avoid overfitting these particular distributions, we need to include 
%\emph{regularization terms} \todo{cite}.  
%oli15
% the regularization terms are needed to capture the inconsistency and
% information deficiency of $\mu$ relative to ${\dg M}$.
%joe14: I put ``regulararization'' in quotes
%the regularization terms are thus necessary to model uncertainty.
%You need to say something about how, although they're not
%regluarization terms as the term is used in the ML community (since
%we are *not* doing machine learning), they have some of the same spirit.
%joe15*: Why are regularization terms necessary to model uncertainty?
%All that you can say is here is that they're necessary to capture
%this particular socirng function
%oli17*: No, this is much weaker than what I want to say. These terms
%are  necessary to have an optimal distribution $\mu$ that is not a
%point mass on a particular world. 
%oli17: Unsure if helpful, but the thermodynamic analogy is
%temperature=0 Kelvin, 
% so that everything is in its very minimum energy state, and the distribution of possible
% configurations is a point mass on the single best one. 
%oli17: Again, the point is: if we want it to even be possible for the minimum
% of this function to be a non-degenerate distribution (i.e., one with uncertainty),
% then we need the regularization terms. 
%oli17: rewrote more explicitly, in the hopes that this was convincing.
%
%the ``regularization'' terms are thus necessary to model uncertainty.
% the ``regularization'' terms are thus necessary to capture $\bbr{\dg
  % M}_\gamma$. 
%joe16*: Cut.  I do not understand what this means, or why it's
%important.  This hurts far more than it helps.
%the ``regularization'' terms are thus necessary for these
%distributions to have any uncertainty. 
%oli19: rewriting.
% By contrasting equation
% \eqref{eq:semantics-breakdown} with the expression for $\GFE_\Phi$, we see
% that, although both $\theta$ and $\beta$ are measures of confidence, the way
% that a factor graph varies with $\theta$ is quite different from the way a PDG
% varies with $\beta$. 
%joe17: Cut this.  I 
%The fact that the local term varies with $\beta_L$ has important
%modeling consequences. 

%joe17*: I don't undersatnd teh rest of this.  I couldn't see anywhere
%in the example where these points were illustrated.  Indeed, the
%exampe doesn't mention \beta at all.  If you could rewrite the exmple
%to illustrate these points, we might consider reinstating this.
%oli20: fair enough, reinstated and added missing details. 
%joe18*: I still can't make sense out of any of this.  As an aside,
%this is *not* what I asked you to do.  I asked you just to focus on
%the technical material, which is all I was (and am) comfortable with.
%$\theta_J$ controls the strength of the potential associated to $J$;
%joe18*: I don't understand this.  Why does increasing one
%terms \theta_J for a specific J, make distributions more
%deterministic.  If you increase all of them, I can see how it wuld
%make distributions more determininstic, but only if all the ffactors 
%were different numbers.  But why do we are about this?  We're writing
%a paper about PDGs.  What does this tell us about PDGs.
%increasing it results in optimal distributions which are more
%deterministic.  
%oli21: You raise a good point about an individual $\theta_J$;
% thanks for adding the context.  
%oli20: added
%joe18*: I have no idea how a generaic atatement about properties. I
%would *strongly* prefer to return this to its previous state (modlo
%the technical concerns I've raised).
%of \theta_J could follow from an Example.  But even if you meant that
%Example 5 illustrates your point, I can't see any way in which you've
%argued that doubling J makes things more deterministic.  The next
%sentence just feels like a complete non sequitur.  
%This follows from \cref{ex:overdet} the fact that doubling $\theta_J$
%is equivalent to duplicating factor $J$. By contrast, $\beta_L$
%controls both the likelihood and the local uncertainty together: as
%described in our motivation for $\Inc$, increasing the reliability of
%$\beta$ in a PDG increases the cost of failing to match the given cpd.  
%\\\contentious{
%joe19*: Please do not spend time doing this, Oliver.  Put your
%efforts dealing with the issues that we agreed were the important
%issues.  I don't understand any of the discussion below. I don't know
%what it means for a factor graph to ``fuse'' things, nor why we have
%to have \alpha_L = \beta_L (indeed, as I pointed out above, we
%definitely do *not* want to assume this). Since we have no intuition
%for \alpha_L, we can make no claim about what is appropriate for it.
%I cut all this.
%The behavior in \cref{ex:overdet} also sheds some light on the nature
%of $\alpha$. 
%Because a factor graph fuses $\alpha_L = \beta_L$, adding a new edge
%effectively doubles both (it doubles $\theta_L$). If one believes
%that both $p$ and $q$ are from independent sources, and thus
%constitute separate qualitative determinations of $X$, then
%effectively doubling $\alpha_p$ is appropriate. If, on the other
%hand, you merely have two distinct sets of observations which agree,
%and have no reason to believe they are independent or have anything
%to do with the causal structure of the world, then effectively
%doubling $\alpha_p$ is not appropriate.   
% }
%
%joe13*: the equation says nothing abut \theta.  This is even more
%confusing because in the tranlsation you take \beta_L = \theta_L.  My
%preference would be to ut the rest of the paragraph, although I didn't
%do it.  I think it hurts more than helps.
%\eqref{eq:semantics-breakdown} shows further that, although
%oli15
% \eqref{eq:semantics-breakdown} helps show  that, although
%oli12: added the technical details and fused with next sentence
%joe11*: typo, I assume.  If not, then it's unclear
%For a factor graph, $\beta$ controls the importance of the likelihood
%term, while in a PDG it also increases the strength of the local
%oli13: you're right. This enables further simplification
% For a factor graph, $\theta$ controls the importance of the likelihood
% term, while in a PDG, $\beta$ it also increases the strength of the
%joe13: I don't know what ``jointly controls'' means.  After changing
%the next few lines, I cut them.   It's true that \beta affets both
%the likelihood term and the local regularaization term, but I don't
%wee why that makes it different from theta; there is no theta in the
%equation above
%oli15: I've reverted because the story makes more sense with this in than out. 
% It's true and I think it won't be too difficult to follow from the equation.
%%$\beta$ jointly controls the importance of the likelihood and local
%oli15 modified
%joe14*: In what sense does it balance them?  I'm lost.  If you say
%``affects'' it's fine.  If not, you must explain what ``balance'' means.
%oli16: the strength of the regularization must match the strength of
%the likelihood. Only then is the function minimized by the cpd on the
%edge, rather than a more deterministic one (if the former is
%stronger) or a more random one (if the latter is stronger). 
    % $\beta$ balances the likelihood and local
    % regularization terms [[REWRITE]], while 
    % $\theta$ affects only the former.
%oli16: rewritten so that now it is clear, but much too long. I like my shortening above,
% and do not think it needs anything else, but I'll let you shorten this one as you like. It also double-covers material from the next sentence.)
%joe15*: Sorry; it is *not* clear (at least, not to me).  I don't know
%what it means that they ``remain balanced''.  I also don't like the
%use of terms of agency (``controls'').  Are you saying anything other
%than \beta affects the difference between the terms.  Is that what
%balance is supposed to mean?  If so, it's an awfully complicated way
%of saying it.  If not, what is it saying that's different?  I cut
%most of it.  I think that what you wrote is a net negative; in
%expectation, it will hurt more than it will help.  Do NOT
%make further changes without discussion (other than correcting typos)
%$\beta$ simultaneously controls both the local likelihood and local
%regularization terms, so that the two remain balanced, and making a
%link more or less random never does better than exactly matching the
%cpt.  
%oli17: \beta preserves the ratio between the first and second 
% terms (in the limit as \gamma -> 0)
%joe16: As a technical matter, I don't see why this is true.  I don't
%see in what sense \beta preserves anything.  It's just a number that
%you miltiply the difference between the terms by.  
%oli17: and the cpds that that optimize it remain the same.
%joe16: The same as what?  I'm totally lost
%oli17: This is very different from just increasing the strength of the 
% likelihood because it makes things more determinisitc. I maintain that this
% is a critical point, but I will not modify the text here to make it until we
% agree...
%joe16: good.  I truly have no idea what you said above, and I
%strongly suspect that I would not understand anything you tried to
%say here.
%oli19: I don't need this anymore either
%joe17: reinstated, because I cut what you added
%oli20: removed again. I don't think this says very much, and I've tried
% to address your concerns again.
    % In (\ref{eq:semantics-breakdown}), 
    % $\beta$ affects the difference between the local likelihood and
    % local regularization terms; on the other hand, in $\GFE_\Phi$,
    % $\theta$ affects only the likliehood.
% %oli17
% $\theta$ affects only the likelihood.
%$\theta$ only controls the likliehood, and so as it
%increases the benefits of reporting an uncertain distribution fade as

%oli21: Here's the paragraph that I didn't cut before, but now it's above so I'm cutting it.
% \Cref{eq:semantics-breakdown} also makes it clear that 
% taking $\beta_L = {$\alpha_L$} \gamma$ for all edges $L$ is
% essentially necessary to get \Cref{thm:pdg-is-fg,thm:fg-is-pdg}.
% However, the analogue of \Cref{prop:consist} does not hold
% in general for such choices, leading to some arguably unacceptable
% behavior in factor graphs trying to capture the same phenomena as PDGs.

%joe9: cut; folded some material into the next section
%oli20: I rewrote this section and brought it up in case 
%joe18: I commented it out again, after making minor changes.  Nothing
%in what you've written explains the connection between PDGs and
%directed factor networks.  THere's no point in giving a (not
%particularly useful) introduction to directed factor graphs.  We're
%not writing a paper about factor graphs.  If you have something
%useful to say about the connection between diredcted factor graphs
%and PDGs (e.g, we can prove an analogue of THeorems 4.4 and 4.5 for
%them), then it would be worth saying.  Otherwise, this is a poor use
%of space.
\begin{inactive}

    \subsection{Directed Factor Graphs}
    % Directed Factor graphs \cite{frey2012extending} impose constraints on factor graphs to resolve some of their issues and bring them more in line with BNs, but they have no semantics if the constraints are not satisfied, making them a way of visualizing factor graphs more than a novel modeling tool.
%joe18
%While PDGs can be thought as a loosening the restrictions on
While PDGs can be thought as a loosening of the restrictions on
    Bayesian Networks, 
%joe18
%Directed Factor Graphs \cite{frey2012extending} go the
\emph{directed Factor Graphs} \cite{frey2012extending} go in the
    opposite direction: they  
%joe18: I don't know what ``better-behaved'' means.  Better behaved
%than what?
%extend factor graphs with better-behaved directed edges,
are variant of factor graphs where edges are directed,
    allowing them to represent a larger class of independencies,
%joe18
%including those of Bayesian Networks. Such an directed edge
including those of Bayesian Networks. A directed edge
    indicates that the product of incoming edges normalize to 1,
    an invariant which must be enforced and maintained as the
    model is changed. As a result, directed factor graphs are
    well-suited to visually describing the structure of
    distributions, but less well suited to modelding beliefs that
    change quickly, or could be inconsistent. 
\end{inactive}

%oli22*: I'm still working on this section. THere's no need to edit it
% at this point, because I plan to reduce it massively myself.
% (also much of it has been heavily modified without marks) 
% I do have a question: is it worth dding a plots which emperically suggest
% that these results are true, even without theorems?
\begin{wip}
\subsection{Dependency Networks}

PDGs are closely related to Dependency Networks, or DNs
\cite{heckerman2000dependency}. The data of a DN is also an unstructured
collection of cpds, but the cpds are attached to nodes rather than edges, so
there must be exactly one per node.  \citeauthor{heckerman2000dependency} also
emphasize the benefit of being able to supply arbitrary data in the cpds,
without enforcing the consistency constraints.


\begin{defn}
%joe20
%A Dependency Network is a tuple $\mathcal D = (\Gr, \mat p) $, where
A \emph{dependency network} is a tuple $\mathcal D = (\Gr, \mat p) $, where
$\Gr$ is a 
%joe20
%directed graph of variables, and $\mat p$, gives for each $X \in \N$ a
directed graph of variables and $\mat p$ gives, for each $X \in \N$ a
%joe20*: misplaced footnote mark.  More importantly, does Heckerman
%require p to be positive in the defintion, or only n the theorem.
%The definitions make perfect sense even without this assmption.
%positive 
%	\footnote{that is, each cpd $\bp[X]$ has no zero entries}
%cpd $\bp[X](X \mid \Pa^{\Gr}(X))$,
positive cpd $\bp[X](X \mid \Pa^{\Gr}(X))$ (i.e., all entries in $\bp[X]$ are
psoitive),
where $\Pa^{\Gr}(X)$ are the parents of $X$ in $\Gr$.
Together with a total order $\prec$ on $\N$,
$\mathcal D$ defines a joint distribution $\Pr^\prec_{\mathcal D}$ on $\V(\N)$ via an \emph{ordered Gibbs sampler}. 
Concretely,
%joe20
%initialize $\mat X^{(0)}$ to any joint setting of variables, and
%for $t= 1,2, \ldots$, and for $i=1,2,\ldots$, we draw
initialize $\mat X^{(0)}$ to an arbitrary joint setting of variables, and
for $t= 1,2, \ldots$ and $i=1,2,\ldots$, define
\[
     X_i^{(t)} \sim \bp[i]\left(X_i ~\big|~ x_1^{(t)}, \ldots,
%joe20
%x_{i-1}^{(t)}, x_{i+1}^{(t-1)}, \ldots, x^{(t-1)}_{n} \right)
x_{i-1}^{(t)}, x_{i+1}^{(t-1)}, \ldots, x^{(t-1)}_{n} \right).  
%		= \bp[i](X_i \mid \mat{pa}^{\Gr}(i)) 
\]
%joe20*: what does rotating through the variables have to do with it?
%After each full rotation through the variables, we get a new sample
%which depends only on $\mat X^{(t)}$> Thus we have a Markov chain of
Note that $\mat X^{(t+1)}$
depends only on $\mat X^{(t)}$. Thus, we have a Markov chain of
joint variable settings 
$ \mat X^{(0)} \to \mat X^{(1)} \to \ldots \to \mat X^{(t)} \to \ldots$.
%joe20
%that, so long as each $\bp[X]$ is positive, ensures that the chain is
Since each matrix $\bp[X]$ is positive,
the chain is
%joe20: you need a reference (perhaps Puterman's book), and need to
%explain ergodic. 
ergodic, 
%joe20
%and thus limits to a unique stationary distribution on $\mat X$,
and thus has a unique stationary distribution on $\mat X$ as its limit.
%joe20
%which we take to be the definition of $\Pr^\prec_{\mathcal D}$.
We define $\Pr^\prec_{\mathcal D}$ as this stationary distribution,
\end{defn}

% << This is what was written in your email. It's still more detailed
% than I have written but I don't want to finish integrating it until after
% I've proven some results. >>
%
%This is clearer than what's written, but it's different from what you 
%wrote in our writeup, as near as I can tell.  But I think I see what's 
%going on, and how it should be explained (at least for me to understand 
%it).  First, you have tell me how to initialize x_1, ..., x_n (i.e., 
%what is x_1^0, ..., x_n^0).  We then define X_n^t by induction on t.  We 
%take X_0^t = (x_1^0, ..., x_n^0) (which I'm assuming is given somehow).  
%If t = nt'+i, we assume that we have defined x_j^0, \ldots, x_j^{t'} for 
%j < i and x_j^0, \ldots, x_j^{t'-1} for j \ge i.  We then define x_i^t' 
%to be the result of drawing a value according to the distribution 
%p_i(.|x_1^{t'}, ..., x_{i-1}^{t'}, x_i^{t'-1}, ...,  x_n^{t'-1}).   Then 
%we define X^t = (x_1^{t'}, x_i^{t'}, x_{i+1}^{t'-1}, ..., x_n^{t'-1}).
%
%Notice how different this description is from years.  Now you can point 
%out that the sequence X^0, X^1, ... can be viewed as a Markov chain.  
%However, you can't just say "you take the limiting distribution", 
%because the limiting distribution does not exist in general.  This is 
%where the fact that all the cpds (not *local distibutions, which is an 
%undefined term) are positive comes in; it ensures that the Markove chain 
%is irreducible, which in turn guarantees that there's a unique 
%stationary distribution. This should be pointed out.

The dependence on the order is artificial in the case that the DN is
``consistent'', which in the authors parlance implies that it also has 
% their favored independencies.\footnote{Interestingly, any BN $\mathcal B$ is an ``inconsistent'' dependency network, despite the fact that the Gibbs sampler whose order $\prec$ is a topological sort  of $\cal B$, generates $\Pr_{\mathcal B}$.}
%joe24
%their favored independencies.\footnote{Interestingly, any BN $\mathcal
their favored independencies.\footnote{Interestingly, a BN $\mathcal
B$ is an ``inconsistent'' dependency network, despite the fact that
the Gibbs sampler whose order $\prec$ is a topological sort  of $\cal
B$, generates $\Pr_{\mathcal B}$.} 
\begthm[\citeauthor{heckerman2000dependency}]{theorem}{thm:dns-uniq}
%Theorem 1. (wrapped into definiton above)
%An ordered Gibbs sampler applied to a dependency network for $\mat X$, where each $X_i$ is discrete and each local distribution $\bp[X](x \mid \mat{pa}(X))$ is positive, has a unique stationary joint distribution for $\mat X$.  
If a dependency network $\mathcal D \!=\! (\Gr, \mat p)$ over variables $\N$ is consistent with a positive joint distribution $p$,
in that $p(X \mid \N\setminus \{X\}) = \bp[X](X \mid \Pa^{\Gr}(X))$ for all $X \in \N$,
 then $\Pr_{\mathcal D}^\prec = p$.
\end{theorem}

%joe24
%A dependency network $\mathcal D$, together with any a vector of
A dependency network $\mathcal D$, together with a vector of
confidences $\beta$ for the cpds on each variable 
% (defaulting to $\beta_X=1$) 
can be naturally regarded as a PDG $\PDGof{\mathcal D, \beta}$ 
via the same translation used for a Bayesian Network.
Although on the surface PDG and DN semantics are quite
different,
the former subsumes the latter, as the next results show.


%\begin{lemma}
%	A distribution $\mu$ that locally minimizes $\IDef{\PDGof{\mathcal D, \beta}}$ satisfies all of the independencies of $\mathcal D$.
%\end{lemma}


\begthm{conj}{thm:dns-are-pdgs}
%oli22: updating theorem statement
% If $\mathcal D = (\N, \Ed, \V, \mathcal P)$ is a consistent dependency network with a positive stationary distribution $p^*$ of its sampling procedure, then the PDG $\PDGof{\mathcal D} = (\N, \mathit{merge}(\Ed), \V, $$\mathcal P, \mat 1{, $\mat 1$})$, then $\bbr{\PDGof{\mathcal D}} = p^*$.
If $\mathcal D$ is a consistent dependency network,
%with positive local distributions,
then for all positive vectors $\beta$, and all orders $\prec$, we have that
%$\SD{\PDGof{\mathcal D, \beta}} =  \{ \Pr_{\cal D}^\prec \}$.
$\bbr{\PDGof{\mathcal D, \beta}}^* =  \Pr_{\cal D}^\prec$.
%, where $\Pr_{\cal D}^\prec$ is the unique stationary distribution of the $\prec$-ordered Gibbs sampling procedure. 
\end{conj}

Inconsistent DNs still define a unique distribution, but we cannot hope that it 
will be the same one as generated by the PDG, owing to the fact that
the Gibbs sampler defined by \citeauthor{heckerman2000dependency} is dependent
on the (arbitrary) fixed order used in sampling.
A PDG can simulate the effect of the order with the reliability $\beta$.

\begin{conj}
If $\mathcal D$ is a dependency network over the variables
$X_1 \prec X_2  \ldots  \prec X_n$ is a total order on the variables, and $\beta$ is a vector over the edges
%oli22: to preemptively eliminate some confusion:
% (which have a 1-1 correspondence with variables in a DN) 
such that  $\beta_1 \ll\beta_2 \ll \ldots \ll \beta_n$, 
then $\Pr_{\cal D}^{\prec} = \bbr{\dg M; \beta}^*$
\end{conj}

A \emph{random scan} Gibbs sampler, rather than cycling though variables in a 
particular order, draws the next variable index to update randomly.
Given a distribution $d$ over $\N$,
let $\mathit{RScan}(d)$ be the random scan Gibbs sampling procedure that draws
the next variable to update from $d$. $\mathit{RScan}(d)$ also has a
unique stationary distribution, $\Pr_{\mathcal D}^d$, which is equal
%joe24
%to $\Pr_{\cal D}^{\prec}$ for any $d$ and $\prec$ if $\mathcal D$ is a
to $\Pr_{\cal D}^{\prec}$ for all $d$ and $\prec$ if $\mathcal D$ is a
consistent DN. This more symmetric procedure is captured nicely by PDG
semantics. 

\begin{conj}\label{thm:dns-are-completely-pdgs}
    Let $d(X) \propto \beta_X$. Then
    $\Pr_{\mathcal D}^d = \bbr{\PDGof{\mathcal D, \beta}}^*$, whether or not $\mathcal D$ is consistent.
\end{conj}

\end{wip}
\end{document}
