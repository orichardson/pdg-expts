\documentclass[the-pdg-manual.tex]{subfiles}
\begin{document}
\section{Raw Probability Distributions}

Probability distributions themselves are a particular kind of PDG;
a  triple $(\Omega, \mathcal F, \Pr)$ is naturally identified with the diagram
\begin{center}
	\begin{tikzpicture}
		\node[dpadded] (1) at (0,0) {$\var 1$};
		\node[dpadded] (W) at (3,0) {$\Omega$};

		\draw[arr] (1) to node[fill=white]{$\Pr$} (W);
	\end{tikzpicture}
\end{center}

% \begin{example}\label{ex:worldsonly}
% \end{example}

Let $\N$ be a set of variables whose values are given by $\V$. When we use the characterization of PDGs based on directed hyper-graphs, a joint distribution $\mu \in \Delta[\V(\N)]$ is naturally identified with a particular unweighted PDG. Specifically, the data of $\mu$ is given by the PDG $(\N, \{ E_0 \}, \V, \mat p)$ containing a single hyper-edge $E_0$ whose source is empty and whose target is all of $\N$, associated with the cpd $\bp[E](\mat x) := \mu(\mat x)$.

\begin{example}
	For a 3-variable
\end{example}
\end{document}
