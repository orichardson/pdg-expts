\documentclass{article}

\usepackage[margin=1in]{geometry}
\usepackage{mathtools,amssymb}
\usepackage{xcolor}
\usepackage{enumitem}

\setlength\parskip{1ex}

\newcommand\ngrad{\hat\nabla}
\newcommand\Ex{\mathop{\mathbb{E}}}


\newcommand\quest[1]{\textsf{\large\color{blue} #1}\par}

\begin{document}
\begin{enumerate}[label={\textsf{\large\color{blue}\arabic*.}}]
\item \quest{
        What is the Fisher information matrix for the Boltzmann parameterization of
            probability distributions, at inverse temperature $\beta = \frac1T$?
    }
    
    The parameter settings are all real-valued functions over the values of the target variable, which we will call $X$. The distribution is then 
    \[
        p_u( )
    \]
    
\item \quest{What are the natural gradients of the various PDG divergences?}
\item \quest{What are the Hessians of the various PDG divergences?}
\item \quest{What are the geodesics in the Fisher metric?}
    
    According to [Itoh \& Satoh 2015]:
    \[
        \mu(t) = \exp_\mu (t \tau) = \Big(\cos \frac t2 + \sin \frac t2 \frac{\mathrm d\tau}{\mathrm d\mu}(x) \Big)^2 \mu
    \]
    is the unique geodesic from $\mu$ to $\mu^*$, where
    \[
        \tau = \frac1{\sin \frac{\ell}2} \left( 
            \sqrt{\frac{d\mu^*}{d\mu}} - \Ex_{\mu}\left[\sqrt{\frac{d\mu^*}{d\mu}}\;\right] 
        \right) \mu
    \]
    and $\ell \in (0, \pi)$, the arc length betwen $\mu$ and $\mu^*$, is given by
    \[
        \cos \frac\ell2 = \Ex_{\mu}\left[\sqrt{\frac{d\mu^*}{d\mu}}\;\right].
    \]
    
\item \quest{Is there a nice embedding that corresponds to the Fisher metric? We know that \emph{some} embedding exists. {\small\color{blue!50!white}(I had thought it was log probability, but that doesn't seem to be right given the formulas above ...)}}
\item \quest{What functions are geodesically linear on the simplex, with the Fisher metric?}
    
    
\end{enumerate}
\end{document}
