
\documentclass{article}

\relax % Controls
    \newif\ifmarginprooflinks
    	\marginprooflinkstrue
    	% \marginprooflinksfalse

\relax % Bibliography, etc
	\usepackage[american]{babel}
	\usepackage{csquotes}
	\usepackage[backend=biber, style=authoryear]{biblatex}
	\DeclareLanguageMapping{american}{american-apa}
	% \usepackage[backend=biber,style=authoryear,hyperref=true]{biblatex}
	% \addbibresource{refs.bib}
	% \addbibresource{conf.bib}

	\DeclareFieldFormat{citehyperref}{%
	  \DeclareFieldAlias{bibhyperref}{noformat}% Avoid nested links
	  \bibhyperref{#1}}

	\DeclareFieldFormat{textcitehyperref}{%
	  \DeclareFieldAlias{bibhyperref}{noformat}% Avoid nested links
	  \bibhyperref{%
	    #1%
	    \ifbool{cbx:parens}
	      {\bibcloseparen\global\boolfalse{cbx:parens}}
	      {}}}

	\savebibmacro{cite}
	\savebibmacro{textcite}

	\renewbibmacro*{cite}{%
	  \printtext[citehyperref]{%
	    \restorebibmacro{cite}%
	    \usebibmacro{cite}}}

	\renewbibmacro*{textcite}{%
	  \ifboolexpr{
	    ( not test {\iffieldundef{prenote}} and
	      test {\ifnumequal{\value{citecount}}{1}} )
	    or
	    ( not test {\iffieldundef{postnote}} and
	      test {\ifnumequal{\value{citecount}}{\value{citetotal}}} )
	  }
	    {\DeclareFieldAlias{textcitehyperref}{noformat}}
	    {}%
	  \printtext[textcitehyperref]{%
	    \restorebibmacro{textcite}%
	    \usebibmacro{textcite}}}

	\DeclareCiteCommand{\brakcite}
	  {\usebibmacro{prenote}}
	  {\usebibmacro{citeindex}%
	   \printtext[bibhyperref]{[\usebibmacro{cite}]}}
	  {\multicitedelim}
	  {\usebibmacro{postnote}}

\relax % Standard Packages
    \usepackage[dvipsnames]{xcolor}
    % \usepackage[utf8]{inputenc}
    \usepackage{mathtools}
    \usepackage{amssymb}
		\DeclareMathSymbol{\shortminus}{\mathbin}{AMSa}{"39}
    % \usepackage{parskip}
    % \usepackage{algorithm}
    \usepackage{bbm}
	\usepackage{lmodern}
	% \usepackage{times}
    \usepackage{faktor}
    % \usepackage{booktabs}
	% \usepackage[margin=1in]{geometry}
    \usepackage{graphicx}
    \usepackage{scalerel}
    \usepackage{enumitem}
    \usepackage{nicefrac}\let\nf\nicefrac

    % \usepackage{color}
    %\usepackage{stmaryrd}
    \usepackage{hyperref} % Load before theorems...
        \hypersetup{colorlinks=true, linkcolor=blue!75!black, urlcolor=magenta, citecolor=green!50!black}

\usepackage{tikz}
	\usetikzlibrary{positioning,fit,calc, decorations, arrows, shapes, shapes.geometric}
	\usetikzlibrary{cd}

	%%%%%%%%%%%%
	\tikzset{AmpRep/.style={ampersand replacement=\&}}
	\tikzset{center base/.style={baseline={([yshift=-.8ex]current bounding box.center)}}}
	\tikzset{paperfig/.style={center base,scale=0.9, every node/.style={transform shape}}}

	% Node Stylings
	\tikzset{dpadded/.style={rounded corners=2, inner sep=0.7em, draw, outer sep=0.3em, fill={black!50}, fill opacity=0.08, text opacity=1}}
	\tikzset{dpad0/.style={outer sep=0.05em, inner sep=0.3em, draw=gray!75, rounded corners=4, fill=black!08, fill opacity=1, align=center}}
	\tikzset{dpadinline/.style={outer sep=0.05em, inner sep=2.5pt, rounded corners=2.5pt, draw=gray!75, fill=black!08, fill opacity=1, align=center, font=\small}}

 	\tikzset{dpad/.style args={#1}{every matrix/.append style={nodes={dpadded, #1}}}}
	\tikzset{light pad/.style={outer sep=0.2em, inner sep=0.5em, draw=gray!50}}

	\tikzset{arr/.style={draw, ->, thick, shorten <=3pt, shorten >=3pt}}
	\tikzset{arr0/.style={draw, ->, thick, shorten <=0pt, shorten >=0pt}}
	\tikzset{arr1/.style={draw, ->, thick, shorten <=1pt, shorten >=1pt}}
	\tikzset{arr2/.style={draw, ->, thick, shorten <=2pt, shorten >=2pt}}

	\newcommand\cmergearr[5][]{
		\draw[arr, #1, -] (#2) -- (#5) -- (#3);
		\draw[arr, #1, shorten <=0] (#5) -- (#4);
		}
	\newcommand\mergearr[4][]{
		\coordinate (center-#2#3#4) at (barycentric cs:#2=1,#3=1,#4=1.2);
		\cmergearr[#1]{#2}{#3}{#4}{center-#2#3#4}
		}
	\newcommand\cunmergearr[5][]{
		\draw[arr, #1, -, shorten >=0] (#2) -- (#5);
		\draw[arr, #1, shorten <=0] (#5) -- (#3);
		\draw[arr, #1, shorten <=0] (#5) -- (#4);
		}
	\newcommand\unmergearr[4][]{
		\coordinate (center-#2#3#4) at (barycentric cs:#2=1.2,#3=1,#4=1);
		\cunmergearr[#1]{#2}{#3}{#4}{center-#2#3#4}
		}

\usepackage{amsthm,thmtools} % Theorem Macros
	\usepackage[noabbrev,nameinlink,capitalize]{cleveref}
    \theoremstyle{plain}
    \newtheorem{theorem}{Theorem}
	\newtheorem{coro}{Corollary}[theorem]
    \newtheorem{prop}[theorem]{Proposition}
    \newtheorem{conj}[theorem]{Conjecture}
    \newtheorem{claim}{Claim}
    \newtheorem{remark}{Remark}
    \newtheorem{lemma}[theorem]{Lemma}
    \theoremstyle{definition}
    % \newtheorem{defn}{Definition}
    % \declaretheorem[name=Definition]{defn}
    \declaretheorem[name=Definition, qed=$\square$]{defn}
    \declaretheorem[name=Example, qed=$\triangle$]{example}

	\crefname{defn}{Definition}{Definitions}
	\crefname{prop}{Proposition}{Propositions}
    \crefname{issue}{Issue}{Issues}

\relax %%%%%%%%% GENERAL MACROS %%%%%%%%
    \let\Horig\H
	\let\H\relax
	\DeclareMathOperator{\H}{\mathrm{H}} % Entropy
	\DeclareMathOperator{\I}{\mathrm{I}} % Information
	\DeclareMathOperator*{\Ex}{\mathbb{E}} % Expectation
	\DeclareMathOperator*{\EX}{\scalebox{1.5}{$\mathbb{E}$}}

    \newcommand{\mat}[1]{\mathbf{#1}}
    \DeclarePairedDelimiterX{\infdivx}[2]{(}{)}{%
		#1\;\delimsize\|\;#2%
	}
	\newcommand{\thickD}{I\mkern-8muD}
	\newcommand{\kldiv}{\thickD\infdivx}
	\newcommand{\tto}{\rightarrow\mathrel{\mspace{-15mu}}\rightarrow}

	\newcommand{\datadist}[1]{\Pr\nolimits_{#1}}
	% \newcommand{\datadist}[1]{p_\text{data}}

	\makeatletter
	\newcommand{\subalign}[1]{%
	  \vcenter{%
	    \Let@ \restore@math@cr \default@tag
	    \baselineskip\fontdimen10 \scriptfont\tw@
	    \advance\baselineskip\fontdimen12 \scriptfont\tw@
	    \lineskip\thr@@\fontdimen8 \scriptfont\thr@@
	    \lineskiplimit\lineskip
	    \ialign{\hfil$\m@th\scriptstyle##$&$\m@th\scriptstyle{}##$\hfil\crcr
	      #1\crcr
	    }%
	  }%
	}
	\makeatother
	\newcommand\numberthis{\addtocounter{equation}{1}\tag{\theequation}}

\relax %%%%%%%%%   PDG  MACROS   %%%%%%%%
	\newcommand{\ssub}[1]{_{\!_{#1}\!}}
	% \newcommand{\bp}[1][L]{\mat{p}_{\!_{#1}\!}}
	% \newcommand{\bP}[1][L]{\mat{P}_{\!_{#1}\!}}
	\newcommand{\bp}[1][L]{\mat{p}\ssub{#1}}
	\newcommand{\bP}[1][L]{\mat{P}\ssub{#1}}
	\newcommand{\V}{\mathcal V}
	\newcommand{\N}{\mathcal N}
	\newcommand{\Ed}{\mathcal E}

    \newcommand{\balpha}{\boldsymbol\alpha}
    \newcommand{\bbeta}{\boldsymbol\beta}

	\DeclareMathAlphabet{\mathdcal}{U}{dutchcal}{m}{n}
	\DeclareMathAlphabet{\mathbdcal}{U}{dutchcal}{b}{n}
	\newcommand{\dg}[1]{\mathbdcal{#1}}
	\newcommand{\PDGof}[1]{{\dg M}_{#1}}
	\newcommand{\UPDGof}[1]{{\dg N}_{#1}}
	\newcommand\VFE{\mathit{V\mkern-4mu F\mkern-4.5mu E}}

	\newcommand\Inc{\mathit{Inc}}
	\newcommand{\IDef}[1]{\mathit{IDef}_{\!#1}}
	% \newcommand{\ed}[3]{%
	% 	\mathchoice%
	% 	{#2\overset{\smash{\mskip-5mu\raisebox{-3pt}{${#1}$}}}{\xrightarrow{\hphantom{\scriptstyle {#1}}}} #3} %display style
	% 	{#2\overset{\smash{\mskip-5mu\raisebox{-3pt}{$\scriptstyle {#1}$}}}{\xrightarrow{\hphantom{\scriptstyle {#1}}}} #3}% text style
	% 	{#2\overset{\smash{\mskip-5mu\raisebox{-3pt}{$\scriptscriptstyle {#1}$}}}{\xrightarrow{\hphantom{\scriptscriptstyle {#1}}}} #3} %script style
	% 	{#2\overset{\smash{\mskip-5mu\raisebox{-3pt}{$\scriptscriptstyle {#1}$}}}{\xrightarrow{\hphantom{\scriptscriptstyle {#1}}}} #3}} %scriptscriptstyle
	\newcommand{\ed}[3]{#2%
	  \overset{\smash{\mskip-5mu\raisebox{-1pt}{$\scriptscriptstyle
	        #1$}}}{\rightarrow} #3}

    \newcommand{\nhphantom}[2]{\sbox0{\kern-2%
		\nulldelimiterspace$\left.\delimsize#1\vphantom{#2}\right.$}\hspace{-.97\wd0}}
		% \nulldelimiterspace$\left.\delimsize#1%
		% \vrule depth\dp#2 height \ht#2 width0pt\right.$}\hspace{-.97\wd0}}
	\makeatletter
	\newsavebox{\abcmycontentbox}
	\newcommand\DeclareDoubleDelim[5]{
	    \DeclarePairedDelimiterXPP{#1}[1]%
			{% box must be saved in this pre code
				\sbox{\abcmycontentbox}{\ensuremath{##1}}%
			}{#2}{#5}{}%
		    %%% Correct spacing, but doesn't work with externalize.
			% {\nhphantom{#3}{##1}\hspace{1.2pt}\delimsize#3\mathopen{}##1\mathclose{}\delimsize#4\hspace{1.2pt}\nhphantom{#4}{##1}}
			%%% Fast, but wrong spacing.
			% {\nhphantom{#3}{~}\hspace{1.2pt}\delimsize#3\mathopen{}##1\mathclose{}\delimsize#4\hspace{1.2pt}\nhphantom{#4}{~}}
			%%% with savebox.
		    {%
				\nhphantom{#3}{\usebox\abcmycontentbox}%
				\hspace{1.2pt} \delimsize#3%
				\mathopen{}\usebox{\abcmycontentbox}\mathclose{}%
				\delimsize#4\hspace{1.2pt}%
				\nhphantom{#4}{\usebox\abcmycontentbox}%
			}%
	}
	\makeatother
	\DeclareDoubleDelim
		\SD\{\{\}\}
	\DeclareDoubleDelim
		\bbr[[]]
	% \DeclareDoubleDelim
	% 	\aar\langle\langle\rangle\rangle
	\makeatletter
	\newsavebox{\aar@content}
	\newcommand\aar{\@ifstar\aar@one@star\aar@plain}
	\newcommand\aar@one@star{\@ifstar\aar@resize{\aar@plain*}}
	\newcommand\aar@resize[1]{\sbox{\aar@content}{#1}\scaleleftright[3.8ex]
		{\Biggl\langle\!\!\!\!\Biggl\langle}{\usebox{\aar@content}}
		{\Biggr\rangle\!\!\!\!\Biggr\rangle}}
	\DeclareDoubleDelim
		\aar@plain\langle\langle\rangle\rangle
	\makeatother


	% \DeclarePairedDelimiterX{\aar}[1]{\langle}{\rangle}
	% 	{\nhphantom{\langle}{#1}\hspace{1.2pt}\delimsize\langle\mathopen{}#1\mathclose{}\delimsize\rangle\hspace{1.2pt}\nhphantom{\rangle}{#1}}

\relax %%%%% restatables and links
	% \usepackage{xstring} % for expandarg
	\usepackage{xpatch}
	\makeatletter
	\xpatchcmd{\thmt@restatable}% Edit \thmt@restatable
	   {\csname #2\@xa\endcsname\ifx\@nx#1\@nx\else[{#1}]\fi}% Replace this code
	   % {\ifthmt@thisistheone\csname #2\@xa\endcsname\typeout{oiii[#1;#2\@xa;#3;\csname thmt@stored@#3\endcsname]}\ifx\@nx#1\@nx\else[#1]\fi\else\csname #2\@xa\endcsname\fi}% with this code
	   {\ifthmt@thisistheone\csname #2\@xa\endcsname\ifx\@nx#1\@nx\else[{#1}]\fi
	   \else\fi}
	   {}{\typeout{FIRST PATCH TO THM RESTATE FAILED}} % execute on success/failure
	\xpatchcmd{\thmt@restatable}% A second edit to \thmt@restatable
	   {\csname end#2\endcsname}
	   {\ifthmt@thisistheone\csname end#2\endcsname\else\fi}
	   {}{\typeout{FAILED SECOND THMT RESTATE PATCH}}

	% \def\onlyaftercolon#1:#2{#2}
	\newcommand{\recall}[1]{\medskip\par\noindent{\bf \Cref{thmt@@#1}.} \begingroup\em \noindent
	   \expandafter\csname#1\endcsname* \endgroup\par\smallskip}

   	\setlength\marginparwidth{1.55cm}
	\newenvironment{linked}[3][]{%
		\def\linkedproof{#3}%
		\def\linkedtype{#2}%
		% \reversemarginpar
		% \marginpar{%
		% \vspace{1.1em}
		% % \hspace{2em}
		% 	% \raggedleft
		% 	\raggedright
		% 	\hyperref[proof:\linkedproof]{%
		% 	\color{blue!50!white}
		% 	\scaleleftright{$\Big[$}{\,{\small\raggedleft\tt\begin{tabular}{@{}c@{}} proof of \\\linkedtype~\ref*{\linkedtype:\linkedproof}\end{tabular}}\,}{$\Big]$}}
		% 	}%
        % \restatable[#1]{#2}{#2:#3}\label{#2:#3}%
		\ifmarginprooflinks
		\marginpar{%
			% \vspace{-3em}% %% for bottom
			\vspace{1.5em}
			\centering%
			\hyperref[proof:\linkedproof]{%
            % \hyperref[proof:#3]{
			\color{blue!30!white}%
			\scaleleftright{$\Big[$}{\,\mbox{\footnotesize\centering\tt\begin{tabular}{@{}c@{}}
				% proof of \\\,\linkedtype~\ref*{\linkedtype:\linkedproof}
				link to\\[-0.15em]
				proof
			\end{tabular}}\,}{$\Big]$}}~
			}%
		\fi
        \restatable[#1]{#2}{#2:#3}\label{#2:#3}%
        }%
		{\endrestatable%
		}
	\makeatother
		\newcounter{proofcntr}
		\newenvironment{lproof}{\begin{proof}\refstepcounter{proofcntr}}{\end{proof}}

		\usepackage{cancel}
		\newcommand{\Cancel}[2][black]{{\color{#1}\cancel{\color{black}#2}}}

		\usepackage{tcolorbox}
		\tcbuselibrary{most}
		\tcolorboxenvironment{lproof}{
			% fonttitle=\bfseries,
			% top=0.5em,
			enhanced,
			parbox=false,
			boxrule=0pt,
			frame hidden,
			borderline west={4pt}{0pt}{blue!20!black!40!white},
			% coltext={blue!20!black!60!white},
			colback={blue!20!black!05!white},
			sharp corners,
			breakable,
			% bottomsep at break=4cm,
			% enlarge bottom at break by=-4cm,
			% topsep at break=3cm,
			% enlarge top at break by=-3cm
		}
		% \usepackage[framemethod=TikZ]{mdframed}
		% \surroundwithmdframed[ % lproof
		% 	   topline=false,
		% 	   linewidth=3pt,
		% 	   linecolor=gray!20!white,
		% 	   rightline=false,
		% 	   bottomline=false,
		% 	   leftmargin=0pt,
		% 	   % innerleftmargin=5pt,
		% 	   skipabove=\medskipamount,
		% 	   skipbelow=\medskipamount
		% 	]{lproof}
	%oli16: The extra space was because there was extra space in the paragraph, not
	%because this length was too big. By breaking arrays, everything will be better.
	\newcommand{\begthm}[3][]{\begin{#2}[{name=#1},restate=#3,label=#3]}

\relax %TODOs and footnotes
    \newcommand{\TODO}[1][INCOMPLETE]{{\centering\Large\color{red}$\langle$~\texttt{#1}~$\rangle$\par}}
    \newcommand{\dfootnote}[1]{%
        \let\oldthefootnote=\thefootnote%
        % \addtocounter{footnote}{-1}%
		\setcounter{footnote}{999}
        \renewcommand{\thefootnote}{\textdagger}%
        \footnote{#1}%
        \let\thefootnote=\oldthefootnote%
    }
	\newcommand{\dfootnotemark}{
		\footnotemark[999]
	}


% \usepackage[framemethod=TikZ]{mdframed}
% \colorlet{color1}{Emerald}
% \colorlet{color3}{color1>wheel,2,3}
% \colorlet{pinkish}{color3!25!magenta}
\definecolor{brownish}{rgb}{0.5, 0.2, 0.1}
% \newmdenv[roundcorner=5pt, 
%     backgroundcolor=brownish!20!white,
%     % frametitle={$\langle$under construction$\rangle$},
%     frametitle={$\langle$incomplete or buggy$\rangle$},
%     frametitlerule=false,
%     innertopmargin=3pt, frametitlebelowskip=1ex, frametitleaboveskip=1ex,
%     frametitlebackgroundcolor=brownish!40!white,
%     skipabove=1em,skipbelow=1em,
%     frametitlefont={\scshape},leftmargin=-10pt, rightmargin=-10pt]
% 		{wip}
\newtcolorbox{wip}{%
    colback=brownish!20!white,%
    % frametitle={$\langle$under construction$\rangle$},
    title={$\langle$under construction$\rangle$},%
    enhanced jigsaw,
    breakable,
    % frametitlerule=false,
    % innertopmargin=3pt, frametitlebelowskip=1ex, frametitleaboveskip=1ex,
    colframe=brownish!40!white,%
    % skipabove=1em,skipbelow=1em,
    % frametitlefont={\scshape},leftmargin=-10pt, rightmargin=-10pt
}
        
% \tcolorboxenvironment{example}{
\newtcolorbox[use counter=example]{examplex}{
        fonttitle=\bfseries,
        % empty,
        enhanced jigsaw,
        title={Example \thetcbcounter.},
        before=\par\medskip\noindent,
        % top=0.5em,
        % enhanced,
        parbox=false,
        % boxrule=0pt,
        % frame hidden,
        % borderline west={4pt}{0pt}{green!20!black!40!white},
        % coltext={blue!20!black!60!white},
        % colback={blue!20!black!05!white},
        colback=white,
        sharp corners,
        breakable,
        % bottomsep at break=4cm,
        % enlarge bottom at break by=-4cm,
        % topsep at break=3cm,
        % enlarge top at break by=-3cm
    }

% \usepackage{parskip}

\setlength\parskip{1ex}
\usepackage[margin=1.5in]{geometry}

\newcommand\IncD{\mathit{IncDist}}

\title{A Probabilistic Account of Confidence Parameters in PDGs}
\author{Oliver E Richardson, Joseph Y Halpern}
\begin{document}
\maketitle


\textbf{Notation.}

\begin{tabular}{ll}
    $\V(X)$ & the set of possible values for a variable $X$ \\
    $\Delta \V$ & the set of all joint distributions over all variables \\
    $\overline{\mathbb R}$ & the set
        $ \mathbb R \cup \{\infty\}$ of extended reals
        % $\{ r \ge 0 : r \in \mathbb R\} \cup \{\infty\}$ of non-negative extended reals
\end{tabular}

\smallskip \hrule\smallskip

Let's consider two variants of pdgs. The first, which we'll call an o-pdg, is just what we had before. 
Concretely, an o-pdg $\dg M = (\mathcal P, \bbeta)$ is collection of conditional probability distributions (cpds) $\mathcal P = \{p\ssub L \}_{L \in \Ed}$, with associated weights $\bbeta = \{\beta\ssub L \in \overline{\mathbb R}\}_{L \in \Ed}$, typically positive, where $\beta\ssub L=0$ means the associated cpd $p\ssub L(Y|X)$ is totally ignored (and in particular is semantically equivalent to one in which $p\ssub L(Y|X)$ were replaced with a different cpd $p'(Y|X)$). 

While the scale makes sense at the extremes and is mathematiaclly well-motivated, the numerical value of $\beta \in [0,\infty]$ is in some ways hard to make sense of. 
How should we set $\beta$?  When, for instance, should we assign $\beta = 3$?

To shine some light on these questions, we will introduce a variant of a pdg, which we refer to as a j-pdg. A j-pdg $\mathcal M = (\mathcal P, \mat c)$ is again a collection of weighted cpds, but instead the weights $c \ssub L$ for each cpd $p\ssub L$ lie in the interval $[0,1]$, and we interpret them as the ``(independent) probability that the given edge is part of the pdg''.

In both cases, we can make sense of the ``set of joint distributions compatible with a pdg'' in the same way, as a function of $\mathcal P$ alone: 
\[
    \SD{\dg M} = \SD{\mathcal M} = \SD{\mathcal P} 
        := \Big\{ \mu \in \Delta\! \V : \mu(X,Y) = \mu(X)p\ssub L\!(Y|X) \text{ for all } 
            \ed LXY  \in \Ed \Big\}.
\]


% So, while the scoring semantics of an o-pdg 
However, we don't imagine all cpds are equal. 
In the case of o-pdgs, recall that we remedy this by introducing a scoring function semantics 
    % $\bbr{-} : \Delta\!\V \to \overline{\mathbb R}$, 
    $\Inc : \Delta\!\V \to \overline{\mathbb R}$, 
    which makes compatibility with matter of degree. 
Rather than simply being consistent or not, 
    % $\bbr{\dg M}(\mu)$
    $\Inc_{\dg M}(\mu)$
    is a measure of \emph{how incompatible} $\dg M$ is with $\mu$.
 For completeness, the formula is given by the weighted sum of relative entropies
 \[
    \Inc_{\dg M} = \sum_{L \in \Ed^{\dg M}} \beta\ssub L \kldiv{\mu}{p \ssub L}.
 \]

 A j-pdg represents a slightly different continuous generalization. 
 Rather than simply be consistent with a pdg or not, a j-pdg gives rise to a probability of compatibility $\mathrm{C}\!\Pr_{\mathcal M}: \Delta\!\V \to [0,1]$, a measure of \emph{how likely} the distribution $\mu$ is to be compatible with $\dg M$.
% \section{Theory of j-pdgs}
Concretely, given a j-pdg $\mathcal M$, the probabilistic confidences $\mat c$ generate a probability distribution over unweighted sub-pdgs $\dg N$, by
\[ 
    \Pr_{\mathcal M}( \dg N )  = \prod_{L \in \Ed}\Big( c\ssub L ~\text{ if }~ L \in \Ed^{\dg N} \text{ else }1 - c\ssub L \Big)
\]
with which we can describe the probability of $\mu$ being in the set of distributions consistent with a random pdg $\dg M$ as 
% $\displaystyle
\[
    \mathrm{C}\!\Pr_{\mathcal M}(\mu) := \Pr_{\mathcal M}\Big( \mu \in \SD{\dg N}\Big).
\]
% $

\subsection{Correspondence by Expectation}
For now, let's interperet an unweighted pdg $\dg N$ as having $\bbeta = \mat 1$, as we did in the original pdg paper.  For every distribution $\mu$, we can show that the expected incompatibility of $\mu$ with a random unweighted pdg drawn from $\Pr_{\mathcal M}$ is the same as the incompatibility with the o-pdg that has $\bbeta := \mat c$.

% We claim that

\vspace{-2ex}
\begin{linked}{prop}{exp}
    \vspace{2ex}
$\displaystyle\quad
    % \Inc_{\mathcal M}(\mu) = 
    \Ex_{\dg N \sim \Pr_{\mathcal M}} [ \Inc_{\dg N}(\mu) ] 
    = \sum_{L \in \Ed} c\ssub L  \kldiv\mu{p\ssub L}
        = \Inc_{
            % \mathrm{o-pdg}
            (\mathcal P, \mat c)}.
$
\end{linked}

While mathematically sound, what we have just done is slightly deceptive motivationally. 
In the original pdg paper, \emph{why} did we choose to interperet an unweighted pdg as having $\bbeta = \mat 1$? 
Because that was a reasonable default---an \emph{intermediate} value of $\beta$, somewhere between zero and infinity, and a convenient choice of scale because we don't have to write multiplication by one.
% But why did we choose it for the 
In our setting, though, this choice seems less appropriate: a j-pdg was supposed to represent a probability distribution over unweighted pdgs in the sense of ``full-confidence'', not the sense of ``default confidence''. 
From another angle: we have first implicitly identified an unweighted pdgs with one that has $\bbeta = \mat 1$ (default confidence), and then identified this pdg with the j-pdg that selects it with probability 1 ($\mat c = \mat 1$, high confidence). 
But these two numbers mean different things. 

 % is not a convenient default, but rather the upper extreme of confidence: it corresponds to believing that every edges is present in the pdg with probability 1. 

So, what happens if we instead begin by identifying an unweighted pdg with one that has absolute confidence in all edges, i.e., $\bbeta = \boldsymbol\infty$?  Well algebraically, we get the same result but with an additional factor of $\infty$\footnote{by $\infty$, we really mean the first non-finite ordinal $\omega$, which is why we calculate $0 \infty = 0$.}:
\[
    \Ex_{\dg N \sim \Pr_{\mathcal M}} \left[ \Inc_{\dg N}^{\text{high conf}}(\mu) \right] 
    = (\infty) \sum_{L \in \Ed} c\ssub L  \kldiv\mu{p\ssub L}
        = (\infty) \Inc_{
            % \mathrm{o-pdg}
            (\mathcal P, \mat c)}
        = \begin{cases}
            0 & \text{ if } \mu \in \SD{\mathcal P|_{c > 0}}\\
            \infty & \text{otherwise}\end{cases}~.
\]

So effectively, it just reproduces the set-of-distribution semantics (for those cpds that have positive probability $c$). 


\section{Having Both Confidences}
Now, let's consider the simultaneous incorporation of the two different weights.
Let a jo-pdg be a combination of all three parameters $\mathfrak M = (\mathcal P, \mat c, \bbeta)$,
    where $\mat c$ describes is a probabilistic description of confidence (as in a j-pdg), 
    and $\bbeta$ is an inconsistency as a matter of degree (as in an o-pdg).

What is the appropriate generalization of $\Inc$ and $\mathrm{C}\!\Pr$?
Once again, consider a distribution $\mu \in \Delta\!\V$.
Recall that $\Inc_{(\mathcal P, \bbeta)}(\mu)$ is the degree to which $\mu$ is incompatible with the o-pdg $\dg M = (\mathcal P, \bbeta)$, while $\mathrm{C}\!\Pr_{(\mathcal P, \mat c)}$ is the probability that it $\mu$ is compatible with the j-pdg $\mathcal M = (\mathcal P, \mat c)$. 
% way of representing both degree and likelihood of incompatibility of $\mu$ with our jo-pdg,   probability distribution of numerical values for the incompatibility $\mu$.
% The natural description of both degree and likelihood is a probability distribution over possible degrees of inconsistency.  
So, the appropriate generalization, which we call $\IncD_{\mathfrak M}(\mu)$ ought to be a (finitely supported) probability distribution over real numbers---namely, the one that models the process of sampling a subset of edges $\dg N \sim \Pr_{\mathcal M}$ and measuring the inconsistency of the induced subpdg.  
Formally, we have:

\begin{align*}
    \IncD_{\mathfrak M} 
        % &: \Delta\!\V \to \Delta \mathbb R^+
        % \\&= \mu \mapsto ~
        (\mu) &= \sum_{E \subset \Ed} \Pr_{\mathcal M}(E)\, \delta_{%
            \textstyle(\Inc_{\dg M |_E}(\mu))}
            % \qquad= \Inc_{-}^\#(\Pr_{\mathcal M})
            \\&= \sum_{E \subset \Ed} \Pr_{\mathcal P, \mat c}(E) \,\delta_{%
                \textstyle(\Inc_{(\mathcal P, \bbeta) |_E}(\mu))}
\end{align*}
where $\dg M |_E$ is the subpdg of $\dg M$ obtained by discarding all edges outside of $E$, and $\delta_{x}$ is the degenerate distribution that places all mass on $x$. 
{\color{gray}(Equivalently, $\Inc_{\mathfrak M}$ may be defined as the pushforward of the probability $\Pr_{\mathcal M}$ over induced sub-o-pdgs through the function that measures the incompatibility of an o-pdg. )}

There are two aspects of this distribution that are of particular interest to us (because they correspond to the semantics given by our two different flavors of pdgs): the expectation of this distribution, and the probability it places on the number zero. 

\begin{prop}
    Let $\mathfrak M = (\mathcal P, \bbeta, \mat c)$ be a jo-pdg, and $\mu \in \Delta\!\V$ be a joint probability over its variables. 
    Then the expectation of the incompatibility distribution is
    \begin{align*}
        \Ex\!\IncD_{\mathfrak M} 
            &= \mu\mapsto \sum_{L \in \mathcal E} \beta\ssub L c \ssub L \kldiv\mu{p\ssub L} 
            = \Inc_{(\mathcal P,\, \bbeta \odot \mat c)},
    \end{align*}
    \vspace{-1ex}
    where $\odot$ denotes the element-wise product of the two vectors.
\end{prop}

At the same time, 
% using ``$\Inc$'' to denote the random variable 
\begin{prop}
    \begin{align*}
        \IncD_{\mathfrak M}(\mu)(\Inc = 0) = 
        % \sum_{E \subset \Ed} \Pr_{\mat c} 
        \Pr_{\mathcal M}\Big( \Inc_{(\dg N,\bbeta)}(\mu) = 0\Big)
        \ge
        \mathrm{C}\!\Pr_{\mathcal M}(\mu) 
            % \Pr_{\mat c} \left\{ E : \sum_{L \in E} \beta_L = 0 \right\}.
    \end{align*}
    with equality if $\bbeta$ is strictly positive.
\end{prop}


Let's look at some elementary consequences of what we know so far, beginning with how o-pdgs and j-pdgs play out as special cases of jo-pdgs.
\begin{enumerate}
    \item If we regard an o-pdg $\dg M = (\mathcal P, \bbeta)$ as the jo-pdg $\mathfrak M := (\mathcal P, \mat 1, \bbeta)$ in which every edge appears with probability one, then 
    \begin{enumerate}
        \item  $\IncD_{\mathfrak M}(\mu)$ is the degenerate distribution $\delta_{\Inc_{\dg M}(\mu)}$ that places all mass on $\Inc_{\dg M}(\mu)$, the degree to which $\mu$ is incompatible with $\dg M$. 
        \item Consequently, the expected incompatibility $\Ex\!\IncD_{\mathfrak M}(\mu) = \Inc_{\dg M}(\mu)$ reduces to incompatibility with $\dg M$, while
        \item the probability of compatibility reduces to the indicator function for the set of distributions compatible with all cpds, i.e., 
        % $\mathrm{C}\!\Pr_{(\mathcal P, \mat 1)}(\mu) = \mathbbm1_{\SD{\mathcal P}}$.
        \[
            \mathrm{C}\!\Pr_{\mathclap{(\mathcal P, \mat 1)}}\,(\mu) = \mathbbm1_{\textstyle\SD{\mathcal P}}
            := \begin{cases}
                1 & \text { if }\mu \in \SD{p} \\ 0 & \text{ otherwise.}
            \end{cases}
        \]
    \end{enumerate}
    
    So to summarize: regarding an o-pdg as a jo-pdg, the aspect of $\IncD$ that corresponds to an o-pdg (its the expectation) just equals inc $\Inc$, while the probability that it equals zero just reflects the set-of-consistent-distributions semantics (restricted to those cpds with $\beta > 0$).
    
    \item As discussed above, there are two natural choices for how view a j-pdg $\mathcal M = (\mathcal P, \mat c)$ as a jo-pdg. We can either identify it with $\mathfrak M_1 := (\mathcal P, \mat c, \mat 1)$, or $\mathfrak M_\infty := (\mathcal P, \mat c, \boldsymbol\infty)$. 
    Now:
    \begin{enumerate}
        \item  $\IncD_{\mathfrak M_1}(\mu)$ is a distribution with support over various values between $0$ and $\Inc_{(\mathcal P, \mat 1)}$, while
            $\IncD_{\mathfrak M_\infty}(\mu) = (m) \delta_0 + (1-m) \delta_\infty$,
            where 
            \[
            m = \sum_{E \subset \Ed} \Pr_{\mat c}(E) \mathbbm1[ \mu \in \SD{\mathcal P|_E}]
            \]
            is the probability of drawing a set of edges from 
            % $\Pr_{(\mathcal P, \mat c)}$
            $\Pr_{\cal M}$
            that are consistent with $\mu$. 
            
        \item The expected inconsistency of $\mu$ with $\mathfrak M_1$ is the same as the inconsistency of an o-pdg with the same cpds and confidences $\bbeta := \mat c$.
        Meanwhile, the expected inconsistency of $\mu$ with $\mathfrak M_\infty$ is zero if $\mu$ is consistent with all cpds $p\ssub L$ with $c\ssub L> 0$, and $\infty$ if not.  That is, 
        \[
            \Ex\!\IncD_{\mathfrak M_1}(\mu) = \Inc_{(\mathcal P, \mat c)}(\mu)
                \qquad \text{and}\qquad
            \Ex\!\IncD_{\mathfrak M_1}(\mu) = \infty \mathbbm1_{\textstyle\SD{\mathcal P}}.
        \]
        \item The probability of zero is the same in both cases:
        \[
            % \mathrm{C}\!\Pr_{\mathclap{\mathfrak M_1}}(\mu) 
            % = \mathrm{C}\!\Pr_{\mathclap{\mathfrak M_\infty}}(\mu) 
            \IncD_{\mathfrak M_1}(\mu)(\Inc=0)
            = \IncD_{\mathfrak M_\infty}(\mu)(\Inc = 0)
            = \mathrm{C}\!\Pr_{\cal M}(\mu).
            %
        \]
    \end{enumerate}
    So to summarize: regarding a j-pdg as a jo-pdg with $\bbeta = \boldsymbol\infty$ gives a result symmetric to what we had above: the aspect of $\IncD$ which corresponds to j-pdg semantics (the probability of getting zero inconsistency) remains unchanged, while the aspect corresonding to o-pdg semantics (its expectation) just reflects the set-of-distribution semantics (restricted to those cpds with $c > 0$).
    
    However, the other way of regardng a j-pdg as a jo-pdg---with $\bbeta := \mat 1$---gives us a foothold for how to begin using j-pdgs to model o-pdgs.
    In particular, it gives us a way of viewing certain values of $\beta$ as equivalent to probabilities, at least in expectation.
\end{enumerate}

These examples justify a simplification of notation for jo-pdgs:
\begin{align*}
    \Inc_{\mathfrak M}(\mu) &:= \Ex\IncD_{\mathfrak M}(\mu) \\ 
    %\qquad\text{and}\qquad 
    \mathrm{CPr}_{\mathfrak M}(\mu) &:= \IncD_{\mathfrak M}(\mu)(\Inc = 0).
\end{align*}

% \section{Equivalence}
% 
% \begin{prop}
%     For every o-pdg $\dg M = (\mathcal P, \bbeta)$ with finite confidences $\bbeta < \infty$, 
%     there is a j-pdg $\mathcal M = (\mathcal P', )$    
% \end{prop}
% 
% \section{Qualitative Confidences}


\clearpage
\section{}

\[
    \begin{tikzcd}
        (\overline{\mathbb R})^{\Delta\!\V}
            & (\Delta \overline{\mathbb R})^{\Delta \!\V}
        \\  \{0,1\}^{\Delta\!\V}
            &\ [0,1]^{\Delta\!\V}
    \end{tikzcd}
\]

Of course, we have the natural inclusions 
\begin{align*}
    \iota &: \{0, 1\} \to [0,1] \\
\end{align*}


\appendix
\section{Proofs}
\recall{prop:exp}
\begin{lproof}\label{proof:exp}
\begin{align*}
    \Inc_{\mathfrak M}(\mu) &= \Ex_{\dg N \sim \Pr_{\mathfrak M}} [ \Inc_{\dg N}(\mu) ] \\
        &= \sum_{\dg N \subset \mathcal P} \Pr_{\mathfrak M}(\dg N)
            \Inc_{\dg N}(\mu) \\
        &= \sum_{E \subset \Ed} 
            \prod_{L \in \Ed}
            \Big( c\ssub L ~\text{ if }~ L \in E \text{ else }1 - c\ssub L \Big)
            \left(
                \sum_{L \in E}
                    \kldiv\mu{p\ssub L}
                \right) \\
        &= \sum_{E \subset \Ed} 
            \prod_{L \in \Ed \setminus E} (1-c_L)
            \prod_{L \in E} c_L
            \left(
                \sum_{L \in E}
                    \kldiv\mu{p\ssub L}
                \right) \\
    \intertext{collecting terms, we compute}
        &= \sum_{L \in \Ed} \kldiv\mu{p\ssub L} \left(
                \sum_{E : L \in E \subset \Ed}
                \prod_{i \in \Ed \setminus E} (1-c_i)
                \prod_{i \in E} c_i
        \right) \\
        &= \sum_{L \in \Ed} \kldiv\mu{p\ssub L} c_L \Bigg(
            \sum_{E \subset \Ed \setminus \{L\} }
            \prod_{i \in \Ed \setminus E \setminus \{L\} } (1-c_i)
            \prod_{i \in E} c_i
        \Bigg). \\
\intertext{Now, the key is to realize that the two products on the right are simply the probability of the edges $E$ being selected from the sub-j-pdg $\mathfrak M \setminus L$ which has the edge $L$ removed, so we have }
        &= \sum_{L \in \Ed} \kldiv\mu{p\ssub L} c_L \Bigg(
            \sum_{E \subset \Ed \setminus \{L\} }
            \Pr_{\mathfrak M \setminus L} ( E )
        \Bigg),\\
\intertext{but each subset $E$ of edges corresponds to a disjoint event, and together they exhaust all of the posibilities, so the sum of their probabilities must equal 1, giving us}
        &= \sum_{L \in \Ed} c_L \kldiv\mu{p\ssub L}
\end{align*}
as desired.
\end{lproof}

\end{document}
