\relax % Controls
    \newif\ifmarginprooflinks
    	\marginprooflinkstrue
    	% \marginprooflinksfalse

\relax % Standard Packages
    \usepackage[dvipsnames]{xcolor}
    % \usepackage[utf8]{inputenc}
    
    \usepackage{mathtools}
    \usepackage{amssymb}
		\DeclareMathSymbol{\shortminus}{\mathbin}{AMSa}{"39}
    
    % \usepackage{parskip}
    % \usepackage{algorithm}
    \usepackage{bbm}
	\usepackage{lmodern}
	% \usepackage{times}
    \usepackage{faktor}
    \usepackage{booktabs}
    \usepackage{graphicx}
    \usepackage{scalerel}
    \usepackage{enumitem}
    \usepackage{nicefrac}\let\nf\nicefrac

    % \usepackage{algorithm2e}
    % \usepackage{algorithm}
    \usepackage[noend]{algpseudocode}
    \usepackage{algorithm}

    % \usepackage{color}
    %\usepackage{stmaryrd}
    \usepackage{hyperref} % Load before theorems...
        \hypersetup{colorlinks=true, linkcolor=blue!75!black, urlcolor=magenta, citecolor=green!50!black}
\relax %%%%%%%%% GENERAL MACROS %%%%%%%%
    \let\Horig\H
	\let\H\relax
	\DeclareMathOperator{\H}{\mathrm{H}} % Entropy
	\DeclareMathOperator{\I}{\mathrm{I}} % Information
	\DeclareMathOperator*{\Ex}{\mathbb{E}} % Expectation
	\DeclareMathOperator*{\EX}{\scalebox{1.5}{$\mathbb{E}$}}
    
	\DeclareMathOperator{\supp}{\mathrm{supp}} % support
	\DeclareMathOperator*{\argmin}{arg\,min}
    
    \DeclarePairedDelimiterX{\infdivx}[2]{(}{)}{%
		#1\;\delimsize\|\;#2%
	}
	\newcommand{\thickD}{I\mkern-8muD}
	\newcommand{\kldiv}{\thickD\infdivx}
	\newcommand{\tto}{\rightarrow\mathrel{\mspace{-15mu}}\rightarrow}

    
    \newcommand{\Rext}{\mskip1mu\overline{\mskip-1mu\mathbb R\!}\,}
    \newcommand{\mat}[1]{\mathbf{#1}}

	\makeatletter
	\newcommand{\subalign}[1]{%
	  \vcenter{%
	    \Let@ \restore@math@cr \default@tag
	    \baselineskip\fontdimen10 \scriptfont\tw@
	    \advance\baselineskip\fontdimen12 \scriptfont\tw@
	    \lineskip\thr@@\fontdimen8 \scriptfont\thr@@
	    \lineskiplimit\lineskip
	    \ialign{\hfil$\m@th\scriptstyle##$&$\m@th\scriptstyle{}##$\hfil\crcr
	      #1\crcr
	    }%
	  }%
    	}
	\makeatother
	\newcommand\numberthis{\addtocounter{equation}{1}\tag{\theequation}}

%%% Macros for the theorem paper
\newcommand\bmu{\boldsymbol\mu}

\usepackage{amsthm,thmtools} % Theorem Macros
	\usepackage[noabbrev,nameinlink,capitalize]{cleveref}
    \theoremstyle{plain}
    \newtheorem{theorem}{Theorem}
	\newtheorem{coro}{Corollary}[theorem]
    \newtheorem{prop}[theorem]{Proposition}
    \newtheorem{conj}[theorem]{Conjecture}
    \newtheorem{claim}{Claim}
	\declaretheorem[numberwithin=theorem,name=Claim]{iclaim}
    \newtheorem{remark}{Remark}
    \newtheorem{lemma}[theorem]{Lemma}
    \theoremstyle{definition}
    % \newtheorem{defn}{Definition}
    % \declaretheorem[name=Definition]{defn}
    \declaretheorem[name=Definition, qed=$\square$]{defn}
    \declaretheorem[name=Example, qed=$\triangle$]{example}

	\crefname{defn}{Definition}{Definitions}
	\crefname{prop}{Proposition}{Propositions}
    \crefname{issue}{Issue}{Issues}

\relax %%%%% restatables and links
	% \usepackage{xstring} % for expandarg
	\usepackage{xpatch}
	\makeatletter
	\xpatchcmd{\thmt@restatable}% Edit \thmt@restatable
	   {\csname #2\@xa\endcsname\ifx\@nx#1\@nx\else[{#1}]\fi}% Replace this code
	   % {\ifthmt@thisistheone\csname #2\@xa\endcsname\typeout{oiii[#1;#2\@xa;#3;\csname thmt@stored@#3\endcsname]}\ifx\@nx#1\@nx\else[#1]\fi\else\csname #2\@xa\endcsname\fi}% with this code
	   {\ifthmt@thisistheone\csname #2\@xa\endcsname\ifx\@nx#1\@nx\else[{#1}]\fi
	   \else\fi}
	   {}{\typeout{FIRST PATCH TO THM RESTATE FAILED}} % execute on success/failure
	\xpatchcmd{\thmt@restatable}% A second edit to \thmt@restatable
	   {\csname end#2\endcsname}
	   {\ifthmt@thisistheone\csname end#2\endcsname\else\fi}
	   {}{\typeout{FAILED SECOND THMT RESTATE PATCH}}

	% \def\onlyaftercolon#1:#2{#2}
	\newcommand{\recall}[1]{\medskip\par\noindent{\bf \Cref{thmt@@#1}.} \begingroup\em \noindent
	   \expandafter\csname#1\endcsname* \endgroup\par\smallskip}

   	\setlength\marginparwidth{1.55cm}
	\newenvironment{linked}[3][]{%
		\def\linkedproof{#3}%
		\def\linkedtype{#2}%
		% \reversemarginpar
		% \marginpar{%
		% \vspace{1.1em}
		% % \hspace{2em}
		% 	% \raggedleft
		% 	\raggedright
		% 	\hyperref[proof:\linkedproof]{%
		% 	\color{blue!50!white}
		% 	\scaleleftright{$\Big[$}{\,{\small\raggedleft\tt\begin{tabular}{@{}c@{}} proof of \\\linkedtype~\ref*{\linkedtype:\linkedproof}\end{tabular}}\,}{$\Big]$}}
		% 	}%
        % \restatable[#1]{#2}{#2:#3}\label{#2:#3}%
		\ifmarginprooflinks
        % \reversemarginpar
		\marginpar{%
			% \vspace{-3em}% %% for bottom
			\vspace{1.5em}
			\centering%
			\hyperref[proof:\linkedproof]{%
            % \hyperref[proof:#3]{
			\color{blue!30!white}%
			\scaleleftright{$\Big[$}{\,\mbox{\footnotesize\centering\tt\begin{tabular}{@{}c@{}}
				% proof of \\\,\linkedtype~\ref*{\linkedtype:\linkedproof}
				link to\\[-0.15em]
				proof
			\end{tabular}}\,}{$\Big]$}}~
			}%
		\fi
        \restatable[#1]{#2}{#2:#3}\label{#2:#3}%
        }%
		{\endrestatable%
		}
	\makeatother
		\newcounter{proofcntr}
		\newenvironment{lproof}{\begin{proof}\refstepcounter{proofcntr}}{\end{proof}}

		\usepackage{cancel}
		\newcommand{\Cancel}[2][black]{{\color{#1}\cancel{\color{black}#2}}}

		\usepackage{tcolorbox}
		\tcbuselibrary{most}
		\tcolorboxenvironment{lproof}{
			% fonttitle=\bfseries,
			% top=0.5em,
			enhanced,
			parbox=false,
			boxrule=0pt,
			frame hidden,
			borderline west={4pt}{0pt}{blue!20!black!40!white},
			% coltext={blue!20!black!60!white},
			colback={blue!20!black!05!white},
			sharp corners,
			breakable,
			% bottomsep at break=4cm,
			% enlarge bottom at break by=-4cm,
			% topsep at break=3cm,
			% enlarge top at break by=-3cm
		}
		% \usepackage[framemethod=TikZ]{mdframed}
		% \surroundwithmdframed[ % lproof
		% 	   topline=false,
		% 	   linewidth=3pt,
		% 	   linecolor=gray!20!white,
		% 	   rightline=false,
		% 	   bottomline=false,
		% 	   leftmargin=0pt,
		% 	   % innerleftmargin=5pt,
		% 	   skipabove=\medskipamount,
		% 	   skipbelow=\medskipamount
		% 	]{lproof}
	%oli16: The extra space was because there was extra space in the paragraph, not
	%because this length was too big. By breaking arrays, everything will be better.
	\newcommand{\begthm}[3][]{\begin{#2}[{name=#1},restate=#3,label=#3]}

\relax % Writing Tools
    \newcommand{\TODO}[1][INCOMPLETE]{{\color{red}\hangindent=0.5cm\rightskip=0.8cm$\smash{\Big\langle}$~\texttt{#1}~\raisebox{-0.3ex}{${\Big\rangle}$}\hspace{-1.5cm}\par}}

\relax % Bibliography, etc
	\usepackage[american]{babel}
	\usepackage{csquotes}
	\usepackage[backend=biber, style=authoryear]{biblatex}
	\DeclareLanguageMapping{american}{american-apa}
	% \usepackage[backend=biber,style=authoryear,hyperref=true]{biblatex}
	\addbibresource{refs.bib}
	% \addbibresource{conf.bib}

	\DeclareFieldFormat{citehyperref}{%
	  \DeclareFieldAlias{bibhyperref}{noformat}% Avoid nested links
	  \bibhyperref{#1}}

	\DeclareFieldFormat{textcitehyperref}{%
	  \DeclareFieldAlias{bibhyperref}{noformat}% Avoid nested links
	  \bibhyperref{%
	    #1%
	    \ifbool{cbx:parens}
	      {\bibcloseparen\global\boolfalse{cbx:parens}}
	      {}}}

	\savebibmacro{cite}
	\savebibmacro{textcite}

	\renewbibmacro*{cite}{%
	  \printtext[citehyperref]{%
	    \restorebibmacro{cite}%
	    \usebibmacro{cite}}}

	\renewbibmacro*{textcite}{%
	  \ifboolexpr{
	    ( not test {\iffieldundef{prenote}} and
	      test {\ifnumequal{\value{citecount}}{1}} )
	    or
	    ( not test {\iffieldundef{postnote}} and
	      test {\ifnumequal{\value{citecount}}{\value{citetotal}}} )
	  }
	    {\DeclareFieldAlias{textcitehyperref}{noformat}}
	    {}%
	  \printtext[textcitehyperref]{%
	    \restorebibmacro{textcite}%
	    \usebibmacro{textcite}}}

	\DeclareCiteCommand{\brakcite}
	  {\usebibmacro{prenote}}
	  {\usebibmacro{citeindex}%
	   \printtext[bibhyperref]{[\usebibmacro{cite}]}}
	  {\multicitedelim}
	  {\usebibmacro{postnote}}

% \relax %% Other
%     %% Narowing
%     \usepackage{keyval}
%     \makeatletter
%     \define@key{setpar}{left}[0pt]{\leftmargin=#1}
%     \define@key{setpar}{right}[0pt]{\rightmargin=#1}
%     \define@key{setpar}{both}{\leftmargin=#1\relax\rightmargin=#1}
%     \makeatother
% 
%     \newenvironment{narrow}[1][]
%       {\list{}{\setkeys{setpar}{left,right}%
%          \setkeys{setpar}{#1}%
%          \listparindent=\parindent
%          \topsep=0pt
%          \partopsep=0pt
%          \parsep=\parskip}\item\relax\hspace*{\listparindent}\ignorespaces}
%       {\endlist}
%     % \newenvironment{abstract}
%     %     {\narrow[both=1in]\small}         
%     %     {\endnarrow}
