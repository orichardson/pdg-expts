\documentclass[twoside]{article}

\usepackage{aistats2023}

% If your paper is accepted, change the options for the package
% aistats2023 as follows:
%
%\usepackage[accepted]{aistats2023}
%
% This option will print headings for the title of your paper and
% headings for the authors names, plus a copyright note at the end of
% the first column of the first page.

% If you set papersize explicitly, activate the following three lines:
%\special{papersize = 8.5in, 11in}
%\setlength{\pdfpageheight}{11in}
%\setlength{\pdfpagewidth}{8.5in}

% If you use natbib package, activate the following three lines:
%\usepackage[round]{natbib}
%\renewcommand{\bibname}{References}
%\renewcommand{\bibsection}{\subsubsection*{\bibname}}

% If you use BibTeX in apalike style, activate the following line:
%\bibliographystyle{apalike}
%% TODO make pdg.sty file that allows you to import all PDG macros.
%%%%%%%%%%


\relax % Writing Tools
    \newcommand{\TODO}[1][INCOMPLETE]{{\color{red}\hangindent=0.5cm\rightskip=0.8cm$\smash{\Big\langle}$~\texttt{#1}~\raisebox{-0.3ex}{${\Big\rangle}$}\hspace{-1.5cm}\par}}


\relax
	\DeclareMathOperator*{\argmin}{arg\,min}
	\newcommand{\bundle}{\mathbin{+}}
    \newcommand{\Rext}{\mskip1mu\overline{\mskip-1mu\mathbb R\!}\,}

\relax
    %% Narrowing
    \usepackage{keyval}
    \makeatletter
    \define@key{setpar}{left}[0pt]{\leftmargin=#1}
    \define@key{setpar}{right}[0pt]{\rightmargin=#1}
    \define@key{setpar}{both}{\leftmargin=#1\relax\rightmargin=#1}
    \makeatother

    \newenvironment{narrow}[1][]
      {\list{}{\setkeys{setpar}{left,right}%
         \setkeys{setpar}{#1}%
         \listparindent=\parindent
         \topsep=0pt
         \partopsep=0pt
         \parsep=\parskip}\item\relax\hspace*{\listparindent}\ignorespaces}
      {\endlist}
    % \newenvironment{abstract}
    %     {\narrow[both=1in]\small}         
    %     {\endnarrow}


\relax % Bibliography
    \usepackage[backend=biber, style=authoryear]{biblatex}
    % \usepackage[backend=biber,style=authoryear,hyperref=true]{biblatex}
    \addbibresource{refs.bib}

    \DeclareLanguageMapping{american}{american-apa}
    % \renewcommand*{\nameyeardelim}{\addcomma\space}
    \DeclareDelimFormat{nameyeardelim}{\addcomma\space}
    % \listfiles

    \DeclareFieldFormat{citehyperref}{%
      \DeclareFieldAlias{bibhyperref}{noformat}% Avoid nested links
      \bibhyperref{#1}}

    \DeclareFieldFormat{textcitehyperref}{%
      \DeclareFieldAlias{bibhyperref}{noformat}% Avoid nested links
      \bibhyperref{%
        #1%
        \ifbool{cbx:parens}
          {\bibcloseparen\global\boolfalse{cbx:parens}}
          {}}}

    \savebibmacro{cite}
    \savebibmacro{textcite}

    \renewbibmacro*{cite}{%
      \printtext[citehyperref]{%
        \restorebibmacro{cite}%
        \usebibmacro{cite}}}

    \renewbibmacro*{textcite}{%
      \ifboolexpr{
        ( not test {\iffieldundef{prenote}} and
          test {\ifnumequal{\value{citecount}}{1}} )
        or
        ( not test {\iffieldundef{postnote}} and
          test {\ifnumequal{\value{citecount}}{\value{citetotal}}} )
      }
        {\DeclareFieldAlias{textcitehyperref}{noformat}}
        {}%
      \printtext[textcitehyperref]{%
        \restorebibmacro{textcite}%
        \usebibmacro{textcite}}}


\usepackage{tikz}
	\usetikzlibrary{positioning,fit,calc, decorations, arrows, shapes, shapes.geometric}
	\usetikzlibrary{cd}

	%%%%%%%%%%%%
	\tikzset{AmpRep/.style={ampersand replacement=\&}}
	\tikzset{center base/.style={baseline={([yshift=-.8ex]current bounding box.center)}}}
	\tikzset{paperfig/.style={center base,scale=0.9, every node/.style={transform shape}}}

	% Node Stylings
	\tikzset{dpadded/.style={rounded corners=2, inner sep=0.7em, draw, outer sep=0.3em, fill={black!50}, fill opacity=0.08, text opacity=1}}
	\tikzset{dpad0/.style={outer sep=0.05em, inner sep=0.3em, draw=gray!75, rounded corners=4, fill=black!08, fill opacity=1, align=center}}
	\tikzset{dpadinline/.style={outer sep=0.05em, inner sep=2.5pt, rounded corners=2.5pt, draw=gray!75, fill=black!08, fill opacity=1, align=center, font=\small}}

 	\tikzset{dpad/.style args={#1}{every matrix/.append style={nodes={dpadded, #1}}}}
	\tikzset{light pad/.style={outer sep=0.2em, inner sep=0.5em, draw=gray!50}}

	\tikzset{arr/.style={draw, ->, thick, shorten <=3pt, shorten >=3pt}}
	\tikzset{arr0/.style={draw, ->, thick, shorten <=0pt, shorten >=0pt}}
	\tikzset{arr1/.style={draw, ->, thick, shorten <=1pt, shorten >=1pt}}
	\tikzset{arr2/.style={draw, ->, thick, shorten <=2pt, shorten >=2pt}}

	\newcommand\cmergearr[5][]{
		\draw[arr, #1, -] (#2) -- (#5) -- (#3);
		\draw[arr, #1, shorten <=0] (#5) -- (#4);
		}
	\newcommand\mergearr[4][]{
		\coordinate (center-#2#3#4) at (barycentric cs:#2=1,#3=1,#4=1.2);
		\cmergearr[#1]{#2}{#3}{#4}{center-#2#3#4}
		}
	\newcommand\cunmergearr[5][]{
		\draw[arr, #1, -, shorten >=0] (#2) -- (#5);
		\draw[arr, #1, shorten <=0] (#5) -- (#3);
		\draw[arr, #1, shorten <=0] (#5) -- (#4);
		}
	\newcommand\unmergearr[4][]{
		\coordinate (center-#2#3#4) at (barycentric cs:#2=1.2,#3=1,#4=1);
		\cunmergearr[#1]{#2}{#3}{#4}{center-#2#3#4}
		}


\relax %% Double delimeters; I need this for pdg macros \aar and \bbr
    \newcommand{\nhphantom}[2]{\sbox0{\kern-2%
    \nulldelimiterspace$\left.\delimsize#1\vphantom{#2}\right.$}\hspace{-.97\wd0}}
    % \nulldelimiterspace$\left.\delimsize#1%
    % \vrule depth\dp#2 height \ht#2 width0pt\right.$}\hspace{-.97\wd0}}
    \makeatletter
    \newsavebox{\abcmycontentbox}
    \newcommand\DeclareDoubleDelim[5]{
    \DeclarePairedDelimiterXPP{#1}[1]%
        {% box must be saved in this pre code
            \sbox{\abcmycontentbox}{\ensuremath{##1}}%
        }{#2}{#5}{}%
        %%% Correct spacing, but doesn't work with externalize.
        % {\nhphantom{#3}{##1}\hspace{1.2pt}\delimsize#3\mathopen{}##1\mathclose{}\delimsize#4\hspace{1.2pt}\nhphantom{#4}{##1}}
        %%% Fast, but wrong spacing.
        % {\nhphantom{#3}{~}\hspace{1.2pt}\delimsize#3\mathopen{}##1\mathclose{}\delimsize#4\hspace{1.2pt}\nhphantom{#4}{~}}
        %%% with savebox.
        {%
            \nhphantom{#3}{\usebox\abcmycontentbox}%
            \hspace{1.2pt} \delimsize#3%
            \mathopen{}\usebox{\abcmycontentbox}\mathclose{}%
            \delimsize#4\hspace{1.2pt}%
            \nhphantom{#4}{\usebox\abcmycontentbox}%
        }%
    }
    \makeatother

\relax %%%%%%%%%   PDG  MACROS   %%%%%%%%
	\newcommand{\ssub}[1]{_{\!_{#1}\!}}
	% \newcommand{\bp}[1][L]{\mat{p}_{\!_{#1}\!}}
	% \newcommand{\bP}[1][L]{\mat{P}_{\!_{#1}\!}}
    \newcommand{\pdgunit}{\mathrlap{\mathit 1} \mspace{2.3mu}\mathit 1}

	\newcommand{\bp}[1][L]{\mat{p}\ssub{#1}}
	\newcommand{\bP}[1][L]{\mat{P}\ssub{#1}}
	\newcommand{\X}{\mathcal X}
	\newcommand{\V}{\mathcal V}
	\newcommand{\N}{\mathcal N}
	\newcommand{\Ed}{\mathcal E}
	\newcommand{\Ar}{\mathcal A}
    
    
    % \newcommand\Src[1]{X_{{#1}}}
    % \newcommand\Tgt[1]{Y_{{#1}}}
    % \newcommand{\Src}{\mathrm{Src}}
    % \newcommand{\Tgt}{\mathrm{Tgt}}
    % \newcommand\Src[1]{S\mskip-2mu\mathit{r\mskip-3muc}_{{#1}}}
    % \newcommand\Tgt[1]{T\mskip-5mu\mathit{g\mskip-1mut}_{{#1}}}
    % \newcommand\Src[1]{\mathsf{S}\mskip-2mu\vphantom{|}_{{#1}}}
    % \newcommand\Tgt[1]{\mathsf{T}\mskip-3mu\vphantom{|}_{{#1}}}
    \newcommand\Src[1]{S\mskip-2mu\vphantom{|}_{{#1}}}
    \newcommand\Tgt[1]{T\mskip-3mu\vphantom{|}_{{#1}}}
    
    \newcommand{\balpha}{\boldsymbol\alpha}
    \newcommand{\bbeta}{\boldsymbol\beta}

	\DeclareMathAlphabet{\mathdcal}{U}{dutchcal}{m}{n}
	\DeclareMathAlphabet{\mathbdcal}{U}{dutchcal}{b}{n}
	\newcommand{\dg}[1]{\mathbdcal{#1}}
	\newcommand{\PDGof}[1]{{\dg M}_{#1}}
	\newcommand{\UPDGof}[1]{{\dg N}_{#1}}
	\newcommand\VFE{\mathit{V\mkern-4mu F\mkern-4.5mu E}}

	\newcommand\Inc{\mathit{Inc}}
	\newcommand{\IDef}[1]{\mathit{IDef}_{\!#1}}
	\newcommand\OInc{\mathit{O\mskip-2.5muI\mskip-3.5mun\mskip-1.7muc}} % new version of Inc
	\newcommand\CInc{\mathit{C\mskip-3.1muI\mskip-3.5mun\mskip-1.7muc}} % new version of IDef
	% \newcommand{\SInc}{\mathit{S\mskip-1muI\mskip-1mun\mskip-1muc}} % new version of IDef
	% \newcommand{\ed}[3]{%
	% 	\mathchoice%
	% 	{#2\overset{\smash{\mskip-5mu\raisebox{-3pt}{${#1}$}}}{\xrightarrow{\hphantom{\scriptstyle {#1}}}} #3} %display style
	% 	{#2\overset{\smash{\mskip-5mu\raisebox{-3pt}{$\scriptstyle {#1}$}}}{\xrightarrow{\hphantom{\scriptstyle {#1}}}} #3}% text style
	% 	{#2\overset{\smash{\mskip-5mu\raisebox{-3pt}{$\scriptscriptstyle {#1}$}}}{\xrightarrow{\hphantom{\scriptscriptstyle {#1}}}} #3} %script style
	% 	{#2\overset{\smash{\mskip-5mu\raisebox{-3pt}{$\scriptscriptstyle {#1}$}}}{\xrightarrow{\hphantom{\scriptscriptstyle {#1}}}} #3}} %scriptscriptstyle
	\newcommand{\ed}[3]{#2%
	  \overset{\smash{\mskip-5mu\raisebox{-1pt}{$\scriptscriptstyle
	        #1$}}}{\rightarrow} #3}

	\newcommand{\bundle}{\mathbin{+}}
	\DeclareDoubleDelim
		\SD\{\{\}\}
	\DeclareDoubleDelim
		\bbr[[]]
	% \DeclareDoubleDelim
	% 	\aar\langle\langle\rangle\rangle
	\makeatletter
	\newsavebox{\aar@content}
	\newcommand\aar{\@ifstar\aar@one@star\aar@plain}
	\newcommand\aar@one@star{\@ifstar\aar@resize{\aar@plain*}}
	\newcommand\aar@resize[1]{\sbox{\aar@content}{#1}\scaleleftright[3.8ex]
		{\Biggl\langle\!\!\!\!\Biggl\langle}{\usebox{\aar@content}}
		{\Biggr\rangle\!\!\!\!\Biggr\rangle}}
	\DeclareDoubleDelim
		\aar@plain\langle\langle\rangle\rangle
	\makeatother


	% \DeclarePairedDelimiterX{\aar}[1]{\langle}{\rangle}
	% 	{\nhphantom{\langle}{#1}\hspace{1.2pt}\delimsize\langle\mathopen{}#1\mathclose{}\delimsize\rangle\hspace{1.2pt}\nhphantom{\rangle}{#1}}


% \author{$\{$Oliver E Richardson, Joseph Y Halpern, Christopher De Sa$\}$}

\begin{document}
% If your paper is accepted and the title of your paper is very long,
% the style will print as headings an error message. Use the following
% command to supply a shorter title of your paper so that it can be
% used as headings.
%
%\runningtitle{I use this title instead because the last one was very long}

% If your paper is accepted and the number of authors is large, the
% style will print as headings an error message. Use the following
% command to supply a shorter version of the authors names so that
% they can be used as headings (for example, use only the surnames)
%
%\runningauthor{Surname 1, Surname 2, Surname 3, ...., Surname n}

\twocolumn[

%joe1: I would cut the second part of the title
%  \aistatstitle{Inference in Probabilistic Dependency Graphs,\\
  %    via Exponential Cones and Otherwise}
    \aistatstitle{Inference in Probabilistic Dependency Graphs}

%joe1: initials need periods
    %\aistatsauthor{ Oliver E Richardson \And Joseph Y Halpern \And
    \aistatsauthor{ Oliver E. Richardson \And Joseph Y. Halpern \And
  Christopher De Sa } 
% \aistatsaddress{ Institution 1 \And  Institution 2 \And Institution 3 } 
\aistatsaddress{Cornell University \And Cornell University \And Cornell University}
]

\begin{abstract}
    We provide the first tractable inference algorithm for
    Probabilistic Dependency Graphs (PDGs) with discrete variables,
    thereby placing PDGs on asymptotically similar footing as other
%joe1
%    graphical models, such as Bayesian Networks and Factor Graphs.
    %    This may be surprising, because PDGs are more expressive than
        graphical models, such as Bayesian Networks and Factor Graphs,
despite the fact that PDGs are significantly more expressive than
other probabilistic graphical models.
%joe1: I have no idea what you mean by "a PDG inferencd algorithm can
%be used to calibrate a broad class of statistical models."  Since I
%don't think you discuss this issue anywhere in the paper, I just cut it.
%other probabilistic graphical models, and also because a PDG
%    inference algorithm can be used  
%    % for ``inconsistency minimization'', 
%    % which has been argued to be widely useful. 
%    % to resolve inconsistencies, which has  as a generic modeling task. 
%    % as a black box to train statistical models in ML.
%    to calibrate a broad class of statistical models.
%joe1: cut paragraph break
The key to our approach is combining 
%joe
%(1) our finding that inference in PDGs with bounded tree-width can
(1) the observation that inference in PDGs with bounded tree-width can
be reduced to a tractable linear optimization problem with exponential
cone constraints, 
%joe1
%with (2) a recent interior point method that can (provably) solve
with (2) a recent interior-point method that can solve
such problems efficiently (Dahl \& Anderson, 2022). 
%joe1: say something about how you do the evaluation; just showing how
%it does on random PDGs is nbot enough.  I don't think comparing it
%only to belief propagation is enough either.
%    We provide a concrete implementation and empirical evaluation.
    We evaluate our approach by ...
%joe1: I wouldn't worry about hte other approaches now.  There are
%more important issues hou have to deal with first.
%    In addition, we prove auxiliary results about complexity of this
    %    problem, and discuss other approaches to it.
We also characterize the complexity of various components of the
inference problem.
\end{abstract}


% \begin{narrow}
% %%-----------    A FRANK SUMMARY    ---------------
% Measuring / Estimating /  Inconsistency is very useful.
% For instance, (1) propogating it backwards through layers of computation = differentiable learning. 
% 
% Certain localized versions of it can be used to do other algorithms.

% How hard is it? 
% With interior point methods (convex programs with exponential cone constraints) we can do it in $O(n^4 \log n)$ time \& space, worst case for exact inference. So far, this means slightly harder than inference in Graphical models.
% \end{narrow}


% \tableofcontents

\section{INTRODUCTION}

%joe3
%Suppose we have a collection of probabilistic beliefs.
Suppose that we have a collection of probabilistic beliefs. 
%oli1: we answered the first two questions already
% Are they self-consistent? 
How can we tell if they are self-consistent?
How difficult is it to measure how inconsistent they are?
% If not, how far off ar they from being consistent?
How much computation is necessary to synthesize our beliefs into a single joint probability distribution?
This paper provides answers---both
theoretical and practical---to these questions. 
%joe3*: While these are interesting questions, they are not the
%standard questions tyhat have been asked when it comes to inference.
%since no other approach can capture inconsistency well, no one has
%asked the question of how inconsistent beliefs are (to the best of my
%knowledge). Moreover, the notion of inconsistency you're dealing
%with is an idiosyncratic notion, that you tailored to PDGs.   That
%means the leadoff paragraph does not situate this work well in the
%ocntext of what's been done.  It may be better to start with
%inference (what you denote as (Q)) and then move to inconsistency,
%rather than the other way around, as you've done.  I think that would
%make the story read better.


% More concretely, we handle
Probabilistic Dependency Graphs, or PDGs \parencite{pdg-aaai},
are a particularly flexible class of probabilistic graphical models, which subsumes Bayesian Networks (BNs) 
% and Markov Random Fields (MRFs).
and Factor Graphs (FGs). 
%joe1: much too wordy
%The primary force behind the expressiveness of pdgs is their ability
%to capture inconsistent beliefs, and the natural way of measuring the
%degree of this inconsistency that the formalism provides.
PDGs can capture inconsistent beliefs, and have an associated measure
%oli2: I cut "of degree" and added quotes
% of degree of inconsistency that captures how far a PDG is from being consistent.  
of ``inconsistency'' that quantifies how far a PDG is from being consistent.  
% Beyond its role in undergirding the semantics of pdgs,
% Beyond its role in providing the semantics of pdgs, 
% Beyond its central role in the development of PDG semantics, this inconsistency measurement also captures many standard loss functions and statistical divergences \parencite{one-true-loss}.
Beyond its central role in the development of PDG semantics, this measurement of inconsistency has proven to be a natural quantity to minimize.
%joe3
%As shown by \textcite{one-true-loss}, a wide breadth of
As shown by \textcite{one-true-loss}, many
    loss functions and statistical divergences 
    % used in practice
    % arise as the inconsistency measurement
    can be viewed as measuring the inconsistency
    of a PDG that models the appropriate context.  
Consquently, computing and minimizing it seems eminantly useful.
It follows, for instance, that the training process in machine
learning can largely be conceptualized as inconsistency minimization.  
% Furthemore, it is suggested that 
% might generalize to 

% also captures many standard loss functions and statistical divergences \parencite{one-true-loss}.
% All of this paints a story of wanting 
% All of this paints a story of wanting 
%joe1: none of what you said suggests that minimizing inconsistency is
%useful.  Specifically, the fact that Inc can be used to capture a
%number of loss functions and divergences says nothing about why we
%should care about minimizing inconsistency, nor does the fact that 
%the semantics is based on Inc.
% It seems that in a  minimizing inconsistency is a generically useful
 % modeling task.
 %joe1:
 %joe2*: I think here you need to say something about how you can use
 %inconsistency minimization to do inference.  The paper is about
 %inference, after all, and you haven't made the connection.  This is critical
%oli2: I agree, but I'm a little bit furstrated because it was structurally
% much closer to doing this before I accepted your %joe1 edits. 
% I'm rewriting the paragraph.  
% 
% However, the earlier work on PDGs does not provide any
% computational method for calculating whether a PDG is consistent and,
% if not, its degree of inconsistency.  We provide such methods in this paper.
%
%From a pragmatic point of view, though, PDGs are currently not yet
%very useful.  As it stands, they only have conceptual
%applications---one can use them to justify 
%a choice of loss function analytically, or to derive cute diagrammatic proofs
%of inequalitites you likely already know \parencite{one-true-loss},
%but it is impossible to compute with them.
%% What use is a model without an inference algorithm? 
%Until now, PDGs have been a model without an inference algorithm. 

% But how, in general
% However, none of the earlier work provides any 

But how does one \emph{calculate} this degree of inconsistency, in general?
% (let alone minimize it)?
Earlier work does not say. 
%
% Also, how does 
% In a similar vein, 
% while PDGs naturally model inconsistent beliefs
% and generalize other graphical models, i
%joe3: this seems unnecessary
%Setting aside for a moment this unique aspect of PDGs,
%it seems they are not even practical as graphical model, because,
%unlike the models that they
Moreover, unlike the graphical representations of uncertainty that they
generalize, 
% PDGs do not have an inference algorithm.
PDGs have not had an inference algorithm.
% That is to say, there is no guaranteed method for 
% There's no way, for instance, to answer probabilistic queries of the form:
In other words, it was not previously known how to use a PDG
%joe3: this seems irrelevant
%as a probabilistic model,
to answer quetions of the form:
% We  
% If I wanted to know the marginal probability of some variables $\mat Y$ according to a BN, 
\begin{equation}
    \begin{minipage}{0.80\linewidth}
    \it Given that the variables $\mat X$ take the value $\mat x$, 
    what is the distribution of the variables $\mat Y$? 
    \end{minipage}
    % \tag{$\mat Y|\mat X$}
%joe3: why are you using the highly nonstandard tag "Q".  I suppose it
%stands for query, but it is nonstandard.
    \tag{Q}
    \label{q:inf}
\end{equation}
% we don't know how to query  
%
% Although they generalize other graphical models, t
% We 
%joe3
%These two practical gaps
These two issues
% which have made PDGs a purely theoretical construct
are related, and we provide an algorithm
that addresses both.

%joe1: where do we do updating?   What inference problems do we
%consider?  You need to slow down here and explain what we do
%We analyze the complexity of inference and updating in pdgs, and show
%joe2: In the previous paragraph you talked about minimizing
%inconsistency.  Here you talk about the complexity of inference.  You
%have to make the connection.  
In more detail,
we analyze the complexity of inference in PDGs, and illustrate
the close relationship it has with inconsistency minimization.
% that it is equivalent to that of inconsistency minimization. 
%joe1*: I have no idea what exponential-cones constraints are.  Unless
%this is a completely standard notion in the AIStats community, you
%*must* give some intuition.  Also, when you talk about reducing the
%problem to a linear program, (a) I don't know which problem you're
%talking about and (b) we usually talk about reducing one problem to
%another, not reducing a problem to a linear program
%Then, we reduce the problem to a linear program with exponenital cones
% We then reduce the problem to a linear program with exponential-cones
% constraints.
%joe3
%Then, we reduce the problem to a convex optimization problem in standard
We then show that the inference problem can be reduced a convex
%joe3: what does "in standard form" mean?  Can you jsut cut it?
%optimization problem in standard form.
optimization problem.
%joe1
%We then use a powerful recent interior point method
This allows us to use powerful interior-point methods
that can solve such problems in polynomial time \parencite{dahl2022primal}. 
% We then lean heavily on recent work 
% \parencite{dahl2022primal} showing that 
% such problems can be
%joe1: 
%We provide a python implementation of our reduction, 
%and also several generic optimization baselines, and then show
We then evaluate our approach, showing 
% that this exponential-cone-based approach is more precise
%oli2:
%joe3: what did you change?
that this exponential-cone-based approach is more precise
%joe1: what do you mean by "optimization baselines"?
%oli1: it's hard to answer this directly until we've dug into the inconsistency a little.
% and, in some cases, faster than optimization baselines.
%joe3: if you can't answer the question, then this termoinology
%doesn't belong here.  Why throw out terms that you're not even
%prepared to define?
%and, often, faster than generic optimization baselines.
and often faster than generic optimization baselines.
%joe1: this belongs in the conclusion, or needs to be rewritten to
%say "while not currently as fast as inference methods such as belief
%propagation on the models to which belief propagation can be applied,
%we are optimistic that further improvements to our methnod are
%possible.  In any case, our results show that inference on pdgs is feasible.
While not currently as fast as inference methods such as belief
propagation on the models to which belief propagation can be applied,
we are optimistic that further improvements are possible.
In any case, our results show that inference in PDGs is feasible.
%However, this approach that our algorithms are not as fast as exact
%inference methods for existing graphical models, such as beleif
%propogation. 
% However, their asymptotics are not much worse.

\section{PRELIMINARIES}

\textbf{Vector notation.}
% This paper concerns the 
% Unless otherwise specified, all scalar quantites range over the extended reals $\Rext := \mathbb R \cup \{\infty\}$. 
For us, a vector is a map from a finite set to the extended reals
    $\Rext := \mathbb R \cup \{\infty\}$. 
The notation $\mat u := [u_i]_{i \in S}$ defines a vector over the
finite set $S$.
%joe3: I would still prefer to get rid of superscripts.  They're scary.
We will sometimes use superscripts as well, especially when indices
depend on one another. For example, if $\mathcal X$ is a finite set of
finite sets, then 
%joe2: (1) why is this a disjoint union?  You never said that the sets in
%\X were disjoint.  (2) You don't want to include X \in \X and x
%\n X in the notation; it's really ugly.  I would slightly prefer
%u_{x,X}$.  (3) Technically, if it's a vector, you have to specify the
%order of the elements, and the notation doesn't do that.
%oli2: (1) This is one construction of the disjoint union. It doesn't
%matter if the sets 
% X \in \X are disjoint; even if x is a member of X1 and X2, the indices (X1, x) and (X2,x) will be different.  (2) I agree that it's a little bit ugly, but I think leaving it out is far more confusing.  (3) Not necessarily.  Just because the standard basis (e_1, ... e_n) has an order doesn't mean we have to provide an order if we use a different basis. Sure, we need an order to write down a concrete vector without reference to the basis elements, but we won't need to do that. 
% $\mat u := [u^X_x]^{X \in \mathcal X}_{x \in X}$ defines a vector whose indicies range over the disjiont union $\sqcup \mathcal X$.
%oli2: here's a compromise
% $[u^X_x]_{x \in X, X \in \X}$ denotes a vector which has an element
% for each $X \in \X$ and $x \in X$.
%joe3: I still don't like having both a subscript and a superscript.
%How often do you even use this notation?
%$[u^X_x]^{X \in \mathcal X}_{x \in X}$ denotes a vector which has an element
$[u^X_x]^{X \in \mathcal X}_{x \in X}$ denotes a vector that has an element
%oli2: futher clarifying now that you've stripped the "disjoint union" out...
% for each $X \in \mathcal X$ and $x \in X$.
%joe3
%for each pair $(X,x)$, satisfying $x \in X \in \mathcal X$.
for each pair $(x,X)$ such that $x \in X \in \mathcal X$.
% for each $(X, x)$ 
%joe2: now you're really getting into the weeds.  This is a paper
%about inference, not notation.
%oli2: ok, although I think it's clearer to spell it out.
%It is equivalent to $\mat u := [u_{(x,X)}]_{x \in X, X \in \mathcal X}$, but more compact.
%joe2: I don't know what "the subspace where the upper index is $X$"
%means.  If you really need this notation, you need to explain it
%better.  But this is really the wrong place for it.
%oli2: if you can point out a better place for it later, that's ok.
%joe3: The "better place" is where you first use this notation.  Even
%better would be to get rid of the notation altogether
% But I think this notation will be a distraction once we get into the actual
% story we want to tell.
% Supplying just the upper index, $\mat u^{X}$ is the projection of $\mat u$ onto the subspace where the upper index is $X$. 
% Supplying just the upper index, $\mat u^{X_0} := [u^{X_0}_x]_{x \in {X_0}}$
% is the projection of $\mat u$ onto the subspace whose upper index is $X_0$. 
By supplying just the upper index of such a vector, as in $\mat u^{X_0}$,
we mean $[u^{X_0}_x]_{x \in {X_0}}$, the projection of $\mat u$ onto the subspace whose upper index is $X_0$. 
Vectors over the same set can be added (+) and partially ordered
($\le$) pointwise as usual; pointwise multiplication is denoted by
$\odot$.   
$\mat 1$ denotes an all-ones vector, whose dimension will always be clear in context.
$\mat u^{\sf T}$ denotes the transpose of $\mat u$, which we use
%joe3
%primarily to denote the inner product $\mat u^{\sf T} \mat v$.
primarily to define an inner product $\mat u^{\sf T} \mat v$. 
% If $\mat u : A \to \Rext$ is a vetor over $A$, and $\mat v : A \times B \to \Rext$,
% then $\mat u \otimes \mat 1 : A \times \Gamma \to \Rext$, by 
% $(\mat u \otimes \mat 1)(a,\gamma) := \mat u(\gamma)$
If $\mat u = [u_a]_{a \in A}$ is a vector over $A$ and $\mat v = [v_b]_{b \in B}$ is a vector over $B$, then $\mat u \mathbin{\otimes} \mat v := [ u_a \cdot v_b ]_{a \in A, b \in B}$ is a vector over $A \times B$.

% {\color{red}\tt
% TODO: unexplained notation / concepts
% \begin{enumerate}[nosep]
% \raggedright
% \item tensor product $\otimes$  (TODO: nix it)
% \item relative entropy $\kldiv\mu\nu$, conditional entropy $\H(Y|X)$
% \end{enumerate}
% }

\textbf{Probabilities.}
We write $\Delta S$ to denote the set of probability distributions over a finite set $S$.
Every variable $X$ can take on a finite set $\V(X)$ of possible values. 
% If $S$ is a finite set, we write $\Delta S$ for the set of probability distributions over $S$, i.e., the simplex over its elements. 
A conditional probability distribution (cpd) $p(Y|X)$ is a map 
%joe2
%$p : \V(X) \to \Delta \V(Y)$, so it assigns, to every $x \in \V(X)$, a
$p : \V(X) \to \Delta \V(Y)$, so it assigns to each $x \in \V(X)$ a
probability distribution $p(Y|x) \in \Delta Y$, which is shorthand for $p(Y|X\!\!=\!x)$.
Given a joint distribution $\mu$ over many variables including both $X$ and $Y$, 
%joe1: Is this standard notation for a marginal?  \mu)(X) looks like
%the probability of X to me.
%oli1: I'm pretty sure it's standard; at the very least, it agrees
%with the standard notation:  If you had Pr(X,Y), and you wanted to
%talk about the probability of X, you would write Pr(X), which is also
%the marginal of the distribution \Pr.
%joe3: I would never write \Pr(X) (and would also never write \Pr(X,Y)
%to denote the joint distribution; I find this horrible notation),
%confounding the argument to a probability with the description of
%something like the domain of the probability)
we write $\mu(X)$ for its marginal distribution on $X$, 
% $\mu(X,Y)$ for the 
and $\mu(Y|X)$ for the cpd obtained by first conditioning on $X$ and then marginalizing to $Y$. 
We measure information in a distribution with entropy $\H(\mu) := \Ex_{\mu} [\log \frac1\mu]$ and conditional entropy $\H_\mu(Y|X) := \Ex_\mu[\log\frac1{\mu(Y|X)}]$, where $X$ and $Y$ are variables.
% \textbf{Graph Theory.}

% \textbf{Inference for Graphical Models.}
% % A graphical model is a graph whose vertices correspond to 
% % 
% % There is a natural equivalence between hyper-graphs and bipartite graphs
% % \[
% % \]


\textbf{Graphs.}
%joe2: what's the INTUITION for a hyperedge?  
%oli2: I don't get why this is necessary. At this point it's just an
%analogue of a 
% graph. Would you want me to give intuition for what an edge of a
% graph means in  
% general, if it were slightly less standard?  It's useful generally. 
%joe3: For people not familiar with hypergraphs, perhaps an example of
%why we want hyperedges would be very helpful.  Having a good
%intuition helps people think about things.  People are familiar with
%graphs and edges, so no need to give intuition for it.  But I
%certainly do give intuition for edges when I teach graphs in CS 2800
%joe3: why is \V a set and \E a collection?
%A hypergraph $G = (V, \Ed)$ is a set $V$ of vertices, and a collection
A hypergraph $G = (V, \Ed)$ is determined by a set $V$ of vertices
and a set
$\Ed$ of \emph{hyperedges}, which correspond to subsets of $V$.  
%joe3
%An ordinary graph may be regarded as the special case in which every
An ordinary graph can be viewedas the special case of a hypergraph in
which every 
hyperedge contains exactly two vertices. 
% There is a natural bijection between hyper-graphs and bipartite graphs.

\begin{defn}
    A \emph{directed hypergraph} $G = (N, \mathcal A)$ is a set $N$ of
%joe3
%    nodes, and a collection $\mathcal A$ of \emph{hyperarcs}; each $a
    nodes and a set $\mathcal A$ of \emph{hyperarcs}; each $a
    \in \mathcal A$  
%joe3
%    is associated with a set $\Src a \subset N$ of source nodes, and  a
%    set $\Tgt a \subset N$ target nodes.  
    is associated with a set $\Src a \subseteq N$ of source nodes and a
    set $\Tgt \subseteq N$ of target nodes.  
\end{defn}
A directed graph is just a directed hypergraph where the source and target sets of every hyperarc are singletons. 
%joe2
%As one might hope, we can form hypergraph from a directed hypergraph
We can form a hypergraph from a directed hypergraph
by ``forgetting the direction of the arrow'', and taking the hyperedge
to be the union of the source and target sets. 
% There is also a natural bijection between directed hypergraphs and directed bipartite graphs. 

% Given a hyper-graph $(\N, \Ed)$,
Many problems that are intractable for general graphs
are tractable when restricted to trees;
some graphs are closer to trees than others. 
%
%joe3
%A tree decomposition of a (hyper)graph $G = (V, \Ed)$ is a tree
A \emph{tree decomposition} of a (hyper)graph $G = (V, \Ed)$ is a tree
$(\mathcal C, \mathcal T)$ whose vertices $C \in \mathcal C$, called  
%joe2
%``clusters'', are subsets of $V$ such that:
\emph{clusters}, are subsets of $V$ such that: 

\begin{enumerate}[itemsep=0pt]
    % \item The union $\bigcup \mathcal C$ of all clusters contains all vertices of $G$;
    % \item Every vertex $v \in V$ lies in at least one cluster,
    % \item Every hyper-edge $E\in \mathcal E$, there is a 
        % cluster $C \in \mathcal C$ that contains $E$, and
%joe3
  %\item Every vertex $v \in V$ and every hyperedge $E \in \Ed$ is
\item every vertex $v \in V$ and every hyperedge $E \in \Ed$ is
        contained in at least one cluster; 
    % \item For every vertex $v \in V$, the subgraph induced by restricting to clusters that contain $v$ is connected.
%joe2: If there'sa  standard definition, you should use that.  If not,
%use whichever one is more useful in terms of proving results.  if you
%use both, state one, and a proposition saying they're equivalent,
%with a reference.
%joe3: you should give a reference for tree decomposition
%      \item Every cluster $D$ along the unique path from $C_1$ to
%          $C_2$ in $\cal T$, 
      \item every cluster $D$ along the unique path from $C_1$ to
          $C_2$ in $\cal T$
contains $C_1 \cap C_2$. 
    % \item[2'.] {\color{blue}
    %     Equivalently, 
    %         \emph{ the running intersection property:}
    %         Every cluster $D$ along the unique path from $C_1$ to $C_2$ in $\cal T$,
    %         contains $C_1 \cap C_2$. 
    %     }
    % 
    %     \TODO[Which is prefereable?, 2 or 2'?]
    % \item Every hyper-edge $E\in \mathcal E$ is contained in some
    %     cluster $C \in \mathcal C$. 
\end{enumerate}

The \emph{width} of a tree decomposition is one less than the size of
its largest cluster,
%joe3
%and the \emph{treewidth} of a (hyper) graph $G$ is the smallest
and the \emph{treewidth} of a (hyper)graph $G$ is the smallest
possible width of any tree-decomposition of $G$. 
It is NP-hard to determine the tree-width of a graph, but fortunately, if the tree-width is known to be at most some constant, a tree-decomposition may be constructed in linear time \parencite{bodlaender1993linear}.




\textbf{Inference In Standard Graphical Models.}
% Fo
% The trick to doing inference quickly is not to ever represent the the full join
% In the exact form of belief propogation
% When belief propogation is used 
% Belief propogation when run on trees, 
% Message-passing algorithms such belief propogation, when applied trees, run in linear time and are provably correct.
% Running these same algorithms on 
% graphs that are not trees, such as \emph{loopy} belief propogation,
% may not converge, and even if it does, may be incorrect, or even inconsistent \parencite{wainwright2008graphical}. 
%joe2
%Message-passing algorithms such belief propogation, when applied
%oli2: what's wrong with "message-passing algorithms?" I wanted to be
%more precise.
%joe3: What's wrong with it is that you're introducing notions out of
%the blue.  It's OK if 90% of the audience knows what it is, which may
%be true for message-passing algorithms, but if you don't need to
%introduce it, why bother?
% There are some inference algorithms (such as belief propogation) that,
% Message-passing algorithms, such as belief propogation, 
Given a probabilistic model $\cal M$, which represents a joint distribution $\Pr_{\cal M}$, the goal of an inference algorithm 
is to respond to probabilistic queries
\eqref{q:inf} with $\Pr_{\cal M}(\mat Y \mid \mat X \!=\! \mat x)$. 
%joe3
%Many inference algorithms (such as belief propogation),
Many inference algorithms (such as belief propagation), 
%oli2: not just BNs, but graphical models generally
% when applied to BNs that are trees, run in linear time and are
% when applied to graphical models whose underlying structure is a tree,
%joe3: what does it mean to be "tree-shaped"?
%when applied to tree-shaped graphical models,
when applied to ``tree-like'' graphical models,
% are provably correct and 
run in linear time and are
provably correct.  
%joe2: if it's the same algorithm, why does it have a different name
%oli2*: because "belief propogation" run on other models that are not trees
% is ambiguous. Sometimes, it refers to the process of first constructing a 
% tree decomposition, and other times it refers to the process of
%Running these same algorithms on
%graphs that are not trees, such as \emph{loopy} belief propogation,
%oli2: I dislike your rewrite. How can we reference the same algorithm,
% when we started with "there are some algorithms"?
% If the same algorithm is run on BNs that are not trees, then it 
% may not converge, and even if it does, 
%oli2: a compromise? 
%joe3: this is not good.  You've never mentioned "loopy belief
%propagation"  before.  Moreover, "same algorithm" is hard to parse.
%The same as what
%If the same algorithms are na{\"i}vely applied to graphs with cycles
If an algorithm that works for a tree-like graph is na{\"i}vely
applied to graphs with cycles 
%joe3: this loks like loopy belief propagation (not propogation) is
%an instance of a graph with cycles
%(such as loopy belief propogation),
%then they may not converge, and even if they do, 
then it may not converge, and even if it does, 
%joe2: what does it man to be inconsistent?
%oli2: litterally the same thing we mean. It may find a collection of marginals
% for which there is no joint distribution that has those marginals. I think it's
% important to keep this in the paper. 
%may be incorrect, or even inconsistent \parencite{wainwright2008graphical}.  
% it may not give the correct answer \parencite{wainwright2008graphical}.  
%joe3*: As you've defined a query in (Q), it does not reutrn a
%collection of marginals, so "inconsistent" is meaningless.  You
%either have to give a better explanation or (my preference) cut it.
may give an answer that is incorrect, or even inconsistent
\parencite{wainwright2008graphical}.  
Nearly all exact inference algorithms for graphical models (variable
%joe3: what is "devision"?
elimination, clique-tree calibration, message-passing with devision,
%clique tree optimization),
clique-tree optimization),  
implicitly or explicitly, effectively construct a tree-decomposition of the model, and may be viewed as running on a tree \parencite[\S9-11]{koller2009probabilistic}.
%oli2: added. 
This is essentially necessary, because under widely believed
hardness assumptions, the only class of graphical models for which
inference is \emph{not} NP-hard is those that have bounded treewidth
\parencite{chandrasekaran2012complexity}. 
%joe2: you need to give some references and examples of algorithms here
%oli2: done, although I think it's overkill. See above. 

%joe2*: I just fell off a cliff here.  Are you given a BN and then
%consitruct a clique-tree which somehow (how?) encodes all answers to
%all possible queries?  How does it do that?  I'm lost.  You need to
%SLOW DOWN here.
%oli2: that's the general idea, yes. 
% For belief propogation in particular, is possible to save time by
% distilling the answers to all possible queries via a data-structure
% called a \emph{clique tree} 
For fixed evidence $\mat X \!=\! \mat x$, it is possible, with very
%joe3: is this what you meant?
%little overhead, to distill the answers to all queries about variables
%$\mat Y$ in a data-structure called a \emph{clique tree} 
little overhead, to summarize the answers to all queries about variables
$\mat Y$ using a data-structure called a \emph{clique tree} 
% clique tree calibration
\parencite[\S10]{koller2009probabilistic}, which is a tree decomposition $(\cal C, T)$
of the underlying model structure, together with a family $\bmu =
%joe3: I would definitely write \mu_C here, not (the misleading) \mu(C)
%\{\mu(C)\}_{ C \in \mathcal C}$ of probability distributions over
%every cluster.   
\{\mu(C)\}_{ C \in \mathcal C}$ of probability distributions.
A clique tree is said to be \emph{calibrated} if neighboring
%joe3: what does it mean for a cluster to have a belief?
%clusters's beliefs agree on the variables they share,%
clusters' beliefs agree on the variables they share,% 
    \footnote{i.e., if it is consistent, when viewed as a PDG}
in which case it determines a joint distribution by
\begin{equation}
%joe3
  %  \Pr_{\bmu} = \faktor
    {\Pr}_{\bmu} = \faktor
        {\prod_{\mathclap{C \in \cal C}} \mu_C(C)~}
%joe3
        %        {~\prod_{\mathclap{(C,D) \in \cal T}} \mu_{C}(C \cap D)}
                {~\prod_{\mathclap{(C,D) \in \cal T}} \mu_{C}(C \cap D)}.
    \label{eq:cliquedist}
\end{equation}
%joe3
%which has the property that $\Pr_{\bmu}(C) = \mu_C$ for $C \in \cal C$. 
This distribution has the property that $\Pr_{\bmu}(C) = \mu_C$ for $C
\in \cal C$.  
To see why this summarizes query information in a simple case, note that if $\mat Y$ is contained in a single cluster $C$, then $\mu_C(\mat Y) = \Pr_{\bmu}(\mat Y) = \Pr_{\cal M}(\mat Y | \mat X \!=\! \mat x)$.
Note also that, in the extreme case where $\mathcal C$ contains only
one cluster with all variables, a clique tree is just a joint
%joe3: Why use the nonstandard terminology "distills"?  What I wrote
%is shorter, simpler, and (in my opinion) clearer
%distribution, and distils inference about probabilistic queries in the
%sense that finding any marginal distributions 
distribution; in this case, finding a marginal distribution
amounts to computing a sum.  

\textbf{Probabilistic Dependency Graphs.}
% \textbf{PDGs.}
% \textbf{PDGs.}
%joe1: you should decide whether you're going to write PDG or pdg.
%I'm OK either way, but you have to be consistent.
%We now give a quick overview of the PDG formalism,
%following the more carefully motivated
%expositions of \textcite{pdg-aaai,one-true-loss}.
% We now give a quick overview of PDGs; the reader is encouraged to consult 
We now give a quick overview of PDGs. Our presentation is slightly
%joe3
%different than (but equivalent to) that of
different from (but equivalent to) that of 
%joe2: Does your AIStats paper add more useful intuition?  If not,
%it's enough (and better) just to reference the AAAI paper.
%oli2: only a little bit. I'll drop the reference here. 
\textcite{pdg-aaai}, which 
the reader is encouraged to consult for more details and intuition.
% We opt for a slightly different presentation, 
 % which the first work shows to be equivalent.
% We give a slightly different, but equivalent presentation.
%following the more carefully motivated
%expositions of \textcite{pdg-aaai,one-true-loss}.
%oli1
% A probabilistic dependency graph (pdg)
At a high level, a PDG
 % is just a collection of cpds, weighted by two kinds of confidence. More precisely:
is just an arbirary collection of cpds and causal assertions,
    weighted by confidence. More precisely:

\begin{defn}
%joe3
%  a PDG $\dg M = (\N, \Ar, \mathcal P, \balpha, \bbeta )
A PDG $\dg M = (\N, \Ar, \mathcal P, \balpha, \bbeta )
  % = (\mathcal P, \balpha, \bbeta)$ 
    $
    % over $\N$ is a set $\Ed$ of ``directed hyper-edges'', 
    % each $L \in \Ed$ of which is associated with:
%joe2: we didn't talk about hypergraphs in the AAAI paper.  Do you
%talk about them in the AIStats paper?
%oli2: no, this is a different presentation. I think it's cleaner in this
% case because I already had to bring up the hyper-graphs to talk about 
% tree-decompositions intelligently.
    is a directed hypergraph  $(\N, \Ar)$, whose nodes correspond to variables, and
    each $a \in \Ar$ is associated with:
    \begin{itemize}[itemsep=0pt]
        % \item (subsets of) variables $\Src L, \Tgt L \subset \N$, indicating the respective source and target variables of the edge;
        % \item variables $\Src L, \Tgt L \in \N$, the source and target of $L$;
        % \item subsets $\Src L, \Tgt L \subset \N$, which are source and target variables of the edge $L$. For example,   
        %     $$\Src L = \{A, B\} \ed L{}{} \{C\} = \Tgt L$$
        %  intuitively represents a joint dependence of $C$ on the variables $A$ and $B$;
        \item a cpd $p\ssub a (\Tgt a | \Src a)$ on the target variables given the source variables,
        \item a weight $\beta_a \in \Rext$ indicating 
            the modeler's confidence in the cpd $p\ssub L(\Tgt a | \Src a)$, and 
        \item a weight $\alpha_a \in \mathbb R$ indicating 
            the modeler's confidence that the arrow $a$ corresponds to an independent mechanism that determines $\Tgt a$ given $\Src a$. 
        \qedhere
    %     % \item $\mathcal P = \{ p\ssub L (\mat T_L | \mat S_L) \}_{L \in \Ed}$ is an indexed set of cpds   
    %     \item $\bbeta$ 
    \end{itemize}
\end{defn}

%joe3
%One selling point of PDGs is their modularity: if $\dg M_1$ and $\dg
One advantage of PDGs is their modularity: if $\dg M_1$ and $\dg
%joe3
%M_2$ are two PDGs, we can take the dusjoint union of their arcs (and
M_2$ are two PDGs, we can take the disjoint union of their arcs (and
associated data) to get a new PDG, denoted $\dg M_1 + \dg M_2$, 
%joe2: You need intuition both for what the union represents and more
%intjuition for incompatibility.  
%oli2: added this:
which represents the combined information of both $\dg M_1$ and $\dg M_2$.

% For the purposes of adding data to PDGs in this way, we implicitly convert cpds to singleton PDGs that have default weight $\beta = 1$. 

%joe1: You need to add a few sentences of intuition here, giving an
%examples of low and high incompatibility, and explaning that D acts
%as a measure of distance.  Have pity on the poor reader!  Don't be afraid
%to slow down and explain things.
% \TODO[intuition]

The semantics of a PDG are given by two scoring functions over joint distributions $\mu(\N)$ over all variables.
The \emph{incompatibility} of $\mu$ with a PDG $\dg M$, which
measures the discrepancy between $\mu$ and the cpds of $\dg M$,
is given by the weighted sum of relative entropies:
\begin{align*}
    \Inc_{\dg M}(\mu) :=
        % \sum_{L \in \Ed} \beta\ssub L\, \kldiv[\Big]{\mu(\Tgt L,\Src L)}{p\ssub L(\Tgt L | \Src L) \mu(\Src L)}.
        \sum_{a \in \Ar} \beta_a\, \kldiv[\Big]{\mu(\Tgt a,\Src a)}{p\ssub a(\Tgt a | \Src a) \mu(\Src a)}.
        % \Ex_{\mu} \sum_{L \in \Ed} \beta\ssub L 
        %     \log \frac{\mu(\Tgt_L \mid \Src_L)}{p\ssub L(\Tgt_L \mid \Src_L)}
\end{align*}
%joe2: It's strange to talk about the quantitative term, since it's
%nota term in any expression that you've defined
%oli2: What? This isn't right.  I use the term "quantitative limit"
%throughout, and this is where I'm introducing the notion!
%joe3: "quantitative limit" is also a horrible expression, and should
%be replaced by something else.
%$\Inc$ is called the ``quantitative'' term because it measures $\mu$'s
% $\Inc$ is called the ``quantitative'' term because it measures $\mu$'s discrepency
% $\Inc$ measures $\mu$'s
% $\Inc$ is the ``quantitative'' term because it measures $\mu$'s
% discrepency with the quantitative data in the cpds.
%joe3: this is a viewpoint that you've never let the reader in on.   I
%would prefer not to use this terminology, which I find confusing,
%since what you later call qualitative also involves quantities (so
%isn't the least bit qualitative)
%joe3: I would prefer not to give this complicated discussion, which
%we never make use of.
%Using a standard interpretation of the relative entropy
%$\kldiv{\mu}{p} := \Ex_{\mu}[\log \frac\mu p]$, we may view $\Inc_{\dg
%  M}$ as measuring total overhead of using codes optimized for the
%cpds of $\dg M$ (weighted by the confidence we have in them),
%supposing that the world is in fact distributed according to $\mu$.
We can think of the relative entropy $\kldiv$ as measuring, for each
hyperedge $a$, the ``distance'' from the cpd associated with $a$ to the
corresponding marginal probability determined by $\mu$ (weighted by
our confidence in $a$).
%joe1
%Meanwhile, there is also a ``qualitative'' term, called the
% There is also a ``qualitative'' term, called the
%joe2: It's strange to call it a qualitative term, when it's a
%numerical quantitative.  More importantly, you need to go back to the
%intuition you gave for the edges here.  
%oli2: what you wrote takes up a lot more space, and we'll never revisit
% the intuition at all.  
%oli2*: I dislike that you got rid of the ``quantitative''
% and ``qualitative'' descriptors, because I want to refer to the
% quantitative limit!
%joe3: And I do *not* want to refer to the quantitative limit, which
%is (in mhy opinon)


% By contrast, there is also a ``qualitative'' term, which 
% measures 
%oli2: I don't want to go here yet because we haven't worked out the details.
%in particular, I don't like "how far" analogy so well here, because IDef can
%be negative. I tried to rewrite it 
% There is also another aspect of inconsistency of a disribution $\mu$
% with respect to a PDG ${\dg M}$: how far $\mu$ is from modeling the 
% treating the edges in ${\dg M}$ as describing independent mechanisms
% that determine the target given the source.  This is captured by the
% \emph{information deficiency}, given by
%joe3
%The second way in which we evalute a distribution $\mu$ is 
The second way in which we evaluate a distribution $\mu$ is 
%joe3
%treating the edges in ${\dg M}$ as describing independent mechanisms
by viewing  the edges in ${\dg M}$ as describing independent mechanisms
that determine the target given the source.  
%joe3*: this is *not* captured by the information deficiency.  IDef is
%a function of \mu, so at best it's capturing some relationship 
%between \mu and the PDG.   You need to describe what that
%relationship is.  Although IDef can be negative (which is part of why I have
%0 intuition for it), I think what I wrote is far more accurate that
%what you wrote
This is captured by the \emph{information deficiency}, given by
% \begin{align*}
$
    % \IDef{\dg M}(\mu) := - \H(\mu) + \sum_{L \in \Ed} \alpha\ssub L\, \H_\mu(\Tgt L | \Src L),
    \IDef{\dg M}(\mu) := - \H(\mu) + \sum_{a \in \Ar} \alpha_a\, \H_\mu(\Tgt a | \Src a),
$
% Although we won't motivate it here, 
%joe2*: NO!  I don't think IDef models causal structure at all.
%Rather, IDef(\mu) should be a measure of how far away \mu is from
%capturing the causal structure described by M, in the same way that
%Inc is a measure of how far away \mu is from the cpds described by
%M.  If this is not true, we need to talk.   In any case, this must be
%rewritten.  
%oli2: I agree with what you wrote: that \IDef(\mu) is a measure of how far away
% \mu is from capturing the causal structure described by M.  That said, I don't 
% understand why you react so strongly to the words "IDef models
    % causal structure".
%joe3: IDef is a number.  It's not doing any sort of modeling    
% for a first high-level description of the term. 
which
    % , roughly speaking, 
    % is a generalization of maximum entropy that accounts for the
    % Seen from another angle, it
%oli2
% models causal structure, 
%joe3
%measures discrepency between $\mu$ and the causal structure describedn
measures the discrepency between $\mu$ and the causal structure described
by $\dg M$. 
%joe3: cut; IDef doesn't play any role in allowing pdgs to capture
%independencies.  It's just a number.  Can you give a hint as to why
%this number measures the discrepancy?
%and plays a significant role in allowing pdgs to capture
%    (conditional) independencies. 
Note that $\IDef{}$ does not depend on the cpds
%joe3: why introduce this terinology out of the blue here?
%(``quantitative beliefs'')
of $\dg M$, nor even the possible values of the
variables---it is defined purely in terms of the topology of the graph
and the weights $\balpha$.  
%joe2*: the weights are a bit of a red herring.  once we have a clear
%intuition for IDef (and I think I now see how the intuition should
%go)  then we're just multiplying by the confidence, because the
%confidence is indicating the probabiility that the edge is there.  We
%need to make clear the basic intuition without \alpha.
%oli2*: I feel like you're missing the point I was trying to make.
% It happens that the presence or absence of edges can be encoded 
% with ones and zeros in the weights. I'm not emphasizing the continuum
% aspect of the weights. I'm emphasizing that it depends only on the (degree of)
% presence or absence of an edge, which is the weight \alpha --- and not on the
% cpds or the nature of the variables involved. 
%joe3: I was surely missing the point you were trying to make, and am
%still missing it.  
%joe1
%The PDG semantics are then given by a scoring fuction: 
% The semantics of a PDG $\dg M$ are then given by a scoring fuction
The semantics of a PDG $\dg M$ are then given by a family of scoring
%joe3
%fuctions
functions, 
$\bbr{\dg M}_\gamma: \Delta \V\N \to \Rext$
% corresponding to the linear combination
%joe3
%indexed by a trade-off parameter $\gamma \ge 0$ which controls the
indexed by a trade-off parameter $\gamma \ge 0$ that controls the
importance of  
$\IDef{}$ relative to $\Inc$. 
%joe2*: First of all, this is not one semantics, but a family of
%semantics, indexed by \gamma.  Second, you need to give INTUITION for
%\gamma.  
%oli2: OK, I've reworded it, although personally I don't think it's
%important to make such a distinction; it can be a single semantics
%that is a map from pairs (\mu, \gamma) to extended reals, just as
%easily  as it can be a family of semantics, indexed by \gamma, each
%mapping \mu to extended reals.
%joe3*: You have to give some intuition as to why we're interested in
%this tradeoff.    *Why* are we interested in the limit as \gamma ->
%0.  Why is it particularly important
\begin{align*}
    \bbr{\dg M}_\gamma(\mu) &:= \Inc_{\dg M}(\mu) + \gamma \IDef{\dg M}(\mu) 
        \numberthis\label{eqn:scoring-fn}
        \\
        % =& \Ex_{\mu}\left[\, \sum_{L \in \Ed} \log \frac
        %     {\mu(\Tgt L| \Src L)^{\beta\ssub L - \gamma \alpha \ssub L}}
        %     {p\ssub L(\Tgt L | \Src L)^{\beta \ssub L}}
        % \right] - \gamma \H(\mu)
        =& \Ex_{\mu}\left[\, \sum_{a \in \Ar} \log \frac
            {\mu(\Tgt a| \Src a)^{\beta_a - \gamma \alpha_a}}
            {p\ssub a(\Tgt a | \Src a)^{\beta_a}}
        \right] - \gamma \H(\mu)
        .
\end{align*}
%joe2*: We need INTUITION.  why do we care about what happens as
%\gamma -> 0.  If we can't motivate this well, there's no reason for a
%reader to be interested in the rest of the paper, so this is critical.
% $\gamma$ controls the trade-off
% between matching quantitative beliefs and qualitative ones. 

The notation $\bbr{\dg M}^*_\gamma := \argmin_\mu \bbr{\dg M}_\gamma(\mu)$ denotes the set of optimal distributions at a particular $\gamma$.
%joe3: getting rid of "quantitative", which in my opinion adds nothing 
%Of particular interest is the ``quantitative limit'' as $\gamma \to 0$,
Of particular interest is the limit as $\gamma \to 0$,
which corresponds to a fully empirical approach: matching quantitative
observations is the primary concern, and causal information is used
only to break ties.  
Why this limit in particular?

%joe2*: We need INTUITION.  why do we care about what happens as
%\gamma -> 0.  If we can't motivate this well, there's no reason for a
%reader to be interested in the rest of the paper, so this is critical.

\TODO[ In my opinion is way too much intuition for $\gamma$, but I'll put it all here so it can be pared down. ]

  %joe3*: *none* of these bullet points count as intjuition, except
  %possibly the second, which I don't understand.  "Intuition" means
  %explaining why this is a good thing to do (without the
  %retrospective benefit of future results) in a compelling way.   We
  %still need intuition. 
\begin{enumerate}[nosep]
  %\item
  %joe3: this is not intuition for why we want to do this (especially
  %when you say "nearly all")
%        This quantitative limit is what is used to generate nearly
%        all of the loss functions in \textcite{one-true-loss}, which
%        are largely empirical in nature. 
%joe3: I have no idea of what you mean by a calibrated model.  
\item 
        Optimizing inconsistency in this limit guarantees a calibrated model, which is one of the biggest advantages PDGs have over factor graphs
         \parencite[Example 5]{pdg-aaai}.
        
    \item 
%oli2: ok, I'm sure this is overkill on intuition, so I commented it
      %out.
      %joe3: You didn't coment it out, so I did.  This is not
      %intuition for why we want to take the limit as \gamma goes to 0.
%    When $\balpha = \mat 0$, this is the principle of maximum entropy;
%    for other values of $\balpha$ it is a causally sensitive variant
%    which accounts for the fact that cpd constraints themselves carry
%    different amounts of entropy depending on the settings of the
%    variables.  
%joe3: this doesn't explain why we want to take \gamma to 0 either
%    \item 
%    Another reason to focus on the quantitative limit is pragmatic:
%    there is a unique optimal joint distribution, $\bbr{\dg M}^*$ (at
%    least if $\bbeta > \mat 0$). 
%    In any case, this is the way in which PDGs define a unique joint
%    distribution, and hence may be thought of as a graphial model. 
\end{enumerate}
% This distribution uniquely achieves the smallest information deficiency among those distributions maximally compatible with $\dg M$. 

One should be careful to distinguish this joint distribution $\bbr{\dg M}^* \in \Delta\V \N$, which arises in the limit as $\gamma \to 0$, from $\bbr{\dg M}^*_0$, the set of distributions that minimize 
%joe1
%$\Inc_{\dg M}$ which contains $\bbr{\dg M}^*$, and possibly many others.
$\Inc_{\dg M}$; the latter set can be shown to contain $\bbr{\dg
  M}^*$ (see \cite{pdg-aaai}), but may also contain other distributions.

%joe2: There is not a unique "inconsistency" of a PDG.  Again, you're
%giving a family of inconsistencies, indexed by \gamma.  So, at best,
%you can talk about the inconsistency relative to \gamma (which you
%must MOTIVATE). 
The inconsistency of a PDG $\dg M$ is the smallest possible score of any distribution:
\begin{align*}
    \aar{\dg M}_\gamma := \inf_{\mu \in \Delta\!\V\!\N}\, \bbr{\dg M}_\gamma(\mu).
\end{align*}
To parallel the notation for scoring functions, when we omit the subscript, we refer to the limit as $\gamma\to 0$, which, unlike before, obeys $\aar{\dg M} = \aar{\dg M}_0$. 


% \textbf{Exponential Conic Programs.}
% \textbf{Exponential Cones.}
% \textbf{Disciplined Convex Programming, and Exponential Cones.}
%joe1*: you *must* give more intuition here (what do these triples
%represent? why are they of interest?), more background (where has
%this approach been used before?), and more intuition about why
%exponential conexs might be useful.  
% includes linear programming, quadratic programming, semidefinite programming.
%joe2: is an "optimization paradign" just a way of doing optimization?
%If so, please use that terminology.
% Convex programming is an optimization paradigm wherein one searches
% within a convex set to find optima of a linear function, subject to
% certain constraints.  
 % objective function to minimize, 
%joe2
%Most computer scientists are familiar with
%where the constraints are also linear, and likely also the variants in
%which the contstraints can quadratic (QP) or that a matrix be positive
% Linear programming (LP) is just an instance of convex programming
% where the constraints are also linear; in \emph{quadratic programming} (QP(
%joe2: is it true that the constraints can be linear or quadratic?
% the constraints are linear or quadratic;
%joe2*: I'm lost.  Where did the matrix comne from?
%oli2: that's another kind of constraint. You can optimize over a set of matrices,
% that is constrained to be positive semidefinite. 
% or that a matrix be positive semidefinite (SDP). 
% The variant we use is less well-known 

%joe2*: we don't need sociology; added paragaph break
%Exponential cone constraints are less well-known, in part becaus
%provably efficient algorithms for exponential conic programs are
%relatively recent.  


% $K_{\exp}$ is non-symmetric, and cannot .



% \textbf{Convex Programming.}
% A \emph{convex program} is an optimization problem of the form
% \begin{align*}
%     \text{\sf minimize}& \quad f(x) \\
%     \text{\sf subject to}& \quad A x = b,\quad g_j(x) \le 0 ~~\text{for }j = 1, \ldots n_g
% \end{align*}
% where $f$ and every $g_j$ are convex functions. 
% which subsumes linear programming

% \textbf{Conic Programs.}
% A \emph{conic program} is an optimization problem of the form
% \[ 
%     \mathop{\text{\sf minimize}}\limits_{x \in \Rext}~~ c^{\sf T} x
%     \quad\text{\sf subject to}~~ Ax = b, x \in K.
% \]
% where $K \subset \Rext^n$ is a proper cone (i.e., non-empty, closed under non-negative linear combination, closed, and full-dimensional).


% 
% Most computer scientists are familiar with linear programming (LP), where the constraints are also linear, and likely also the variants in which the contstraints can quadratic (QP) or that a matrix be positive semidefinite (SDP).
% Exponential cone constraints are less well-known, in part because provably efficient algorithms for exponential conic programs are relatively recent. 
\textbf{Exponential Conic Programs.}

\TODO[ Chris: Can you take a look at this section? 
    I suspect you will be able to address Joe's concerns about intuition
    in a more satisfactory way. ]
%joe3: We need something.  I have not clue as to why we should care about the
%exponential cone, and nothing that you've said helps.
% \emph{Exponential-cone constraints} are another class of constraints,
% which will be of particular interest to this paper.
%joe2: Yes, the intjuition is critical
%oli2: I feel like asking for intuition here is a little bit like
% asking for intuition about . What's below is for 
%
% \TODO[TODO: add more intuition + references]
%
%joe2
%The exponential cone is the convex set
% An exponential cone is a convex set of the form
%oli2: no, there's only one exponential cone
The exponential cone is the convex set
%joe2: the layout here looks strange.
%oli2: maybe this looks better to you?
%joe3: no; it still has the same problem, with the \R^3 in the middle
%of two lines.
\begin{align*}
    K_{\mskip-1mu\exp} &:=\!\!\!\!\!
        \begin{aligned}
        \big\{ (x_1, x_2, x_3) &: 
                x_1 \ge x_2 e^{x_3 / x_2}\!,\, x_2 > 0 \big\} 
            \\[-0.2ex]\quad \mathbin{\cup}\, 
        \big\{ (x_1, 0, x_3&) : x_1 \ge 0,\, x_3 \le 0 \big\} 
            % \cup\\
        % \\\quad \mathbin{\cup}\, 
    \end{aligned}
    \subset \mathbb R^3.
\end{align*}
It is also sometimes called the relative entropy cone, because
$(-u, m, p) \in K_{\exp}$ iff $u$ is an upper bound on $m \log \frac m p$, 
which is very close to the expression for relative entropy.
%
%joe3: is this standard terminology?  It's not a program
An \emph{exponential conic program} is an optimization problem of the
form 
\[ 
    \mathop{\text{\sf minimize}}\limits_{x \in \Rext}~~ c^{\sf T} x
    \quad\text{\sf subject to}~~ Ax = b,~~x \in K_{\exp}^n.
\]
% Recent work in interior point methods has provided a way to solve linear optimization problems with such constraints in polynomial time \parencite{dahl2022primal}.
Recent work in interior point methods has shown that such problems may be solved in polynomial time \parencite{dahl2022primal}. 
% provided that $x_1, x_2$ and $x_3$ are affine transformations of the program variables. 

% %joe2*: I'm lost again.  You need to SLOW DOWN; give INUITION and MOTIVATION.
Disciplined Convex Programming \parencite{dcp-thesis} is a
    compositional approach to convex optimization, which imposes
    certain conventions on how an optimization program may be written;
%joe3: this doesn't parse well.  How can a program be "disciplined
%convex programming".  At best it's an instance of dcp.
    %    a program that satsfies these conventions is said to be dcp.
        a program that satisfies these conventions is said to be dcp.  
% This approach to convex programing allows us to articulate problems compositionally 
A disciplined convex program can be compiled to a convex optimization
%joe3: what is "standard form"?  Is this standard terminology?  If you
%can't assume that 90% of readers know it, you have to explain it.
problem in standard form \parencite{agrawal2018rewriting}, which can
be handled efficiently. 
Of particular relevance for us is the rule for 
%joe3: know you talk about a constraint satisfying dcp.  Before it was
%a program being dcp.  You have to be consistent
exponential cones: an exponential cone constraint satisfies dcp 
%joe2: I'm lost.  What are the "program variables".
% provided that $x_1, x_2$ and $x_3$ are affine transformations of the program variables. 
%oli2: maybe this will help
the constraint $(x,y,z) \in K_{\exp}$ is dcp, iff $x$, $y$, and $z$
are affine transformations of the variables being optimized.  

% So long as the arguments to the constraints satisfy certain rules they are said to be dcp; dcp program can be compiled to a convex optimization problem that can be handled efficiently.
% So long as the constraints satisfy certain rules they are said to be dcp; dcp program can be compiled to a convex optimization problem that can be handled efficiently.
% The rule for the exponential cone is simple: the constraint $(x,y,z) \in K_{\exp}$ is dcp, iff $a$, $b$, and $c$ are affine transformations of the program variables. 


\section{INFERENCE VIA INCONSISTENCY MINIMIZATION}
    \label{sec:inf-via-inc}

We are now equipped to talk more technically about inference in PDGs. 
% When working with traditional graphical models, the meaning is clear. 
%k
Since PDG semantics are already given in terms of a scoring function,
% the obvious thing to do is to find a distribution that minimizes it. 
the obvious thing to do is to find a distribution that minimizes it.
%joe2*: But what does this minimization have to do with the inference
%problem you defined above?
%oli2*: if you have a concrete joint distribution, computing marginal
%probabilities of it is trivial, and conditioning is not much
%harder. Did my additions help?
%joe3*: What additions are you talking about?  I don't see anywhere
%where you've connected minimization with inference.  How could such a
%distribution \muy be used to answer queries?  Why does it give the
%"right" answer?  Is it the right answer bhy definition?  That is, is
%it the case that, by definition, are you looking for \mu(Y) for the
%\mu that is M^*?  If so, you need to say that somewhere.
Such a distribution $\mu$ could straightforwardly be used to answer probabilistic queries \eqref{q:inf} and compute inconsistency.  
% There are several immediate difficulties.
There are some immediate difficulties with this approach.
% We immediately run into some obstacles.
% There are some obstacles to this.


\begin{enumerate}[nosep, label=\textbf{D\arabic*.}]
    \item Even writing down a distribution $\mu$, let alone evaluating its score $\bbr{\dg M}_\gamma (\mu)$, or minimizing it, takes exponential time.
    
    \item Generally speaking, optimization is computationally difficult.
%joe3
%      Even our most powerful optimization techniques only provably
      Even our most powerful optimization techniques are only
      guaranteed to
      find optima in certain special cases. 
    Unfortunately, while standard optimization techniques seem to work in practice
%joe1: why don't the standard tools apply in our setting?
%oli1: the next sentence explains it: the standard tools require Lipshitz-ness or
% self-concordance. How can I write this more clearly, if I don't really want to go into
% either but still want to mention the names so that people know what doesn't work? 
%  (actually, the way the optimizer works is by using a self-concordant barrier function,
%       which is realted, but we can't use self-concordance directly in the obvious way.)
%
%      (more-or-less; see \cref{sec:expts}), the standard theoretical
      (more or less; see \cref{sec:expts}), the standard theoretical
      tools do not apply in our setting.
        % and even if it can be 
%joe2*: What does it even mean for a pdg to be strictly convex?  
%oli2*: The scoring function is convex, in its argument $\mu$. Is this unclear from the wording? [[M]]_\gamma is the scoring function. 
%oli2: adding wording to try to abdicate responsibility for motivation
        % Despite being strictly convex,
        To be technical: despite being strictly convex,
        %oli2: removing because I have too many duplicate citations.
        % \parencite[at least for small $\gamma$,][]{pdg-aaai},
        and even $C^\infty$ smooth, 
%joe1: I have no idea what self-concordant means.  Unless you're sure
%that over 90% of AIStats folks will now, you must define it and
%explain why it's relevant.  At least I know what the Lipshitz
%condition is, but it couldn't hurt to explain that too.
%joe2*: You still need to explain these terms and MOTIVATE why we care
%about them.
%oli2: I don't really. I want this to be a passing remark for those who know about self-concordance (which I'd guess is 40% of AISTATS folks) and Lipshitz-ness (I'd guess > 95% of AISTATS), but I really don't wan't to make a big deal of either of these two things. They're not important to me, they're just the standard tools that I don't think work. What do you think I should do?
        in general $\bbr{\dg M}_\gamma$ seems to be neither Lipshitz nor self-concordant.
         % convex (in $\mu$, which is exponentially large),
    \item Even if we coulld easily find optimizers of the function $\bbr{\dg M}_\gamma$ for fixed $\gamma > 0$, it's still not obvious that this would allow us to calculate the unique limiting distribution $\bbr{\dg M}^*$.
\end{enumerate}

We will ultimately address each of these issues, but before we do so, 
% let's start with a shift in perspective.
let's start by trying to do inference as
suggested in the final section of \textcite{pdg-aaai}, which meshes well with 
the persepctive taken in \textcite{one-true-loss}. 

%joe2*: As I said, this should already be in the intro
%oli2*: As I responded below with %oli1, I disagree because it doesn't pan out;
% it's just an interesting perspective. I think this is the right place for it. 
The argument there is that one can do modeling as follows:
    represent all of the relevant information as cpds, 
    form a PDG out of them, and 
%joe2*: I'm lost.  What knobs do we have?  This comes out of the blue.
%oli2: in this case, just $p(Y|X)$, but I was trying to reerence the
%general approach 
% of my first AISTATS paper, not talk about this in particular. I've done a minor
% tweak to the wording to try to make this clear.
    % play with the knobs you have to to minimize the resulting inconsistency.
%joe3*: My comment 
    play with whatever knobs are available, so as to minimize the overall inconsistency.
% Well, we have a PDG $\dg M$, and we also have a candidate joint distribution $\mu$.
% What if we put them both on the same footing, in a new PDG, and measure its inconsistency?
%joe1
%It is not hard to show that the distribution any distribution $\mu$
%that minimizes this quantity must also satisfy $\mu \in \bbr{\dg
% It is not hard to show that any distribution 
% that minimizes this quantity must be in $\bbr{\dg
  % M}_\gamma^*$. More generally,  
Well, we have a PDG $\dg M$, and we wanted to know the probablility of $Y$ given $X$. 
What happens if we extend $\dg M$ with a guess, say $p(Y|X)$, and then
%oli2: added, to address your concern below
alter $p$ so as to minimize inconsistency? 
As hinted in \textcite{pdg-aaai},
this
%oli2: modified
% would indeed perform our inference task,
would indeed suffice for inference,
%oli2: also added
since the optimal value of $p(Y|X)$ would be the cpd we wanted.
 
%joe2*: What does this proposition have to  say about the inference
%task you defined earlier.  A priori, it doesn't look that closely related.
%oli2: did my additions above help?
\begin{linked}{prop}{optimalYgivenX}
	% \label{prop:optimalYgivenX}
	% For all $\dg M$, $X,Y\in\N^{\dg M}$, and $\gamma > 0$, we have that
    For all variables $X,Y$, and $\gamma > 0$, 
	$$\displaystyle
		% \argmin_{p : X \to \Delta Y}
		\argmin_{p(Y|X)}\,
        \aar{\dg M + p}_\gamma =
		\Big\{ \mu(Y | X) :  \mu \in \bbr{\dg M}_\gamma^* \Big\}
	,$$
% \end{linked}
% 
% In the limit of small $\gamma$, since there is only one such distribution,
% the expression beomes simpler.
% 
% \begin{linked}{coro}{smallgammaopt}
%joe1: What's a "quantiative limit"? 
%oli1: it's the limit as \gamma -> 0; 
% I defined it when I defined [[M]]^*, and also it's not necessary
% to remember the worlds, because the symbols are sufficient to get the meaning.
%joe3: I still see no reason why this is a "quantitative limit", and I
%believe that this terminology will confuse the reader.  And again,
%you have to explain why we care about this limit
                %                and in the quantiative limit,
                                and in the  limit
	$\displaystyle
		\bbr{\dg M}^*(Y | X)
	$ is the unique minimizer of the function
$
    p(Y|X) \mapsto \aar{\dg M + p}
$
% which  a conditional probability on $Y$ given $X$ into the PDG $\dg M$. 
%joe1: I have no idea what you're trying to say here.  What does it
%mean that it "includes a conditional probability on $Y$ given $X$
%into the PDG"    
%oli1: good point--- I'm not just adding ("includig") the new probability, but
% also measuring its inconsistency.
% which measures the inconsistency of 
\end{linked}

%joe1*: The claim that we can use inconsistency to compute the
%marginal probability of somne (small) set of variables is an
%important part of the story of why inconsistency is important, and
%should have come *much* earlier (i.e., in the introduction)
%
%oli1: If this was what gave the reason we don't discuss it in the introduction
Consequently, if we are interested only in querying the marginal probability of some small subset $Y$ of variables conditioned on other ones $X$ (i.e., the usual form of a query to a graphical model), and we had an efficient way to estimate the inconsistency of a guess $p(Y|X)$ with the rest of the model, we would have successfully cleared obstacle \textbf{D1}.

% Since inference in other graphical models is already NP-hard, 
% and the class of PDGs subsumes capture them, it should be no surprise 
% that inference in PDGs is NP hard as well.
%
%
% One might imagine that \emph{resolving} the inconsistency is the hard part,
%     as opposed to noticing it. 
% Might it easier to simply determine whether or not a PDG is inconsistent?

Since inference in other graphical models is already NP-hard, 
and PDGs subsume them, it should come as no surprise 
that inference in PDGs is NP hard as well.
%
Still, one might imagine that \emph{resolving} the inconsistency is the hard part,
    as opposed to noticing it. 
Might it easier to simply determine whether or not a PDG is inconsistent?
% If this seems reasonable, you might suspect that this reformulation could increase the difficulty of optimization (\textbf{D2})---that we might lose the several nice properties we do have (strict convexity and smoothness)---but this concern turns out not to be substantiated. 
%
%joe2: we don't need to try to get into the reader's head
%oli2: the problem is that it doesn't (clearly) increase the difficulty of the optimization, so "this could increase" is wrong. I only mean to suggest that one might suspect that. 
If this were the case, one might imagine that the reformulation
could increase the difficulty of optimization (\textbf{D2}); 
%joe2: why do we care if we lose them.  We're trying to do inference.
%This is copletely unnmotivated.
%oli2: we're trying to find an optimal joint distribution, which is one
% first pass at inference. Is it better motivated now?
% might lose the several nice properties we do have (strict convexity
might we lose the nice properties we do have (strict convexity
%joe2
%and smoothness)---but this concern turns out not to be substantiated.  
% and smoothness).  This turns not to be the case.
and smoothness)?  It turns out that we don't.

%joe2*: Again, why do we care about this result?  How does it fit into
%the story.
%oli2: smoothness and strict convexity means gradient descent with 
% an infinitely small step size is guaranteed to find the unique optimum distribution. 
% It fits into the story because functions with these properties are relatively easy to
% optimize. At the outset, we know inference is hard. We can ask ourselves: is it the
% difficulty of optimizing the function, or the difficulty of evaluating it? The answer is
% the latter.  How can I make this come through?
\begin{linked}{prop}{smooth-and-strictly-cvx}
	The map $p \mapsto \aar{\dg M \bundle p}_\gamma$ is smooth and
		strictly convex 
        %for $\gamma$%
	% (concretely: all $\gamma$ less than $\min (\{1\}\cup\{ \beta^{\dg M}_L : L \in \Ed^{\dg M}\})$
    when $\gamma < \min \{1\} \cup \{\beta_L\}_{ L \in \Ed}$.
	% .
\end{linked}

Operationally, though, we still haven't made much progress, since
%joe3
%we still don't have an easy way to compute $\aar{ ~\cdot~ }_\gamma$. 
we still don't have an easy way to compute $\aar{ \dg M }_\gamma$. 
% In \cref{sec:complexity}, we will see why.
This is because there isn't one. 
% For now, 

\begin{linked}{prop}{consistent-NP-hard}\label{sharp-p-hard}
    \begin{enumerate}[nosep,label={\rm{(\alph*)}}]
    \item Deciding if $\dg M$ is consistent is NP-hard.
    \item Computing $\aar{\dg M}_\gamma$ is \#P-hard, for all $\gamma > 0$.
    \end{enumerate}
\end{linked}

% Since inference in other graphical models is already NP-hard, 
% and the class of PDGs subsumes capture them, it should be no surprise 
% that inference in PDGs is NP hard as well.

%joe3: Instead of what?  Since this is being rewritten, I won't try to
%suggest something better.
%Instead, let's focus instead on first directly addressing \textbf{D2},
%an approach 
%joe2*: More fruitful for what?  How does this relate to inference?
%I'm missing the story here (because you haven't told it).
%oli2: more fruitful in that it will yield an inference
%algorithm. With all of the extra 
%context that I added, is this closer?
%joe3: this is bad English.  When you say "more fruitful", it makes
%sense only as a comparison.  I still have no idea what it's more
%fruitful than, and you didn't answer that question in your response above.
%which will ultimately be more fruitful.
 
% \begin{linked}{prop}{sementics-via-inconsistency}
% 	$\displaystyle
% 		\bbr{\dg M}_\gamma(\mu)
% 			% =  \aar*{\dg M \bundle \mu!}_\gamma
%             =\aar[\Big]{\dg M \bundle \overset{(\beta: \infty)}\mu}_{\!\!\gamma}.
% 			% \qquad\Big(~= \lim_{t \to \infty} \aar[\Big]{\dg M \bundle \overset{(\beta:t)}\mu}_{\!\!\gamma}
% 			% 	~\Big)
% 	% \qquad\text{and}\qquad
% 	% \bbr{\dg M}(\\)
% 	$
% \end{linked}
% 
% So it seems that generic optimization algorithms will
% give us a foothold on inference.
% However, the PDG objective \eqref{eqn:scoring-fn} 

% \begin{prop}
% \begin
%     	% \label{prop:optimalYgivenX}
%     	% For all $\dg M$, $X,Y\in\N^{\dg M}$, and $\gamma > 0$, we have that
%     	$\displaystyle
%     		\argmin_{p : X \to \Delta Y} \aar{\dg M + p}_\gamma =
%     		\Big\{ \mu(Y | X) :  \mu \in \bbr{\dg M}_\gamma^* \Big\}
%     	$.
% \end{prop}

%joe1
%\section{REDUCTIONS TO CONVEX PROGRAMS WITH EXPONENTIAL CONE CONSTRAINTS}
% \section{REDUCING TO CONVEX PROGRAMS WITH EXPONENTIAL-CONE CONSTRAINTS}
\section{REDUCING TO EXPONENTIAL CONIC PROGRAMS}

    \label{sec:reductions}
%joe1
%We now present the central finding of our paper:  observation that the
%oli1: it's the central finding in that it's the lynchpin, but this is
%not the main result; nobody should care about it by itself. It's more
%like the key lemma.  
% We now prove our main result: that the
%joe3
%    We now present the key technical finding of our paper: 
    We now present the key technical result of our paper: 
    that the
%joe1: what do you mean by "the PDG objective"?
%oli 1: I mean the scoring funciton. Because it's the "optimization objective"
% Chris has been calling it that and I adopted it. But ``scoring function'' works too.
% PDG objective $\bbr{\dg M}_\gamma$ can be written 
the scoring function $\bbr{\dg M}_\gamma$
%joe1
%as a linear optimization problem with exponential cone constraints.
%oli1
% as a linear optimization problem with exponential-ggcone constraints.
can be written as an exponental conic program. 
%
%joe2
%We will proceed as follows:
We proceed as follows: 
\begin{enumerate}[itemsep=0pt]
\item
  %joe3
  We
    illustrate how to find the minimizers of $\Inc$ in a simple 
%joe3: given your changes
    setting with only one variable and no conditional distributions;
    setting with only one variable and no conditional distributions.

  \item
    %joe3
    We
    show how the same approach can be generalized to find minimizers 
%joe3
    of $\Inc$ in general PDGs (addressing \textbf{D2});
    of $\Inc$ in general PDGs (addressing \textbf{D2});.
    \item \label{item:+idef}
    %joe1 I cut this; we 'already defined {\dg M}^*.  In any case, I still
    % show how to find $\bbr{\dg M}^*$, the unique distribution
    % specified by $\dg M$ in the quantitative limit. 
      %joe3
%      show how to find $\bbr{\dg M}^*$ (addressing \textbf{D3});
      We       show how to find $\bbr{\dg M}^*$ (addressing \textbf{D3}).
    % \item \label{item:cccp}
    % employ the convex-concave procedure 
    % \parencite{yuille2003concave}, to find some minimizer $\mu^* \in \bbr{\dg M}^*_\gamma$ (although it may not be unique), for fixed $\gamma > 0$.
    % 
    \item 
    % Finally, show how \cref{item:+idef,item:cccp} can also be achieved 
%joe3
      %      finally, show how this can all be done
            Finally, we show how this can all be done
%joe2
%    with a more efficient compact representation for PDGs that have
%oli2: reverted + added some words; I meant what I originally said. 
    % with a more efficient compact representation for PDGs that has
%joe3
            %            more efficiently, for PDGs that happen to have
                        more efficiently for PDGs that happen to have
    bounded tree-width (addressing \textbf{D1}).
\end{enumerate}

\subsection{Warm-Up}\label{sec:illust}

%joe2
%To illustrate the idea, consider the special case in which our PDG
To illustrate the idea, consider the special case of a PDG that
contains only one variable $X$, which takes values $\V(X) = \{1, \ldots, n\}$. 
Suppose further that for every edge $j \in \Ar = \{1, \ldots, k\}$, the cpd $p_j(X)$ is an unconditional dsitribution over $X$.
That is, $\Tgt j = \{ X \}$, and $\Src j = \emptyset$.
% Such unconditional probabilities may be identified with unit vectors $\mat p\ssub L \in \mathbb R^n$.  Similarly a candidate (``joint'') distribution $\mu(X)$
Such unconditional probabilities may be identified with vectors $\mat p_j \in [0,1]^n$, and all $k$ of them may conjoined to form a 
%joe1: what isn't this just a matrix?  What makes is "stochastic"?
%oli1: a stochastic matrix is a matrix in which every column (or row) sums to one,
% like a conditional probability table.  It's a more precise word, but I definitely
% don't need the jargon so I'll remove it. 
matrix $\mat P = [\,p_{ij}] \in [0,1]^{n \times k}$.
%joe2
%Of course, a candidate (``joint'') distribution $\mu(X)$
Of course, a candidate (joint) distribution $\mu(X)$
% may be represented as a vector $\mat m \in \mathbb R^n$. 
may be represented as a vector $\mat m \in [0,1]^n$. 
%
% Now, consider another collection of vectors $\{\mat t\ssub {\,L}\,\in \mathbb R^n\}_{L \in \Ed}$ and notice that:
% \begin{align*}
%     \forall  L &\in \Ed.~~ 
%     (-\mat t\ssub L\,, \mat m, \mat p\ssub L) \in K_{\exp}^n \\
%         &\iff 
%             \forall  L \in \Ed.~~
%             \mat t \succeq {\mat m} \log \frac{\mat m}{\mat p}
%         \\&\implies \sum_{L \in \Ed}\sum_{i=1}^n t_i  \ge \kldiv{\mat m}{\mat p}
% \end{align*}
Now consider a matrix $\mat U = [u_{i,j}] \in \Rext^{n \times k}$,
and observe that:
\begin{align*}
%joe2: hyou still haven't defined \otimes.  I'd be happy if you could
%get rid of it altogether.
    &(- \mat U,~ \mat m \otimes \mat 1,~ \mat P) \in K_{\exp}^{n \times k} \\
    &\iff \forall  i,j \in [n]\!\times\![k].~~ 
        (- u_{ij}, m_{i}, p_{ij}) \in K_{\exp} \\
    &\iff \forall  i,j \in [n]\!\times\![k].~~ 
            u_{ij} \ge m_i \log \frac{m_i}{p_{ij}} \\
    &\implies \forall j \in [n].~~  {\textstyle\sum_i} u_{ij}  \ge \kldiv{\mu}{p_j} \\
    &\implies \sum_{i,j} \beta_j u_{ij}  \ge \beta_j \kldiv{\mu}{p_j} \\
    &\iff \mat 1^{\sf T} \mat U \bbeta \ge \Inc(\mu)
    .
    % &\implies \sum_{L \in \Ed}\sum_{i=1}^n t_i  \ge \kldiv{\mat m}{\mat p}
\end{align*}

%joe2
%So now, if $(\mat U, \mat m)$ are a solution to the convex program
So now, if $(\mat U, \mat m)$ is a solution to the convex program
\begin{align*}
    \mathop{\text{\sf minimize}}
    % \min
    \limits_{\mat m, \mat U}~~
        \mat 1^{\sf T} \mat U \bbeta 
    \quad\text{\sf subject to}\quad &
        \mat 1^{\sf T} \mat m  = 1, \\[-2ex]
    % \begin{cases}
        (-\mat U,\;\, &\mat m\otimes \mat 1,\; \mat P) \in K_{\exp}^{n \times m},
    % \end{cases},
\end{align*}
then (a) the inconsistency $\aar{\dg M} = \mat 1^{\sf T} \mat U \bbeta$ equals the optimal objective value, and 
(b) $\mu \in \bbr{\dg M}^*_0$ is maximally compatible with $\mu$. 
% This illustrates the general principle, but 

\subsection{Adding More Variables, Conditionals, Marginals}

Having seen some of the details of the matrix computations, let's now move up a level, and identify finitely supported distributions with their simplex representations. 
% For example, we will implicitly identify a joint distribution $\mu$ 
% with the appropriate vector 
We now tackle the general case of a pdg 
$\dg M = (\N, \Ar, \mathcal P, \balpha, \bbeta)$.
%
% For each $a \in \Ar$, let $N_a := |\V(\Src a, \Tgt a)|$ be the dimension of joint settings of the source and target values of $a$, i.e., the dimension of the cpd $p \ssub a$,
% let $K := \sum_{a \in \Ar} N_a$ be the total dimension, and
%joe2: As I said, I really don't like the subscript-superscript
%notation.  
Let $\mat u = [u^a_{s,t}]^{a \in \Ar,}_{ (s,t) \in \V(\Src a, \Tgt a)}$
be a vector.
% be a free variable in $\Rext^{K}$.
% of all of the condtional probabilities in $\dg M$.
% Let $t = (t^L)_{L \in \Ed} \in \Rext^{K}$.
% {\color{red} Let $\Pi_{X}$ be the projection map that marginalizes to the variables $X$. }
%
Consider the problem
% \begin{align*}
%     % \min_{\mu, t} &\qquad
%     \min_{\mat m, \mat u} &\quad
%         % \Vert t \Vert_1  
%         \sum_{L}\beta_L\, | \mat u\ssub {\mskip 1muL}\mskip 1mu |
%     \\
%     \text{subject to:}&\;\;
%         % (-t^L, \mu(\Src L, \Tgt L), \mu(\Src L) p\ssub L(\Tgt L | \Src L)) 
%         % (-t^L, \Pi_{(\Src L \Tgt L)}\mu, \Pi_{(\Src L)} (\mu) p\ssub L(\Tgt L | \Src L)) 
%         \big(\!\shortminus\! \mat u\ssub L,~ \Pi_{\Src L\Tgt L}\!(\mat m),~
%             \mat P\!\ssub L (\Pi_{\Src L}\!(\mat m) \otimes \mat 1) \big) 
%         % (-t^L_{xy}, \mu(x,y), )
%             \in K_{\exp},\\
%         &\qquad 
%             % \mu \ge 0, ~~ | \mu | = 1.
%             \mat m \ge 0, ~~ | \mat m | = 1.
%             \numberthis\label{prob:joint-inc}
% \end{align*}
\begin{align*}
    % \min_{\mu, t} &\qquad
    % \mathop{\text{\sf minimize}}\limits_{\mat m, \mat u} &\quad
    \mathop{\text{\sf minimize}}\limits_{\mu, \mat u} &\quad
        % \Vert t \Vert_1  
        % \sum_{L}\beta_L\, | \mat u\ssub {\mskip 1muL}\mskip 1mu |
        \sum_{a \in \Ed}\beta_a \, \sum_{\mathrlap{s,t \in \V(\Src a, \Tgt a)}} u^a_{s,t}
    \numberthis\label{prob:joint-inc}\\
    \text{\sf subject to:}&\quad \mu \in \Delta\V\N, \\[0.2ex]
        \forall L \in \Ed.~&\big(-\mat u^a, \mu( \Tgt a,\Src a),p\ssub a(\Tgt a | \Src a)  \mu(\Src a) \big) \in K_{\exp}.
    % \\
    % \color{red}\text{previously}&\color{red}
    %     % (-t^L, \mu(\Src L, \Tgt L), \mu(\Src L) p\ssub L(\Tgt L | \Src L)) 
    %     % (-t^L, \Pi_{(\Src L \Tgt L)}\mu, \Pi_{(\Src L)} (\mu) p\ssub L(\Tgt L | \Src L)) 
    %     \big(\!\shortminus\! \mat u\ssub L,~ \Pi_{\Src L\Tgt L}\!(\mat m),~
    %         \mat P\!\ssub L (\Pi_{\Src L}\!(\mat m) \otimes \mat 1) \big) 
    %     % (-t^L_{xy}, \mu(x,y), )
    %         \in K_{\exp},\\
    %     &\qquad \color{red}
    %         % \mu \ge 0, ~~ | \mu | = 1.
    %         \mat m \ge 0, ~~ | \mat m | = 1.
\end{align*}

% This convex program has $K+1$ constraints 
% \TODO[ I've rewritten this a couple times, and it's always pretty ugly.
%     One higher level question: should we convert this to a different form? 
%     The fact that we can even write constraints this way hinges on the fact
%     that the arguments to the exponential cone are
%     affine transformations of the program variables, which 
%     this presentation sweeps under the rug entirely.
%     \hfill ]
%oli1: this ``more explicit'' presentation is not helpful.
% {\color{gray}
% To be more explicit, if $|\V(\Src L)| = s$ and $|\V(\Tgt L)| = t$, 
% the quantity $p\ssub L(\Tgt L | \Src L)\mu(\Src L)$ represents the 
% flattened matrix
% \[
%     (\mat 1 \otimes \mu(\Src L)) \odot p \ssub L(\Tgt L | \Src L)
%     =
%     \mu(\Src L)|_{i}\; p \ssub L(\Tgt L | \Src L)|_{i,j} \in \mathbb R^{t \times s}.
% \]
% }
Note that the marginals $\mu(\Src a, \Tgt a)$ and $\mu(\Src a)$ are
linear functions of $\mu$'s simplex representation, as required by the 
disciplined convex programming conditions for exponential cones. 
Logic similar to that in \cref{sec:illust} yields: 
\begin{prop}
    % If $(\mu, t)$ are a solution to \eqref{prob:joint-inc}, then
    % $\mu \in \bbr{\dg M}_0^*$,
    % % i.e., is maximally compatible with $\dg M$, and
    % and
    % $\sum_{L}\beta_L |t^L| = \aar{\dg M}$.
    % If $(\mu, \mat u)$ are a solution to \eqref{prob:joint-inc}, then
    If $(\mu, \mat u)$ is a solution to \eqref{prob:joint-inc}, then
    $\mu \in \bbr{\dg M}_0^*$,
    and
    $%\displaystyle
        \sum_{a}\beta_a \sum_{(s,t) \in \V(\Src a, \Tgt a)} u^a_{s,t} = \aar{\dg M}$.
\end{prop}

This is a start, but what we were really after was the unique distribution
$\bbr{\dg M}^*$ that also minimizes $\IDef{\dg M}$.

\subsection{Incorporating IDef}
    \label{sec:also-idef}
% To do this second pass, we will need this second property
% As $\gamma \to 0$, the limit of $\bbr{\dg M}_\gamma$
So far, we have only found \emph{some} distribution that minimizes $\Inc$; 
we really wanted to find the unique distribution $\bbr{\dg M}^*$.
It turns out that a solution to \eqref{prob:joint-inc} can be used to construct a second optimization problem of a similar size, that also optimizes $\IDef{}$ subject to these constraints, finally addressing \textbf{D3}. 
% To justify our approach, we will to prove two more results.
%joe2
%To justify our approach, we need a little more math.
To justify our approach, we need several more results.
% First, a characterization of 
First, a characterization of the set $\bbr{\dg M}^*_0$ of distributions that are maximally compatible with $\dg M$. 

\begin{prop}\label{prop:marginonly}
    For a PDG $\dg M$ with arcs $\cal A$,
	the highest-compatibility distributions (the minimizers $\bbr{\dg M}_0^*$ of $\Inc_{\dg M}$) all have the same conditional probabilities along the edges of $\dg M$.   
	That is, if there is an arc $\ed aXY \in \Ar$, and $\mu_1, \mu_2 \in \bbr{\dg M}_0^*$,
%joe2: You haven't defined "quantitatively optimal"
%oli2: refactored.
    % are quantitatively optimal distributions,
    % (i.e., both $\mu_1$ and $\mu_2$ minimize $\Inc_{\dg M}$),
    then $\mu_1(Y|X) = \mu_2(Y|X)$.  
    % then $\mu_1(Y|X)\mu_2(X) = \mu_2(Y|X) \mu_1(X)$.  
\end{prop}

As a result, having already found one minimizer of $\Inc_{\dg M}$ via \eqref{prob:joint-inc}, it suffices to constrain distributions that have the same conditional marginals along the edges, and now optimize $\IDef{}$.%
    \footnote{
        Technically, to deal with the possibility that variables we
%joe2
%        are conditioning on might have probability zero,
        are conditioning on have probability zero,  
        we require that $\mu(X,Y)\nu(X) = \mu(X) \nu(X,Y)$, rather than 
        $\mu(Y|X) = \nu(Y|X)$.
    }

We now run into a second issue: IDef is not convex in $\mu$. 
Fortunately, when we constrain to distributions that optimize $\Inc$, 
% it equals another function that is. 
it is.
Moreover, this function can also be represented with exponential cones.

\begin{prop}\label{prop:idef-frozen}
If $\mu \in \bbr{\dg M}_0^*$, 
then
\vspace{-1ex}
\[
    \IDef{\dg M}(\mu) = 
        % \kldiv[\bigg]{\mu}{ \prod_{L \in \Ed} \nu(\Tgt L | \Src L) }
            % + K(\dg M)
        % \Ex_\mu 
        % \left[
        \sum_{\mathclap{ w \in \V(\N)} }
            % \log \frac{\mu}{\prod_{L \in \Ed} \nu(\Tgt L | \Src L)}
            \mu(\mskip-1mu w \mskip-1mu)
            \log  \bigg(
                \faktor{\mu(\mskip-1mu w\mskip-1mu )\,}{\,\prod_{a \in \Ar} \nu(\Tgt aw | \Src aw)^{\alpha_a}}
            \bigg)
        % \right]
        ,
\]
%
where $\{ \nu(\Tgt a | \Src a ) \}_{a \in \Ar}$ are the
conditional marginals along the arcs $\Ar$ 
shared by all distributions in $\bbr{\dg M}^*_0$\
(per \cref{prop:marginonly}),
%joe2: have we talked about worlds before?  If not, you can't
%introduce this terminology out of the blue here.
%and $\Src a w, \Tgt a w$ are the respective values of the variables
%$\Src a$ and $\Tgt a$ in the world $w$, which is a joint setting of
%setting of all variables. 
and $\Src a w, \Tgt a w$ are the values of the variables
$\Src a$ and $\Tgt a$
% in the world $w$ (a joint setting of all variables),
in the joint setting $w$ of all variables,
respectively.
% and $K(\dg M)$ does not depend on $\mu$. 
% and $K(\dg M)$ does not depend on $\mu$. 
\end{prop}

Having already computed a solution to \eqref{prob:joint-inc},
the denominator of the expression in \cref{prop:idef-frozen}
(apart from its dependence on joint values $w \in \V\N$ of all variables)
%joe2
%is a constant in our optimiziation problem; let's call it
is a constant in the optimiziation problem; call it
\begin{equation}
    % \psi(w) := \prod_{L \in \Ed} \nu(\Tgt Lw | \Src Lw)^{\alpha\ssub L}.
    \boldsymbol\psi := \bigg[~\prod_{a \in \Ar} \nu(\Tgt a w | \Src a w)^{\alpha\ssub L} \bigg]_{w \in \V\N}\quad.
    \label{eq:cm-product}
\end{equation}
% These constants can $\boldsymbol\psi$ be the vector representation of

We now have our first reliable way of computing the optimal
%joe3: in what sense is M^* optimal?
%distribution $\bbr{\dg M}^*$ in the quantitative limit.
distribution $\bbr{\dg M}^*$.
\begin{prop}
% A solution to % \eqref{eqn:joint+idef}
If $\nu \in \bbr{\dg M}_0^*$, 
and $(\mu, \mat u)$ is a solution to the convex problem 
\begin{align*}
    \mathop{\text{\sf minimize}}\limits_{\mu, \mat u} & \quad
        \sum_{w \in \V \N} u_w
        % |\,\mat u\,| 
        \numberthis\label{eqn:joint+idef}\\
    \text{\sf subject to} &\quad 
        (-\mat u,  \mu, \boldsymbol\psi) \in K_{\exp},~~ \mu \in \Delta\V\N, \\
            &\forall a \in \Ar.~~\mu(\Src a, \Tgt a) \nu(\Src a) = \mu(\Src a) \nu(\Src a, \Tgt a),
\end{align*}
then $\bbr{\dg M}^* = \mu$ 
and $\mat 1^{\sf T} \mat u = \IDef{\dg M}(\mu)$. 
\end{prop}


% \begin{algorithm}
% \begin{algorithmic}
%     \State $X$
% \end{algorithmic}
% \end{algorithm}


% \footnote{Indeed, $K_{\exp}$ is sometimes called the ``relative entropy cone'' for this reason.} 


% \subsection{Tree Deomposition
% \subsection{A Polynomial Algorithm for the Case of Bounded Tree-Width}
% \section{TREE DECOMPOSITION IN PDGS}
% \section{TREE DECOMPOSITION IN PDGS}
% \section{USING A TREE DECOMPOSITION}
\section{EXPONENTIAL CONIC PROGRAMS OVER A TREE DECOMPOSITION }
% \subsection{A Less Expensive Representation for PDGs with Small Tree-Width}
    \label{sec:clique-tree-expcone}

We now show how to optimize over clique-trees $\bmu$, instead of a full joint distribution $\mu$. What makes this possible is the fact that PDGs have similar independence properties to other graphical models. 

\begin{linked}[Markov Property for PDGs]{prop}{markov-property}
	% Suppose $\dg M_1$ and $\dg M_2$ are compatible PDGs, and let $\mathbf X$ denote the variables they have in common.
	% Then for all $\gamma > 0$, we have that
	% \[
	%  	\bbr{\dg M_1 \bundle \dg M_2}^*_\gamma
	% 		% \subset
	% 		~\models~
	% 	% \mathrm I( \N_1 ; \N_2 \mid \mathbf X)
	% 	\N_1 \mathbin{\bot\!\!\!\bot} \N_2 \mid \mat X
	% \]
	% That is: in every optimizing distribution, for any value of $\gamma$, the variables of $\dg M_1$ and the variables of $\dg M_2$ are conditionally independent given their shared variables $\mat X$.
	% Suppose $\dg M_1$ and $\dg M_2$ are value-compatible PDGs,
	% with respective sets of nodes $\mat X_1 := \N^{\dg M_1}$ and
	% $\mat X_2 := \N^{\dg M_2}$.
%joe2
%  Suppose $\dg M_1$ and $\dg M_2$ are PDGs
    Suppose that $\dg M_1$ and $\dg M_2$ are PDGs
    over the sets of variables $\N_1$ and $\N_2$, respectively.
	 % and let $\mathbf X$ denote the variables they have in common.
	Then for all $\gamma > 0$, we have that
	\[
	 	\bbr{\dg M_1 \bundle \dg M_2}^*_\gamma
			% \subset
			~\models~
		% \mathrm I( \N_1 ; \N_2 \mid \mathbf X)
		\N_1 \mathbin{\bot\!\!\!\bot} \N_2 \mid \N_1 \cap \N_2
		% \N^{\dg M_1} \mathbin{\bot\!\!\!\bot} \N^{\dg M_2} \mid \mat X
		% \mat X_1 \mathbin{\bot\!\!\!\bot} \mat X_2 \mid \mat X_1 \cap \mat X_2.
	\]
	That is: in every optimal distribution $\mu^* \in \bbr{\dg M_1 \bundle \dg M_2}^*_\gamma$,
     % for some $\gamma>0$, 
    the variables of $\dg M_1$ and of $\dg M_2$ are conditionally independent given the variables they have in common.
\end{linked}

%One major consequence of \cref{prop:markov-property} is that, in our
%search for optimizers of \eqref{eqn:scoring-fn} we only have to
%consider distributions $\mu$ that come from cluster trees. 
One significant consequence of \cref{prop:markov-property} is that, in the
search for optimizers of \eqref{eqn:scoring-fn}, we have to
consider only distributions $\mu$ that come from clique trees.

Suppose that we are given a tree decomposition $(\mathcal C, \mathcal T)$
of $\dg M$'s underlying hypergraph $\Ar$. 
Since $(\cal C, T)$ is a tree-decomposition of $\Ar$, we are guaranteed
that the source and target of every arc $a$ lie entirely within at least one cluster.
Fix a mapping from arcs to clusters, and let $C_a \in \cal C$ be the cluster that corresponds to the edge $a$.


We now optimize over possible
% cluster marginals
clique trees
$\bmu = \{\mu_C \in \Delta\V(C) \}_{C \in \mathcal C}$,
%joe2*: I'm lost.  Whose simplex representation are you referring to
%when you say "its simplex representation"?
%oli2: oops, I meant something a little more complex. Thanks for pointing it out.
% which we identify with its simplex representation
% as a vector of demension $\sum_{C \in \mathcal C}|\V(C)|$.
which we identify with the vector
$[\mu_C(c)]^{C \in \mathcal C}_{c \in C}$.
As before, consider
% let $K := \sum_{L \in \Ed} |\V(\Src L, \Tgt L)|$ and
$\mat u := [u^a_{s,t}]^{a \in \Ar,}_{ (s,t) \in \V(\Src a, \Tgt a)}$
 % in $\Rext^{K}$.
%
% Let $\mat U = [u^C_{\mat x}]^{C \in \mathcal C,}_{\mat x \in \V(C)}$ be a free vector in 
% $\Rext^K$.
%
% Consider the problem
in the convex problem
%
\begin{align*}
    \mathop{\text{\sf minimize}}\limits_{\bmu, \mat u} &\quad
        \sum_{a \in \Ed}\beta_a \, \sum_{\mathrlap{s,t \in \V(\Src a, \Tgt a)}} u^a_{s,t}
    \numberthis\label{prob:cluster-inc}\\
    \text{\sf subject to:}&\quad
        \forall C \in \mathcal C.~\mu_C \in \Delta\V(C), \\[-0.2ex]
        % exponential constraints
        \forall a \in \Ar.~ \big(&\!- \! \mat u^L\!,\, \mu_{C_a}\!(\Src a,\mskip-2mu \Tgt L), \mu_{C_a}\!(\Src a) p\ssub L(\Tgt a | \Src a)\big) \in K_{\exp} \\
        % marginal constraints
        \forall (C,D) &\in \mathcal T.~~ \mu_{C}(C \cap D) = \mu_{D}(C \cap D).
\end{align*}

%joe3
%Because it is a relative entropy optimization over calibrated clique trees, 
Because it is a relative-entropy optimization over calibrated clique trees, 
\eqref{prob:cluster-inc} is essentially an analogue of
%joe3
%CTree-Optimize-KL from \textcite[pg. 384]{koller2009probabilistic},
CTree-Optimize-KL from \textcite[pg. 384]{koller2009probabilistic}
%joe3: you have no limit here, so this is not the right way to
%describe the next reusult.  In fact, I don't understand the sense in
%which an analogue of clique-tree optimization for PDGs 
%for PDGs in the quantitative limit.
for PDGs, and determines the distribution $\bbr{\dg M}^*_0$.

\begin{prop} \label{prop:cluster-idef}
    If $(\bmu, \mat u)$ is a solution to \eqref{prob:cluster-inc}, then 
%joe3
%    $\bmu$ is a calibrated clique tree, whose coresponding joint
    $\bmu$ is a calibrated clique tree whose coresponding joint
    distribution      is in $\bbr{\dg M}^*_0$.
\end{prop}
% Since $(\cal C, T)$ is a tree-decomposition of $\Ed$, we are guaranteed
% that the source and target of every edge $L$ lie entirely within at least one cluster.
% Fix a mapping of edges to clusters, and let $C_L \in \cal C$ be the cluster that corresponds to the edge $L$.


\subsection{Incorporating IDef, Again}
% \textbf{Incorporating IDef, in the quantitative limit.}
%
% One nice feature of $\Inc$, which was integral to 
In the construction of \eqref{prob:cluster-inc}
and proof of \cref{prop:cluster-idef}, we rely
heavily on the fact that 
each term 
of $\Inc_{\dg M}$
depends only on local conditional marginal distributions $\mu_{C_a}(\Tgt a | \Src a)$,
which can easily be computed from the clique tree $\bmu$.
However, the same cannot be said of $\IDef{}$, because it depends on the joint entropy $\H(\Pr_{\bmu})$ of the entire distribution.
% Now, there is still hope because teh
Because $\Pr_{\bmu}$ has the Markov property with respect to
the tree decomposition $\mathcal T$, 
% Because the clusters form a tree with respect to which $\bmu$ has the Markov property,
the joint entropy may computed exactly in terms of the cluster marginals \parencite{wainwright2008graphical}, by
\begin{equation}\label{eq:bethe-entropy}
    % \H_{\text{bethe}}(\mu) 
    \H(\mu) 
        = \sum_{C \in \mathcal C} \H_\mu(C) 
        - \sum_{(C,D) \in \mathcal T} \H_{\bmu}(C \cap D).
\end{equation}
% is exactly $\H(\bmu)$. 
Nevertheless, there are some subtleties applying this in our setting.
In particular, it may not be obvious that the negation of \eqref{eq:bethe-entropy} can be 
captured with exponential cone constraints in which every argument is affine. 
% In particular, .?
We now illustrate how to overcome these difficulties.

% \def\Par{P^{\mathit{a}}}
\def\Par{\mathrm{Par}}
% \def\Pash{\mathrm{Parsh}}
% \def\Pash{\mathrm{P}^{\mathrm{arShare}}}
\def\Pash{\mathit{P\!S}}

Choose a root node $C_0$ of the tree decomposition $\mathcal T$, and orient each edge of $\mathcal T$ so that it points away from $C_0$. 
Now each cluster $C \in \cal C$, except for $C_0$, has a parent cluster $\Par_C$;
define $\Par_{C_0} := \emptyset$ to be an empty cluster, since $C_0$ has no parent. 
Finally, for each $C \in \mathcal C$, let $\Pash_C := C \cap \Par_C$ denote the
the set of variables that the cluster $C$ has in common with its parent cluster.
% Now, we can compute the Bethe Entropy,
Because $\cal T$ is a tree, we can decompose the entropy of $\Pr_{\bmu}$ as 
%
\begin{align*}
    \H(\Pr_{\bmu}) &= 
        \H(\mu_{C_0}) + 
        % \sum_{(C, D) \in \mathcal T}
        \sum_{(C \to D) \in \mathcal T}
        \H_{\Pr_{\bmu}}(D \mid C)\\
    &= \sum_{C \in \cal C} 
        \sum_{c \in \V(C)} \mu_C(c) \log \frac
            {\mu_C(c)}{\mu_C( \Pash_C(c) )}
\end{align*}
%
% This means that
Note that the denominator $\mu_C(\Pash_C)$ is an affine transformation of $\mu_C$.
The upshot is that we have written the joint entropy in a way that 
splits as a sum of terms over the clusters, each of which can be captured as a dcp exponential cone constraint. 

Suppose that $\boldsymbol\nu = \{\nu_C : C \in \mathcal C\}$ is a calibrated clique tree over the tree-decomposition $(\mathcal C, \mathcal T)$ representing a distribution $\Pr_{\boldsymbol\nu} \in \bbr{\dg M}^*_0$, say obtained by solving \eqref{prob:cluster-inc}. 
 % representing
%
For $C \in \mathcal C$, let $\Ar_C:= \{ a \in \Ar : C_a = C\}$ be the set of 
edges assigned to cluster $C$, and let
% $$
% \boldsymbol\psi_C  := \prod_{\substack{L \in \Ed\\C_L = C}} \nu_C (\Tgt Lw | \Src Lw)^{\alpha\ssub L}
% $$
$$
\boldsymbol\psi_C  := \bigg[ \prod_{a \in \Ar_C} \nu_C (\Tgt a w | \Src a w)^{\alpha\ssub L} \bigg]_{w \in \V(C)}
$$
be the analogue of \eqref{eq:cm-product} local to the cluster $C$.
Once again, let $\mat u := [ u^C_c ]^{C \in \mathcal C}_{c \in \V(C)}$,
and consider the following optimization problem.
%
\begin{align*}
\mathop{\text{\sf minimize}}\limits_{\bmu, \mat u} & \quad
    % \sum_{w \in \V\N} u_w
    \mat 1^{\sf T} \mat u
    % |\,\mat u\,| 
    \numberthis\label{prob:cluster+idef}\\
\text{\sf subject to} &\quad 
    \forall C \in \mathcal C.~\mu_C \in \Delta\V(C), \\[-0.2ex]
    \forall C \in \mathcal C.&~~
        (-\mat u^C,  \mu_C(C), \boldsymbol\psi_C \odot 
            \mu_C(\Pash_C) ) \in K_{\exp}, \\
    % conditional marginal matching
    \forall a \in \Ar.&~~\mu_{C_a}\!(\Src a, \Tgt a) \nu_{C_a}\!(\Src a) = \mu_{C_a}\!(\Src a) \nu_{C_a}\!(\Src a, \Tgt a)\\
    % marginal constraints
    \forall (C,D) &\in \mathcal T.~~ \mu_{C}(C \cap D) = \mu_{D}(C \cap D).
\end{align*}

\begin{prop} \label{prop:cluster+idef}
    If $(\bmu, \mat u)$ is a solution to \eqref{prob:cluster+idef}, 
    then $\bmu$ is a calibrated cluster tree representing the unique
    joint distribution $\bbr{\dg M}^*$, and $\mat 1^{\sf T} \mat u = \aar{\dg M}$.
\end{prop}


\subsection{A Polynomial Time Algorithm For PDGs of Bounded Tree-Width}

\cref{prop:cluster+idef} shows that finding distributions that optimize the PDG scoring function \eqref{eqn:scoring-fn} can be written as a convex optimization problem with a
polynomial number of variables and constraints and with an objective that is also polynomial-sized.
All that remains to achieve polynomial-time inference is to show that a problem of this form can be solved in polynomial time. For this, we turn to interior point methods, which are known \parencite{badenbroek2021algorithm} to converge in polynomial time.
Specifically, \eqref{prob:cluster+idef} can be transformed via established methods \parencite{agrawal2018rewriting} into a standard form called an \emph{exponential conic program} which itself can be solved in polynomial time by commercial solvers \parencite{mosek}. 

\TODO[TODO: refactor the main theorem. I don't know how to get rid of the ``residual norm'' part.]

\begin{linked}{theorem}{main}
    % For   O( 3 n * ((3n+m+1)^3 ) * log(√n / ϵ) )  =   O( n^4 log(n)  log(1 / ϵ) )
There is an algorithm that takes $O(n^4 \log n  \log \nf1\epsilon )$ time,
and finds a point $\epsilon$-close in residual norm to the optimal distribution
$\bbr{\dg M}^*$, where $n$ is the total number of parameters in a clique tree.
\end{linked}

\begin{coro}
    PDG inference can be solved to machine precision in $O( N^4 V^4 \log(N V) \exp(V T) )$ time, where $T$ is the tree-width of the graph, and $N$ is the total number of variables, $V$ is the number of values per variable, and $T$ is the tree-width of the PDG's underlying hypergraph.
\end{coro}

 
% \TODO[ Finding the optimal clique tree is NP-hard. 
%     So how can we guarantee we won't get a bad clique tree (i.e., much wider than the tree-width), so as to guarantee polynomial time?
%     Do we have to know the bound?
%     I imagine there's a standard answer, because exactly the same issue 
%     arises for other graphical models. 
%     \hskip-1.1em ]

The proof rests almost entirely on the analysis of \textcite{badenbroek2021algorithm},
which certifies that the optimization algorithm 
proposed in \textcite{dahl2022primal} works in polynomial time. 

\begin{table}
    % Let $m$ denote the 
    \centering
    \begin{tabular}{ccc}
        \toprule
        & BP &  ExpCone \\\cmidrule(lr){2-3}
        Time & $O(m t)$ & $O( m^4 \log m )$ \\
        Memory  & $O(m + )$ {\color{red}??} & $O( m^2 )${\color{red}????}\\    \bottomrule
    \end{tabular}
    
    \TODO[fill this in properly]
    
    \caption{ }
\end{table}

%joe2*: Why do we care about this?  How does it relate to what you
%called the inference problem on p. 3 (which does *not* involve
%minimizing inconsistency).
%oli2: is this now clearer now that I've connected the two a little better?
\section{INFERENCE WHEN
    \texorpdfstring{$\boldsymbol\gamma \boldsymbol> \mat 0$}{gamma > 0},
    VIA THE CONVEX-CONCAVE PROCEDURE }

We have now given an algorithm that provably finds the distribution $\bbr{\dg M}^*$ in polynomial time. 
What about optimal distributions for fixed $\gamma > 0$.



To do this, we re-use our work in \cref{sec:reductions} employ the convex-concave procedure 
\parencite{yuille2003concave} to find some minimizer $\mu^* \in \bbr{\dg M}^*_\gamma$ (although it may not be unique), for fixed $\gamma > 0$.

The PDG scoring function can be written as \parencite[Proposition 4.6]{pdg-aaai}
\begin{align*}
    \bbr{\dg M}_\gamma(\mu) = 
        -\gamma\H(\mu) + 
            \sum_{L \in \Ed}
                % \left[
                \beta\ssub L\, \Ex_\mu 
                    \log \frac1{p\ssub L(\Tgt L | \Src L)}
                % \right]
                \\
            + \sum_{L \in \Ed}
            (\gamma \alpha \ssub L - \beta\ssub L)
                \Ex_\mu \log \mu(\Tgt L | \Src L)
\end{align*}
The first line is the sum of a linear term and a convex one,
and each individual term on the second line is either convex or
concave, depending on the sign of the quantity $\gamma \alpha\ssub L -
\beta\ssub L$.  
Once we sort the terms into convex terms $f(\mu)$ and concave terms $g(\mu)$, we can choose an initial guess $\mu_0$, and iteratively use the convex solver to compute
%
\begin{align*}
    \mu_{t+1} &:= \argmin_{\mu} f(\mu) + (\mu - \mu_{t})^{\sf T}
        \nabla g(\mu_t)
\end{align*}

As we will see in \cref{sec:expts}, the aproach presented here section is 
not very fast, but it is guaranteed to make progress, since
\begin{align*}
    f(\mu_{t+1}) + g(\mu_{t+1}) &\le  f(\mu_{t+1}) + (\mu_{t+1}-\mu_t)^{\sf T} \nabla g(\mu_t) + g(\mu_t)
        % &\text{(concavity of $g$)}
        \\
    &\le  f(\mu_t) + (\mu_t - \mu_{t})^{\sf T}\nabla g(\mu_t)  + g(\mu_t)
        % &\text{(defn of argmin)}
        \\
    &= f(\mu_t) + g(\mu_t)
\end{align*}
and eventually find an optimum, because the concave terms are bounded.
%
% The slightly trickier part is combining this with the clique tree representation of \cref{sec:clique-tree-expcone}.
% \TODO[ TODO: Show the fancier spanning aborescence trick ]
%
% To combine this tech this with the clique tree representation of 
The only remaining difficulty is to do the convex optimization over
calibrated clique-trees. Fortunately, we already dealt with the 
tricky part (rewriting the joint entropy term as a disciplined convex program) in 
\cref{sec:clique-tree-expcone},
and so we defer the details to the appendix.
    
\section{OTHER APPROACHES TO PDG INFERENCE} \label{sec:other-inference}

\subsection{}
A stronger result than \cref{prop:markov-property} holds as well.
\begin{prop}\label{prop:same-set-dists}
    % For all $\gamma > 0$ and in the limit as $\gamma \to 0$, 
    Let $\Ed$ be a set of (hyper)edges over $\N$. 
    For every PDG $\dg M$ over $\N$ with edges $\Ed$, every $\gamma > 0$, and every optimum $\mu^* \in \bbr{\dg M}_\gamma^*$ of $\dg M$'s scoring function at $\gamma$, 
    there is a factor graph $\Phi$ with factors along $\Ed$ such that $\Pr_\Phi = \mu^*$. 
\end{prop}

In other words: every distribution that a PDG can pick out as optimal (for any choice of $\gamma > 0$ and also in the limit as $\gamma \to 0$), can also be described as a factor graph with the same structure as that PDG.
How do we square this with the \citeauthor{pdg-aaai}'s claim that PDGs are more general than factor graphs?

% This may be surprising, given how \citeauthor{pdg-aaai} position their model as strictly more expressive than other graphical models, because it implies that the optimal distribution 
\TODO[TODO: answer this question.\\
    The short answer: PDGs still compose differently, and in a way that respects the meaning of the probabilities. And just because you can find a factor graph that would have given you the right distribution after the fact, doesn't mean you could have specified the component factors.]
% The answer is simply that 


\TODO[Also: don't get lost; figure out how to continue as below:]
\cref{prop:same-set-dists} suggests another approach to avoiding an exponential representation of $\mu$: given a PDG, fit a factor graph that has the same structure to it. 

\subsection{Approximate Inference}
% \subsubsection{Relaxing the Marginal Polytope}
\textbf{Relaxing the marginal polytope.}
Just as it is possible to do belief propogation on cluster graphs that are not trees (e.g., loopy belief propogation)
so too is it possible to drop the requirement that the cluster that we use is indeed a tree-decomposition.
This program is smaller, and will converge, but it will only be an approximate solution. 
Like the original PDG itself, it might be inconsistent. 

\subsubsection{Variational Approaches}

% Because of the deep connection between variational approaches 
% shown in \parencite{one-true-loss}, there's 



% \section{IMPLEMENTATION} \label{sec:implementation}
\section{EMPIRICAL EVALUATION} \label{sec:expts}

\begin{figure}
    \includegraphics[width=\linewidth]{figs/resources-fine}
    \caption{
        The amount of resources: computation time (top) and maximum memory usage (bottom) for the various optimization methods (by color), as the size of the PDG increases, as measured by \texttt{n\_worlds} (right) and \texttt{n\_params} (left).
     }\label{fig:resources}
\end{figure}

\begin{figure}
    \includegraphics[width=\linewidth]{figs/gamma-vs-gap-bettergap}
    \caption{
        A graph of the gap (the difference between the attained objective value, and the best objective value obtained across all methods for that value of $\gamma$), 
        as $\gamma$ varies. As before, colors indicate method. 
        The size of the circle illustrates the relative number of worlds.
    }\label{fig:gamma-v-gap}
\end{figure}


\begin{figure}
    \includegraphics[width=\linewidth]{figs/2}
    \caption{
        A fine-grained variant of \cref{fig:gamma-v-gap}, which splits each method into sub-groups.
        The ExpCone methods \texttt{cvx\_opt\_joint} are split into two variants, depending on whether or not it also computed the second step described in \cref{sec:also-idef} to account for $\IDef{}$.
        The CCCP variants are \texttt{cccp\_opt\_joint} split into regimes where the entire problem is convex, and the entire problem is concave. The optimization approaches \texttt{opt\_dist} are split into three different optimizers: LBFGS, Adam, and accelerated Gradient Descent.
    }\label{fig:gamma-v-gap-fine}
\end{figure}

\begin{figure}
    \includegraphics[width=\linewidth]{figs/1}
    \caption{
        A fine-grained variant of the right half of \cref{fig:resources}, 
        with gap information on the left. 
    }\label{fig:gap-resource-fine}
\end{figure}


\begin{figure}
    \includegraphics[width=\linewidth]{figs/inc-idef2}
    \caption{An illustration of the trade-off between $\Inc$ and $\IDef{}$. Darker collors correspond to larger $\gamma$.}\label{fig:inc-idef}
\end{figure}

\subsection{Comparison To Belief Propogation}

Since PDGs generalize other graphical models, one might wonder how our method stacks up against them. 
We benchmarked against the small networks, and some of the medium-sized ones, from the \href{https://www.bnlearn.com/bnrepository/}{\texttt{bnlearn}} repository. 



\subsection{Evaluations On Random PDGSs}
We start by focusing on empirical properties of the optimization over joint distributions.

We generated several hundred PDGs with various properties: 9 or 10 variables, each of which can take 2-3 values. Each PDG contains 7-15 hyperedges, with 1-2 target nodes and 0-3 source nodes. The cpds are chosen by taking uniformly random numbers from [0,1] and normalizing appropriately, and every $\beta$ is set to 1.
For each PDG $\dg M$, we measure its complexity by:
\begin{itemize}[nosep]
    \item \texttt{n\_edges}, the number of edges in $\dg M$,
    \item \texttt{n\_params}, the total number of parameters across all the cpds of $\dg M$, and
    \item \texttt{n\_worlds}, the size of the joint distributions on the variables of $\dg M$.
\end{itemize}

\textbf{Capacity.} 
The black-box py-torch based approaches clearly have an edge in that they can handle larger models; see the cut-offs on the right sides of \cref{fig:resources,fig:gap-resource-fine}.

\textbf{Resource Costs.} 
Look at \cref{fig:resources}. 
Note that the exponential cone methods without the CCCP (blue and green) are actually faster than LBFGS, which was the best-performing torch optimizer. 
However, they use \emph{significantly} more memory, and cannot handle more than 8000 worlds. 


\textbf{Accuracy.}
In addition to being faster, the exponential cone techniques are also more preicse.
Note that the CCCP is typically more precise than the black-box
optimizers when the problem is fully convex $\gamma \le 1$, and
mirrors the performance of the exp-cone algorithms for the
%joe3
%quantitative limit on the left, in blue.  For combinations of larger
limit on the left, in blue.  For combinations of larger
$\gamma$ and more worlds however, the 20 iteration maximum we imposed
is not nearly enough to get convergence, and the black-box optimizers
are both faster and attain better objective values. 

\section{DISCUSSION}

% Our anaysis 
Our analysis shows that inference in PDGs with bounded tree-width can be done .


\TODO[ the below is a transplant without context; fix it ]
% Some queries are more difficult than others. 
Although we the question the same way, we also want to point out that there are other reasonable ways to answer that question once we move to PDGs.
Suppose we were looking at a BN in which it just so happens that $\mat X$ contains only a single variable and $\mat Y$ are the parents of $\mat X$.
In this case, our representation already contains the probabilities we are looking for, and we would be happy returning that row of the conditional probability table. 
But in a PDG, that cpd $p(\mat Y \,|\,\mat X)$, even if it is the only one
attached to an edge from $X$ to $Y$, may not be the same as $\bbr{\dg M}^*(Y|X)$.
As a consequence, 




\subsubsection*{Acknowledgements}
% All acknowledgments go at the end of the paper, including thanks to reviewers who gave useful comments, to colleagues who contributed to the ideas, and to funding agencies and corporate sponsors that provided financial support. 
% To preserve the anonymity, please include acknowledgments \emph{only} in the camera-ready papers.

\subsubsection*{References}
\printbibliography


\clearpage
\onecolumn
\appendix
\section{Proofs}

\recall{theorem:main}
\begin{lproof}\label{proof:main}
    We apply the analysis of \textcite{badenbroek2021algorithm}.
    The primal conic problem
    \[
        \inf_{x} \{\langle c, x\rangle : Ax = b, x \in K \tag{D}
    \]
    
\end{lproof}

\subsection{}

\recall{prop:smooth-and-strictly-cvx}
\begin{lproof}\label{proof:smooth-and-strictly-cvx}
	% First, we deal with the convexity, for which we make use of \cref{lem:cvx2}.
	% \commentout{
	% 	\def\mw#1{{\mat w}_{\!_{#1}}}
	% 	\def\ofmw(#1|#2){(\mw{#1} | \mw{#2})}
	% 	\begin{align*}
	% 		\aar{\dg M \bundle p}_\gamma &= \inf_\mu \Big[ \Inc_{\dg M \bundle p}(\mu)
	% 			+ \IDef{\dg M \bundle p}(\mu) \Big] \\
	% 		&=  \inf_{\mu} \Ex_{\mat w \sim \mu}
	% 			\left[\log \mu(\mat w) +
	% 			 	\beta_p \log \frac{\mu\ofmw(Y|X)}{p\ofmw(Y|X)} \; +  \!\sum_{\ed LAB} \beta_L \log \frac{\mu\ofmw(B|A)}{\bp\ofmw(B|A)} + \alpha_L \log \frac{0}{\mu\ofmw(B|A)}\right] \\
	% 		&= f
	% 	\end{align*}
	% }
	We start by expanding the definitions, obtaining
	\begin{align*}
		\aar{\dg M \bundle p}_\gamma &= \inf_\mu ~\bbr{\dg M \bundle p}_\gamma(\mu) \\
			&= \inf_\mu \left[ \bbr{\dg M }_\gamma(\mu)
				+ \Ex_{x\sim\mu_{\!_X}} \kldiv[\Big]{\mu(Y\mid x)}{p(Y\mid x)} \right]\\
			&= \inf_\mu \left[ \bbr{\dg M }_\gamma(\mu)
				+  \kldiv[\Big]{\mu(X,Y)}{p(Y \mid X)\, \mu(X)} \right].
	\end{align*}
	% % Choose $\gamma < \min (\{1\}\cup\{ \beta^{\dg M}_L : L \in \Ed^{\dg M}\})$.
	% Since $\bbr{\dg M}_\gamma$ is a $\gamma$-strongly convex function of $\mu$ for all
	% such $\gamma < \min_L \beta_L$, and
	% $\kldiv{\mu_{XY}}{\mu_X \; p_{Y\mid X}}$ is 1-strongly
	% convex in $p$ for fixed $\mu$ (\cref{lem:Dstrongcvx}),
	% % $\thickD$ is convex in both of its arguments,
	% their sum is $\gamma$-strongly convex in $\mu$ and in $p$.
	% By \cref{lem:cvx2} taking an infemum preserves this convexity,
	% and so
	% $
	%  	\inf_\mu \left[ \bbr{\dg M }_\gamma(\mu)
	% 	+  \kldiv[\big]{\mu_{XY}}{p_{Y \mid X}\; \mu_X} \right]
	% $, which equals $\aar{\dg M \bundle p}_\gamma$,
	% is $\gamma$-strongly convex in $p$.
	% % $\aar{\dg M \bundle p}_\gamma$ is smooth
	% % Smoothness.


	% Choose $\gamma < \min (\{1\}\cup\{ \beta^{\dg M}_L : L \in \Ed^{\dg M}\})$.
	Fix $\gamma < \min_L \beta_L$. Then we know that $\bbr{\dg X}_\gamma(\mu)$ is a $\gamma$-strongly convex function for every PDG $\dg X$, and hence there is a unique joint distribution which minimizes it.

	\textbf{Strict Convexity.}
	Suppose $p_1(Y \mid X)$ and $p_2(Y\mid X)$ are two cpds on $Y$ given $X$.
	Fix $\lambda \in [0,1]$, and set $p_\lambda = (1-\lambda) p_1 + \lambda p_2$.
	Let $\mu_1, \mu_2$ and $\mu_\lambda$ be the joint distributions that minimze $\bbr{\dg M \bundle p_1}_\gamma$, $\bbr{\dg M \bundle p_2}_\gamma$ and $\bbr{\dg M \bundle p_\lambda}_\gamma$, respectively.  Then we have
	\begin{equation*}
		\aar{\dg M \bundle p_\lambda}_\gamma
			= \bbr{\dg M}_\gamma(\mu_\lambda) + \kldiv[\Big]{\mu_\lambda(X,Y)}{p_\lambda(Y\mid X) \mu_\lambda( X)}.
	\end{equation*}
	By convexity of $\bbr{\dg M}$ and $\thickD$, we have
	\begin{align}
		\bbr{\dg M}_\gamma(\mu_\lambda)
		 	&\le (\lambda-1)\bbr{\dg M}_\gamma(\mu_1) + \lambda \bbr{\dg M}_\gamma(\mu_2)
			 	\label{eqn:score-cvx}\\
		\text{and}\qquad \kldiv[\Big]{\mu_\lambda(XY)}{p_\lambda(Y | X) \mu_\lambda( X)}
			&\le (1-\lambda)\kldiv[\Big]{\mu_1(XY)}{p_1(Y | X) \mu_1( X)} \nonumber \\
			&\qquad+ \lambda\;\;\kldiv[\Big]{\mu_2(XY)}{p_2(Y | X) \mu_2( X)}.
				\label{eqn:D-cvx}
	\end{align}
	If $\mu_1 \ne \mu_2$ then since $\bbr{\dg M}$ is strictly convex, \eqref{eqn:score-cvx} must
	be a strict inequality. On the other hand, if $\mu_1 = \mu_2$, then since $\mu_\lambda = \mu_1 = \mu_2$ and $\thickD$ is stricly convex in its second argument when its first argument is fixed (\Cref{lem:Dstrongcvx}), \eqref{eqn:D-cvx} must be a strict inequality.
	In either case, the sum of the two inequalities must be strict, giving us
	\begin{align*}
		\aar{\dg M \bundle p_\lambda}_\gamma &=
		\bbr{\dg M}_\gamma(\mu_\lambda) + \kldiv[\Big]{\mu_\lambda(XY)}{p_\lambda(Y | X) \mu_\lambda( X)} \\
		&<
		 (\lambda-1) \left[\bbr{\dg M}_\gamma(\mu_1)
			 	+ \kldiv[\Big]{\mu_1(XY)}{p_1(Y | X) \mu_1( X)} \right]
			 \\[-0.3em]&\qquad\qquad
			 + \lambda \left[ \bbr{\dg M}_\gamma(\mu_2)
			 	+ \kldiv[\Big]{\mu_2(XY)}{p_2(Y | X) \mu_2( X)}
			 	\right] \\
		 &= (\lambda-1) \aar{\dg M \bundle p_1} + \lambda\,\aar{\dg M \bundle p_2},
	\end{align*}
	which shows that $\aar{\dg M \bundle p}$ is \emph{strictly} convex in $p$, as desired.


	\textbf{Smoothness.}
	If $\bbr{\dg M \bundle p}_\gamma^*$ is a positive distribution, then by definition $\bbr{\dg M \bundle p}$ achieves its minimum on the interior of the probability simplex $\Delta \V(\dg M \bundle p)$, and so by \Cref{lem:cvx4}, we immediately find that $\aar{\dg M \bundle p}_\gamma$ is smooth in $p$.

	Now, suppose that $\bbr{\dg M \bundle p}_\gamma^*(\mat w) = 0$,  for some $\mat w \in \V(\dg M \bundle p)$.

	Applying \Cref{lem:cvx4} to the function $f = \bbr{\dg M}_\gamma$

	Now for the second case.

	\TODO

	If $x^*_b \in \partial X$, then we claim that either
	\begin{enumerate}[nosep]
		\item There is a subspace $T \subseteq \mathbb R^{m}$ with
			$\SD{}$
	 	\item There is a subspace $S \subseteq \mathbb R^{n}$ with
			$x^*_b \in S \cap \partial X$ such

	\end{enumerate}

\end{lproof}

\begin{lemma}\label{lem:cvx4}
	Let $X$ and $Y$ be convex sets, and
	$f : X \times Y \to \mathbb R$ be a smooth $(C^\infty)$, convex function.
	If $f$ is strictly convex in $X$, and for some $y_0 \in Y$, $f(x, y_0)$ achieves its infemum on the interior of $X$.
	then $y\mapsto \inf_x f(x, y)$ is smooth $(C^\infty)$ at the point $y_0$.
\end{lemma}

\begin{lproof}%[Proof of \Cref{lem:cvx4}]
	% Let $f_y(x) = f(x,y)$.
	% Since $f$ is smooth and stritly convex, each restriction $f_y$ of $f$ to a
	% particular $y$ is also smooth and strictly convex.
	% As a result, each $f_y$ has a unique minimum $m_y := \inf_{x} f_y(x)$.
	% As $f_y$ is smooth, $m_y$ is either a boundary point, or
	% at a point where $\nabla f_y = 0$.
	%
	% Moreover, it is a constrained optimization problem, so
	% $\nabla_{x,y,\lambda} [ f(x,y) + \lambda (y_0 - y)] = 0$.
	%
	% \TODO
	Let $x_0^* := \arg\min_x f(x,y_0)$, which is achieved by assumption, and is unique because $f(-,y_0)$ is strictly convex.

	We will ultimately apply the implicit function theorem to give us a smooth function which is equal to this infemum, but to do so we must deal with the technicality that it requires an open set; the boundary is the most complicated part of this result.
	Here we have essentially required that the domain be open by fiat for $X$, but for $Y$ (which is a possibly non-open subset of $\mathbb R^m$), we use the Extension Lemma for smooth functions \cite[Lemma 2.26]{Lee.SmoothManifolds}. In our context, it states that
	for every open set $U$ with $\overline{Y} \subseteq U \subseteq \mathbb R^m$,
	there exists a function $\tilde f : X \times \mathbb R^m \to \mathbb R$, such that $\tilde f |_{Y} = f$ (and $\supp \tilde f \subseteq U$).
	We only need a small fraction of this power: that we can smoothly extend $f$ to \emph{some} open set of $\mathbb R^m$, which we fix and call $\tilde Y$.

	% Similarly, for other $y \in Y$, let $x^*_y$ be the unique value of $x$ which minimizes $f(x,y)$.

	% \textbf{Smoothness.}
	% By assumption, $x^*_b$ is not a boundary point of $X$.
	%
	We claim that now all conditions for the Implicit Function Theorem are met if invoked with
		$\phi(y,x) := \vec\nabla_x \tilde f(x,y)$ and $(\mat b,\mat a) = (y_0, x^*_0)$.
	Concretely, we have $m = \mathop{dim} X$, $n = \mathop{dim} Y$, and $Z = (\tilde Y \times X)^\circ$, i.e., the interior of $\tilde Y \times X$, which is open and contains $(\mat b, \mat a)$.
	 Becuase $\phi$ is smooth, it is $k$-times differentiable for all $k$. We have $\vec\nabla_x \tilde f (y_0, x^*_0) = \vec 0$ because $x^*_0$ is a local minimum of the smooth function $\tilde f(-, y_0)$ which lies on the interior of $X$.

	Moreover, the Jacobian matrix
	\[ \mat J_{\nabla\!\tilde f, x}(y_0, x_0^*) = \left[ \frac{\partial^2 f}{\partial x_i \partial x_j}(x^*_0, y_0) \right]\]
	is the Hessian of the strictly convex funtion $f(-, b)$, and therefore positive definite (and in particular non-singular).
	Therefore, the Implicit Function Theorem guarantees us the existence of a neighborhood $U \subset \tilde Y$ of $y_0$ for which
	there is a unique $k$-times differentiable function $g: U \to X$ such that $g(y_0) = x^*_0$ and $\vec\nabla_x \tilde f(y, g(y)) = 0$ for all $y \in U$. Of course, this implies $g(y) = \argmin_x f(x,y)$ at every such point, and $\inf_x f(x,y) = f(g(y),y)$ is a composition of the smooth function $f$ with the $k$-times differentiable function $g \otimes \mathrm{id}_Y$.
	Therefore, $\inf_x f(x,y)$ is itself $k$-times continuously differentiable at $y_0$ for all $k$, or in other words, $\inf_x f(x,y)$ is smooth at $y=y_0$.
\end{lproof}

\recall{prop:markov-property}
\begin{lproof}
	Choose $\mu \in \bbr{\dg M_1 \bundle \dg M_2}^*_\gamma$.
	% Choose $\mu \in \mu^*_\gamma (\dg M_1 \bundle \dg M_2)$.
	Let $\mu' := \mu(\N_1) \mu(\N_2)$
	
	\TODO[Finish Transcribing Proof]
\end{lproof}


\subsection{Hardness Results}

\recall{prop:consistent-NP-hard}
\begin{lproof} \label{proof:consistent-NP-hard}
	We can directly encode SAT problems as PDGs.
	Specifically, let
	$$\varphi := \bigwedge_{j \in \mathcal J} \bigvee_{i \in \mathcal I(j)} (X_{j,i})$$
	be a CNF formula over binary variables $\mat X := \bigcup_{j,i} X_{j,i}$. Let
	$\dg M_\varphi$ be the PDG containing every variable $X \in \mat X$ and a binary
	variable $C_j$ (taking the value 0 or 1) for each clause $j \in \mathcal J$, as well as the following edges, for each $j \in \mathcal J$:
	%\{$``$\varphi(\mat X)$''$\}$ with $\V(\varphi) = \{0,1\}$, and
	\begin{itemize}
		\item a hyperedge $\{X_{j,i} : i \in \mathcal I(j)\} \tto C_j$, together with a degenerate cpd
			encoding the boolean OR function (i.e., the truth of $C_j$ given $\{X_{j,i}\}$);
		\item an edge $\pdgunit \tto C_j$, together with a cpd asserting $C_j$ be equal to 1.
	\end{itemize}
	% We give each edge $\alpha = 0$ and $\beta = 1$.
	First, note that the number of nodes, edges, and non-zero entries in the cpds are polynomial in the $|\mathcal J|, |\mat X|$, and the total number of parameters in a simple matrix representation of the cpds is also polynomial if $\mathcal I$ is bounded (e.g., if $\varphi$ is a 3-CNF formula).
	A satisfying assignment $\mat x \models \varphi$ of the variables $\mat X$ can be regarded as a degenerate joint distribution $\delta_{\mat X = \mat x}$ on $\mat X$, and extends uniquely to a full joint distribution $\mu_{\mat x} \in \Delta \V(\dg M_\varphi)$ consistent with all of the edges, by
	\[ \mu_{\mat x} = \delta_{\mat x} \otimes \delta_{\{C_j = \vee_i  x_{j,i}\}} \]

 	Conversely, if $\mu$ is a joint distribution consistent with the edges above, then any point $\mat x$ in the support of $\mu(\mat X)$ must be a satisfying assignment, since the two classes of edges respectively ensure that $1 =\mu(C_j\!=\! 1 \mid \mat X \!=\! \mat x) = \bigvee_{i \in \mathcal I(j)} \mat x_{j,i}$ for all $j \in \mathcal J$, and so $\mat x \models \varphi$.

	Thus, $\SD{\dg M_\varphi} \ne \emptyset$ if and only if $\varphi$ is satisfiable, so
	an algorithm for determining if a PDG is consistent can also be adapted (in polynomial space and time) for use as a SAT solver, and so the problem of determining if a PDG consistent is NP-hard.

% \end{lproof}
% \recall{prop:sharp-p-hard}
% \begin{lproof}\label{proof:sharp-p-hard}
    
    \medskip\hrule\smallskip
    
	\textbf{PART (b).}
    We prove this by reduction to \#SAT. Again, let $\varphi$ be some CNF formula over $\mat X$, and construct
	$\dg M_\varphi$ as in \hyperref[proof:consistent-NP-hard]{the proof} of
	\Cref{prop:consistent-NP-hard}.
	Furthemore, let $\bbr{\varphi} := \{ \mat x : \mat x \models \varphi \}$ be the set of  assingments to $\mat X$ satisfying $\varphi$, and $\#_\varphi := |\bbr{\dg M}|$ denote the number such assignments. We now claim that
	\begin{equation}\label{eqn:number-of-solns}
		\#_\varphi = \exp \left[- \frac1\gamma \aar{ \dg M_\varphi }_\gamma \right].
	\end{equation}
 	If true, we would have a reduced the \#P-hard problem of computing $\#_\varphi$ to the problem of computing $\aar{\dg M}_\gamma$ for fixed $\gamma$. We now proceed with proof \eqref{eqn:number-of-solns}.
	By definition, we have
	\[ \aar{\dg M_\varphi}_\gamma = \inf_\mu \Big[ \Inc_{\dg M_\varphi}(\mu) + \gamma \IDef{\dg M_\varphi}(\mu) \Big]. \]
	We start with a claim about first term.
	% For the particular PDG $\dg M_\varphi$, the

	\begin{iclaim} \label{claim:separate-inc-varphi}
		% $\Inc(\dg M_\varphi)$ is finite if and only if $\varphi$ is statisfiable.
		$\Inc_{\dg M_\varphi}\!(\mu) =
		% \begin{cases}
		% 	0 & \text{if}~  \mat x \models \varphi~\text{and}~\mat c = \mat 1
		% 	 	~\text{for all}~(\mat x, \mat c) \in \supp \mu\\
		% 	\infty & \text{otherwise}
		% \end{cases}
		\begin{cases}
			0 & \text{if}~  \supp \mu \subseteq \bbr{\varphi} \times \{ \mat 1\} \\
			\infty & \text{otherwise}
		\end{cases}$.
	\end{iclaim}
	\vspace{-1em}
	\begin{lproof}
		Writing out the definition explicitly, the first can be written as
		\begin{equation}
			\Inc_{\dg M_\varphi}\!(\mu) = \sum_{j} \left[ \kldiv[\Big]{\mu(C_j)}{\delta_1} +
				\Ex_{\mat x \sim \mu(\mat X_j)} \kldiv[\Big]{\mu(C_j \mid \mat X_j = \mat x)}{\delta_{\lor_i \mat x_{j,i}}} \right], \label{eqn:explicit-INC-Mvarphi}
				% &= \sum_{j} \left[
				% 	\begin{matrix} \mu(C_j\!=\!0) (\infty) \\
				% 	 	+ \mu(C_j \!=\! 1) \log \mu(C_j \!=\! 1)
				% 	\end{matrix} +
				% 	\Ex_{\mat x \sim \mu(\mat X_j)} \kldiv[\Big]{\mu(C_j \mid \mat X_j = \mat x)}{\delta_{\lor_i \mat x_i}} \right],
		\end{equation}
		where $\mat X_j = \{X_{ij} : j \in \mathcal I(j)\}$ is the set of variables that
		appear in clause $j$, and $\delta_{(-)}$ is the probability distribution placing all mass on the point indicated by its subscript.
		As a reminder, the relative entropy is given by
		\[ \kldiv[\Big]{\mu(\Omega)}{\nu(\Omega)} := \Ex_{\omega \sim \mu} \log \frac{\mu(\omega)}{\nu(\omega)},
		\quad\parbox{1.4in}{\centering and in particular, \\ if $\Omega$ is binary,}\quad
			\kldiv[\big]{\mu(\Omega)}{\delta_\omega} = \begin{cases}
				0 &  \text{if}~\mu(\omega) = 1 ; \\
				\infty & \text{otherwise}.
		\end{cases} \]
		Applying this to \eqref{eqn:explicit-INC-Mvarphi}, we find that either:
		\begin{enumerate}[itemsep=0pt]
			\item Every term of \eqref{eqn:explicit-INC-Mvarphi} is finite (and zero) so $\Inc_{\dg M_\varphi}(\mu) = 0$, which happens when $\mu(C_j = 1) = 1$ and $\mu(C_j = \vee_i~ x_{j,i}) = 1$ for all $j$.  In this case, $\mat c = \mat 1 = \{ \vee_i~x_{j,i} \}_j$ so $\mat x \models \varphi$ for every $(\mat{c,x}) \in \supp \mu$;
			\item Some term of \eqref{eqn:explicit-INC-Mvarphi} is infinite, so that $\Inc_{\dg M_\varphi}(\mu) = \infty$, which happens if some $j$, either

			\begin{enumerate}
				\item $\mu(C_j \ne 1) > 0$ --- in which case there is some $(\mat{x,c}) \in \supp \mu$ with $\mat c \ne 1$, or
				\item $\supp \mu(\mat C) = \{\mat 1\}$, but $\mu(C_j \ne \vee_i~ x_{j,i}) > 0$ --- in which case there is some $(\mat{x,1}) \in \supp \mu$ for which $1 = c_j \ne \vee_i~x_{j,i}\;$, and so $\mat x \not\models \varphi$.
			\end{enumerate}
		\end{enumerate}
		Condensing and rearranging slightly, we have shown that
		\[
			\Inc_{\dg M_\varphi}(\mu) =
			\begin{cases}
				0 & \text{if}~  \mat x \models \varphi~\text{and}~\mat c = \mat 1
				 	~\text{for all}~(\mat x, \mat c) \in \supp \mu\\
				\infty & \text{otherwise}
			\end{cases}~.
		\]
		% So if $\mat x \models \varphi$ for all $\mat x \in \supp \mu(X)$,
		%
		% $\Inc_{\dg M_\varphi}(\mu) = 0$
		% The first term is infinite if $\mu(C_j = 1) < 1$, and the second is infinite
		% if $\mu(C_j = \lor_i X_{i,j}) < 1$. Thus, if $\Inc_{\dg M_\varphi}(\mu)$ is finite, then $\mat x \sim \mu(\mat X)$ satisfies $\varphi$ with probability 1, and $\varphi$ must be satisfiable.
		% Conversely,
	\end{lproof}

	% Thus, if $\Inc_{\dg M_\varphi}(\mu)$ is finite, then every $\mat x \in \supp \mu$ is a satisfying assignment of $\varphi$.
	Because $\IDef{}$ is bounded, it follows immediately that
 	$\aar{\dg M_\varphi}_\gamma$, is finite if and only if
	there is some distribution $\mu \in \Delta\V(\mat X,\mat C)$ for which $\Inc_{\dg M_\varphi}(\mu)$ is finite, or equivalently, by \Cref{claim:separate-inc-varphi}, iff there exists some $\mu(\mat X) \in \Delta \V(\mat X)$ for which $\supp \mu(\mat X) \subseteq \bbr{\varphi}$, which in turn is true if and only if $\varphi$ is satisfiable.

	In particular, if $\varphi$ is not satisfiable (i.e., $\#_\varphi = 0$), then $\aar{\dg M_\varphi}_\gamma = +\infty$, and
	\[
		\exp \left[ -\frac1\gamma \aar{\dg M_\varphi}_\gamma \right] =
	 		\exp [ - \infty ] = 0 = \#_\varphi,
	\]
	so in this case \eqref{eqn:number-of-solns} holds as promised. On the other hand, if $\varphi$ \emph{is} satisfiable, then, again by \Cref{claim:separate-inc-varphi}, every $\mu$ minimizing $\bbr{\dg M_\varphi}_\gamma$, (i.e., every $\mu \in \bbr{\dg M_\varphi}_\gamma^*$) must be supported entirely on $\bbr{\varphi}$ and have $\Inc_{\dg M_\varphi}\!(\mu) = 0$.  As a result, we have
	\[
		\aar{\dg M_\varphi}_\gamma =
			\inf\nolimits_{\mu \in \Delta \big[\bbr{\varphi} \times \{\mat 1\}\big]} \gamma\; \IDef{\dg M_\varphi}(\mu) .
	\]
	A priori, by the definition of $\IDef{\dg M_\varphi}$, we have
	\[
		\IDef{\dg M_\varphi}(\mu) =
		 	- \H(\mu) + \sum_{j} \Big[ \alpha_{j,1} \H_\mu(C_j \mid \mat X_j)
						+ \alpha_{j,0} \H_\mu(C_j) \Big],
	\]
	where $\alpha_{j,0}$ and $\alpha_{j,1}$ are values of $\alpha$ for the edges of $\dg M_\varphi$, which we have not specified because they are rendered irrelevant by the fact that their corresponding cpds are deterministic. We now show how this plays out in the present case.
	Any $\mu \in \Delta\big[\bbr{\varphi} \times \{\mat 1\}\big]$ we consider has a degenerate marginal on $\mat C$. Specifcally, for every $j$, we have $\mu(C_j) = \delta_1$, and since entropy is non-negative and never increased by conditioning,
	$$
		0 \le \H_\mu(C_j \mid \mat X_j) \le \H_\mu(C_j) = 0.
	$$
	Therefore, $\IDef{\dg M_\varphi}(\mu)$ reduces to the negative entropy of $\mu$.
	Finally, making use of the fact that the maximum entropy distribution $\mu^*$ supported on a finite set $S$ is the uniform distribution on $S$, and has $\H(\mu^*) = \log | S |$, we have
	\begin{align*}
		\aar{\dg M_\varphi}_\gamma &= \inf\nolimits_{\mu \in \Delta \big[\bbr{\varphi} \times \{\mat 1\}\big]} \gamma\; \IDef{\dg M_\varphi}(\mu) \\
			&= \inf\nolimits_{\mu \in \Delta \big[\bbr{\varphi} \times \{\mat 1\}\big]} -\, \gamma\, \H(\mu) \\
			&= - \gamma\, \sup\nolimits_{\mu \in \Delta \big[\bbr{\varphi} \times \{\mat 1\}\big]}  \H(\mu) \\
			&= - \gamma\, \log (\#_\varphi),
	\end{align*}
	\hspace{1in}giving us
	$$
		\#_\varphi = \exp \left[- \frac1\gamma \aar{ \dg M_\varphi }_\gamma \right],
	$$
	as desired. We have now reduced \#SAT to computing $\aar{\dg M}_\gamma$, for $\gamma \in \mathbb R^{>0}$ and an arbitrary PDG $\dg M$, which is therefore \#P-hard.
\end{lproof}


\section{}
\begin{conj}
    Inference in a PDG---that is, computing conditional marginals of $\bbr{\dg M}^*$---%
    and the computing inconsistency $\aar{\dg M}$ are equally difficult:
        there are polynomial-time reductions from each to the other.
\end{conj}

If we do not restrict to finite variables, then the problem is much worse.

\begin{linked}{conj}{incomputable}
    The problem of deciding whether a PDG whose variables take values in $\mathbb N$ is not computable.
\end{linked}

\end{document}
