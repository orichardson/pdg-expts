%%%%%%%%%%% DEFINITION OF IDEF %%%%%%%%%%%%%
% By contrast, there is also a ``qualitative'' term, which  measures
%oli2: I don't want to go here yet because we haven't worked out the details.
%in particular, I don't like "how far" analogy so well here, because IDef can
%be negative. I tried to rewrite it
% There is also another aspect of inconsistency of a disribution $\mu$
% with respect to a PDG ${\dg M}$: how far $\mu$ is from modeling the
% treating the edges in ${\dg M}$ as describing independent mechanisms
% that determine the target given the source.  This is captured by the
% \emph{information deficiency}, given by
% The second way in which we evalute a distribution $\mu$ is
% treating the edges in ${\dg M}$ as describing independent mechanisms
% that determine the target given the source.
%
%joe3*: this is *not* captured by the information deficiency.  IDef is
%a function of \mu, so at best it's capturing some relationship
%between \mu and the PDG.   You need to describe what that
%relationship is.  Although IDef can be negative (which is part of why I have
%0 intuition for it), I think what I wrote is far more accurate that
%what you wrote
%oli3*: updated
% The second way in which we score a distribution $\mu$,
% There is also a second aspect of how well a distribution $\mu$ fits $\dg M$:
% There is a second kind of discrepency between $\mu$ and $\dg M$, of a more structural flavor:


%joe2*: the weights are a bit of a red herring.  once we have a clear
%intuition for IDef (and I think I now see how the intuition should
%go)  then we're just multiplying by the confidence, because the
%confidence is indicating the probabiility that the edge is there.  We
%need to make clear the basic intuition without \alpha.
%oli2*: I feel like you're missing the point I was trying to make.
% It happens that the presence or absence of edges can be encoded
% with ones and zeros in the weights. I'm not emphasizing the continuum
% aspect of the weights. I'm emphasizing that it depends only on the (degree of)
% presence or absence of an edge, which is the weight \alpha --- and not on the
% cpds or the nature of the variables involved.


%%%%%%% COMBINING INTO SCORING FNS %%%%%%%
%joe2*: First of all, this is not one semantics, but a family of
%semantics, indexed by \gamma.  Second, you need to give INTUITION for
%\gamma.
%oli2: OK, I've reworded it, although personally I don't think it's important to make such a distinction; it can be a single semantics that is a map from pairs (\mu, \gamma) to extended reals, just as easily  as it can be a family of semantics, indexed by \gamma, each mapping \mu to extended reals.



%joe2*: We need INTUITION.  why do we care about what happens as
%\gamma -> 0.  If we can't motivate this well, there's no reason for a
%reader to be interested in the rest of the paper, so this is critical.
% $\gamma$ controls the trade-off
% between matching quantitative beliefs and qualitative ones.
%
%joe4
% The notation 



%%% observational limit
% $\epsilon$-inference, on the other hand, is different.
% One case of particular interest is the limit as $\gamma \to 0$,
% which corresponds to a fully empirical approach: matching quantitative observations is the primary concern, and causal information is used only to break ties.
% in which observational data dominates, and
% structural information is used only to break ties.
% observational data is strictly more important than raw structural information.
% raw structural information is used only to break ties.
% When $\gamma$ is small enough.
% So long as $\bbeta > \mat 0$,
% So long as $\bbeta \gg \balpha$,
% It also has an interesting property.
% It is also of practical value, because under


% reading of $\dg M$, in which concrete observations and data trump causal structure.
% primary objective of this paper is to do inference with respect to this distribution.
% Why this limit in particular?
%
%joe2*: We need INTUITION.  why do we care about what happens as
%\gamma -> 0.  If we can't motivate this well, there's no reason for a
%reader to be interested in the rest of the paper, so this is critical.
%
% \TODO[ In my opinion is way too much intuition for $\gamma$, but I'll put it all here so it can be pared down. ]
%
% \begin{enumerate}[nosep]
%     \item %1.
% This quantitative limit is what is used to generate nearly all of the loss functions in \textcite{one-true-loss}, which are largely empirical in nature.
%     \item %2.
% Optimizing inconsistency in this limit guarantees a calibrated model, which is one of the biggest advantages PDGs have over factor graphs
%          \parencite[Example 5]{pdg-aaai}.
%
%     \item %3.
% When $\balpha = \mat 0$, this is the principle of maximum entropy; for other values of $\balpha$ it is a causally sensitive variant which accounts for the fact that cpd constraints themselves carry different amounts of entropy depending on the settings of the variables.
%     \item %4.
% Another reason to focus on the quantitative limit is pragmatic: there is a unique optimal joint distribution, $\bbr{\dg M}^*$ (at least if $\bbeta > \mat 0$).
%  In any case, this is the way in which PDGs define a unique joint distribution, and hence may be thought of as a graphial model.
% \end{enumerate}
% This distribution uniquely achieves the smallest information deficiency among those distributions maximally compatible with $\dg M$.

%%% inconsistency
%joe2: There is not a unique "inconsistency" of a PDG.  Again, you're
%giving a family of inconsistencies, indexed by \gamma.  So, at best,
%you can talk about the inconsistency relative to \gamma (which you
%must MOTIVATE).



%%%%%%%%%%%%%%%%%%%%% PDG INFERENCE AS CVX PROGRAM %%%%%%%%%%%%%%%%%%%%%%%%%%%
%Is it a convex program?
% More importantly, is it a \emph{disciplined} convex program,
%     which would mean it can be solved in polynomial time?
% That depends on $\gamma$.
%
%
%joe1
%We now present the central finding of our paper:  observation that the
%oli1: it's the central finding in that it's the lynchpin, but this is not the main result; nobody should care about it by itself. It's more like the key lemma.
% We now prove our main result: that the
% We now present the key technical finding of our paper: that the
% PDG scoring function $\bbr{\dg M}_\gamma$
% \eqref{eqn:scoring-fn}
% can be written as an exponental conic program, or sequence thereof.
% This requires different approaches, depending on the value of $\gamma$.
