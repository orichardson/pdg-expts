\documentclass[twoside]{article}

\usepackage{aistats2023}

% If your paper is accepted, change the options for the package
% aistats2023 as follows:
%
%\usepackage[accepted]{aistats2023}
%
% This option will print headings for the title of your paper and
% headings for the authors names, plus a copyright note at the end of
% the first column of the first page.

% If you set papersize explicitly, activate the following three lines:
%\special{papersize = 8.5in, 11in}
%\setlength{\pdfpageheight}{11in}
%\setlength{\pdfpagewidth}{8.5in}

% If you use natbib package, activate the following three lines:
%\usepackage[round]{natbib}
%\renewcommand{\bibname}{References}
%\renewcommand{\bibsection}{\subsubsection*{\bibname}}

% If you use BibTeX in apalike style, activate the following line:
%\bibliographystyle{apalike}
%% TODO make pdg.sty file that allows you to import all PDG macros.
%%%%%%%%%%


\relax % Writing Tools
    \newcommand{\TODO}[1][INCOMPLETE]{{\color{red}\hangindent=0.5cm\rightskip=0.8cm$\smash{\Big\langle}$~\texttt{#1}~\raisebox{-0.3ex}{${\Big\rangle}$}\hspace{-1.5cm}\par}}


\relax
	\DeclareMathOperator*{\argmin}{arg\,min}
	\newcommand{\bundle}{\mathbin{+}}
    \newcommand{\Rext}{\mskip1mu\overline{\mskip-1mu\mathbb R\!}\,}

\relax
    %% Narrowing
    \usepackage{keyval}
    \makeatletter
    \define@key{setpar}{left}[0pt]{\leftmargin=#1}
    \define@key{setpar}{right}[0pt]{\rightmargin=#1}
    \define@key{setpar}{both}{\leftmargin=#1\relax\rightmargin=#1}
    \makeatother

    \newenvironment{narrow}[1][]
      {\list{}{\setkeys{setpar}{left,right}%
         \setkeys{setpar}{#1}%
         \listparindent=\parindent
         \topsep=0pt
         \partopsep=0pt
         \parsep=\parskip}\item\relax\hspace*{\listparindent}\ignorespaces}
      {\endlist}
    % \newenvironment{abstract}
    %     {\narrow[both=1in]\small}         
    %     {\endnarrow}


\relax % Bibliography
    \usepackage[backend=biber, style=authoryear]{biblatex}
    % \usepackage[backend=biber,style=authoryear,hyperref=true]{biblatex}
    \addbibresource{refs.bib}

    \DeclareLanguageMapping{american}{american-apa}
    % \renewcommand*{\nameyeardelim}{\addcomma\space}
    \DeclareDelimFormat{nameyeardelim}{\addcomma\space}
    % \listfiles

    \DeclareFieldFormat{citehyperref}{%
      \DeclareFieldAlias{bibhyperref}{noformat}% Avoid nested links
      \bibhyperref{#1}}

    \DeclareFieldFormat{textcitehyperref}{%
      \DeclareFieldAlias{bibhyperref}{noformat}% Avoid nested links
      \bibhyperref{%
        #1%
        \ifbool{cbx:parens}
          {\bibcloseparen\global\boolfalse{cbx:parens}}
          {}}}

    \savebibmacro{cite}
    \savebibmacro{textcite}

    \renewbibmacro*{cite}{%
      \printtext[citehyperref]{%
        \restorebibmacro{cite}%
        \usebibmacro{cite}}}

    \renewbibmacro*{textcite}{%
      \ifboolexpr{
        ( not test {\iffieldundef{prenote}} and
          test {\ifnumequal{\value{citecount}}{1}} )
        or
        ( not test {\iffieldundef{postnote}} and
          test {\ifnumequal{\value{citecount}}{\value{citetotal}}} )
      }
        {\DeclareFieldAlias{textcitehyperref}{noformat}}
        {}%
      \printtext[textcitehyperref]{%
        \restorebibmacro{textcite}%
        \usebibmacro{textcite}}}


\usepackage{tikz}
	\usetikzlibrary{positioning,fit,calc, decorations, arrows, shapes, shapes.geometric}
	\usetikzlibrary{cd}

	%%%%%%%%%%%%
	\tikzset{AmpRep/.style={ampersand replacement=\&}}
	\tikzset{center base/.style={baseline={([yshift=-.8ex]current bounding box.center)}}}
	\tikzset{paperfig/.style={center base,scale=0.9, every node/.style={transform shape}}}

	% Node Stylings
	\tikzset{dpadded/.style={rounded corners=2, inner sep=0.7em, draw, outer sep=0.3em, fill={black!50}, fill opacity=0.08, text opacity=1}}
	\tikzset{dpad0/.style={outer sep=0.05em, inner sep=0.3em, draw=gray!75, rounded corners=4, fill=black!08, fill opacity=1, align=center}}
	\tikzset{dpadinline/.style={outer sep=0.05em, inner sep=2.5pt, rounded corners=2.5pt, draw=gray!75, fill=black!08, fill opacity=1, align=center, font=\small}}

 	\tikzset{dpad/.style args={#1}{every matrix/.append style={nodes={dpadded, #1}}}}
	\tikzset{light pad/.style={outer sep=0.2em, inner sep=0.5em, draw=gray!50}}

	\tikzset{arr/.style={draw, ->, thick, shorten <=3pt, shorten >=3pt}}
	\tikzset{arr0/.style={draw, ->, thick, shorten <=0pt, shorten >=0pt}}
	\tikzset{arr1/.style={draw, ->, thick, shorten <=1pt, shorten >=1pt}}
	\tikzset{arr2/.style={draw, ->, thick, shorten <=2pt, shorten >=2pt}}

	\newcommand\cmergearr[5][]{
		\draw[arr, #1, -] (#2) -- (#5) -- (#3);
		\draw[arr, #1, shorten <=0] (#5) -- (#4);
		}
	\newcommand\mergearr[4][]{
		\coordinate (center-#2#3#4) at (barycentric cs:#2=1,#3=1,#4=1.2);
		\cmergearr[#1]{#2}{#3}{#4}{center-#2#3#4}
		}
	\newcommand\cunmergearr[5][]{
		\draw[arr, #1, -, shorten >=0] (#2) -- (#5);
		\draw[arr, #1, shorten <=0] (#5) -- (#3);
		\draw[arr, #1, shorten <=0] (#5) -- (#4);
		}
	\newcommand\unmergearr[4][]{
		\coordinate (center-#2#3#4) at (barycentric cs:#2=1.2,#3=1,#4=1);
		\cunmergearr[#1]{#2}{#3}{#4}{center-#2#3#4}
		}


\relax %% Double delimeters; I need this for pdg macros \aar and \bbr
    \newcommand{\nhphantom}[2]{\sbox0{\kern-2%
    \nulldelimiterspace$\left.\delimsize#1\vphantom{#2}\right.$}\hspace{-.97\wd0}}
    % \nulldelimiterspace$\left.\delimsize#1%
    % \vrule depth\dp#2 height \ht#2 width0pt\right.$}\hspace{-.97\wd0}}
    \makeatletter
    \newsavebox{\abcmycontentbox}
    \newcommand\DeclareDoubleDelim[5]{
    \DeclarePairedDelimiterXPP{#1}[1]%
        {% box must be saved in this pre code
            \sbox{\abcmycontentbox}{\ensuremath{##1}}%
        }{#2}{#5}{}%
        %%% Correct spacing, but doesn't work with externalize.
        % {\nhphantom{#3}{##1}\hspace{1.2pt}\delimsize#3\mathopen{}##1\mathclose{}\delimsize#4\hspace{1.2pt}\nhphantom{#4}{##1}}
        %%% Fast, but wrong spacing.
        % {\nhphantom{#3}{~}\hspace{1.2pt}\delimsize#3\mathopen{}##1\mathclose{}\delimsize#4\hspace{1.2pt}\nhphantom{#4}{~}}
        %%% with savebox.
        {%
            \nhphantom{#3}{\usebox\abcmycontentbox}%
            \hspace{1.2pt} \delimsize#3%
            \mathopen{}\usebox{\abcmycontentbox}\mathclose{}%
            \delimsize#4\hspace{1.2pt}%
            \nhphantom{#4}{\usebox\abcmycontentbox}%
        }%
    }
    \makeatother

\relax %%%%%%%%%   PDG  MACROS   %%%%%%%%
	\newcommand{\ssub}[1]{_{\!_{#1}\!}}
	% \newcommand{\bp}[1][L]{\mat{p}_{\!_{#1}\!}}
	% \newcommand{\bP}[1][L]{\mat{P}_{\!_{#1}\!}}
    \newcommand{\pdgunit}{\mathrlap{\mathit 1} \mspace{2.3mu}\mathit 1}

	\newcommand{\bp}[1][L]{\mat{p}\ssub{#1}}
	\newcommand{\bP}[1][L]{\mat{P}\ssub{#1}}
	\newcommand{\X}{\mathcal X}
	\newcommand{\V}{\mathcal V}
	\newcommand{\N}{\mathcal N}
	\newcommand{\Ed}{\mathcal E}
	\newcommand{\Ar}{\mathcal A}
    
    
    % \newcommand\Src[1]{X_{{#1}}}
    % \newcommand\Tgt[1]{Y_{{#1}}}
    % \newcommand{\Src}{\mathrm{Src}}
    % \newcommand{\Tgt}{\mathrm{Tgt}}
    % \newcommand\Src[1]{S\mskip-2mu\mathit{r\mskip-3muc}_{{#1}}}
    % \newcommand\Tgt[1]{T\mskip-5mu\mathit{g\mskip-1mut}_{{#1}}}
    % \newcommand\Src[1]{\mathsf{S}\mskip-2mu\vphantom{|}_{{#1}}}
    % \newcommand\Tgt[1]{\mathsf{T}\mskip-3mu\vphantom{|}_{{#1}}}
    \newcommand\Src[1]{S\mskip-2mu\vphantom{|}_{{#1}}}
    \newcommand\Tgt[1]{T\mskip-3mu\vphantom{|}_{{#1}}}
    
    \newcommand{\balpha}{\boldsymbol\alpha}
    \newcommand{\bbeta}{\boldsymbol\beta}

	\DeclareMathAlphabet{\mathdcal}{U}{dutchcal}{m}{n}
	\DeclareMathAlphabet{\mathbdcal}{U}{dutchcal}{b}{n}
	\newcommand{\dg}[1]{\mathbdcal{#1}}
	\newcommand{\PDGof}[1]{{\dg M}_{#1}}
	\newcommand{\UPDGof}[1]{{\dg N}_{#1}}
	\newcommand\VFE{\mathit{V\mkern-4mu F\mkern-4.5mu E}}

	\newcommand\Inc{\mathit{Inc}}
	\newcommand{\IDef}[1]{\mathit{IDef}_{\!#1}}
	\newcommand\OInc{\mathit{O\mskip-2.5muI\mskip-3.5mun\mskip-1.7muc}} % new version of Inc
	\newcommand\CInc{\mathit{C\mskip-3.1muI\mskip-3.5mun\mskip-1.7muc}} % new version of IDef
	% \newcommand{\SInc}{\mathit{S\mskip-1muI\mskip-1mun\mskip-1muc}} % new version of IDef
	% \newcommand{\ed}[3]{%
	% 	\mathchoice%
	% 	{#2\overset{\smash{\mskip-5mu\raisebox{-3pt}{${#1}$}}}{\xrightarrow{\hphantom{\scriptstyle {#1}}}} #3} %display style
	% 	{#2\overset{\smash{\mskip-5mu\raisebox{-3pt}{$\scriptstyle {#1}$}}}{\xrightarrow{\hphantom{\scriptstyle {#1}}}} #3}% text style
	% 	{#2\overset{\smash{\mskip-5mu\raisebox{-3pt}{$\scriptscriptstyle {#1}$}}}{\xrightarrow{\hphantom{\scriptscriptstyle {#1}}}} #3} %script style
	% 	{#2\overset{\smash{\mskip-5mu\raisebox{-3pt}{$\scriptscriptstyle {#1}$}}}{\xrightarrow{\hphantom{\scriptscriptstyle {#1}}}} #3}} %scriptscriptstyle
	\newcommand{\ed}[3]{#2%
	  \overset{\smash{\mskip-5mu\raisebox{-1pt}{$\scriptscriptstyle
	        #1$}}}{\rightarrow} #3}

	\newcommand{\bundle}{\mathbin{+}}
	\DeclareDoubleDelim
		\SD\{\{\}\}
	\DeclareDoubleDelim
		\bbr[[]]
	% \DeclareDoubleDelim
	% 	\aar\langle\langle\rangle\rangle
	\makeatletter
	\newsavebox{\aar@content}
	\newcommand\aar{\@ifstar\aar@one@star\aar@plain}
	\newcommand\aar@one@star{\@ifstar\aar@resize{\aar@plain*}}
	\newcommand\aar@resize[1]{\sbox{\aar@content}{#1}\scaleleftright[3.8ex]
		{\Biggl\langle\!\!\!\!\Biggl\langle}{\usebox{\aar@content}}
		{\Biggr\rangle\!\!\!\!\Biggr\rangle}}
	\DeclareDoubleDelim
		\aar@plain\langle\langle\rangle\rangle
	\makeatother


	% \DeclarePairedDelimiterX{\aar}[1]{\langle}{\rangle}
	% 	{\nhphantom{\langle}{#1}\hspace{1.2pt}\delimsize\langle\mathopen{}#1\mathclose{}\delimsize\rangle\hspace{1.2pt}\nhphantom{\rangle}{#1}}


% \author{$\{$Oliver E Richardson, Joseph Y Halpern, Christopher De Sa$\}$}

\begin{document}
% If your paper is accepted and the title of your paper is very long,
% the style will print as headings an error message. Use the following
% command to supply a shorter title of your paper so that it can be
% used as headings.
%
%\runningtitle{I use this title instead because the last one was very long}

% If your paper is accepted and the number of authors is large, the
% style will print as headings an error message. Use the following
% command to supply a shorter version of the authors names so that
% they can be used as headings (for example, use only the surnames)
%
%\runningauthor{Surname 1, Surname 2, Surname 3, ...., Surname n}

\twocolumn[

%joe1: I would cut the second part of the title
%  \aistatstitle{Inference in Probabilistic Dependency Graphs,\\
  %    via Exponential Cones and Otherwise}
    \aistatstitle{Inference in Probabilistic Dependency Graphs}

%joe1: initials need periods
    %\aistatsauthor{ Oliver E Richardson \And Joseph Y Halpern \And
    \aistatsauthor{ Oliver E. Richardson \And Joseph Y. Halpern \And
  Christopher De Sa } 
% \aistatsaddress{ Institution 1 \And  Institution 2 \And Institution 3 } 
\aistatsaddress{Cornell University \And Cornell University \And Cornell University}
]

\begin{abstract}
    We provide the first tractable inference algorithm for
    Probabilistic Dependency Graphs (PDGs) with discrete variables,
    thereby placing PDGs on asymptotically similar footing as other
%joe1
%    graphical models, such as Bayesian Networks and Factor Graphs.
    %    This may be surprising, because PDGs are more expressive than
        graphical models, such as Bayesian Networks and Factor Graphs,
despite the fact that PDGs are significantly more expressive than
other probabilistic graphical models.
%joe1: I have no idea what you mean by "a PDG inferencd algorithm can
%be used to calibrate a broad class of statistical models."  Since I
%don't think you discuss this issue anywhere in the paper, I just cut it.
%other probabilistic graphical models, and also because a PDG
%    inference algorithm can be used  
%    % for ``inconsistency minimization'', 
%    % which has been argued to be widely useful. 
%    % to resolve inconsistencies, which has  as a generic modeling task. 
%    % as a black box to train statistical models in ML.
%    to calibrate a broad class of statistical models.
%joe1: cut paragraph break
The key to our approach is combining 
%joe
%(1) our finding that inference in PDGs with bounded tree-width can
(1) the observation that inference in PDGs with bounded tree-width can
be reduced to a tractable linear optimization problem with exponential
cone constraints, 
%joe1
%with (2) a recent interior point method that can (provably) solve
with (2) a recent interior-point method that can solve
such problems efficiently (Dahl \& Anderson, 2022). 
%joe1: say something about how you do the evaluation; just showing how
%it does on random PDGs is nbot enough.  I don't think comparing it
%only to belief propagation is enough either.
%    We provide a concrete implementation and empirical evaluation.
    We evaluate our approach by ...
%joe1: I wouldn't worry about hte other approaches now.  There are
%more important issues hou have to deal with first.
%    In addition, we prove auxiliary results about complexity of this
    %    problem, and discuss other approaches to it.
We also characterize the complexity of various components of the
inference problem.
\end{abstract}


% \begin{narrow}
% %%-----------    A FRANK SUMMARY    ---------------
% Measuring / Estimating /  Inconsistency is very useful.
% For instance, (1) propogating it backwards through layers of computation = differentiable learning. 
% 
% Certain localized versions of it can be used to do other algorithms.

% How hard is it? 
% With interior point methods (convex programs with exponential cone constraints) we can do it in $O(n^4 \log n)$ time \& space, worst case for exact inference. So far, this means slightly harder than inference in Graphical models.
% \end{narrow}


% \tableofcontents

\section{INTRODUCTION}

Suppose we have a collection of probabilistic beliefs. 
%oli1: we answered the first two questions already
% Are they self-consistent? 
How can we tell if they are self-consistent?
How difficult is it to measure how inconsistent they are?
% If not, how far off ar they from being consistent?
How much computation is necessary to synthesize our beliefs into a single joint probability distribution?
This paper provides answers---both
theoretical and practical---to these questions. 


% More concretely, we handle
Probabilistic Dependency Graphs, or PDGs \parencite{pdg-aaai},
are a particularly flexible class of probabilistic graphical models, which subsumes Bayesian Networks (BNs) 
% and Markov Random Fields (MRFs).
and Factor Graphs (FGs). 
%joe1: much too wordy
%The primary force behind the expressiveness of pdgs is their ability
%to capture inconsistent beliefs, and the natural way of measuring the
%degree of this inconsistency that the formalism provides.
PDGs can capture inconsistent beliefs, and have an associated measure
of degree of inconsistency that captures how far a PDG is from being
consistent.  
%
%Beyond its role in undergirding the semantics of pdgs,
% Beyond its role in providing the semantics of pdgs, 
Beyond its central role in the development of PDG semantics,
this inconsistency measurement also captures many standard
%joe1: I don't know what "in the appropriate contexts" means.  I would
%cut it.
%loss functions and statistical divergences, in the appropriate contexts
loss functions and statistical divergences
 \parencite{one-true-loss}.
% All of this paints a story of wanting 
% All of this paints a story of wanting 
%joe1: none of what you said suggests that minimizing inconsistency is
%useful.  Specifically, the fact that Inc can be used to capture a
%number of loss functions and divergences says nothing about why we
%should care about minimizing inconsistency, nor does the fact that 
%the semantics is based on Inc.
% It seems that in a  minimizing inconsistency is a generically useful
 % modeling task.
 %joe1:
However, the earlier work on PDGs does not provide any
computational method for calculating whether a PDG is consistent and,
if not, its degree of inconsistency.  We provide such methods in this paper.

\TODO[TODO: now that I've adopted all of the changes, 
    the motivation of ``why minimize inconsistency'' is 
    entirely gone. I need to find a better way to put it back in]

%joe1: cut this paragraph.  Replaced it by the previous sentence
%From a pragmatic point of view, though, PDGs are currently not yet
%very useful.  As it stands, they only have conceptual
%applications---one can use them to justify 
%a choice of loss function analytically, or to derive cute diagrammatic proofs
%of inequalitites you likely already know \parencite{one-true-loss},
%but it is impossible to compute with them.
%% What use is a model without an inference algorithm? 
%Until now, PDGs have been a model without an inference algorithm. 

%joe1: where do we do updating?   What inference problems do we
%consider?  You need to slow down here and explain what we do
%We analyze the complexity of inference and updating in pdgs, and show
In more detail,
we analyze the complexity of inference in PDGs, and show
%joe1
%that it is equivalent to inconsistency minimization.
that it is equivalent to that of inconsistency minimization. 
%joe1*: I have no idea what exponential-cones constraints are.  Unless
%this is a completely standard notion in the AIStats community, you
%*must* give some intuition.  Also, when you talk about reducing the
%problem to a linear program, (a) I don't know which problem you're
%talking about and (b) we usually talk about reducing one problem to
%another, not reducing a problem to a linear program
%Then, we reduce the problem to a linear program with exponenital cones
% We then reduce the problem to a linear program with exponential-cones
% constraints.
Then, we reduce the problem to a convex optimization problem in standard
form.
%joe1
%We then use a powerful recent interior point method
This allows us to use powerful interior-point methods
that can solve such problems in polynomial time \parencite{dahl2022primal}. 
% We then lean heavily on recent work 
% \parencite{dahl2022primal} showing that 
% such problems can be
%joe1: 
%We provide a python implementation of our reduction, 
%and also several generic optimization baselines, and then show
We then evaluate our approach, showing 
that this exponential-cone-based approach is more precise
%joe1: what do you mean by "optimization baselines"?
%oli1: it's hard to answer this directly until we've dug into the inconsistency a little.
and, in some cases, faster than optimization baselines.
%joe1: this belongs in the conclusion, or needs to be rewritten to
%say "while not currently as fast as inference methods such as belief
%propagation on the models to which belief propagation can be applied,
%we are optimistic that further improvements to our methnod are
%possible.  In any case, our results show that inference on pdgs is feasible.
While not currently as fast as inference methods such as belief
propagation on the models to which belief propagation can be applied,
we are optimistic that further improvements are possible.
In any case, our results show that inference in PDGs is feasible.
%However, this approach that our algorithms are not as fast as exact
%inference methods for existing graphical models, such as beleif
%propogation. 
% However, their asymptotics are not much worse.

\section{PRELIMINARIES}

\textbf{Vector notation.}
% This paper concerns the 
% Unless otherwise specified, all scalar quantites range over the extended reals $\Rext := \mathbb R \cup \{\infty\}$. 
For us, a vector is a map from a finite set to the extended reals
    $\Rext := \mathbb R \cup \{\infty\}$. 
The notation $\mat u := [u_i]_{i \in S}$ defines a vector over the finite set $S$.
We will sometimes use superscripts as well, especially when indicies depend on one another. For example, if $\mathcal X$ is a set of finite sets, then
$\mat u := [u^X_x]^{X \in \mathcal X}_{x \in X}$ defines a vector whose indicies range over the disjiont union $\sqcup \mathcal X$.
It is equivalent to $\mat u := [u_{(x,X)}]_{x \in X, X \in \mathcal X}$, but more compact.
Supplying just the upper index, $\mat u^{X}$ is the projection of $\mat u$ onto the subspace where the upper index is $X$. 
Vectors over the same set can be added pointwise as usual, and pointwise multiplication is denoted by $\odot$.  
$\mat 1$ denotes an all-ones vector, whose dimension will always be clear in context.

{\color{red}\tt
TODO: unexplained notation / concepts
\begin{enumerate}[nosep]
\raggedright
\item tensor product $\otimes$  (TODO: nix it)
\item relative entropy $\kldiv\mu\nu$, conditional entropy $\H(Y|X)$
\end{enumerate}
}

\textbf{Probabilities.}
We write $\Delta S$ to denote the set of probability distributions over a finite set $S$.
Every variable $X$ can take on a finite set $\V(X)$ of possible values. 
% If $S$ is a finite set, we write $\Delta S$ for the set of probability distributions over $S$, i.e., the simplex over its elements. 
A conditional probability distribution (cpd) $p(Y|X)$ is a map 
$p : \V(X) \to \Delta \V(Y)$, so it assigns, to every $x \in \V(X)$, a probability distribution $p(Y|x) \in \Delta Y$, which is shorthand for $p(Y|X\!\!=\!x)$.
Given a joint distribution $\mu$ over many variables including both $X$ and $Y$, 
%joe1: Is this standard notation for a marginal?  \mu)(X) looks like
%the probability of X to me.
%oli1: I'm pretty sure it's standard; at the very least, it agrees with the standard notation:  If you had Pr(X,Y), and you wanted to talk about the probability of X, you would write Pr(X), which is also the marginal of the distribution \Pr. 
we write $\mu(X)$ for its marginal distribution on $X$,
% $\mu(X,Y)$ for the 
and $\mu(Y|X)$ for the cpd obtained by first conditioning on $X$ and then marginalizing to $Y$. 

% \textbf{Graph Theory.}

% \textbf{Inference for Graphical Models.}
% % A graphical model is a graph whose vertices correspond to 
% % 
% % There is a natural equivalence between hyper-graphs and bipartite graphs
% % \[
% % \]


\textbf{Graphs.}
A hypergraph $G = (V, \Ed)$ is a set $V$ of vertices, and a collection $\Ed$ of ``hyperedges'', which correspond to subsets of $V$. 
A graph may be regarded as the special case in which every hyper-edge contains two vertices.
% There is a natural bijection between hyper-graphs and bipartite graphs.

\begin{defn}
    A \emph{directed hypergraph} $G = (N, \mathcal A)$ is a set $N$ of nodes, and a collection $\mathcal A$ of ``hyperarcs''; each $a \in \mathcal A$ 
    is associated with a set $\Src a \subset N$ of source nodes, and a set $\Tgt a \subset N$ target nodes. 
\end{defn}
A directed graph is just a directed hypergraph where the source and target sets of every hyper-arc are singletons. 
As one might hope, we can form hypergraph from a directed hypergraph by ``forgetting the direction of the arrow'', and taking the hyperedge to be the union of the source and target sets.
% There is also a natural bijection between directed hypergraphs and directed bipartite graphs. 

% Given a hyper-graph $(\N, \Ed)$,
Many problems that are intractable for general graphs
are tractable when restricted to trees. 
Some graphs are closer to trees than others. 

A tree decomposition of a (hyper-)graph $G = (V, \Ed)$ is a tree $(\mathcal C, \mathcal T)$ whose vertices $C \in \mathcal C$, called ``clusters'', are subsets of $V$ such that:
\begin{enumerate}[itemsep=0pt]
    % \item The union $\bigcup \mathcal C$ of all clusters contains all vertices of $G$;
    % \item Every vertex $v \in V$ lies in at least one cluster,
    % \item Every hyper-edge $E\in \mathcal E$, there is a 
        % cluster $C \in \mathcal C$ that contains $E$, and
    \item Every vertex $v \in V$ and every hyper-edge $E \in \Ed$ is contained in at least one cluster;
    % \item The subtree $\mathcal T$ 
    \item For every vertex $v \in V$, the subgraph induced by restricting to clusters that contain $v$ is connected.
    
    \item[2'.] {\color{blue}Equivalently, \emph{ the running intersection property:}
            Every cluster $D$ along the unique path from $C_1$ to $C_2$ in $\cal T$,
            contains $C_1 \cap C_2$. 
        }
        
        \TODO[Which is prefereable?, 2 or 2'?]
    % \item Every hyper-edge $E\in \mathcal E$ is contained in some
    %     cluster $C \in \mathcal C$. 
\end{enumerate}

The \emph{width} of a tree decomposition is one less than the size of its largest cluster,
and the \emph{treewidth} of a (hyper) graph $G$ is the smallest possible width of any tree-decomposition of $G$.
It is NP-hard to determine the tree-width of a graph, but fortunately, if the tree-width is known to be at most some constant, a tree-decomposition may be constructed in linear time \parencite{badenbroek2021algorithm}.

% A \emph{directed} hyper-graph is a 

\textbf{Probabilistic Dependency Graphs.}
% \textbf{PDGs.}
% \textbf{PDGs.}
%joe1: you should decide whether you're going to write PDG or pdg.
%I'm OK either way, but you have to be consistent.
%We now give a quick overview of the PDG formalism,
%following the more carefully motivated
%expositions of \textcite{pdg-aaai,one-true-loss}.
We now give a quick overview of PDGs; the reader is encouraged to
consult \textcite{pdg-aaai,one-true-loss} for more details and intuition.
% We opt for a slightly different presentation, 
 % which the first work shows to be equivalent.
% We give a slightly different, but equivalent presentation.
%following the more carefully motivated
%expositions of \textcite{pdg-aaai,one-true-loss}.
%oli1
% A probabilistic dependency graph (pdg)
At a high level, a PDG
 % is just a collection of cpds, weighted by two kinds of confidence. More precisely:
is just an arbirary collection of cpds and causal assertions,
    weighted by confidence. More precisely:

\begin{defn}
    a PDG $\dg M = (\N, \Ar, \mathcal P, \balpha, \bbeta )
     % = (\mathcal P, \balpha, \bbeta)$ 
    $
    % over $\N$ is a set $\Ed$ of ``directed hyper-edges'', 
    % each $L \in \Ed$ of which is associated with:
    is a directed hypergraph  $(\N, \Ar)$, whose nodes correspond to variables, and
    each $a \in \Ar$ is associated with:
    \begin{itemize}[itemsep=0pt]
        % \item (subsets of) variables $\Src L, \Tgt L \subset \N$, indicating the respective source and target variables of the edge;
        % \item variables $\Src L, \Tgt L \in \N$, the source and target of $L$;
        % \item subsets $\Src L, \Tgt L \subset \N$, which are source and target variables of the edge $L$. For example,   
        %     $$\Src L = \{A, B\} \ed L{}{} \{C\} = \Tgt L$$
        %  intuitively represents a joint dependence of $C$ on the variables $A$ and $B$;
        \item a cpd $p\ssub a (\Tgt a | \Src a)$ on the target variables given the source variables,
        \item a weight $\beta_a \in \Rext$ indicating 
            the modeler's confidence in the cpd $p\ssub L(\Tgt a | \Src a)$, and 
        \item a weight $\alpha_a \in \mathbb R$ indicating 
            the modeler's confidence that the arrow $a$ corresponds to an independent mechanism that determines $\Tgt a$ given $\Src a$. 
        \qedhere
    %     % \item $\mathcal P = \{ p\ssub L (\mat T_L | \mat S_L) \}_{L \in \Ed}$ is an indexed set of cpds   
    %     \item $\bbeta$ 
    \end{itemize}
\end{defn}

One selling point of PDGs is their modularity: if $\dg M_1$ and $\dg M_2$ are two PDGs, we can take the union of their edge sets to get a new PDG, denoted $\dg M_1 + \dg M_2$. 
% For the purposes of adding data to PDGs in this way, we implicitly convert cpds to singleton PDGs that have default weight $\beta = 1$. 

%joe1: You need to add a few sentences of intuition here, giving an
%examples of low and high incompatibility, and explaning that D acts
%as a measure of distance.  Have pity on the poor reader!  Don't be afraid
%to slow down and explain things.
\TODO[intuition]

The incompatibility of a joint distribution $\mu(\N)$ over all variables, with such a PDG is given by a weighted sum of relative entropies:
\begin{align*}
    \Inc_{\dg M}(\mu) :=
        % \sum_{L \in \Ed} \beta\ssub L\, \kldiv[\Big]{\mu(\Tgt L,\Src L)}{p\ssub L(\Tgt L | \Src L) \mu(\Src L)}.
        \sum_{a \in \Ar} \beta_a\, \kldiv[\Big]{\mu(\Tgt a,\Src a)}{p\ssub a(\Tgt a | \Src a) \mu(\Src a)}.
        % \Ex_{\mu} \sum_{L \in \Ed} \beta\ssub L 
        %     \log \frac{\mu(\Tgt_L \mid \Src_L)}{p\ssub L(\Tgt_L \mid \Src_L)}
\end{align*}
$\Inc$ is called the ``quantitative'' term because it measures $\mu$'s discrepency
with the quantitative data in the cpds. 
%joe1
%Meanwhile, there is also a ``qualitative'' term, called the
There is also a ``qualitative'' term, called the
\emph{information deficiency}, given by
% \begin{align*}
$
    % \IDef{\dg M}(\mu) := - \H(\mu) + \sum_{L \in \Ed} \alpha\ssub L\, \H_\mu(\Tgt L | \Src L),
    \IDef{\dg M}(\mu) := - \H(\mu) + \sum_{a \in \Ar} \alpha_a\, \H_\mu(\Tgt a | \Src a),
$
% Although we won't motivate it here, 
which
    % , roughly speaking, 
    % is a generalization of maximum entropy that accounts for the
    % Seen from another angle, it
    models causal structure, and plays a significant role in allowing pdgs to capture (conditional) independencies.
Note that $\IDef{}$ does not depend on the cpds (``quantitative beliefs'') of $\dg M$, nor even the possible values of the variables---it is defined purely in terms of the topology of the graph and the weights $\balpha$. 
% \end{align*}
%joe1
%The PDG semantics are then given by a scoring fuction: 
The semantics of a PDG $\dg M$ are then given by a scoring fuction
$\bbr{\dg M}: \Delta \V\N \to \Rext$,
the linear combination
\begin{align*}
    \bbr{\dg M}_\gamma(\mu) &:= \Inc_{\dg M}(\mu) + \gamma \IDef{\dg M}(\mu) 
        \numberthis\label{eqn:scoring-fn}
        \\
        % =& \Ex_{\mu}\left[\, \sum_{L \in \Ed} \log \frac
        %     {\mu(\Tgt L| \Src L)^{\beta\ssub L - \gamma \alpha \ssub L}}
        %     {p\ssub L(\Tgt L | \Src L)^{\beta \ssub L}}
        % \right] - \gamma \H(\mu)
        =& \Ex_{\mu}\left[\, \sum_{a \in \Ar} \log \frac
            {\mu(\Tgt a| \Src a)^{\beta_a - \gamma \alpha_a}}
            {p\ssub a(\Tgt a | \Src a)^{\beta_a}}
        \right] - \gamma \H(\mu)
        .
\end{align*}

The notation $\bbr{\dg M}^*_\gamma := \argmin_\mu \bbr{\dg M}_\gamma(\mu)$ denotes the set of optimal distributions at a particular $\gamma$.
Of particular interest is the ``quantitative limit'' as $\gamma \to 0$, 
at which (at least if $\bbeta > 0$) there is a unique optimal joint distribution, $\bbr{\dg M}^*$.
This distribution uniquely achieves the smallest information deficiency among those distributions maximally compatible with $\dg M$. 

One should be careful to distinguish this joint distribution $\bbr{\dg M}^* \in \Delta\V \N$, which arises in the limit as $\gamma \to 0$, from $\bbr{\dg M}^*_0$, the set of distributions that minimize 
%joe1
%$\Inc_{\dg M}$ which contains $\bbr{\dg M}^*$, and possibly many others.
$\Inc_{\dg M}$; the latter set can be shown to contain $\bbr{\dg
  M}^*$ (see \cite{pdg-aaai}), but may also contain other distributions.

The inconsistency of a PDG $\dg M$ is the smallest possible score of any distribution:
\begin{align*}
    \aar{\dg M}_\gamma := \inf_{\mu \in \Delta\!\V\!\N}\, \bbr{\dg M}_\gamma(\mu).
\end{align*}
To parallel the notation for scoring functions, when we omit the subscript, we refer to the limit as $\gamma\to 0$, which, unlike before, obeys $\aar{\dg M} = \aar{\dg M}_0$. 


\textbf{Inference In Graphical Models.}
% When we speak of inference, we mean providing answers to questions of the form:
An inference algorithm for a probabilistic model $\mathcal M$ one that answers questions of the form:
\begin{center}
    \it Given that the variables $\mat X$ take a particular value $\mat x$, 
    what is the distribution of the variables $\mat Y$? 
\end{center}

% The meaning of this question is obvious 
% While it's obvious what this means for other graphical models,
% it's not so clear for PDGs. 
% Although perhaps less-so for PDGs, the meaning of this is usually clear:
Although perhaps murkier for PDGs, the desired output is generally clear:
$\cal M$ represents some joint distribution $\Pr_{\cal M}$ over all variables, and an inference algorithm calculates the conditional marginal $\Pr_{\cal M}(\mat Y| \mat X \!=\! \mat x)$ of that distribution.
% For a Bayesian network in particular, 


% Queries can be 
% There are many approaches 
% Many years of graphical models research have converged on 

% Typically, exact inference is done by 
% If one is interested in distilling the answers to all

% Fo
% The trick to doing inference quickly is not to ever represent the the full join

% In the exact form of belief propogation
% When belief propogation is used 
% Belief propogation when run on trees, 
Message-passing algorithms such belief propogation, when applied trees, run in linear time and are provably correct.
Running these same algorithms on 
graphs that are not trees, such as \emph{loopy} belief propogation,
may not converge, and even if it does, may be incorrect, or even inconsistent \parencite{wainwright2008graphical}. 
Nearly all exact inference algorithms for graphical models, 
implicitly or explicitly, effectively construct a tree-decomposition of the model, and may be viewed as running on a tree.

For belief propogation in particular, is possible to save time by distilling the answers to all possible queries via a data-structure called a \emph{clique tree}
% clique tree calibration
\parencite{koller2009probabilistic}, which is a tree decomposition $(\cal C, T)$
of the underlying model structure, together with a family $\bmu = \{\mu(C)\}_{ C \in \mathcal C}$ of probability distributions over each cluster $C$. 
A clique tree is said to be \emph{calibrated} if neighboring clusters's beliefs agree on the variables they share.

\textbf{Convex Programming.}
% \textbf{Exponential Cones.}
% \textbf{Disciplined Convex Programming, and Exponential Cones.}
%joe1*: you *must* give more intuition here (what do these triples
%represent? why are they of interest?), more background (where has
%this approach been used before?), and more intuition about why
%exponential conexs might be useful.  
% includes linear programming, quadratic programming, semidefinite programming.
Convex programming is an optimization paradigm wherein one searches within a convex set to find optima of a linear function, subject to certain constraints. 
 % objective function to minimize, 
Most computer scientists are familiar with linear programming (LP), where the constraints are also linear, and likely also the variants in which the contstraints can quadratic (QP) or that a matrix be positive semidefinite (SDP).
% The variant we use is less well-known 
Exponential cone constraints are less well-known, in part because provably efficient algorithms for exponential conic programs are relatively recent. 

\TODO[TODO: add more intuition + references]

The exponential cone is the convex set
\begin{align*}
    K_{\mskip-1mu\exp} &:=\!\!\!\!\!\!
        \begin{aligned}
        \big\{ (x_1, x_2, x_3) &: 
                x_1 \ge x_2 e^{x_3 / x_2},\, x_2 > 0 \big\} 
        \\\quad \mathbin{\cup}\, \big\{ (x_1, 0, x_3) &: x_1 \ge 0,\, x_3 \le 0 \big\} 
    \end{aligned}
    \subset \mathbb R^3.
\end{align*}
% $K_{\exp}$ is non-symmetric, and cannot .
Recent work in interior point methods has
provided a way to solve linear optimization problems with such constraints in polynomial time \parencite{dahl2022primal},
provided that $x_1, x_2$ and $x_3$ are affine transformations of the program variables. 

The disciplined convex programming \parencite{dcp-thesis} approach to convex programing allows users to articulate composite constraints. 
So long as the arguments to the constraints satisfy certain rules they are said to be dcp, and a dcp program can be compiled to a convex optimization problem that can be handled efficiently.
The rule for the exponential cone is simple: the constraint $(x,y,z) \in K_{\exp}$ is dcp, iff $a$, $b$, and $c$ are affine transformations of the program variables. 

% \TODO[Should I talk about disciplined convex programming here? 
%     It might go something like this:]
% 
% In the discipline c
% Exponential cone constraints constraints of the form $(a,b,c) \in K_{\exp}$ may be added to convex programs, so long as $a,b,$ and $c$ are affine transformations of the program variables. 



\section{INFERENCE VIA INCONSISTENCY MINIMIZATION}
    \label{sec:inf-via-inc}

We are now equipped to talk more technically about inference in PDGs. 
% When working with traditional graphical models, the meaning is clear. 
%k
Since PDG semantics are already given in terms of a scoring function,
the obvious thing to do is to find a distribution that minimizes it. 
% There are some immediate 
There are several immediate difficulties.
% We immediately run into some obstacles.
% There are some obstacles to this.


\begin{enumerate}[nosep, label=\textbf{D\arabic*.}]
    \item Even writing down a distribution $\mu$, let alone evaluating its score $\bbr{\dg M}_\gamma (\mu)$, or minimizing it, takes exponential time.
    
    \item Generally speaking, optimization is computationally difficult.
        Even our most powerful optimization techniques only provably find optima in certain special cases.
    Unfortunately, while standard optimization techniques seem to work in practice
%joe1: why don't the standard tools apply in our setting?
%oli1: the next sentence explains it: the standard tools require Lipshitz-ness or
% self-concordance. How can I write this more clearly, if I don't really want to go into
% either but still want to mention the names so that people know what doesn't work? 
%  (actually, the way the optimizer works is by using a self-concordant barrier function,
%       which is realted, but we can't use self-concordance directly in the obvious way.)
%
%      (more-or-less; see \cref{sec:expts}), the standard theoretical
      (more or less; see \cref{sec:expts}), the standard theoretical
      tools do not apply in our setting.  
        % and even if it can be 
        Despite being strictly convex \parencite{pdg-aaai},
        and even $C^\infty$ smooth, 
%joe1: I have no idea what self-concordant means.  Unless you're sure
%that over 90% of AIStats folks will now, you must define it and
%explain why it's relevant.  At least I know what the Lipshitz
%condition is, but it couldn't hurt to explain that too.
        in general $\bbr{\dg M}_\gamma$ is neither Lipshitz nor self-concordant.
         % convex (in $\mu$, which is exponentially large),
            
    \item Even if we coulld easily find optimizers of the function $\bbr{\dg M}_\gamma$ for fixed $\gamma > 0$, it's still not obvious that this would allow us to calculate the unique limiting distribution $\bbr{\dg M}^*$.
\end{enumerate}

We will ultimately address each of these issues, but before we do so, 
% let's start with a shift in perspective.
let's start by trying to do inference as
suggested in the final section of \textcite{pdg-aaai}, which meshes well with 
the persepctive taken in \textcite{one-true-loss}. 

The argument there is that one can do modeling as follows:
    represent all of the relevant information as cpds, 
    form a PDG out of them, and 
    play with the knobs you have to to minimize the resulting inconsistency.
% Well, we have a PDG $\dg M$, and we also have a candidate joint distribution $\mu$.
% What if we put them both on the same footing, in a new PDG, and measure its inconsistency?
%joe1
%It is not hard to show that the distribution any distribution $\mu$
%that minimizes this quantity must also satisfy $\mu \in \bbr{\dg
% It is not hard to show that any distribution 
% that minimizes this quantity must be in $\bbr{\dg
  % M}_\gamma^*$. More generally,  
Well, we have a PDG $\dg M$, and we wanted to know the probablility of $Y$ given $X$. 
What happens if we extend $\dg M$ with a guess, say $p(Y|X)$, and then minimize inconsistency? As hinted in \textcite{pdg-aaai}, this would indeed perform our inference task.
 
 
\begin{linked}{prop}{optimalYgivenX}
	% \label{prop:optimalYgivenX}
	% For all $\dg M$, $X,Y\in\N^{\dg M}$, and $\gamma > 0$, we have that
    For all variables $X,Y$, and $\gamma > 0$, 
	$$\displaystyle
		% \argmin_{p : X \to \Delta Y}
		\argmin_{p(Y|X)}\,
        \aar{\dg M + p}_\gamma =
		\Big\{ \mu(Y | X) :  \mu \in \bbr{\dg M}_\gamma^* \Big\}
	,$$
% \end{linked}
% 
% In the limit of small $\gamma$, since there is only one such distribution,
% the expression beomes simpler.
% 
% \begin{linked}{coro}{smallgammaopt}
%joe1: What's a "quantiative limit"? 
%oli1: it's the limit as \gamma -> 0; 
% I defined it when I defined [[M]]^*, and also it's not necessary
% to remember the worlds, because the symbols are sufficient to get the meaning.
and in the quantiative limit, 
	$\displaystyle
		\bbr{\dg M}^*(Y | X)
	$ is the unique minimizer of the function
$
    p(Y|X) \mapsto \aar{\dg M + p}
$
% which  a conditional probability on $Y$ given $X$ into the PDG $\dg M$. 
%joe1: I have no idea what you're trying to say here.  What does it
%mean that it "includes a conditional probability on $Y$ given $X$
%into the PDG"    
%oli1: good point--- I'm not just adding ("includig") the new probability, but
% also measuring its inconsistency.
% which measures the inconsistency of 
\end{linked}

%joe1*: The claim that we can use inconsistency to compute the
%marginal probability of somne (small) set of variables is an
%important part of the story of why inconsistency is important, and
%should have come *much* earlier (i.e., in the introduction)
%
%oli1: If this was what gave the reason we don't discuss it in the introduction
Consequently, if we are only interested in querying the marginal probability of some small subset $Y$ of variables conditioned on other ones $X$ (i.e., the usual form of a query to a graphical model), and we had an efficient way to estimate the inconsistency of a guess $p(Y|X)$ with the rest of the model, we would have successfully cleared obstacle \textbf{D1}.

% Since inference in other graphical models is already NP-hard, 
% and the class of PDGs subsumes capture them, it should be no surprise 
% that inference in PDGs is NP hard as well.
%
%
% One might imagine that \emph{resolving} the inconsistency is the hard part,
%     as opposed to noticing it. 
% Might it easier to simply determine whether or not a PDG is inconsistent?


One might imagine that \emph{resolving} the inconsistency is the hard part,
    as opposed to noticing it. 
Might it easier to simply determine whether or not a PDG is inconsistent?
If this seems reasonable, you might suspect that this reformulation could increase the difficulty of optimization (\textbf{D2})---that we might lose the several nice properties we do have (strict convexity and smoothness)---but this concern turns out not to be substantiated. 

\begin{linked}{prop}{smooth-and-strictly-cvx}
	The map $p \mapsto \aar{\dg M \bundle p}_\gamma$ is smooth and
		strictly convex 
        %for $\gamma$%
	% (concretely: all $\gamma$ less than $\min (\{1\}\cup\{ \beta^{\dg M}_L : L \in \Ed^{\dg M}\})$
    when $\gamma < \min \{1\} \cup \{\beta_L\}_{ L \in \Ed}$.
	% .
\end{linked}

Operationally, though, we still haven't made much progress, since
we still don't have an easy way to compute $\aar{ ~\cdot~ }_\gamma$. 
% In \cref{sec:complexity}, we will see why.
This is because there isn't one. 
% For now, 

\begin{linked}{prop}{consistent-NP-hard}\label{sharp-p-hard}
    \begin{enumerate}[nosep,label={\rm{(\alph*)}}]
    \item Deciding if $\dg M$ is consistent is NP-hard.
    \item Computing $\aar{\dg M}_\gamma$ is \#P-hard, for all $\gamma > 0$.
    \end{enumerate}
\end{linked}

% Since inference in other graphical models is already NP-hard, 
% and the class of PDGs subsumes capture them, it should be no surprise 
% that inference in PDGs is NP hard as well.

Instead, let's focus instead on first directly addressing \textbf{D2}, an approach
which will ultimately be more fruitful.
 
% \begin{linked}{prop}{sementics-via-inconsistency}
% 	$\displaystyle
% 		\bbr{\dg M}_\gamma(\mu)
% 			% =  \aar*{\dg M \bundle \mu!}_\gamma
%             =\aar[\Big]{\dg M \bundle \overset{(\beta: \infty)}\mu}_{\!\!\gamma}.
% 			% \qquad\Big(~= \lim_{t \to \infty} \aar[\Big]{\dg M \bundle \overset{(\beta:t)}\mu}_{\!\!\gamma}
% 			% 	~\Big)
% 	% \qquad\text{and}\qquad
% 	% \bbr{\dg M}(\\)
% 	$
% \end{linked}
% 
% So it seems that generic optimization algorithms will
% give us a foothold on inference.
% However, the PDG objective \eqref{eqn:scoring-fn} 

% \begin{prop}
% \begin
%     	% \label{prop:optimalYgivenX}
%     	% For all $\dg M$, $X,Y\in\N^{\dg M}$, and $\gamma > 0$, we have that
%     	$\displaystyle
%     		\argmin_{p : X \to \Delta Y} \aar{\dg M + p}_\gamma =
%     		\Big\{ \mu(Y | X) :  \mu \in \bbr{\dg M}_\gamma^* \Big\}
%     	$.
% \end{prop}

%joe1
%\section{REDUCTIONS TO CONVEX PROGRAMS WITH EXPONENTIAL CONE CONSTRAINTS}
% \section{REDUCING TO CONVEX PROGRAMS WITH EXPONENTIAL-CONE CONSTRAINTS}
\section{REDUCING TO EXPONENTIAL CONIC PROGRAMS}

    \label{sec:reductions}
%joe1
%We now present the central finding of our paper:  observation that the
%oli1: it's the central finding in that it's the lynchpin, but this is not the main result; nobody should care about it by itself. It's more like the key lemma. 
% We now prove our main result: that the
We now present the the key 
%joe1: what do you mean by "the PDG objective"?
%oli 1: I mean the scoring funciton. Because it's the "optimization objective"
% Chris has been calling it that and I adopted it. But ``scoring function'' works too.
% PDG objective $\bbr{\dg M}_\gamma$ can be written 
the scoring function $\bbr{\dg M}_\gamma$
%joe1
%as a linear optimization problem with exponential cone constraints.
%oli1
% as a linear optimization problem with exponential-ggcone constraints.
can be written as an exponental conic program. 

We will proceed as follows: 
\begin{enumerate}[itemsep=0pt]
    \item
    illustrate how to find the minimizers of $\Inc$, in a simple 
    setting with only one variable and no conditional distributions; 

    \item
    show how the same approach can be generalized to find minimizers 
    of $\Inc$ in general PDGs; 
    \item \label{item:+idef}
    %joe1 I cut this; we 'already defined {\dg M}^*.  In any case, I still
    % show how to find $\bbr{\dg M}^*$, the unique distribution specified by $\dg M$ in the quantitative limit.
    show how to find $\bbr{\dg M}^*$;
    % \item \label{item:cccp}
    % employ the convex-concave procedure 
    % \parencite{yuille2003concave}, to find some minimizer $\mu^* \in \bbr{\dg M}^*_\gamma$ (although it may not be unique), for fixed $\gamma > 0$.
    % 
    \item 
    % Finally, show how \cref{item:+idef,item:cccp} can also be achieved 
    finally, show how this can all be done
    with a more efficient compact representation for PDGs that have
    bounded tree-width.
\end{enumerate}

\subsection{Warm-Up}\label{sec:illust}

To illustrate the idea, consider the special case in which our PDG contains only one variable $X$, which takes values $\V(X) = \{1, \ldots, n\}$. 
Suppose further that for every edge $j \in \Ar = \{1, \ldots, k\}$, the cpd $p_j(X)$ is an unconditional dsitribution over $X$.
That is, $\Tgt j = \{ X \}$, and $\Src j = \emptyset$.
% Such unconditional probabilities may be identified with unit vectors $\mat p\ssub L \in \mathbb R^n$.  Similarly a candidate (``joint'') distribution $\mu(X)$
Such unconditional probabilities may be identified with vectors $\mat p_j \in \Rext^n$, and all $k$ of them may conjoined to form a 
%joe1: what isn't this just a matrix?  What makes is "stochastic"?
%oli1: a stochastic matrix is a matrix in which every column (or row) sums to one,
% like a conditional probability table.  It's a more precise word, but I definitely
% don't need the jargon so I'll remove it. 
matrix $\mat P = [\,p_{ij}] \in [0,1]^{n \times k}$.
Of course, a candidate (``joint'') distribution $\mu(X)$
may be represented as a unit vector $\mat m \in \mathbb R^n$. 
%
% Now, consider another collection of vectors $\{\mat t\ssub {\,L}\,\in \mathbb R^n\}_{L \in \Ed}$ and notice that:
% \begin{align*}
%     \forall  L &\in \Ed.~~ 
%     (-\mat t\ssub L\,, \mat m, \mat p\ssub L) \in K_{\exp}^n \\
%         &\iff 
%             \forall  L \in \Ed.~~
%             \mat t \succeq {\mat m} \log \frac{\mat m}{\mat p}
%         \\&\implies \sum_{L \in \Ed}\sum_{i=1}^n t_i  \ge \kldiv{\mat m}{\mat p}
% \end{align*}
Now consider a matrix $\mat U = [u_{i,j}] \in \Rext^{n \times k}$,
and observe that:
\begin{align*}
    &(- \mat U,~ \mat m \otimes \mat 1,~ \mat P) \in K_{\exp}^{n \times k} \\
    &\iff \forall  i,j \in [n]\!\times\![k].~~ 
        (- u_{ij}, m_{i}, p_{ij}) \in K_{\exp} \\
    &\iff \forall  i,j \in [n]\!\times\![k].~~ 
            u_{ij} \ge m_i \log \frac{m_i}{p_{ij}} \\
    &\implies \forall j \in [n].~~  {\textstyle\sum_i} u_{ij}  \ge \kldiv{\mu}{p_j} \\
    &\implies \sum_{i,j} \beta_j u_{ij}  \ge \beta_j \kldiv{\mu}{p_j} \\
    &\iff \mat 1^{\sf T} \mat U \bbeta \ge \Inc(\mu)
    .
    % &\implies \sum_{L \in \Ed}\sum_{i=1}^n t_i  \ge \kldiv{\mat m}{\mat p}
\end{align*}

So now, if $(\mat U, \mat m)$ are a solution to the convex program
\begin{align*}
    \mathop{\text{\sf minimize}}
    % \min
    \limits_{\mat m, \mat T}~~
        \mat 1^{\sf T} \mat U \bbeta 
    \quad\text{\sf subject to}\quad &
        \mat 1^{\sf T} \mat m  = 1, \\[-2ex]
    % \begin{cases}
        (-\mat U,\;\, &\mat m\otimes \mat 1,\; \mat P) \in K_{\exp}^{n \times m},
    % \end{cases},
\end{align*}
then (a) the inconsistency $\aar{\dg M} = \mat 1^{\sf T} \mat U \bbeta$ equals the optimal objective value, and 
(b) $\mu \in \bbr{\dg M}^*_0$ is maximally compatible with $\mu$. 
% This illustrates the general principle, but 

\subsection{Adding More Variables, Conditionals, Marginals}

Having seen some of the details of the maxtrix computations, let's now move up a level, and identify finitely supported distributions with their simplex representations. 
% For example, we will implicitly identify a joint distribution $\mu$ 
% with the appropriate vector 
We now tackle the general case of a pdg 
$\dg M = (\N, \Ar, \mathcal P, \balpha, \bbeta)$.
%
% For each $a \in \Ar$, let $N_a := |\V(\Src a, \Tgt a)|$ be the dimension of joint settings of the source and target values of $a$, i.e., the dimension of the cpd $p \ssub a$,
% let $K := \sum_{a \in \Ar} N_a$ be the total dimension, and
Let $\mat u = [u^a_{s,t}]^{a \in \Ar,}_{ (s,t) \in \V(\Src a, \Tgt a)}$
be a vector.
% be a free variable in $\Rext^{K}$.
% of all of the condtional probabilities in $\dg M$.
% Let $t = (t^L)_{L \in \Ed} \in \Rext^{K}$.
% {\color{red} Let $\Pi_{X}$ be the projection map that marginalizes to the variables $X$. }
%
Consider the problem
% \begin{align*}
%     % \min_{\mu, t} &\qquad
%     \min_{\mat m, \mat u} &\quad
%         % \Vert t \Vert_1  
%         \sum_{L}\beta_L\, | \mat u\ssub {\mskip 1muL}\mskip 1mu |
%     \\
%     \text{subject to:}&\;\;
%         % (-t^L, \mu(\Src L, \Tgt L), \mu(\Src L) p\ssub L(\Tgt L | \Src L)) 
%         % (-t^L, \Pi_{(\Src L \Tgt L)}\mu, \Pi_{(\Src L)} (\mu) p\ssub L(\Tgt L | \Src L)) 
%         \big(\!\shortminus\! \mat u\ssub L,~ \Pi_{\Src L\Tgt L}\!(\mat m),~
%             \mat P\!\ssub L (\Pi_{\Src L}\!(\mat m) \otimes \mat 1) \big) 
%         % (-t^L_{xy}, \mu(x,y), )
%             \in K_{\exp},\\
%         &\qquad 
%             % \mu \ge 0, ~~ | \mu | = 1.
%             \mat m \ge 0, ~~ | \mat m | = 1.
%             \numberthis\label{prob:joint-inc}
% \end{align*}
\begin{align*}
    % \min_{\mu, t} &\qquad
    % \mathop{\text{\sf minimize}}\limits_{\mat m, \mat u} &\quad
    \mathop{\text{\sf minimize}}\limits_{\mu, \mat u} &\quad
        % \Vert t \Vert_1  
        % \sum_{L}\beta_L\, | \mat u\ssub {\mskip 1muL}\mskip 1mu |
        \sum_{a \in \Ed}\beta_a \, \sum_{\mathrlap{s,t \in \V(\Src a, \Tgt a)}} u^a_{s,t}
    \numberthis\label{prob:joint-inc}\\
    \text{\sf subject to:}&\quad \mu \in \Delta\V\N, \\[0.2ex]
        \forall L \in \Ed.~&\big(-\mat u^a, \mu( \Tgt a,\Src a),p\ssub a(\Tgt a | \Src a)  \mu(\Src a) \big) \in K_{\exp}.
    % \\
    % \color{red}\text{previously}&\color{red}
    %     % (-t^L, \mu(\Src L, \Tgt L), \mu(\Src L) p\ssub L(\Tgt L | \Src L)) 
    %     % (-t^L, \Pi_{(\Src L \Tgt L)}\mu, \Pi_{(\Src L)} (\mu) p\ssub L(\Tgt L | \Src L)) 
    %     \big(\!\shortminus\! \mat u\ssub L,~ \Pi_{\Src L\Tgt L}\!(\mat m),~
    %         \mat P\!\ssub L (\Pi_{\Src L}\!(\mat m) \otimes \mat 1) \big) 
    %     % (-t^L_{xy}, \mu(x,y), )
    %         \in K_{\exp},\\
    %     &\qquad \color{red}
    %         % \mu \ge 0, ~~ | \mu | = 1.
    %         \mat m \ge 0, ~~ | \mat m | = 1.
\end{align*}

% This convex program has $K+1$ constraints 
% \TODO[ I've rewritten this a couple times, and it's always pretty ugly.
%     One higher level question: should we convert this to a different form? 
%     The fact that we can even write constraints this way hinges on the fact
%     that the arguments to the exponential cone are
%     affine transformations of the program variables, which 
%     this presentation sweeps under the rug entirely.
%     \hfill ]
%oli1: this ``more explicit'' presentation is not helpful.
% {\color{gray}
% To be more explicit, if $|\V(\Src L)| = s$ and $|\V(\Tgt L)| = t$, 
% the quantity $p\ssub L(\Tgt L | \Src L)\mu(\Src L)$ represents the 
% flattened matrix
% \[
%     (\mat 1 \otimes \mu(\Src L)) \odot p \ssub L(\Tgt L | \Src L)
%     =
%     \mu(\Src L)|_{i}\; p \ssub L(\Tgt L | \Src L)|_{i,j} \in \mathbb R^{t \times s}.
% \]
% }
Note that the marginals $\mu(\Src a, \Tgt a)$ and $\mu(\Src a)$ are
linear functions of $\mu$'s simplex representation, as required by the 
disciplined convex programming conditions for exponential cones. 
Logic similar to that in \cref{sec:illust} gives yields: 
\begin{prop}
    % If $(\mu, t)$ are a solution to \eqref{prob:joint-inc}, then
    % $\mu \in \bbr{\dg M}_0^*$,
    % % i.e., is maximally compatible with $\dg M$, and
    % and
    % $\sum_{L}\beta_L |t^L| = \aar{\dg M}$.
    If $(\mu, \mat u)$ are a solution to \eqref{prob:joint-inc}, then
    $\mu \in \bbr{\dg M}_0^*$,
    and
    $%\displaystyle
        \sum_{a}\beta_a \sum_{(s,t) \in \V(\Src a, \Tgt a)} u^a_{s,t} = \aar{\dg M}$.
\end{prop}

This is a start, but what we were really after was the unique distribution
$\bbr{\dg M}^*$ that also minimizes $\IDef{\dg M}$.

\subsection{Incorporating IDef}
    \label{sec:also-idef}
% To do this second pass, we will need this second property
% As $\gamma \to 0$, the limit of $\bbr{\dg M}_\gamma$

So far, we have only found \emph{some} distribution that minimizes $\Inc$; 
we really wanted to find the unique distribution $\bbr{\dg M}^*$.
% Fortunately, this can be done by simply running a second optimization problem, 
It turns out that a solution to \eqref{prob:joint-inc}  to construct a second optimization problem of a similar size.
% To justify our approach, we will to prove two more results.
To justify our approach, we need a little more math. 
% First, a characterization of 
First, a characterization of the set $\bbr{\dg M}^*_0$ of distributions that are maximally compatible with $\dg M$. 

\begin{prop}\label{prop:marginonly}
	For any PDG $\dg M$, with arcs $\cal A$,
	the highest-compatibility distributions (the minimizers $\bbr{\dg M}_0^*$ of $\Inc_{\dg M}$) all have the same conditional probabilities along the edges of $\dg M$.   
	That is to say, if there is an edge $\ed aXY \in \Ar$, and $\mu_1, \mu_2 \in \bbr{\dg M}_0^*$ are quantitatively optimal distributions, then $\mu_1(Y|X) = \mu_2(Y|X)$.  
\end{prop}

As a result, having already found one minimizer of $\Inc_{\dg M}$ via \eqref{prob:joint-inc}, it suffices to constrain distributions that have the same conditional marginals along the edges, and now optimize $\IDef{}$.%
    \footnote{
        Technically, to deal with the possibility that variables we are conditioning on might have probability zero, 
        we require that $\mu(X,Y)\nu(X) = \mu(X) \nu(X,Y)$, rather than 
        $\mu(Y|X) = \nu(Y|X)$.
    }

We now run into a second issue: IDef is not convex in $\mu$. 
Fortunately, when we constrain to distributions that optimize $\Inc$, 
% it equals another function that is. 
it is.
Moreover, this function can also be represented with exponential cones.

\begin{prop}\label{prop:idef-frozen}
If $\mu \in \bbr{\dg M}_0^*$, 
then
\vspace{-1ex}
\[
    \IDef{\dg M}(\mu) = 
        % \kldiv[\bigg]{\mu}{ \prod_{L \in \Ed} \nu(\Tgt L | \Src L) }
            % + K(\dg M)
        % \Ex_\mu 
        % \left[
        \sum_{\mathclap{ w \in \V(\N)} }
            % \log \frac{\mu}{\prod_{L \in \Ed} \nu(\Tgt L | \Src L)}
            \mu(\mskip-1mu w \mskip-1mu)
            \log  \bigg(
                \faktor{\mu(\mskip-1mu w\mskip-1mu )\,}{\,\prod_{a \in \Ar} \nu(\Tgt aw | \Src aw)^{\alpha_a}}
            \bigg)
        % \right]
        ,
\]
%
where $\{ \nu(\Tgt a | \Src a ) \}_{a \in \Ar}$ are the
conditional marginals along the arcs $\Ar$ 
shared by all distributions in $\bbr{\dg M}^*_0$\
(per \cref{prop:marginonly}),
and $\Src a w, \Tgt a w$ are the respective values of the variables $\Src a$ and $\Tgt a$ in the world $w$, which is a joint setting of all variables.
% and $K(\dg M)$ does not depend on $\mu$. 
\end{prop}

Having already computed a solution to \eqref{prob:joint-inc},
the denominator of the expression in \cref{prop:idef-frozen},
(apart from its dependence on joint values $w \in \V\N$ of all variables)
is a constant in our optimiziation problem; let's call it
\begin{equation}
    % \psi(w) := \prod_{L \in \Ed} \nu(\Tgt Lw | \Src Lw)^{\alpha\ssub L}.
    \boldsymbol\psi := \bigg[~\prod_{a \in \Ar} \nu(\Tgt a w | \Src a w)^{\alpha\ssub L} \bigg]_{w \in \V\N}\quad.
    \label{eq:cm-product}
\end{equation}
% These constants can $\boldsymbol\psi$ be the vector representation of
\begin{prop}
% A solution to % \eqref{eqn:joint+idef}
If $\nu \in \bbr{\dg M}_0^*$, then $\bbr{\dg M}^*$ is the unique value of
$\mu$ that can 
% e an optimum of
solve
the convex problem
\begin{align*}
    \mathop{\text{\sf minimize}}\limits_{\mu, \mat u} & \quad
        \sum_{w \in \V \N} u_w
        % |\,\mat u\,| 
        \numberthis\label{eqn:joint+idef}\\
    \text{\sf subject to} &\quad 
        (-\mat u,  \mu, \boldsymbol\psi) \in K_{\exp},~~ \mu \in \Delta\V\N, \\
            &\forall a \in \Ar.~~\mu(\Src a \Tgt a) \nu(\Src a) = \mu(\Src a) \nu(\Src a, \Tgt a).
\end{align*}
\end{prop}


% \begin{algorithm}
% \begin{algorithmic}
%     \State $X$
% \end{algorithmic}
% \end{algorithm}


% \footnote{Indeed, $K_{\exp}$ is sometimes called the ``relative entropy cone'' for this reason.} 


% \subsection{Tree Deomposition
% \subsection{A Polynomial Algorithm for the Case of Bounded Tree-Width}
\section{TREE DECOMPOSITION }
\subsection{A Less Expensive Representation for PDGs with Small Tree-Width}
    \label{sec:clique-tree-expcone}

The first property that makes this possible is 

\begin{linked}[Markov Property for PDGs]{prop}{markov-property}
	% Suppose $\dg M_1$ and $\dg M_2$ are compatible PDGs, and let $\mathbf X$ denote the variables they have in common.
	% Then for all $\gamma > 0$, we have that
	% \[
	%  	\bbr{\dg M_1 \bundle \dg M_2}^*_\gamma
	% 		% \subset
	% 		~\models~
	% 	% \mathrm I( \N_1 ; \N_2 \mid \mathbf X)
	% 	\N_1 \mathbin{\bot\!\!\!\bot} \N_2 \mid \mat X
	% \]
	% That is: in every optimizing distribution, for any value of $\gamma$, the variables of $\dg M_1$ and the variables of $\dg M_2$ are conditionally independent given their shared variables $\mat X$.
	% Suppose $\dg M_1$ and $\dg M_2$ are value-compatible PDGs,
	% with respective sets of nodes $\mat X_1 := \N^{\dg M_1}$ and
	% $\mat X_2 := \N^{\dg M_2}$.
	Suppose $\dg M_1$ and $\dg M_2$ are PDGs
    over the sets of variables $\N_1$ and $\N_2$, respectively.
	 % and let $\mathbf X$ denote the variables they have in common.
	Then for all $\gamma > 0$, we have that
	\[
	 	\bbr{\dg M_1 \bundle \dg M_2}^*_\gamma
			% \subset
			~\models~
		% \mathrm I( \N_1 ; \N_2 \mid \mathbf X)
		\N_1 \mathbin{\bot\!\!\!\bot} \N_2 \mid \N_1 \cap \N_2
		% \N^{\dg M_1} \mathbin{\bot\!\!\!\bot} \N^{\dg M_2} \mid \mat X
		% \mat X_1 \mathbin{\bot\!\!\!\bot} \mat X_2 \mid \mat X_1 \cap \mat X_2.
	\]
	That is to say: in every optimal distribution $\mu^* \in \bbr{\dg M_1 \bundle \dg M_2}^*_\gamma$,
     % for some $\gamma>0$, 
    the variables of $\dg M_1$ and of $\dg M_2$ are conditionally independent given the variables they have in common.
\end{linked}

One major consequence of \cref{prop:markov-property} is that, in our search for optimizers of \eqref{eqn:scoring-fn} we only have to consider distributions $\mu$ that come from cluster trees.

Suppose we are given a tree decomposition $(\mathcal C, \mathcal T)$
of $\dg M$'s underlying hypergraph $\Ar$. 
Since $(\cal C, T)$ is a tree-decomposition of $\Ar$, we are guaranteed
that the source and target of every arc $a$ lie entirely within at least one cluster.
Fix a mapping from arcs to clusters, and let $C_a \in \cal C$ be the cluster that corresponds to the edge $a$.


We will now optimize over possible
cluster marginals $\bmu = \{\mu_C \in \Delta\V(C) \}_{C \in \mathcal C}$,
which we identify with its simplex representation
as a vector of demension $\sum_{C \in \mathcal C}|\V(C)|$.
As before, consider
% let $K := \sum_{L \in \Ed} |\V(\Src L, \Tgt L)|$ and
$\mat u := [u^a_{s,t}]^{a \in \Ar,}_{ (s,t) \in \V(\Src a, \Tgt a)}$
 % in $\Rext^{K}$.
%
% Let $\mat U = [u^C_{\mat x}]^{C \in \mathcal C,}_{\mat x \in \V(C)}$ be a free vector in 
% $\Rext^K$.
%
% Consider the problem
in the convex problem

\begin{align*}
    \mathop{\text{\sf minimize}}\limits_{\bmu, \mat u} &\quad
        \sum_{a \in \Ed}\beta_a \, \sum_{\mathrlap{s,t \in \V(\Src a, \Tgt a)}} u^a_{s,t}
    \numberthis\label{prob:cluster-inc}\\
    \text{\sf subject to:}&\quad
        \forall C \in \mathcal C.~\mu_C \in \Delta\V(C), \\[-0.2ex]
        % exponential constraints
        \forall a \in \Ar.~ \big(&\!- \! \mat u^L\!,\, \mu_{C_a}\!(\Src a,\mskip-2mu \Tgt L), \mu_{C_a}\!(\Src a) p\ssub L(\Tgt a | \Src a)\big) \in K_{\exp} \\
        % marginal constraints
        \forall (C,D) &\in \mathcal T.~~ \mu_{C}(C \cap D) = \mu_{D}(C \cap D).
\end{align*}

Because it is a relative entropy optimization over calibrated clique trees, 
\eqref{prob:cluster-inc} is essentially an analogue of
CTree-Optimize-KL from \textcite[pg. 384]{koller2009probabilistic},
for PDGs in the quantitative limit. 

\begin{prop}
    If $(\bmu, \mat u)$ is a solution to \eqref{prob:cluster-inc}, then 
    $\bmu$ is a calibrated clique tree, whose coresponding joint distribution 
    is in $\bbr{\dg M}^*_0$.
\end{prop}
% Since $(\cal C, T)$ is a tree-decomposition of $\Ed$, we are guaranteed
% that the source and target of every edge $L$ lie entirely within at least one cluster.
% Fix a mapping of edges to clusters, and let $C_L \in \cal C$ be the cluster that corresponds to the edge $L$.


\subsection{Incorporating IDef, Again}

Suppose $\boldsymbol\nu = \{\nu_C : C \in \mathcal C\}$ is a calibrated clique tree representing


For $C \in \mathcal C$, let $\Ar_C:= \{ a \in \Ar : C_a = C\}$ be the set of 
edges assigned to cluster $C$, and let
% $$
% \boldsymbol\psi_C  := \prod_{\substack{L \in \Ed\\C_L = C}} \nu_C (\Tgt Lw | \Src Lw)^{\alpha\ssub L}
% $$
$$
\boldsymbol\psi_C  := \bigg[ \prod_{a \in \Ar_C} \nu_C (\Tgt a w | \Src a w)^{\alpha\ssub L} \bigg]_{w \in \V(C)}
$$
be the analogue of \eqref{eq:cm-product} local to the cluster $C$.

Let $\mat u = [ u^C_c ]^{C \in \mathcal C}_{c \in \V(C)}$.

\TODO[ TODO: There's a lot that needs to be filled in here,
    but I think I know what to do. ]

Choose a root node $C_0$ of the tree decomposition $\mathcal T$, and orient each edge of $\mathcal T$ so that it points away from $C_0$. 
Now, we can compute the Bethe Entropy,
\begin{align*}
    \H(\bmu) = \sum_{(C, D) \in \mathcal T} \H()
\end{align*}



\begin{align*}
\mathop{\text{\sf minimize}}\limits_{\bmu, \mat u} & \quad
    \sum_{w \in \V\N} u_w
    % |\,\mat u\,| 
    \numberthis\label{eqn:joint+idef}\\
\text{\sf subject to} &\quad 
    \forall C \in \mathcal C.~~ (-\mat u^C,  \mu_C(C), \boldsymbol\psi_C ) \in K_{\exp}, \\
        &\forall a \in \Ed.~~\mu(\Src a \Tgt a) \nu(\Src a) = \mu(\Src a) \nu(\Src a, \Tgt a)\\
        &
\end{align*}


\subsection{A Polynomial Time Algorithm For PDGs of Bounded Tree-Width}
\TODO[ TODO: I like the text Chris wrote and think it can be worked in here ]
{\color{blue}
... shows that we can write the PDG inference task (1) as a convex optimization problem with a polynomial number of variables and constraints and with an objective that is somehow polynomial-sized. All that remains to achieve polynomial-time inference is to show that a problem of this form can be solved in polynomial time. For this, we turn to interior point methods, which are known [cite] to converge in polynomial time. Specifically, [previous equation] can be transformed via established methods \parencite{agrawal2018rewriting} into a standard form called an \emph{exponential conic program} [cite] which itself can be solved in polynomial time by commercial solvers [cite mosek]. 
}

\TODO[TODO: refactor the main theorem. I don't know how to get rid of the ``residual norm'' part.]

\begin{linked}{theorem}{main}
    % For   O( 3 n * ((3n+m+1)^3 ) * log(√n / ϵ) )  =   O( n^4 log(n)  log(1 / ϵ) )
There is an algorithm that takes $O(n^4 \log n  \log \nf1\epsilon )$ time,
and finds a point $\epsilon$-close in residual norm to the optimal distribution
$\bbr{\dg M}^*$, where $n$ is the total number of parameters in a clique tree.
\end{linked}

\begin{coro}
    PDG inference can be solved to machine precision in $O( N^4 V^4 \log(N V) \exp(V T) )$ time, where $T$ is the tree-width of the graph, and $N$ is the total number of variables, $V$ is the number of values per variable, and $T$ is the tree-width of the PDG's underlying hypergraph.
\end{coro}

 
% \TODO[ Finding the optimal clique tree is NP-hard. 
%     So how can we guarantee we won't get a bad clique tree (i.e., much wider than the tree-width), so as to guarantee polynomial time?
%     Do we have to know the bound?
%     I imagine there's a standard answer, because exactly the same issue 
%     arises for other graphical models. 
%     \hskip-1.1em ]

The proof relies almost entirely on the analysis of \textcite{badenbroek2021algorithm},
which certifies that the optimization algorithm 
proposed in \textcite{dahl2022primal} works in polynomial time. 

\begin{table}
    % Let $m$ denote the 
    \centering
    \begin{tabular}{ccc}
        \toprule
        & BP &  ExpCone \\\cmidrule(lr){2-3}
        Time & $O(m t)$ & $O( m^4 \log m )$ \\
        Memory  & $O(m + )$ {\color{red}??} & $O( m^2 )${\color{red}????}\\    \bottomrule
    \end{tabular}
    
    \TODO[fill this in properly]
    
    \caption{ }
\end{table}


\section{INFERENCE WHEN
    \texorpdfstring{$\boldsymbol\gamma \boldsymbol> \mat 0$}{gamma > 0},
    VIA THE CONVEX-CONCAVE PROCEDURE }

We have now given an algorithm that provably finds the distribution $\bbr{\dg M}^*$ in polynomial time. 
What about optimal distributions for fixed $\gamma > 0$.



To do this, we re-use our work in \cref{sec:reductions} employ the convex-concave procedure 
\parencite{yuille2003concave}. to find some minimizer $\mu^* \in \bbr{\dg M}^*_\gamma$ (although it may not be unique), for fixed $\gamma > 0$.

The PDG objective can be written as \parencite[Proposition 4.6]{pdg-aaai}
\begin{align*}
    \bbr{\dg M}_\gamma(\mu) = 
        -\gamma\H(\mu) + 
            \sum_{L \in \Ed}
                % \left[
                \beta\ssub L\, \Ex_\mu 
                    \log \frac1{p\ssub L(\Tgt L | \Src L)}
                % \right]
                \\
            + \sum_{L \in \Ed}
            (\gamma \alpha \ssub L - \beta\ssub L)
                \Ex_\mu \log \mu(\Tgt L | \Src L)
\end{align*}
The first line is the sum of a linear term and a convex one,
and each individual term on the second line is either convex or concave, depending on the sign of the quantity $\gamma \alpha\ssub L - \beta\ssub L$. 
Once we sort the terms in to convex terms $f(\mu)$ and concave terms $g(\mu)$, we can choose an initial guess $\mu_0$, and iteratively use the convex solver to compute
%
\begin{align*}
    \mu_{t+1} &:= \argmin_{\mu} f(\mu) + (\mu - \mu_{t})^{\sf T}
        \nabla g(\mu_t)
\end{align*}

As we will see in \cref{sec:expts}, the aproach presented here section is 
not very fast, but it is guaranteed to make progress, since
\begin{align*}
    f(\mu_{t+1}) + g(\mu_{t+1}) &\le  f(\mu_{t+1}) + (\mu_{t+1}-\mu_t)^{\sf T} \nabla g(\mu_t) + g(\mu_t)
        % &\text{(concavity of $g$)}
        \\
    &\le  f(\mu_t) + (\mu_t - \mu_{t})^{\sf T}\nabla g(\mu_t)  + g(\mu_t)
        % &\text{(defn of argmin)}
        \\
    &= f(\mu_t) + g(\mu_t)
\end{align*}
and eventually find an optimum. 

The slightly trickier part is combining this with the clique tree representation of \cref{sec:clique-tree-expcone}.

\TODO[ TODO: Show the fancier spanning aborescence trick ]
        
\section{OTHER APPROACHES TO PDG INFERENCE} \label{sec:other-inference}

\subsection{}
A stronger result than \cref{prop:markov-property} holds as well.
\begin{prop}\label{prop:same-set-dists}
    % For all $\gamma > 0$ and in the limit as $\gamma \to 0$, 
    Let $\Ed$ be a set of (hyper-)edges over $\N$. 
    For every PDG $\dg M$ over $\N$ with edges $\Ed$, every $\gamma > 0$, and every optimum $\mu^* \in \bbr{\dg M}_\gamma^*$ of $\dg M$'s scoring function at $\gamma$, 
    there is a factor graph $\Phi$ with factors along $\Ed$ such that $\Pr_\Phi = \mu^*$. 
\end{prop}

In other words: every distribution that a PDG can pick out as optimal (for any choice of $\gamma > 0$ and also in the limit as $\gamma \to 0$), can also be described as a factor graph with the same structure as that PDG.
How do we square this with the \citeauthor{pdg-aaai}'s claim that PDGs are more general than factor graphs?

% This may be surprising, given how \citeauthor{pdg-aaai} position their model as strictly more expressive than other graphical models, because it implies that the optimal distribution 
\TODO[TODO: answer this question.\\
    The short answer: PDGs still compose differently, and in a way that respects the meaning of the probabilities. And just because you can find a factor graph that would have given you the right distribution after the fact, doesn't mean you could have specified the component factors.]
% The answer is simply that 


\TODO[Also: don't get lost; figure out how to continue as below:]
\cref{prop:same-set-dists} suggests another approach to avoiding an exponential representation of $\mu$: given a PDG, fit a factor graph that has the same structure to it. 

\subsection{Approximate Inference}
\subsubsection{Relaxing the Marginal Polytope}
Just as it is possible to do belief propogation on cluster graphs that are not trees (i.e., belief propogation)
so too is it possible to drop the requirement that the cluster that we use is indeed a tree-decomposition.
This program is smaller, and will converge, but it will only be an approximate solution. 
Like the original PDG itself, it might be inconsistent. 

\subsubsection{Variational Approaches}

% Because of the deep connection between variational approaches 
% shown in \parencite{one-true-loss}, there's 



\section{IMPLEMENTATION} \label{sec:implementation}
\section{EMPIRICAL EVALUATION} \label{sec:expts}

\begin{figure}
    \includegraphics[width=\linewidth]{figs/resources-fine}
    \caption{
        The amount of resources: computation time (top) and maximum memory usage (bottom) for the various optimization methods (by color), as the size of the PDG increases, as measured by \texttt{n\_worlds} (right) and \texttt{n\_params} (left).
     }\label{fig:resources}
\end{figure}

\begin{figure}
    \includegraphics[width=\linewidth]{figs/gamma-vs-gap-bettergap}
    \caption{
        A graph of the gap (the difference between the attained objective value, and the best objective value obtained across all methods for that value of $\gamma$), 
        as $\gamma$ varies. As before, colors indicate method. 
        The size of the circle illustrates the relative number of worlds.
    }\label{fig:gamma-v-gap}
\end{figure}


\begin{figure}
    \includegraphics[width=\linewidth]{figs/2}
    \caption{
        A fine-grained variant of \cref{fig:gamma-v-gap}, which splits each method into sub-groups.
        The ExpCone methods \texttt{cvx\_opt\_joint} are split into two variants, depending on whether or not it also computed the second step described in \cref{sec:also-idef} to account for $\IDef{}$.
        The CCCP variants are \texttt{cccp\_opt\_joint} split into regimes where the entire problem is convex, and the entire problem is concave. The optimization approaches \texttt{opt\_dist} are split into three different optimizers: LBFGS, Adam, and accelerated Gradient Descent.
    }\label{fig:gamma-v-gap-fine}
\end{figure}

\begin{figure}
    \includegraphics[width=\linewidth]{figs/1}
    \caption{
        A fine-grained variant of the right half of \cref{fig:resources}, 
        with gap information on the left. 
    }\label{fig:gap-resource-fine}
\end{figure}


\begin{figure}
    \includegraphics[width=\linewidth]{figs/inc-idef2}
    \caption{An illustration of the trade-off between $\Inc$ and $\IDef{}$. Darker collors correspond to larger $\gamma$.}\label{fig:inc-idef}
\end{figure}

\subsection{Comparison To Belief Propogation}

Since PDGs generalize other graphical models, one might wonder how our method stacks up against them. 
We benchmarked against the small networks, and some of the medium-sized ones, from the \href{https://www.bnlearn.com/bnrepository/}{\texttt{bnlearn}} repository. 



\subsection{Evaluations On Random PDGSs}
We start by focusing on empirical properties of the optimization over joint distributions.

We generated several hundred PDGs with various properties: 9 or 10 variables, each of which can take 2-3 values. Each PDG contains 7-15 hyper-edges, with 1-2 target nodes and 0-3 source nodes. The cpds are chosen by taking uniformly random numbers from [0,1] and normalizing appropriately, and every $\beta$ is set to 1.
For each PDG $\dg M$, we measure its complexity by:
\begin{itemize}[nosep]
    \item \texttt{n\_edges}, the number of edges in $\dg M$,
    \item \texttt{n\_params}, the total number of parameters across all the cpds of $\dg M$, and
    \item \texttt{n\_worlds}, the size of the joint distributions on the variables of $\dg M$.
\end{itemize}

\textbf{Capacity.} 
The black-box py-torch based approaches clearly have an edge in that they can handle larger models; see the cut-offs on the right sides of \cref{fig:resources,fig:gap-resource-fine}

\textbf{Resource Costs.} 
Look at \cref{fig:resources}. 
Note that the exponential cone methods without the CCCP (blue and green) are actually faster than LBFGS, which was the best-performing torch optimizer. 
However, they use \emph{significantly} more memory, and cannot handle more than 8000 worlds. 


\textbf{Accuracy.}
In addition to being faster, the exponential cone techniques are also more preicse.
Note that the CCCP is typically more precise than the black-box optimizers when the problem is fully convex $\gamma \le 1$, and mirrors the performance of the exp-cone algorithms for the quantitative limit on the left, in blue.  For combinations of larger $\gamma$ and more worlds however, the 20 iteration maximum we imposed is not nearly enough to get convergence, and the black-box optimizers are both faster and attain better objective values.

\section{DISCUSSION}

% Our anaysis 
Our analysis shows that inference in PDGs with bounded tree-width can be done .


\TODO[ the below is a transplant without context; fix it ]
% Some queries are more difficult than others. 
Although we the question the same way, we also want to point out that there are other reasonable ways to answer that question once we move to PDGs.
Suppose we were looking at a BN in which it just so happens that $\mat X$ contains only a single variable and $\mat Y$ are the parents of $\mat X$.
In this case, our representation already contains the probabilities we are looking for, and we would be happy returning that row of the conditional probability table. 
But in a PDG, that cpd $p(\mat Y \,|\,\mat X)$, even if it is the only one
attached to an edge from $X$ to $Y$, may not be the same as $\bbr{\dg M}^*(Y|X)$.
As a consequence, 




\subsubsection*{Acknowledgements}
% All acknowledgments go at the end of the paper, including thanks to reviewers who gave useful comments, to colleagues who contributed to the ideas, and to funding agencies and corporate sponsors that provided financial support. 
% To preserve the anonymity, please include acknowledgments \emph{only} in the camera-ready papers.

\subsubsection*{References}
\printbibliography


\clearpage
\onecolumn
\appendix
\section{Proofs}

\recall{theorem:main}
\begin{lproof}\label{proof:main}
    We apply the analysis of \textcite{badenbroek2021algorithm}.
    The primal conic problem
    \[
        \inf_{x} \{\langle c, x\rangle : Ax = b, x \in K \tag{D}
    \]
    
\end{lproof}

\subsection{}

\recall{prop:smooth-and-strictly-cvx}
\begin{lproof}\label{proof:smooth-and-strictly-cvx}
	% First, we deal with the convexity, for which we make use of \cref{lem:cvx2}.
	% \commentout{
	% 	\def\mw#1{{\mat w}_{\!_{#1}}}
	% 	\def\ofmw(#1|#2){(\mw{#1} | \mw{#2})}
	% 	\begin{align*}
	% 		\aar{\dg M \bundle p}_\gamma &= \inf_\mu \Big[ \Inc_{\dg M \bundle p}(\mu)
	% 			+ \IDef{\dg M \bundle p}(\mu) \Big] \\
	% 		&=  \inf_{\mu} \Ex_{\mat w \sim \mu}
	% 			\left[\log \mu(\mat w) +
	% 			 	\beta_p \log \frac{\mu\ofmw(Y|X)}{p\ofmw(Y|X)} \; +  \!\sum_{\ed LAB} \beta_L \log \frac{\mu\ofmw(B|A)}{\bp\ofmw(B|A)} + \alpha_L \log \frac{0}{\mu\ofmw(B|A)}\right] \\
	% 		&= f
	% 	\end{align*}
	% }
	We start by expanding the definitions, obtaining
	\begin{align*}
		\aar{\dg M \bundle p}_\gamma &= \inf_\mu ~\bbr{\dg M \bundle p}_\gamma(\mu) \\
			&= \inf_\mu \left[ \bbr{\dg M }_\gamma(\mu)
				+ \Ex_{x\sim\mu_{\!_X}} \kldiv[\Big]{\mu(Y\mid x)}{p(Y\mid x)} \right]\\
			&= \inf_\mu \left[ \bbr{\dg M }_\gamma(\mu)
				+  \kldiv[\Big]{\mu(X,Y)}{p(Y \mid X)\, \mu(X)} \right].
	\end{align*}
	% % Choose $\gamma < \min (\{1\}\cup\{ \beta^{\dg M}_L : L \in \Ed^{\dg M}\})$.
	% Since $\bbr{\dg M}_\gamma$ is a $\gamma$-strongly convex function of $\mu$ for all
	% such $\gamma < \min_L \beta_L$, and
	% $\kldiv{\mu_{XY}}{\mu_X \; p_{Y\mid X}}$ is 1-strongly
	% convex in $p$ for fixed $\mu$ (\cref{lem:Dstrongcvx}),
	% % $\thickD$ is convex in both of its arguments,
	% their sum is $\gamma$-strongly convex in $\mu$ and in $p$.
	% By \cref{lem:cvx2} taking an infemum preserves this convexity,
	% and so
	% $
	%  	\inf_\mu \left[ \bbr{\dg M }_\gamma(\mu)
	% 	+  \kldiv[\big]{\mu_{XY}}{p_{Y \mid X}\; \mu_X} \right]
	% $, which equals $\aar{\dg M \bundle p}_\gamma$,
	% is $\gamma$-strongly convex in $p$.
	% % $\aar{\dg M \bundle p}_\gamma$ is smooth
	% % Smoothness.


	% Choose $\gamma < \min (\{1\}\cup\{ \beta^{\dg M}_L : L \in \Ed^{\dg M}\})$.
	Fix $\gamma < \min_L \beta_L$. Then we know that $\bbr{\dg X}_\gamma(\mu)$ is a $\gamma$-strongly convex function for every PDG $\dg X$, and hence there is a unique joint distribution which minimizes it.

	\textbf{Strict Convexity.}
	Suppose $p_1(Y \mid X)$ and $p_2(Y\mid X)$ are two cpds on $Y$ given $X$.
	Fix $\lambda \in [0,1]$, and set $p_\lambda = (1-\lambda) p_1 + \lambda p_2$.
	Let $\mu_1, \mu_2$ and $\mu_\lambda$ be the joint distributions that minimze $\bbr{\dg M \bundle p_1}_\gamma$, $\bbr{\dg M \bundle p_2}_\gamma$ and $\bbr{\dg M \bundle p_\lambda}_\gamma$, respectively.  Then we have
	\begin{equation*}
		\aar{\dg M \bundle p_\lambda}_\gamma
			= \bbr{\dg M}_\gamma(\mu_\lambda) + \kldiv[\Big]{\mu_\lambda(X,Y)}{p_\lambda(Y\mid X) \mu_\lambda( X)}.
	\end{equation*}
	By convexity of $\bbr{\dg M}$ and $\thickD$, we have
	\begin{align}
		\bbr{\dg M}_\gamma(\mu_\lambda)
		 	&\le (\lambda-1)\bbr{\dg M}_\gamma(\mu_1) + \lambda \bbr{\dg M}_\gamma(\mu_2)
			 	\label{eqn:score-cvx}\\
		\text{and}\qquad \kldiv[\Big]{\mu_\lambda(XY)}{p_\lambda(Y | X) \mu_\lambda( X)}
			&\le (1-\lambda)\kldiv[\Big]{\mu_1(XY)}{p_1(Y | X) \mu_1( X)} \nonumber \\
			&\qquad+ \lambda\;\;\kldiv[\Big]{\mu_2(XY)}{p_2(Y | X) \mu_2( X)}.
				\label{eqn:D-cvx}
	\end{align}
	If $\mu_1 \ne \mu_2$ then since $\bbr{\dg M}$ is strictly convex, \eqref{eqn:score-cvx} must
	be a strict inequality. On the other hand, if $\mu_1 = \mu_2$, then since $\mu_\lambda = \mu_1 = \mu_2$ and $\thickD$ is stricly convex in its second argument when its first argument is fixed (\Cref{lem:Dstrongcvx}), \eqref{eqn:D-cvx} must be a strict inequality.
	In either case, the sum of the two inequalities must be strict, giving us
	\begin{align*}
		\aar{\dg M \bundle p_\lambda}_\gamma &=
		\bbr{\dg M}_\gamma(\mu_\lambda) + \kldiv[\Big]{\mu_\lambda(XY)}{p_\lambda(Y | X) \mu_\lambda( X)} \\
		&<
		 (\lambda-1) \left[\bbr{\dg M}_\gamma(\mu_1)
			 	+ \kldiv[\Big]{\mu_1(XY)}{p_1(Y | X) \mu_1( X)} \right]
			 \\[-0.3em]&\qquad\qquad
			 + \lambda \left[ \bbr{\dg M}_\gamma(\mu_2)
			 	+ \kldiv[\Big]{\mu_2(XY)}{p_2(Y | X) \mu_2( X)}
			 	\right] \\
		 &= (\lambda-1) \aar{\dg M \bundle p_1} + \lambda\,\aar{\dg M \bundle p_2},
	\end{align*}
	which shows that $\aar{\dg M \bundle p}$ is \emph{strictly} convex in $p$, as desired.


	\textbf{Smoothness.}
	If $\bbr{\dg M \bundle p}_\gamma^*$ is a positive distribution, then by definition $\bbr{\dg M \bundle p}$ achieves its minimum on the interior of the probability simplex $\Delta \V(\dg M \bundle p)$, and so by \Cref{lem:cvx4}, we immediately find that $\aar{\dg M \bundle p}_\gamma$ is smooth in $p$.

	Now, suppose that $\bbr{\dg M \bundle p}_\gamma^*(\mat w) = 0$,  for some $\mat w \in \V(\dg M \bundle p)$.

	Applying \Cref{lem:cvx4} to the function $f = \bbr{\dg M}_\gamma$

	Now for the second case.

	\TODO

	If $x^*_b \in \partial X$, then we claim that either
	\begin{enumerate}[nosep]
		\item There is a subspace $T \subseteq \mathbb R^{m}$ with
			$\SD{}$
	 	\item There is a subspace $S \subseteq \mathbb R^{n}$ with
			$x^*_b \in S \cap \partial X$ such

	\end{enumerate}

\end{lproof}

\begin{lemma}\label{lem:cvx4}
	Let $X$ and $Y$ be convex sets, and
	$f : X \times Y \to \mathbb R$ be a smooth $(C^\infty)$, convex function.
	If $f$ is strictly convex in $X$, and for some $y_0 \in Y$, $f(x, y_0)$ achieves its infemum on the interior of $X$.
	then $y\mapsto \inf_x f(x, y)$ is smooth $(C^\infty)$ at the point $y_0$.
\end{lemma}

\begin{lproof}%[Proof of \Cref{lem:cvx4}]
	% Let $f_y(x) = f(x,y)$.
	% Since $f$ is smooth and stritly convex, each restriction $f_y$ of $f$ to a
	% particular $y$ is also smooth and strictly convex.
	% As a result, each $f_y$ has a unique minimum $m_y := \inf_{x} f_y(x)$.
	% As $f_y$ is smooth, $m_y$ is either a boundary point, or
	% at a point where $\nabla f_y = 0$.
	%
	% Moreover, it is a constrained optimization problem, so
	% $\nabla_{x,y,\lambda} [ f(x,y) + \lambda (y_0 - y)] = 0$.
	%
	% \TODO
	Let $x_0^* := \arg\min_x f(x,y_0)$, which is achieved by assumption, and is unique because $f(-,y_0)$ is strictly convex.

	We will ultimately apply the implicit function theorem to give us a smooth function which is equal to this infemum, but to do so we must deal with the technicality that it requires an open set; the boundary is the most complicated part of this result.
	Here we have essentially required that the domain be open by fiat for $X$, but for $Y$ (which is a possibly non-open subset of $\mathbb R^m$), we use the Extension Lemma for smooth functions \cite[Lemma 2.26]{Lee.SmoothManifolds}. In our context, it states that
	for every open set $U$ with $\overline{Y} \subseteq U \subseteq \mathbb R^m$,
	there exists a function $\tilde f : X \times \mathbb R^m \to \mathbb R$, such that $\tilde f |_{Y} = f$ (and $\supp \tilde f \subseteq U$).
	We only need a small fraction of this power: that we can smoothly extend $f$ to \emph{some} open set of $\mathbb R^m$, which we fix and call $\tilde Y$.

	% Similarly, for other $y \in Y$, let $x^*_y$ be the unique value of $x$ which minimizes $f(x,y)$.

	% \textbf{Smoothness.}
	% By assumption, $x^*_b$ is not a boundary point of $X$.
	%
	We claim that now all conditions for the Implicit Function Theorem are met if invoked with
		$\phi(y,x) := \vec\nabla_x \tilde f(x,y)$ and $(\mat b,\mat a) = (y_0, x^*_0)$.
	Concretely, we have $m = \mathop{dim} X$, $n = \mathop{dim} Y$, and $Z = (\tilde Y \times X)^\circ$, i.e., the interior of $\tilde Y \times X$, which is open and contains $(\mat b, \mat a)$.
	 Becuase $\phi$ is smooth, it is $k$-times differentiable for all $k$. We have $\vec\nabla_x \tilde f (y_0, x^*_0) = \vec 0$ because $x^*_0$ is a local minimum of the smooth function $\tilde f(-, y_0)$ which lies on the interior of $X$.

	Moreover, the Jacobian matrix
	\[ \mat J_{\nabla\!\tilde f, x}(y_0, x_0^*) = \left[ \frac{\partial^2 f}{\partial x_i \partial x_j}(x^*_0, y_0) \right]\]
	is the Hessian of the strictly convex funtion $f(-, b)$, and therefore positive definite (and in particular non-singular).
	Therefore, the Implicit Function Theorem guarantees us the existence of a neighborhood $U \subset \tilde Y$ of $y_0$ for which
	there is a unique $k$-times differentiable function $g: U \to X$ such that $g(y_0) = x^*_0$ and $\vec\nabla_x \tilde f(y, g(y)) = 0$ for all $y \in U$. Of course, this implies $g(y) = \argmin_x f(x,y)$ at every such point, and $\inf_x f(x,y) = f(g(y),y)$ is a composition of the smooth function $f$ with the $k$-times differentiable function $g \otimes \mathrm{id}_Y$.
	Therefore, $\inf_x f(x,y)$ is itself $k$-times continuously differentiable at $y_0$ for all $k$, or in other words, $\inf_x f(x,y)$ is smooth at $y=y_0$.
\end{lproof}

\recall{prop:markov-property}
\begin{lproof}
	Choose $\mu \in \bbr{\dg M_1 \bundle \dg M_2}^*_\gamma$.
	% Choose $\mu \in \mu^*_\gamma (\dg M_1 \bundle \dg M_2)$.
	Let $\mu' := \mu(\N_1) \mu(\N_2)$
	
	\TODO[Finish Transcribing Proof]
\end{lproof}


\subsection{Hardness Results}

\recall{prop:consistent-NP-hard}
\begin{lproof} \label{proof:consistent-NP-hard}
	We can directly encode SAT problems as PDGs.
	Specifically, let
	$$\varphi := \bigwedge_{j \in \mathcal J} \bigvee_{i \in \mathcal I(j)} (X_{j,i})$$
	be a CNF formula over binary variables $\mat X := \bigcup_{j,i} X_{j,i}$. Let
	$\dg M_\varphi$ be the PDG containing every variable $X \in \mat X$ and a binary
	variable $C_j$ (taking the value 0 or 1) for each clause $j \in \mathcal J$, as well as the following edges, for each $j \in \mathcal J$:
	%\{$``$\varphi(\mat X)$''$\}$ with $\V(\varphi) = \{0,1\}$, and
	\begin{itemize}
		\item a hyper-edge $\{X_{j,i} : i \in \mathcal I(j)\} \tto C_j$, together with a degenerate cpd
			encoding the boolean OR function (i.e., the truth of $C_j$ given $\{X_{j,i}\}$);
		\item an edge $\pdgunit \tto C_j$, together with a cpd asserting $C_j$ be equal to 1.
	\end{itemize}
	% We give each edge $\alpha = 0$ and $\beta = 1$.
	First, note that the number of nodes, edges, and non-zero entries in the cpds are polynomial in the $|\mathcal J|, |\mat X|$, and the total number of parameters in a simple matrix representation of the cpds is also polynomial if $\mathcal I$ is bounded (e.g., if $\varphi$ is a 3-CNF formula).
	A satisfying assignment $\mat x \models \varphi$ of the variables $\mat X$ can be regarded as a degenerate joint distribution $\delta_{\mat X = \mat x}$ on $\mat X$, and extends uniquely to a full joint distribution $\mu_{\mat x} \in \Delta \V(\dg M_\varphi)$ consistent with all of the edges, by
	\[ \mu_{\mat x} = \delta_{\mat x} \otimes \delta_{\{C_j = \vee_i  x_{j,i}\}} \]

 	Conversely, if $\mu$ is a joint distribution consistent with the edges above, then any point $\mat x$ in the support of $\mu(\mat X)$ must be a satisfying assignment, since the two classes of edges respectively ensure that $1 =\mu(C_j\!=\! 1 \mid \mat X \!=\! \mat x) = \bigvee_{i \in \mathcal I(j)} \mat x_{j,i}$ for all $j \in \mathcal J$, and so $\mat x \models \varphi$.

	Thus, $\SD{\dg M_\varphi} \ne \emptyset$ if and only if $\varphi$ is satisfiable, so
	an algorithm for determining if a PDG is consistent can also be adapted (in polynomial space and time) for use as a SAT solver, and so the problem of determining if a PDG consistent is NP-hard.

% \end{lproof}
% \recall{prop:sharp-p-hard}
% \begin{lproof}\label{proof:sharp-p-hard}
    
    \medskip\hrule\smallskip
    
	\textbf{PART (b).}
    We prove this by reduction to \#SAT. Again, let $\varphi$ be some CNF formula over $\mat X$, and construct
	$\dg M_\varphi$ as in \hyperref[proof:consistent-NP-hard]{the proof} of
	\Cref{prop:consistent-NP-hard}.
	Furthemore, let $\bbr{\varphi} := \{ \mat x : \mat x \models \varphi \}$ be the set of  assingments to $\mat X$ satisfying $\varphi$, and $\#_\varphi := |\bbr{\dg M}|$ denote the number such assignments. We now claim that
	\begin{equation}\label{eqn:number-of-solns}
		\#_\varphi = \exp \left[- \frac1\gamma \aar{ \dg M_\varphi }_\gamma \right].
	\end{equation}
 	If true, we would have a reduced the \#P-hard problem of computing $\#_\varphi$ to the problem of computing $\aar{\dg M}_\gamma$ for fixed $\gamma$. We now proceed with proof \eqref{eqn:number-of-solns}.
	By definition, we have
	\[ \aar{\dg M_\varphi}_\gamma = \inf_\mu \Big[ \Inc_{\dg M_\varphi}(\mu) + \gamma \IDef{\dg M_\varphi}(\mu) \Big]. \]
	We start with a claim about first term.
	% For the particular PDG $\dg M_\varphi$, the

	\begin{iclaim} \label{claim:separate-inc-varphi}
		% $\Inc(\dg M_\varphi)$ is finite if and only if $\varphi$ is statisfiable.
		$\Inc_{\dg M_\varphi}\!(\mu) =
		% \begin{cases}
		% 	0 & \text{if}~  \mat x \models \varphi~\text{and}~\mat c = \mat 1
		% 	 	~\text{for all}~(\mat x, \mat c) \in \supp \mu\\
		% 	\infty & \text{otherwise}
		% \end{cases}
		\begin{cases}
			0 & \text{if}~  \supp \mu \subseteq \bbr{\varphi} \times \{ \mat 1\} \\
			\infty & \text{otherwise}
		\end{cases}$.
	\end{iclaim}
	\vspace{-1em}
	\begin{lproof}
		Writing out the definition explicitly, the first can be written as
		\begin{equation}
			\Inc_{\dg M_\varphi}\!(\mu) = \sum_{j} \left[ \kldiv[\Big]{\mu(C_j)}{\delta_1} +
				\Ex_{\mat x \sim \mu(\mat X_j)} \kldiv[\Big]{\mu(C_j \mid \mat X_j = \mat x)}{\delta_{\lor_i \mat x_{j,i}}} \right], \label{eqn:explicit-INC-Mvarphi}
				% &= \sum_{j} \left[
				% 	\begin{matrix} \mu(C_j\!=\!0) (\infty) \\
				% 	 	+ \mu(C_j \!=\! 1) \log \mu(C_j \!=\! 1)
				% 	\end{matrix} +
				% 	\Ex_{\mat x \sim \mu(\mat X_j)} \kldiv[\Big]{\mu(C_j \mid \mat X_j = \mat x)}{\delta_{\lor_i \mat x_i}} \right],
		\end{equation}
		where $\mat X_j = \{X_{ij} : j \in \mathcal I(j)\}$ is the set of variables that
		appear in clause $j$, and $\delta_{(-)}$ is the probability distribution placing all mass on the point indicated by its subscript.
		As a reminder, the relative entropy is given by
		\[ \kldiv[\Big]{\mu(\Omega)}{\nu(\Omega)} := \Ex_{\omega \sim \mu} \log \frac{\mu(\omega)}{\nu(\omega)},
		\quad\parbox{1.4in}{\centering and in particular, \\ if $\Omega$ is binary,}\quad
			\kldiv[\big]{\mu(\Omega)}{\delta_\omega} = \begin{cases}
				0 &  \text{if}~\mu(\omega) = 1 ; \\
				\infty & \text{otherwise}.
		\end{cases} \]
		Applying this to \eqref{eqn:explicit-INC-Mvarphi}, we find that either:
		\begin{enumerate}[itemsep=0pt]
			\item Every term of \eqref{eqn:explicit-INC-Mvarphi} is finite (and zero) so $\Inc_{\dg M_\varphi}(\mu) = 0$, which happens when $\mu(C_j = 1) = 1$ and $\mu(C_j = \vee_i~ x_{j,i}) = 1$ for all $j$.  In this case, $\mat c = \mat 1 = \{ \vee_i~x_{j,i} \}_j$ so $\mat x \models \varphi$ for every $(\mat{c,x}) \in \supp \mu$;
			\item Some term of \eqref{eqn:explicit-INC-Mvarphi} is infinite, so that $\Inc_{\dg M_\varphi}(\mu) = \infty$, which happens if some $j$, either

			\begin{enumerate}
				\item $\mu(C_j \ne 1) > 0$ --- in which case there is some $(\mat{x,c}) \in \supp \mu$ with $\mat c \ne 1$, or
				\item $\supp \mu(\mat C) = \{\mat 1\}$, but $\mu(C_j \ne \vee_i~ x_{j,i}) > 0$ --- in which case there is some $(\mat{x,1}) \in \supp \mu$ for which $1 = c_j \ne \vee_i~x_{j,i}\;$, and so $\mat x \not\models \varphi$.
			\end{enumerate}
		\end{enumerate}
		Condensing and rearranging slightly, we have shown that
		\[
			\Inc_{\dg M_\varphi}(\mu) =
			\begin{cases}
				0 & \text{if}~  \mat x \models \varphi~\text{and}~\mat c = \mat 1
				 	~\text{for all}~(\mat x, \mat c) \in \supp \mu\\
				\infty & \text{otherwise}
			\end{cases}~.
		\]
		% So if $\mat x \models \varphi$ for all $\mat x \in \supp \mu(X)$,
		%
		% $\Inc_{\dg M_\varphi}(\mu) = 0$
		% The first term is infinite if $\mu(C_j = 1) < 1$, and the second is infinite
		% if $\mu(C_j = \lor_i X_{i,j}) < 1$. Thus, if $\Inc_{\dg M_\varphi}(\mu)$ is finite, then $\mat x \sim \mu(\mat X)$ satisfies $\varphi$ with probability 1, and $\varphi$ must be satisfiable.
		% Conversely,
	\end{lproof}

	% Thus, if $\Inc_{\dg M_\varphi}(\mu)$ is finite, then every $\mat x \in \supp \mu$ is a satisfying assignment of $\varphi$.
	Because $\IDef{}$ is bounded, it follows immediately that
 	$\aar{\dg M_\varphi}_\gamma$, is finite if and only if
	there is some distribution $\mu \in \Delta\V(\mat X,\mat C)$ for which $\Inc_{\dg M_\varphi}(\mu)$ is finite, or equivalently, by \Cref{claim:separate-inc-varphi}, iff there exists some $\mu(\mat X) \in \Delta \V(\mat X)$ for which $\supp \mu(\mat X) \subseteq \bbr{\varphi}$, which in turn is true if and only if $\varphi$ is satisfiable.

	In particular, if $\varphi$ is not satisfiable (i.e., $\#_\varphi = 0$), then $\aar{\dg M_\varphi}_\gamma = +\infty$, and
	\[
		\exp \left[ -\frac1\gamma \aar{\dg M_\varphi}_\gamma \right] =
	 		\exp [ - \infty ] = 0 = \#_\varphi,
	\]
	so in this case \eqref{eqn:number-of-solns} holds as promised. On the other hand, if $\varphi$ \emph{is} satisfiable, then, again by \Cref{claim:separate-inc-varphi}, every $\mu$ minimizing $\bbr{\dg M_\varphi}_\gamma$, (i.e., every $\mu \in \bbr{\dg M_\varphi}_\gamma^*$) must be supported entirely on $\bbr{\varphi}$ and have $\Inc_{\dg M_\varphi}\!(\mu) = 0$.  As a result, we have
	\[
		\aar{\dg M_\varphi}_\gamma =
			\inf\nolimits_{\mu \in \Delta \big[\bbr{\varphi} \times \{\mat 1\}\big]} \gamma\; \IDef{\dg M_\varphi}(\mu) .
	\]
	A priori, by the definition of $\IDef{\dg M_\varphi}$, we have
	\[
		\IDef{\dg M_\varphi}(\mu) =
		 	- \H(\mu) + \sum_{j} \Big[ \alpha_{j,1} \H_\mu(C_j \mid \mat X_j)
						+ \alpha_{j,0} \H_\mu(C_j) \Big],
	\]
	where $\alpha_{j,0}$ and $\alpha_{j,1}$ are values of $\alpha$ for the edges of $\dg M_\varphi$, which we have not specified because they are rendered irrelevant by the fact that their corresponding cpds are deterministic. We now show how this plays out in the present case.
	Any $\mu \in \Delta\big[\bbr{\varphi} \times \{\mat 1\}\big]$ we consider has a degenerate marginal on $\mat C$. Specifcally, for every $j$, we have $\mu(C_j) = \delta_1$, and since entropy is non-negative and never increased by conditioning,
	$$
		0 \le \H_\mu(C_j \mid \mat X_j) \le \H_\mu(C_j) = 0.
	$$
	Therefore, $\IDef{\dg M_\varphi}(\mu)$ reduces to the negative entropy of $\mu$.
	Finally, making use of the fact that the maximum entropy distribution $\mu^*$ supported on a finite set $S$ is the uniform distribution on $S$, and has $\H(\mu^*) = \log | S |$, we have
	\begin{align*}
		\aar{\dg M_\varphi}_\gamma &= \inf\nolimits_{\mu \in \Delta \big[\bbr{\varphi} \times \{\mat 1\}\big]} \gamma\; \IDef{\dg M_\varphi}(\mu) \\
			&= \inf\nolimits_{\mu \in \Delta \big[\bbr{\varphi} \times \{\mat 1\}\big]} -\, \gamma\, \H(\mu) \\
			&= - \gamma\, \sup\nolimits_{\mu \in \Delta \big[\bbr{\varphi} \times \{\mat 1\}\big]}  \H(\mu) \\
			&= - \gamma\, \log (\#_\varphi),
	\end{align*}
	\hspace{1in}giving us
	$$
		\#_\varphi = \exp \left[- \frac1\gamma \aar{ \dg M_\varphi }_\gamma \right],
	$$
	as desired. We have now reduced \#SAT to computing $\aar{\dg M}_\gamma$, for $\gamma \in \mathbb R^{>0}$ and an arbitrary PDG $\dg M$, which is therefore \#P-hard.
\end{lproof}


\section{}
\begin{conj}
    Inference in a PDG---that is, computing conditional marginals of $\bbr{\dg M}^*$---%
    and the computing inconsistency $\aar{\dg M}$ are equally difficult:
        there are polynomial-time reductions from each to the other.
\end{conj}

If we do not restrict to finite variables, then the problem is much worse.

\begin{linked}{conj}{incomputable}
    The problem of deciding whether a PDG whose variables take values in $\mathbb N$ is not computable.
\end{linked}

\end{document}
