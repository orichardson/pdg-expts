We then evaluate our approach, showing 
that this exponential-cone-based approach is more precise
and, often, faster than generic optimization baselines.
While not currently as fast as inference methods such as belief
propagation on the models to which belief propagation can be applied,
we are optimistic that further improvements are possible.
In any case, our results show that inference in PDGs is feasible.

%%%%%%%%%%%==================================================%%%%%%%%%%%%
 
 %joe2*: I think here you need to say something about how you can use
 %inconsistency minimization to do inference.  The paper is about
 %inference, after all, and you haven't made the connection.  This is critical
%oli2: I agree, but I'm a little bit furstrated because it was structurally
% much closer to doing this before I accepted your %joe1 edits. 
% I'm rewriting the paragraph.  
% 
% However, the earlier work on PDGs does not provide any
% computational method for calculating whether a PDG is consistent and,
% if not, its degree of inconsistency.  We provide such methods in this paper.
%
%From a pragmatic point of view, though, PDGs are currently not yet
%very useful.  As it stands, they only have conceptual
%applications---one can use them to justify 
%a choice of loss function analytically, or to derive cute diagrammatic proofs
%of inequalitites you likely already know \parencite{one-true-loss},
%but it is impossible to compute with them.
%% What use is a model without an inference algorithm? 
%Until now, PDGs have been a model without an inference algorithm. 
%
%joe1: where do we do updating?   What inference problems do we
%consider?  You need to slow down here and explain what we do
%We analyze the complexity of inference and updating in pdgs, and show
%joe2: In the previous paragraph you talked about minimizing
%inconsistency.  Here you talk about the complexity of inference.  You
%have to make the connection.  
In more detail,
we analyze the complexity of inference in PDGs, and illustrate
the close relationship it has with inconsistency minimization.
% that it is equivalent to that of inconsistency minimization. 
%joe1*: I have no idea what exponential-cones constraints are.  Unless
%this is a completely standard notion in the AIStats community, you
%*must* give some intuition.  Also, when you talk about reducing the
%problem to a linear program, (a) I don't know which problem you're
%talking about and (b) we usually talk about reducing one problem to
%another, not reducing a problem to a linear program
%
Then, we reduce the problem to a convex optimization problem in standard
form.
This allows us to use powerful interior-point methods
that can solve such problems in polynomial time \parencite{dahl2022primal}. 

%%%%%%%%%%%==================================================%%%%%%%%%%%%
