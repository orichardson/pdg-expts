
\documentclass{article}


\usepackage[margin=1.2in]{geometry}
\usepackage{parskip}

\relax % Controls
    \newif\ifmarginprooflinks
    	\marginprooflinkstrue
    	% \marginprooflinksfalse

\relax % Bibliography, etc
	\usepackage[american]{babel}
	\usepackage{csquotes}
	\usepackage[backend=biber, style=authoryear]{biblatex}
	\DeclareLanguageMapping{american}{american-apa}
	% \usepackage[backend=biber,style=authoryear,hyperref=true]{biblatex}
	% \addbibresource{refs.bib}
	% \addbibresource{conf.bib}

	\DeclareFieldFormat{citehyperref}{%
	  \DeclareFieldAlias{bibhyperref}{noformat}% Avoid nested links
	  \bibhyperref{#1}}

	\DeclareFieldFormat{textcitehyperref}{%
	  \DeclareFieldAlias{bibhyperref}{noformat}% Avoid nested links
	  \bibhyperref{%
	    #1%
	    \ifbool{cbx:parens}
	      {\bibcloseparen\global\boolfalse{cbx:parens}}
	      {}}}

	\savebibmacro{cite}
	\savebibmacro{textcite}

	\renewbibmacro*{cite}{%
	  \printtext[citehyperref]{%
	    \restorebibmacro{cite}%
	    \usebibmacro{cite}}}

	\renewbibmacro*{textcite}{%
	  \ifboolexpr{
	    ( not test {\iffieldundef{prenote}} and
	      test {\ifnumequal{\value{citecount}}{1}} )
	    or
	    ( not test {\iffieldundef{postnote}} and
	      test {\ifnumequal{\value{citecount}}{\value{citetotal}}} )
	  }
	    {\DeclareFieldAlias{textcitehyperref}{noformat}}
	    {}%
	  \printtext[textcitehyperref]{%
	    \restorebibmacro{textcite}%
	    \usebibmacro{textcite}}}

	\DeclareCiteCommand{\brakcite}
	  {\usebibmacro{prenote}}
	  {\usebibmacro{citeindex}%
	   \printtext[bibhyperref]{[\usebibmacro{cite}]}}
	  {\multicitedelim}
	  {\usebibmacro{postnote}}

\relax % Standard Packages
    \usepackage[dvipsnames]{xcolor}
    % \usepackage[utf8]{inputenc}
    \usepackage{mathtools}
    \usepackage{amssymb}
		\DeclareMathSymbol{\shortminus}{\mathbin}{AMSa}{"39}
    % \usepackage{parskip}
    % \usepackage{algorithm}
    \usepackage{bbm}
	\usepackage{lmodern}
	% \usepackage{times}
    \usepackage{faktor}
    % \usepackage{booktabs}
	% \usepackage[margin=1in]{geometry}
    \usepackage{graphicx}
    \usepackage{scalerel}
    \usepackage{enumitem}
    \usepackage{nicefrac}\let\nf\nicefrac

    % \usepackage{color}
    %\usepackage{stmaryrd}
    \usepackage{hyperref} % Load before theorems...
        \hypersetup{colorlinks=true, linkcolor=blue!75!black, urlcolor=magenta, citecolor=green!50!black}

\usepackage{tikz}
	\usetikzlibrary{positioning,fit,calc, decorations, arrows, shapes, shapes.geometric}
	\usetikzlibrary{cd}

	%%%%%%%%%%%%
	\tikzset{AmpRep/.style={ampersand replacement=\&}}
	\tikzset{center base/.style={baseline={([yshift=-.8ex]current bounding box.center)}}}
	\tikzset{paperfig/.style={center base,scale=0.9, every node/.style={transform shape}}}

	% Node Stylings
	\tikzset{dpadded/.style={rounded corners=2, inner sep=0.7em, draw, outer sep=0.3em, fill={black!50}, fill opacity=0.08, text opacity=1}}
	\tikzset{dpad0/.style={outer sep=0.05em, inner sep=0.3em, draw=gray!75, rounded corners=4, fill=black!08, fill opacity=1, align=center}}
	\tikzset{dpadinline/.style={outer sep=0.05em, inner sep=2.5pt, rounded corners=2.5pt, draw=gray!75, fill=black!08, fill opacity=1, align=center, font=\small}}

 	\tikzset{dpad/.style args={#1}{every matrix/.append style={nodes={dpadded, #1}}}}
	\tikzset{light pad/.style={outer sep=0.2em, inner sep=0.5em, draw=gray!50}}

	\tikzset{arr/.style={draw, ->, thick, shorten <=3pt, shorten >=3pt}}
	\tikzset{arr0/.style={draw, ->, thick, shorten <=0pt, shorten >=0pt}}
	\tikzset{arr1/.style={draw, ->, thick, shorten <=1pt, shorten >=1pt}}
	\tikzset{arr2/.style={draw, ->, thick, shorten <=2pt, shorten >=2pt}}

	\newcommand\cmergearr[5][]{
		\draw[arr, #1, -] (#2) -- (#5) -- (#3);
		\draw[arr, #1, shorten <=0] (#5) -- (#4);
		}
	\newcommand\mergearr[4][]{
		\coordinate (center-#2#3#4) at (barycentric cs:#2=1,#3=1,#4=1.2);
		\cmergearr[#1]{#2}{#3}{#4}{center-#2#3#4}
		}
	\newcommand\cunmergearr[5][]{
		\draw[arr, #1, -, shorten >=0] (#2) -- (#5);
		\draw[arr, #1, shorten <=0] (#5) -- (#3);
		\draw[arr, #1, shorten <=0] (#5) -- (#4);
		}
	\newcommand\unmergearr[4][]{
		\coordinate (center-#2#3#4) at (barycentric cs:#2=1.2,#3=1,#4=1);
		\cunmergearr[#1]{#2}{#3}{#4}{center-#2#3#4}
		}

\usepackage{amsthm,thmtools} % Theorem Macros
	\usepackage[noabbrev,nameinlink,capitalize]{cleveref}
    \theoremstyle{plain}
    \newtheorem{theorem}{Theorem}
	\newtheorem{coro}{Corollary}[theorem]
    \newtheorem{prop}[theorem]{Proposition}
    \newtheorem{conj}[theorem]{Conjecture}
    \newtheorem{claim}{Claim}
    \newtheorem{remark}{Remark}
    \newtheorem{lemma}[theorem]{Lemma}
    \theoremstyle{definition}
    % \newtheorem{defn}{Definition}
    % \declaretheorem[name=Definition]{defn}
    \declaretheorem[name=Definition, qed=$\square$]{defn}
    \declaretheorem[name=Example, qed=$\triangle$]{example}

    \definecolor{openQcolor}{rgb}{0.9,0.2,0.9}
    \declaretheorem[preheadhook=\color{openQcolor},name=Open Question]{openQ}

	\crefname{defn}{Definition}{Definitions}
	\crefname{prop}{Proposition}{Propositions}
    \crefname{issue}{Issue}{Issues}

\relax %%%%%%%%% GENERAL MACROS %%%%%%%%
    \let\Horig\H
	\let\H\relax
	\DeclareMathOperator{\H}{\mathrm{H}} % Entropy
	\DeclareMathOperator{\I}{\mathrm{I}} % Information
	\DeclareMathOperator*{\Ex}{\mathbb{E}} % Expectation
	\DeclareMathOperator*{\EX}{\scalebox{1.5}{$\mathbb{E}$}}
    \newcommand{\ifrac}[2]{{#1}/{#2}}
    
    % \DeclareMathOperator{\argmin}{\arg\min}
    \DeclareMathOperator*{\argmin}{\arg\!\min}
    \DeclareMathOperator*{\argmax}{\arg\!\max}

    \newcommand{\mat}[1]{\mathbf{#1}}
    \DeclarePairedDelimiterX{\infdivx}[2]{(}{)}{%
		#1\;\delimsize\|\;#2%
	}
	\newcommand{\thickD}{I\mkern-8muD}
	\newcommand{\kldiv}{\thickD\infdivx}
	\newcommand{\tto}{\rightarrow\mathrel{\mspace{-15mu}}\rightarrow}

	\newcommand{\datadist}[1]{\Pr\nolimits_{#1}}
	% \newcommand{\datadist}[1]{p_\text{data}}

	\makeatletter
	\newcommand{\subalign}[1]{%
	  \vcenter{%
	    \Let@ \restore@math@cr \default@tag
	    \baselineskip\fontdimen10 \scriptfont\tw@
	    \advance\baselineskip\fontdimen12 \scriptfont\tw@
	    \lineskip\thr@@\fontdimen8 \scriptfont\thr@@
	    \lineskiplimit\lineskip
	    \ialign{\hfil$\m@th\scriptstyle##$&$\m@th\scriptstyle{}##$\hfil\crcr
	      #1\crcr
	    }%
	  }%
	}
	\makeatother
	\newcommand\numberthis{\addtocounter{equation}{1}\tag{\theequation}}

\relax %%%%%%%%%   PDG  MACROS   %%%%%%%%
	\newcommand{\ssub}[1]{_{\!_{#1}\!}}
	% \newcommand{\bp}[1][L]{\mat{p}_{\!_{#1}\!}}
	% \newcommand{\bP}[1][L]{\mat{P}_{\!_{#1}\!}}
	\newcommand{\bp}[1][L]{\mat{p}\ssub{#1}}
	\newcommand{\bP}[1][L]{\mat{P}\ssub{#1}}
	\newcommand{\V}{\mathcal V}
	\newcommand{\N}{\mathcal N}
	\newcommand{\Ed}{\mathcal E}

    \newcommand{\balpha}{\boldsymbol\alpha}
    \newcommand{\bbeta}{\boldsymbol\beta}

	\DeclareMathAlphabet{\mathdcal}{U}{dutchcal}{m}{n}
	\DeclareMathAlphabet{\mathbdcal}{U}{dutchcal}{b}{n}
	\newcommand{\dg}[1]{\mathbdcal{#1}}
	\newcommand{\PDGof}[1]{{\dg M}_{#1}}
	\newcommand{\UPDGof}[1]{{\dg N}_{#1}}
	\newcommand\VFE{\mathit{V\mkern-4mu F\mkern-4.5mu E}}

	\newcommand\Inc{\mathit{Inc}}
	\newcommand{\IDef}[1]{\mathit{IDef}_{\!#1}}
	% \newcommand{\ed}[3]{%
	% 	\mathchoice%
	% 	{#2\overset{\smash{\mskip-5mu\raisebox{-3pt}{${#1}$}}}{\xrightarrow{\hphantom{\scriptstyle {#1}}}} #3} %display style
	% 	{#2\overset{\smash{\mskip-5mu\raisebox{-3pt}{$\scriptstyle {#1}$}}}{\xrightarrow{\hphantom{\scriptstyle {#1}}}} #3}% text style
	% 	{#2\overset{\smash{\mskip-5mu\raisebox{-3pt}{$\scriptscriptstyle {#1}$}}}{\xrightarrow{\hphantom{\scriptscriptstyle {#1}}}} #3} %script style
	% 	{#2\overset{\smash{\mskip-5mu\raisebox{-3pt}{$\scriptscriptstyle {#1}$}}}{\xrightarrow{\hphantom{\scriptscriptstyle {#1}}}} #3}} %scriptscriptstyle
	\newcommand{\ed}[3]{#2%
	  \overset{\smash{\mskip-5mu\raisebox{-1pt}{$\scriptscriptstyle
	        #1$}}}{\rightarrow} #3}

    \newcommand{\nhphantom}[2]{\sbox0{\kern-2%
		\nulldelimiterspace$\left.\delimsize#1\vphantom{#2}\right.$}\hspace{-.97\wd0}}
		% \nulldelimiterspace$\left.\delimsize#1%
		% \vrule depth\dp#2 height \ht#2 width0pt\right.$}\hspace{-.97\wd0}}
	\makeatletter
	\newsavebox{\abcmycontentbox}
	\newcommand\DeclareDoubleDelim[5]{
	    \DeclarePairedDelimiterXPP{#1}[1]%
			{% box must be saved in this pre code
				\sbox{\abcmycontentbox}{\ensuremath{##1}}%
			}{#2}{#5}{}%
		    %%% Correct spacing, but doesn't work with externalize.
			% {\nhphantom{#3}{##1}\hspace{1.2pt}\delimsize#3\mathopen{}##1\mathclose{}\delimsize#4\hspace{1.2pt}\nhphantom{#4}{##1}}
			%%% Fast, but wrong spacing.
			% {\nhphantom{#3}{~}\hspace{1.2pt}\delimsize#3\mathopen{}##1\mathclose{}\delimsize#4\hspace{1.2pt}\nhphantom{#4}{~}}
			%%% with savebox.
		    {%
				\nhphantom{#3}{\usebox\abcmycontentbox}%
				\hspace{1.2pt} \delimsize#3%
				\mathopen{}\usebox{\abcmycontentbox}\mathclose{}%
				\delimsize#4\hspace{1.2pt}%
				\nhphantom{#4}{\usebox\abcmycontentbox}%
			}%
	}
	\makeatother
	\DeclareDoubleDelim
		\SD\{\{\}\}
	\DeclareDoubleDelim
		\bbr[[]]
	% \DeclareDoubleDelim
	% 	\aar\langle\langle\rangle\rangle
	\makeatletter
	\newsavebox{\aar@content}
	\newcommand\aar{\@ifstar\aar@one@star\aar@plain}
	\newcommand\aar@one@star{\@ifstar\aar@resize{\aar@plain*}}
	\newcommand\aar@resize[1]{\sbox{\aar@content}{#1}\scaleleftright[3.8ex]
		{\Biggl\langle\!\!\!\!\Biggl\langle}{\usebox{\aar@content}}
		{\Biggr\rangle\!\!\!\!\Biggr\rangle}}
	\DeclareDoubleDelim
		\aar@plain\langle\langle\rangle\rangle
	\makeatother


	% \DeclarePairedDelimiterX{\aar}[1]{\langle}{\rangle}
	% 	{\nhphantom{\langle}{#1}\hspace{1.2pt}\delimsize\langle\mathopen{}#1\mathclose{}\delimsize\rangle\hspace{1.2pt}\nhphantom{\rangle}{#1}}

\relax %%%%% restatables and links
	% \usepackage{xstring} % for expandarg
	\usepackage{xpatch}
	\makeatletter
	\xpatchcmd{\thmt@restatable}% Edit \thmt@restatable
	   {\csname #2\@xa\endcsname\ifx\@nx#1\@nx\else[{#1}]\fi}% Replace this code
	   % {\ifthmt@thisistheone\csname #2\@xa\endcsname\typeout{oiii[#1;#2\@xa;#3;\csname thmt@stored@#3\endcsname]}\ifx\@nx#1\@nx\else[#1]\fi\else\csname #2\@xa\endcsname\fi}% with this code
	   {\ifthmt@thisistheone\csname #2\@xa\endcsname\ifx\@nx#1\@nx\else[{#1}]\fi
	   \else\fi}
	   {}{\typeout{FIRST PATCH TO THM RESTATE FAILED}} % execute on success/failure
	\xpatchcmd{\thmt@restatable}% A second edit to \thmt@restatable
	   {\csname end#2\endcsname}
	   {\ifthmt@thisistheone\csname end#2\endcsname\else\fi}
	   {}{\typeout{FAILED SECOND THMT RESTATE PATCH}}

	% \def\onlyaftercolon#1:#2{#2}
	\newcommand{\recall}[1]{\medskip\par\noindent{\bf \Cref{thmt@@#1}.} \begingroup\em \noindent
	   \expandafter\csname#1\endcsname* \endgroup\par\smallskip}

   	\setlength\marginparwidth{1.55cm}
	\newenvironment{linked}[3][]{%
		\def\linkedproof{#3}%
		\def\linkedtype{#2}%
		% \reversemarginpar
		% \marginpar{%
		% \vspace{1.1em}
		% % \hspace{2em}
		% 	% \raggedleft
		% 	\raggedright
		% 	\hyperref[proof:\linkedproof]{%
		% 	\color{blue!50!white}
		% 	\scaleleftright{$\Big[$}{\,{\small\raggedleft\tt\begin{tabular}{@{}c@{}} proof of \\\linkedtype~\ref*{\linkedtype:\linkedproof}\end{tabular}}\,}{$\Big]$}}
		% 	}%
        % \restatable[#1]{#2}{#2:#3}\label{#2:#3}%
		\ifmarginprooflinks
		\marginpar{%
			% \vspace{-3em}% %% for bottom
			\vspace{1.5em}
			\centering%
			\hyperref[proof:\linkedproof]{%
            % \hyperref[proof:#3]{
			\color{blue!30!white}%
			\scaleleftright{$\Big[$}{\,\mbox{\footnotesize\centering\tt\begin{tabular}{@{}c@{}}
				% proof of \\\,\linkedtype~\ref*{\linkedtype:\linkedproof}
				link to\\[-0.15em]
				proof
			\end{tabular}}\,}{$\Big]$}}~
			}%
		\fi
        \restatable[#1]{#2}{#2:#3}\label{#2:#3}%
        }%
		{\endrestatable%
		}
	\makeatother
		\newcounter{proofcntr}
		\newenvironment{lproof}{\begin{proof}\refstepcounter{proofcntr}}{\end{proof}}

		\usepackage{cancel}
		\newcommand{\Cancel}[2][black]{{\color{#1}\cancel{\color{black}#2}}}

		\usepackage{tcolorbox}
		\tcbuselibrary{most}
		\tcolorboxenvironment{lproof}{
			% fonttitle=\bfseries,
			% top=0.5em,
			enhanced,
			parbox=false,
			boxrule=0pt,
			frame hidden,
			borderline west={4pt}{0pt}{blue!20!black!40!white},
			% coltext={blue!20!black!60!white},
			colback={blue!20!black!05!white},
			sharp corners,
			breakable,
			% bottomsep at break=4cm,
			% enlarge bottom at break by=-4cm,
			% topsep at break=3cm,
			% enlarge top at break by=-3cm
		}
		% \usepackage[framemethod=TikZ]{mdframed}
		% \surroundwithmdframed[ % lproof
		% 	   topline=false,
		% 	   linewidth=3pt,
		% 	   linecolor=gray!20!white,
		% 	   rightline=false,
		% 	   bottomline=false,
		% 	   leftmargin=0pt,
		% 	   % innerleftmargin=5pt,
		% 	   skipabove=\medskipamount,
		% 	   skipbelow=\medskipamount
		% 	]{lproof}
	%oli16: The extra space was because there was extra space in the paragraph, not
	%because this length was too big. By breaking arrays, everything will be better.
	\newcommand{\begthm}[3][]{\begin{#2}[{name=#1},restate=#3,label=#3]}

\relax %TODOs and footnotes
    \newcommand{\TODO}[1][INCOMPLETE]{{\centering\Large\color{red}$\langle$~\texttt{#1}~$\rangle$\par}}
    \newcommand{\dfootnote}[1]{%
        \let\oldthefootnote=\thefootnote%
        % \addtocounter{footnote}{-1}%
		\setcounter{footnote}{999}
        \renewcommand{\thefootnote}{\textdagger}%
        \footnote{#1}%
        \let\thefootnote=\oldthefootnote%
    }
	\newcommand{\dfootnotemark}{
		\footnotemark[999]
	}


% \usepackage[framemethod=TikZ]{mdframed}
% \colorlet{color1}{Emerald}
% \colorlet{color3}{color1>wheel,2,3}
% \colorlet{pinkish}{color3!25!magenta}
\definecolor{brownish}{rgb}{0.5, 0.2, 0.1}
% \newmdenv[roundcorner=5pt, 
%     backgroundcolor=brownish!20!white,
%     % frametitle={$\langle$under construction$\rangle$},
%     frametitle={$\langle$incomplete or buggy$\rangle$},
%     frametitlerule=false,
%     innertopmargin=3pt, frametitlebelowskip=1ex, frametitleaboveskip=1ex,
%     frametitlebackgroundcolor=brownish!40!white,
%     skipabove=1em,skipbelow=1em,
%     frametitlefont={\scshape},leftmargin=-10pt, rightmargin=-10pt]
% 		{wip}
\newtcolorbox{wip}{%
    colback=brownish!20!white,%
    % frametitle={$\langle$under construction$\rangle$},
    title={$\langle$under construction$\rangle$},%
    enhanced jigsaw,
    breakable,
    % frametitlerule=false,
    % innertopmargin=3pt, frametitlebelowskip=1ex, frametitleaboveskip=1ex,
    colframe=brownish!40!white,%
    % skipabove=1em,skipbelow=1em,
    % frametitlefont={\scshape},leftmargin=-10pt, rightmargin=-10pt
}
        
% \tcolorboxenvironment{example}{
% \newtcolorbox[use counter=example]{examplex}{
\newtcbtheorem[use counter=example]{examplex}{Example}{
        label type=example,
        fonttitle=\bfseries,
        % empty,
        enhanced jigsaw,
        % title={Example \thetcbcounter.},
        before=\par\medskip\noindent,
        % top=0.5em,
        % enhanced,
        parbox=false,
        % boxrule=0pt,
        % frame hidden,
        % borderline west={4pt}{0pt}{green!20!black!40!white},
        % coltext={blue!20!black!60!white},
        % colback={blue!20!black!05!white},
        colback=white,
        sharp corners,
        breakable,
        % bottomsep at break=4cm,
        % enlarge bottom at break by=-4cm,
        % topsep at break=3cm,
        % enlarge top at break by=-3cm
    }
    %
    {ex}


\makeatletter
\newcommand{\@minipagerestore}{\setlength{\parskip}{\medskipamount}}
\makeatother

\usepackage{xparse}
\let\realItem\item % save a copy of the original item
\makeatletter
\NewDocumentCommand\myItemboldperiod{ o }{%
   \IfNoValueTF{#1}%
      {\realItem}% add an item
      {\realItem[\textbf{#1.}]\def\@currentlabel{#1}%
        % \def\cref@currentlabel{#1}
        }% add an item and update label
}
\makeatother


% \usepackage{enumitem}
\newlist{URaxioms}{enumerate}{1}
\setlist[URaxioms]{
    resume,%
    label=\textbf{UR\arabic{*}.},
    % label=UR\arabic{*},
    ref={UR\arabic*},
    leftmargin=2cm,
    before=\let\item\myItemboldperiod,
    topsep=1ex
    }
% \crefname{URaxiomsi}{axiom}{axioms}
\crefname{URaxiomsi}{}{}
\newcommand\commentout[1]{}

\addbibresource{conf.bib}


\title{Measures of Confidence}
\author{Oliver E Richardson, Joseph Y Halpern}
\begin{document}
    \maketitle

\section{Introduction}
% \def\stmt{$\phi$}
\def\stmt{$A$}

Being able to express variable confidence is an important aspect of knowledge representation,
% because we want to be able to deal with information that we aren't certain about.

%
% It is extremely useful to be able to express variable confidence in information.
%%
%%
%%
% Among computer scientists, the notions of confidence and probability are often identified.
Among computer scientists, ``confidence'' is often a synonym for probability.
% ---%
% Although the two notions are closely related, it is important to distinguish between confidence and likelihood.
For instance, it is common to say that one has ``high confidence in a statement \stmt'' to indicate a belief that
% (the content of) \stmt is likely to be true.
\stmt\ is likely to be true.
But
% However,
having high confidence in \stmt\ can also mean something subtly different: that we should \emph{trust} \stmt, in the sense of taking it seriously in updating our beliefs.
% The two notions might seem indistinghishable at first,
% This distinction may not at first seem meaningful, because if we take \stmt\ seriously in updating our beliefs, we will ultimately believe \stmt\ is likely.
If we take \stmt\ seriously in updating our beliefs, we will ultimately believe \stmt\ is likely, and so the distinction might not seem meaningful at first.
% However, the two are quite different at the other end of the spectrum. If you believe that \stmt\ is
% However, thinking that one should \emph{not} trust \stmt\ is quite different from thinking that \stmt\ is unlikely.
% But \emph{low} probability is quite different from low probability.
% At the other end of the scale, however, the two notions differ a great deal.
But at the other end of the scale, the two notions are quite different.
If we have little trust in \stmt, we should \emph{ignore} \stmt, rather than coming to believe that \stmt\ is unlikely.
So, if an untrusted party tells you something that you happen to already believe, you might reasonably ascribe it high probability but low confidence.

% We argue that confidence is a useful concept, distinct from probability.
% We will show how confidences
In this paper, we give a formal framework for talking about confidence in the context of updating,
which we then use to interperet quantities across a wide variety of settings as confidences.
% In each case, we will see the following features:
Here are some recurring themes:

\begin{enumerate}[nosep]
    \item Updating with zero confidence leaves the belief state unchanged.
    \item Updating with absolute certainty (the limit as confidence becomes large) causes you to (permanently) gain a certainty, and acts like a projection.
    % \item A zero-confidence update does nothing, while
    % \item Zero confidence is neutral.
    % updating with certainty (the limit of large confidence) is a projection.
    % \item Updating with certainty (the limit of large confidence) is a projection.
    \item Confidence measures involve an arbitrary choice of scale, and so only ratios of confidences are meaningful.
    % \item $\exp(- \beta E)$
    % \item Confidence tends to end up as the scaling term for an exponential functions
    \item Update rules are exponential in confidence.
\end{enumerate}


% Theorem. Certainties can never be gained or lost with any finite sequence of differentiable updates.
% Theorem. Certainty is always infinitely far away, as measured by the Riemannian metric.

\section{Update Rules}
\def\X{\mathcal X}

% We now proceed with the formal details.
Consider a space $\Theta$
of possible belief states,
and a set $\Phi$ of ``statements'', i.e., the things one can have confidence in.
An \emph{update rule} (or more precisely, a \emph{$\Theta$-updating rule on $\Phi$})
is then a function of the form
\[
    F :  (\mathbb R \times \Phi) \to \Big( \Theta \to \Theta \Big)
\]
which describe how to update beliefs about $X$, with the new information, at a certain level of trust.
Given a piece of information $\phi \in \Phi$, and a number $\beta$ measuring the change in confidence of $\phi$, we write
% $F^\beta_\phi : \Delta\X \to \Delta X$
$F^\beta_\phi : \Theta \to \Theta$
for the \emph{update} prescribed by $F$.
Note that updates can be composed.
We will insist that update rules obey certain properties, especially the following two.
\begin{URaxioms}
    % \item \textbf{(zero)} $F^{0}_A(\Pr) = (\Pr)$
    % \item  $F^{0}_A  = {1}_{\Delta\X}$. (That is, $F^{0}_A(\Pr) = \Pr$ for all $\Pr \in \Delta\X$.)
    % \item  $F^{0}_\phi  = {1}_{\Delta\X}$. (That is, $F^{0}_\phi(\Pr) = \Pr$ for all $\Pr \in \Delta\X$.)
    %     \hfill \textbf{(zero)} \label{ax:zero}
    \item  $F^{0}_\phi  = 1_{\Theta}$.
        (That is, $F^{0}_\phi(\theta) = \theta$ for all $\theta \in \Theta$.)
        \hfill \textbf{(zero)} \label{ax:zero}
    % \item $F^{\beta_1}_A \circ F^{\beta_2}_A = F^{\beta_1 + \beta_2}_A$
    \item For all $\beta_1, \beta_2 \in \mathbb R_{\ge 0}$,~
        $F^{\beta_1}_\phi \circ F^{\beta_2}_\phi = F^{\beta_1 + \beta_2}_\phi$
        \hfill \textbf{(additivity)} \label{ax:additivity}
\end{URaxioms}


We are primarily interested in the setting where $\Theta$ parameterizes a family of probaility distributions.
To that end, suppose that $\X = (X, \mathcal A)$ be a measurable space, so that $X$ is a set and $\mathcal A$ be a $\sigma$-algebra over it, let $\Delta \X$ denote the set of probability measures over $\X$,
and keep in the back of our heads an indexed family
% $\{ p_\theta ( X_\theta ) : \theta \in \Theta\}$.
$
    \mathcal P =
    \{ p_\theta \in\Delta\X \mid \theta \in \Theta \}
$ of probability distributions.
% If we take

% For instance, starting with a distribution $\mu_0 \in \Delta \X$, we can \emph{compose} updates.
%
% % \[
% %     \mu_0
% %         \xmapsto{\displaystyle F^{.3}_{\mathit{Height}=5'11''}} \mu_1
% %         \xmapsto{\displaystyle F^{.6}_{\Pr(Y=1|X=3)=0.4}} \mu_2
% %         \xmapsto{\displaystyle F^{2.1}_{K_i(\varphi)}} \mu_3.
% % \]


\begin{examplex}{Balls in Urns}{colorballs}%\label{ex:colorballs}
    \def\red{{\mathsf{red}}}
    \def\green{{\mathsf{green}}}
    \def\blue{{\mathsf{blue}}}
    % Perhaps we're uncertain about our shoe size.
% Imagine a ball has been drawn from an urn,
% and we are uncertain about its color, which we know to be either red, green, or blue.
Suppose there is a game show in which, at some point, a ball of unknown color
(red, green, or blue) is drawn from an urn.
Let  $X := \{ \red,\green,\blue \}$ be the three-element set corresponding to possible colors,
 and $\mathcal A = 2^X$ be all possible subsets of it, so that a robability $\mu \in \Delta \X$
can be specified as a
% three-dimensional vector $\mat p \in \Theta
point in the simplex
$$\Theta
:= \Delta \{ \red,\green,\blue \}
= \{\mat p = (p_1, p_2, p_3)  \in \mathbb R^3_{\ge 0} : p_1 + p_2 + p_3 = 1\},$$
interpreting
the coordinates as the respective probabilities of drawing $\red$, $\green$, and $\blue$.
% the a point
% $\langle p_0, p_1, p_2 \rangle$
% in the standard basis as
% $(p_\red, p_\green, p_\blue \rangle$, where $p_x$ is the probability of drawing color $x$.
% an interpretation of $\langle p_0, p_1, p_2 \langle$ as the respective probabilities
%
     % is just a 3-dimensional vector $\mat p$.
% For the assertion language, consider the set
Now consider the set
 $\Phi := \{$``the ball is $\mathsf c$''$ : \mathsf c \in X\} \cong X$ of possible assertions of the color of the ball.
% , so that you can make assertions such as ``the ball is red''.

% Now, consider the update rule $F$ defined by
One possible update rule $F$ for $\Phi$ and $\Theta$ can be defined by increasing the probability of the given statement, by a factor exponential in the confidence, and then renormalizing.
For instance,
    \[
        % F^{\beta}_h( \mat p ) =
        % \frac{1}{\mat p \cdot \mat e_h} \mat p \odot \exp(\beta )
        F^{\beta}_{\text{``the ball is $\red$''}} \left(
            \begin{bmatrix} p_\red \\ p_\green \\ p_\blue \\ \end{bmatrix}
        \right) :=
            \underbrace{\vphantom{\Bigg|}
                \frac{1}{1 + p_\red (e^{\beta} - 1) }
            }_{\text{normalizing constant}}
            % \begin{bmatrix}
            %     \vdots \\
            %     \hphantom{e^\beta} p_{h-1} \\
            %     e^\beta \cdot p_h  \\
            %     \hphantom{e^\beta} p_{h+1}\\
            %     \vdots
            % \end{bmatrix}
            \begin{bmatrix}
                e^\beta  p_\red \\
                p_\green  \\
                p_\blue
            \end{bmatrix},
    \]
    and analogously for green and blue.
    Note that an exponential of some sort is necessary to get \cref{ax:additivity}, for update rules of this kind; we cannot simply replace
    % it with another increasing function of $\beta$.
    $e^{\beta}$ with $\beta^2$, for instance.

    As mentioned above, you can compose udpates. For instance, suppose we had a prior $\mat p_0$ over colors, and then three people come up to us and make assertions about the ball color---%
    % first an aquaintance, then our friend, and then our tailor.
    first someone we don't recognize, who claims the ball is red, then a hobyist with a statistical model who says it's green, and finally someone who says they witnessed the draw, who says blue.
    We trust them to different degrees, and so in response  chain of updates as follows:
    \[
        \mat p_0
            \xmapsto{\displaystyle F^{\,0.3}_{\text{``ball is red''}}}
        \mat p_1
            \xmapsto{\displaystyle F^{\,0.6}_{\text{``ball is green''}}}
        \mat p_2
            \xmapsto{\displaystyle F^{\,7.1}_{\text{``ball is blue''}}}
        \mat p_3.
    \]

    % How should one decide the confidences $\beta$?
    % That depeends on the update rule.

    % The choices of confidence $\beta=0.3, 0.6, 7.1$ seem meaningless, but one aim of this paper is to argue that they are meaningful relative to an update rule.

    % We make the following observations.
    We now make some observations.
    \begin{enumerate}
        % \item The order might matter.
        \item For this particular update rule, the order of the updates does not matter.
        \item If we happen to already know the color of the ball (e.g., if $\mat p_0$ is a point mass on $\red$), then none of these updates will have an effect; this will still be the case after the updates.
        % \item betas are meaningless if we mix and match update rules
        \item Starting from any strictly positive distribution, one can get to any distribution at all.
        Concretely, for every $\mat p$ and $\mat q > 0$, there is a vector
        % $\boldsymbol\beta^{\mat q \to \mat p}
        % = (\beta_{\red}^{\mat q \to \mat p}, \beta_{\green}^{\mat q \to \mat p}, \beta_{\blue}^{\mat q \to \mat p})$
        $\boldsymbol\beta
        = (\beta_{\red}, \beta_{\green}, \beta_{\blue})$
        such that applying the updates
        $F^{\beta_{\red}
            % ^{\mat q \to \mat p}
            }_{\red}$,
        $F^{\beta_{\green}
            % ^{\mat q \to \mat p}
            }_{\green}$, and
        $F^{\beta_{\blue}
            % ^{\mat q \to \mat p}
            }_{\blue}$,
        to $\mat q$, in any order, yields $\mat p$. Moreover, this vector is unique.

        % In general, it will turn out that .

        % By the Radon-Nikodim Theorem, there if $\mu \ll \nu$, there is a measure $\frac{\mathrm d \mu}{\mathrm d\nu}$ such that $\int_{A} \frac{\mathrm d \mu}{\mathrm d\nu} \mathrm d \nu = \mu(A)$
    \end{enumerate}

\end{examplex}

\subsection{Basic Axioms for Update Rules}

We now detail some other properties we might want an update rule to have.

\textbf{Differentiability}.
% A primary goal of this paper is to study how updates are made in low-confidence settings.
Insofar as confidence is meant to smoothly interpolating between fully incorporating information and ignoring it, we would like the interpolation to be differentiable.

Here are two ways of capturing this, depending on the structure one has in hand.


\begin{URaxioms}
    \item
    \begin{enumerate}
    \item If $\Theta$ itself has differentiable structure, we require that
        for fixed $\theta$ and $\phi$, the function $\beta \mapsto F^{\beta}_\phi(\theta)$
        is continuously differentiable. \label{ax:diffble}
    \item If $\Theta$ parameterizes a family of probabilities over $(\X, \mathcal A)$,
        via $\{ \Pr_\theta \}_{\theta \in \Theta}$, we can avoid talking abot a differentiable structure on $\Theta$ by simply requiring that the update rule be differentiable from the perspective of every event $A \in \mathcal A$.
        That is,
        %  is a family of probability distributions.
        for all $\theta \in \Theta$, $\phi \in \Phi$, and  $A \in \mathcal A$,
        the function $\beta \mapsto \Pr_{F^{\beta}_\phi(\theta)}(A)
        : \mathbb R_{\ge 0} \to \mathbb [0,1]$ is
        continuously differentiable.
        %(in pairs $(\beta,\Pr)$).
            % \hfill \textbf{(differentiability)}
            \label{ax:diffble2}
    \end{enumerate}
    \hfill \textbf{(differentiability)}
\end{URaxioms}

% Suppose that the space $\Theta$ is actually a differentiable manifold.
% In this case, we might want want $F$ to be compatible with the differentiable structure.
% \begin{URaxioms}
%     \item $\Theta$ is a differentiable manifold.
%         For fixed $\theta$ and $\phi$, the function $\beta \mapsto F^{\beta}_\phi(\theta)$
%         is continuously differentiable.
%             \hfill \textbf{(differentiability)} \label{ax:diffble2}
% \end{URaxioms}



Some update rules, such as the one in \cref{ex:colorballs} have a particularly conveneint property: the result of applying many updates does not depend on the order in which the information arives.

\begin{URaxioms}
    \item For all $\phi_1, \phi_2 \in \Phi$ and $\beta_1, \beta_2 \in \mathbb R$, and
    % $\mu \in \Delta\X$,
    $\theta \in \Theta$,
    we have that
    $
        F^{\beta_1}_{\phi_1} ( F^{\beta_2}_{\phi_2}(\theta)) =
            F^{\beta_2}_{\phi_2} ( F^{\beta_1}_{\phi_1}(\theta)).
    $
    \hfill\textbf{(Commutativity)} \label{ax:commute}
\end{URaxioms}

We will not always want to insist on commutativity. Human belief updates, for example, are notoriously non-commutative, in part due to confirmation bias:
if you are already fairly certain that $P$ is false, you are likely to disregard
information that $P$ is true. Thus earlier updates tend to be more impactful.%
\footnote{
    It is also possible to model this effect with a commutative update rule,
    by expressing reduced confidence in later inputs.
}


We would also like update rules to preserve any symmetries shared by the state space $\X$ and the assertion language $\Phi$, so that updates are not sensitive to irrelevant labelings of points.
% Concretely, let $\mathrm{Aut}(X, \Phi)$ be the set of automorphisms $\sigma : X \to X$, together with an action on assertions, so that $\sigma\phi \in \Phi$ is the appropriately relabeled assertion equivalent to $\phi$ after the relabeling.
Concretely, let $\mathrm{Aut}(\Theta, \Phi)$ be the set of automorphisms $\sigma : \Theta \to \Theta$ (say, rotations of the simplex of distributions), that also have an associated action on assertions, so that $\sigma\phi \in \Phi$ is the corresponding relabeling of $\phi$ under $\sigma$.  The symmetry condition can now be captured by:

\begin{URaxioms}
    \item
        % All symmetries of $\X$ are reflected in the update rule.
        % Concretely, for all  $\sigma
        For all $\sigma
            % : X \to X
            % \in \mathrm{Aut}(X, \Phi)$, we have
            \in \mathrm{Aut}(\Theta, \Phi)$, we have
        % \\ \indent\hspace{2em}
        % $F^\beta_A(\sigma_\#(\Pr)) = \sigma_\#\Big(F^\beta_{\{\sigma(a) : a \in A \}}(\Pr)\Big)$,
        % $F^\beta_\phi (\sigma_\#(\Pr)) = \sigma_\#\Big(F^\beta_{\sigma\phi}(\Pr)\Big)$,
        $F^\beta_{\sigma\phi} (\sigma(\theta)) = \sigma \Big( F^\beta_{\phi}(\Pr)\Big)$.
            \hfill \textbf{(symmetry)} \label{ax:symmetry}
        \\
        % where $\sigma_\#(\Pr)$ is the pushforward of $\Pr$ under $\sigma$.
\end{URaxioms}

Intuitively, this axiom states that doing an update is equivalent to changing to an equivalent representaton, doing the appropriately transformed update, and then transforming back.



\section{}
\subsection{Case Study: Jeffrey's Rule}
\subsection{Mixture of Experts}

\subsection{Weight of Evidence}
\subsection{Loss Functions}
\subsection{Bolzman Rationality}
% Loss Functions
% Total \# of Updates
\section{Confidence in PDGs}
\subsection{}


\end{document}
