\documentclass{article}

\relax % Controls
    \newif\ifmarginprooflinks
    	\marginprooflinkstrue
    	% \marginprooflinksfalse

\relax % Bibliography, etc
	\usepackage[american]{babel}
	\usepackage{csquotes}
	\usepackage[backend=biber, style=authoryear]{biblatex}
	\DeclareLanguageMapping{american}{american-apa}
	% \usepackage[backend=biber,style=authoryear,hyperref=true]{biblatex}
	% \addbibresource{refs.bib}
	% \addbibresource{conf.bib}

	\DeclareFieldFormat{citehyperref}{%
	  \DeclareFieldAlias{bibhyperref}{noformat}% Avoid nested links
	  \bibhyperref{#1}}

	\DeclareFieldFormat{textcitehyperref}{%
	  \DeclareFieldAlias{bibhyperref}{noformat}% Avoid nested links
	  \bibhyperref{%
	    #1%
	    \ifbool{cbx:parens}
	      {\bibcloseparen\global\boolfalse{cbx:parens}}
	      {}}}

	\savebibmacro{cite}
	\savebibmacro{textcite}

	\renewbibmacro*{cite}{%
	  \printtext[citehyperref]{%
	    \restorebibmacro{cite}%
	    \usebibmacro{cite}}}

	\renewbibmacro*{textcite}{%
	  \ifboolexpr{
	    ( not test {\iffieldundef{prenote}} and
	      test {\ifnumequal{\value{citecount}}{1}} )
	    or
	    ( not test {\iffieldundef{postnote}} and
	      test {\ifnumequal{\value{citecount}}{\value{citetotal}}} )
	  }
	    {\DeclareFieldAlias{textcitehyperref}{noformat}}
	    {}%
	  \printtext[textcitehyperref]{%
	    \restorebibmacro{textcite}%
	    \usebibmacro{textcite}}}

	\DeclareCiteCommand{\brakcite}
	  {\usebibmacro{prenote}}
	  {\usebibmacro{citeindex}%
	   \printtext[bibhyperref]{[\usebibmacro{cite}]}}
	  {\multicitedelim}
	  {\usebibmacro{postnote}}

\relax % Standard Packages
    \usepackage[dvipsnames]{xcolor}
    % \usepackage[utf8]{inputenc}
    \usepackage{mathtools}
    \usepackage{amssymb}
		\DeclareMathSymbol{\shortminus}{\mathbin}{AMSa}{"39}
    % \usepackage{parskip}
    % \usepackage{algorithm}
    \usepackage{bbm}
	\usepackage{lmodern}
	% \usepackage{times}
    \usepackage{faktor}
    % \usepackage{booktabs}
	% \usepackage[margin=1in]{geometry}
    \usepackage{graphicx}
    \usepackage{scalerel}
    \usepackage{enumitem}
    \usepackage{nicefrac}\let\nf\nicefrac

    % \usepackage{color}
    %\usepackage{stmaryrd}
    \usepackage{hyperref} % Load before theorems...
        \hypersetup{colorlinks=true, linkcolor=blue!75!black, urlcolor=magenta, citecolor=green!50!black}

\usepackage{tikz}
	\usetikzlibrary{positioning,fit,calc, decorations, arrows, shapes, shapes.geometric}
	\usetikzlibrary{cd}

	%%%%%%%%%%%%
	\tikzset{AmpRep/.style={ampersand replacement=\&}}
	\tikzset{center base/.style={baseline={([yshift=-.8ex]current bounding box.center)}}}
	\tikzset{paperfig/.style={center base,scale=0.9, every node/.style={transform shape}}}

	% Node Stylings
	\tikzset{dpadded/.style={rounded corners=2, inner sep=0.7em, draw, outer sep=0.3em, fill={black!50}, fill opacity=0.08, text opacity=1}}
	\tikzset{dpad0/.style={outer sep=0.05em, inner sep=0.3em, draw=gray!75, rounded corners=4, fill=black!08, fill opacity=1, align=center}}
	\tikzset{dpadinline/.style={outer sep=0.05em, inner sep=2.5pt, rounded corners=2.5pt, draw=gray!75, fill=black!08, fill opacity=1, align=center, font=\small}}

 	\tikzset{dpad/.style args={#1}{every matrix/.append style={nodes={dpadded, #1}}}}
	\tikzset{light pad/.style={outer sep=0.2em, inner sep=0.5em, draw=gray!50}}

	\tikzset{arr/.style={draw, ->, thick, shorten <=3pt, shorten >=3pt}}
	\tikzset{arr0/.style={draw, ->, thick, shorten <=0pt, shorten >=0pt}}
	\tikzset{arr1/.style={draw, ->, thick, shorten <=1pt, shorten >=1pt}}
	\tikzset{arr2/.style={draw, ->, thick, shorten <=2pt, shorten >=2pt}}

	\newcommand\cmergearr[5][]{
		\draw[arr, #1, -] (#2) -- (#5) -- (#3);
		\draw[arr, #1, shorten <=0] (#5) -- (#4);
		}
	\newcommand\mergearr[4][]{
		\coordinate (center-#2#3#4) at (barycentric cs:#2=1,#3=1,#4=1.2);
		\cmergearr[#1]{#2}{#3}{#4}{center-#2#3#4}
		}
	\newcommand\cunmergearr[5][]{
		\draw[arr, #1, -, shorten >=0] (#2) -- (#5);
		\draw[arr, #1, shorten <=0] (#5) -- (#3);
		\draw[arr, #1, shorten <=0] (#5) -- (#4);
		}
	\newcommand\unmergearr[4][]{
		\coordinate (center-#2#3#4) at (barycentric cs:#2=1.2,#3=1,#4=1);
		\cunmergearr[#1]{#2}{#3}{#4}{center-#2#3#4}
		}

\usepackage{amsthm,thmtools} % Theorem Macros
	\usepackage[noabbrev,nameinlink,capitalize]{cleveref}
    \theoremstyle{plain}
    \newtheorem{theorem}{Theorem}
	\newtheorem{coro}{Corollary}[theorem]
    \newtheorem{prop}[theorem]{Proposition}
    \newtheorem{conj}[theorem]{Conjecture}
    \newtheorem{claim}{Claim}
    \newtheorem{remark}{Remark}
    \newtheorem{lemma}[theorem]{Lemma}
    \theoremstyle{definition}
    % \newtheorem{defn}{Definition}
    % \declaretheorem[name=Definition]{defn}
    \declaretheorem[name=Definition, qed=$\square$]{defn}
    \declaretheorem[name=Example, qed=$\triangle$]{example}

	\crefname{defn}{Definition}{Definitions}
	\crefname{prop}{Proposition}{Propositions}
    \crefname{issue}{Issue}{Issues}

\relax %%%%%%%%% GENERAL MACROS %%%%%%%%
    \let\Horig\H
	\let\H\relax
	\DeclareMathOperator{\H}{\mathrm{H}} % Entropy
	\DeclareMathOperator{\I}{\mathrm{I}} % Information
	\DeclareMathOperator*{\Ex}{\mathbb{E}} % Expectation
	\DeclareMathOperator*{\EX}{\scalebox{1.5}{$\mathbb{E}$}}

    \newcommand{\mat}[1]{\mathbf{#1}}
    \DeclarePairedDelimiterX{\infdivx}[2]{(}{)}{%
		#1\;\delimsize\|\;#2%
	}
	\newcommand{\thickD}{I\mkern-8muD}
	\newcommand{\kldiv}{\thickD\infdivx}
	\newcommand{\tto}{\rightarrow\mathrel{\mspace{-15mu}}\rightarrow}

	\newcommand{\datadist}[1]{\Pr\nolimits_{#1}}
	% \newcommand{\datadist}[1]{p_\text{data}}

	\makeatletter
	\newcommand{\subalign}[1]{%
	  \vcenter{%
	    \Let@ \restore@math@cr \default@tag
	    \baselineskip\fontdimen10 \scriptfont\tw@
	    \advance\baselineskip\fontdimen12 \scriptfont\tw@
	    \lineskip\thr@@\fontdimen8 \scriptfont\thr@@
	    \lineskiplimit\lineskip
	    \ialign{\hfil$\m@th\scriptstyle##$&$\m@th\scriptstyle{}##$\hfil\crcr
	      #1\crcr
	    }%
	  }%
	}
	\makeatother
	\newcommand\numberthis{\addtocounter{equation}{1}\tag{\theequation}}

\relax %%%%%%%%%   PDG  MACROS   %%%%%%%%
	\newcommand{\ssub}[1]{_{\!_{#1}\!}}
	% \newcommand{\bp}[1][L]{\mat{p}_{\!_{#1}\!}}
	% \newcommand{\bP}[1][L]{\mat{P}_{\!_{#1}\!}}
	\newcommand{\bp}[1][L]{\mat{p}\ssub{#1}}
	\newcommand{\bP}[1][L]{\mat{P}\ssub{#1}}
	\newcommand{\V}{\mathcal V}
	\newcommand{\N}{\mathcal N}
	\newcommand{\Ed}{\mathcal E}

    \newcommand{\balpha}{\boldsymbol\alpha}
    \newcommand{\bbeta}{\boldsymbol\beta}

	\DeclareMathAlphabet{\mathdcal}{U}{dutchcal}{m}{n}
	\DeclareMathAlphabet{\mathbdcal}{U}{dutchcal}{b}{n}
	\newcommand{\dg}[1]{\mathbdcal{#1}}
	\newcommand{\PDGof}[1]{{\dg M}_{#1}}
	\newcommand{\UPDGof}[1]{{\dg N}_{#1}}
	\newcommand\VFE{\mathit{V\mkern-4mu F\mkern-4.5mu E}}

	\newcommand\Inc{\mathit{Inc}}
	\newcommand{\IDef}[1]{\mathit{IDef}_{\!#1}}
	% \newcommand{\ed}[3]{%
	% 	\mathchoice%
	% 	{#2\overset{\smash{\mskip-5mu\raisebox{-3pt}{${#1}$}}}{\xrightarrow{\hphantom{\scriptstyle {#1}}}} #3} %display style
	% 	{#2\overset{\smash{\mskip-5mu\raisebox{-3pt}{$\scriptstyle {#1}$}}}{\xrightarrow{\hphantom{\scriptstyle {#1}}}} #3}% text style
	% 	{#2\overset{\smash{\mskip-5mu\raisebox{-3pt}{$\scriptscriptstyle {#1}$}}}{\xrightarrow{\hphantom{\scriptscriptstyle {#1}}}} #3} %script style
	% 	{#2\overset{\smash{\mskip-5mu\raisebox{-3pt}{$\scriptscriptstyle {#1}$}}}{\xrightarrow{\hphantom{\scriptscriptstyle {#1}}}} #3}} %scriptscriptstyle
	\newcommand{\ed}[3]{#2%
	  \overset{\smash{\mskip-5mu\raisebox{-1pt}{$\scriptscriptstyle
	        #1$}}}{\rightarrow} #3}

    \newcommand{\nhphantom}[2]{\sbox0{\kern-2%
		\nulldelimiterspace$\left.\delimsize#1\vphantom{#2}\right.$}\hspace{-.97\wd0}}
		% \nulldelimiterspace$\left.\delimsize#1%
		% \vrule depth\dp#2 height \ht#2 width0pt\right.$}\hspace{-.97\wd0}}
	\makeatletter
	\newsavebox{\abcmycontentbox}
	\newcommand\DeclareDoubleDelim[5]{
	    \DeclarePairedDelimiterXPP{#1}[1]%
			{% box must be saved in this pre code
				\sbox{\abcmycontentbox}{\ensuremath{##1}}%
			}{#2}{#5}{}%
		    %%% Correct spacing, but doesn't work with externalize.
			% {\nhphantom{#3}{##1}\hspace{1.2pt}\delimsize#3\mathopen{}##1\mathclose{}\delimsize#4\hspace{1.2pt}\nhphantom{#4}{##1}}
			%%% Fast, but wrong spacing.
			% {\nhphantom{#3}{~}\hspace{1.2pt}\delimsize#3\mathopen{}##1\mathclose{}\delimsize#4\hspace{1.2pt}\nhphantom{#4}{~}}
			%%% with savebox.
		    {%
				\nhphantom{#3}{\usebox\abcmycontentbox}%
				\hspace{1.2pt} \delimsize#3%
				\mathopen{}\usebox{\abcmycontentbox}\mathclose{}%
				\delimsize#4\hspace{1.2pt}%
				\nhphantom{#4}{\usebox\abcmycontentbox}%
			}%
	}
	\makeatother
	\DeclareDoubleDelim
		\SD\{\{\}\}
	\DeclareDoubleDelim
		\bbr[[]]
	% \DeclareDoubleDelim
	% 	\aar\langle\langle\rangle\rangle
	\makeatletter
	\newsavebox{\aar@content}
	\newcommand\aar{\@ifstar\aar@one@star\aar@plain}
	\newcommand\aar@one@star{\@ifstar\aar@resize{\aar@plain*}}
	\newcommand\aar@resize[1]{\sbox{\aar@content}{#1}\scaleleftright[3.8ex]
		{\Biggl\langle\!\!\!\!\Biggl\langle}{\usebox{\aar@content}}
		{\Biggr\rangle\!\!\!\!\Biggr\rangle}}
	\DeclareDoubleDelim
		\aar@plain\langle\langle\rangle\rangle
	\makeatother


	% \DeclarePairedDelimiterX{\aar}[1]{\langle}{\rangle}
	% 	{\nhphantom{\langle}{#1}\hspace{1.2pt}\delimsize\langle\mathopen{}#1\mathclose{}\delimsize\rangle\hspace{1.2pt}\nhphantom{\rangle}{#1}}

\relax %%%%% restatables and links
	% \usepackage{xstring} % for expandarg
	\usepackage{xpatch}
	\makeatletter
	\xpatchcmd{\thmt@restatable}% Edit \thmt@restatable
	   {\csname #2\@xa\endcsname\ifx\@nx#1\@nx\else[{#1}]\fi}% Replace this code
	   % {\ifthmt@thisistheone\csname #2\@xa\endcsname\typeout{oiii[#1;#2\@xa;#3;\csname thmt@stored@#3\endcsname]}\ifx\@nx#1\@nx\else[#1]\fi\else\csname #2\@xa\endcsname\fi}% with this code
	   {\ifthmt@thisistheone\csname #2\@xa\endcsname\ifx\@nx#1\@nx\else[{#1}]\fi
	   \else\fi}
	   {}{\typeout{FIRST PATCH TO THM RESTATE FAILED}} % execute on success/failure
	\xpatchcmd{\thmt@restatable}% A second edit to \thmt@restatable
	   {\csname end#2\endcsname}
	   {\ifthmt@thisistheone\csname end#2\endcsname\else\fi}
	   {}{\typeout{FAILED SECOND THMT RESTATE PATCH}}

	% \def\onlyaftercolon#1:#2{#2}
	\newcommand{\recall}[1]{\medskip\par\noindent{\bf \Cref{thmt@@#1}.} \begingroup\em \noindent
	   \expandafter\csname#1\endcsname* \endgroup\par\smallskip}

   	\setlength\marginparwidth{1.55cm}
	\newenvironment{linked}[3][]{%
		\def\linkedproof{#3}%
		\def\linkedtype{#2}%
		% \reversemarginpar
		% \marginpar{%
		% \vspace{1.1em}
		% % \hspace{2em}
		% 	% \raggedleft
		% 	\raggedright
		% 	\hyperref[proof:\linkedproof]{%
		% 	\color{blue!50!white}
		% 	\scaleleftright{$\Big[$}{\,{\small\raggedleft\tt\begin{tabular}{@{}c@{}} proof of \\\linkedtype~\ref*{\linkedtype:\linkedproof}\end{tabular}}\,}{$\Big]$}}
		% 	}%
        % \restatable[#1]{#2}{#2:#3}\label{#2:#3}%
		\ifmarginprooflinks
		\marginpar{%
			% \vspace{-3em}% %% for bottom
			\vspace{1.5em}
			\centering%
			\hyperref[proof:\linkedproof]{%
            % \hyperref[proof:#3]{
			\color{blue!30!white}%
			\scaleleftright{$\Big[$}{\,\mbox{\footnotesize\centering\tt\begin{tabular}{@{}c@{}}
				% proof of \\\,\linkedtype~\ref*{\linkedtype:\linkedproof}
				link to\\[-0.15em]
				proof
			\end{tabular}}\,}{$\Big]$}}~
			}%
		\fi
        \restatable[#1]{#2}{#2:#3}\label{#2:#3}%
        }%
		{\endrestatable%
		}
	\makeatother
		\newcounter{proofcntr}
		\newenvironment{lproof}{\begin{proof}\refstepcounter{proofcntr}}{\end{proof}}

		\usepackage{cancel}
		\newcommand{\Cancel}[2][black]{{\color{#1}\cancel{\color{black}#2}}}

		\usepackage{tcolorbox}
		\tcbuselibrary{most}
		\tcolorboxenvironment{lproof}{
			% fonttitle=\bfseries,
			% top=0.5em,
			enhanced,
			parbox=false,
			boxrule=0pt,
			frame hidden,
			borderline west={4pt}{0pt}{blue!20!black!40!white},
			% coltext={blue!20!black!60!white},
			colback={blue!20!black!05!white},
			sharp corners,
			breakable,
			% bottomsep at break=4cm,
			% enlarge bottom at break by=-4cm,
			% topsep at break=3cm,
			% enlarge top at break by=-3cm
		}
		% \usepackage[framemethod=TikZ]{mdframed}
		% \surroundwithmdframed[ % lproof
		% 	   topline=false,
		% 	   linewidth=3pt,
		% 	   linecolor=gray!20!white,
		% 	   rightline=false,
		% 	   bottomline=false,
		% 	   leftmargin=0pt,
		% 	   % innerleftmargin=5pt,
		% 	   skipabove=\medskipamount,
		% 	   skipbelow=\medskipamount
		% 	]{lproof}
	%oli16: The extra space was because there was extra space in the paragraph, not
	%because this length was too big. By breaking arrays, everything will be better.
	\newcommand{\begthm}[3][]{\begin{#2}[{name=#1},restate=#3,label=#3]}

\relax %TODOs and footnotes
    \newcommand{\TODO}[1][INCOMPLETE]{{\centering\Large\color{red}$\langle$~\texttt{#1}~$\rangle$\par}}
    \newcommand{\dfootnote}[1]{%
        \let\oldthefootnote=\thefootnote%
        % \addtocounter{footnote}{-1}%
		\setcounter{footnote}{999}
        \renewcommand{\thefootnote}{\textdagger}%
        \footnote{#1}%
        \let\thefootnote=\oldthefootnote%
    }
	\newcommand{\dfootnotemark}{
		\footnotemark[999]
	}


% \usepackage[framemethod=TikZ]{mdframed}
% \colorlet{color1}{Emerald}
% \colorlet{color3}{color1>wheel,2,3}
% \colorlet{pinkish}{color3!25!magenta}
\definecolor{brownish}{rgb}{0.5, 0.2, 0.1}
% \newmdenv[roundcorner=5pt, 
%     backgroundcolor=brownish!20!white,
%     % frametitle={$\langle$under construction$\rangle$},
%     frametitle={$\langle$incomplete or buggy$\rangle$},
%     frametitlerule=false,
%     innertopmargin=3pt, frametitlebelowskip=1ex, frametitleaboveskip=1ex,
%     frametitlebackgroundcolor=brownish!40!white,
%     skipabove=1em,skipbelow=1em,
%     frametitlefont={\scshape},leftmargin=-10pt, rightmargin=-10pt]
% 		{wip}
\newtcolorbox{wip}{%
    colback=brownish!20!white,%
    % frametitle={$\langle$under construction$\rangle$},
    title={$\langle$under construction$\rangle$},%
    enhanced jigsaw,
    breakable,
    % frametitlerule=false,
    % innertopmargin=3pt, frametitlebelowskip=1ex, frametitleaboveskip=1ex,
    colframe=brownish!40!white,%
    % skipabove=1em,skipbelow=1em,
    % frametitlefont={\scshape},leftmargin=-10pt, rightmargin=-10pt
}
        
% \tcolorboxenvironment{example}{
\newtcolorbox[use counter=example]{examplex}{
        fonttitle=\bfseries,
        % empty,
        enhanced jigsaw,
        title={Example \thetcbcounter.},
        before=\par\medskip\noindent,
        % top=0.5em,
        % enhanced,
        parbox=false,
        % boxrule=0pt,
        % frame hidden,
        % borderline west={4pt}{0pt}{green!20!black!40!white},
        % coltext={blue!20!black!60!white},
        % colback={blue!20!black!05!white},
        colback=white,
        sharp corners,
        breakable,
        % bottomsep at break=4cm,
        % enlarge bottom at break by=-4cm,
        % topsep at break=3cm,
        % enlarge top at break by=-3cm
    }


\usepackage[margin=1in]{geometry}
\usepackage{parskip}


\DeclareMathOperator{\supp}{\mathrm{Supp}}

\begin{document}





\section{General Thoughts}
Confidence is an important aspect of knowledge representation.

% The notion of confidence also has
% Confidence is the opposite of uncertainty: you cannot be uncertain about $X$ .
Confidence is often thought of as the opposite of uncertainty.


% Confidence is dual to information.
One has confidence in information.
Confidence and Information are thermodynamically ``conjugate variables''.


``cool-headed'' means calculated (a vote of confidence) while ``hot-headed'' means rash (such an assessment indicates a lack of confidence). Confidence is like inverse temperature.

Can you can be confident that $X$ is uncertain?
There is a huge difference between being certain that a coin has is fair, and knowing nothing about it.


% Here we draw a distinction between confidence and certainty:
% while you are certain that [proposition],

Confidence is intimately related to probability, and our development will largely be couched in (higher-order) probabilistic terms.
Nevertheless, we submit that in many cases---and especially in more subjective or computationally restricted settings---probability alone is not enough, and the higher-order probabilistic picture is far too extreme.




% We want to be able to entertain and reason about models we aren't sure about.

\subsection*{Our Approach}
% Here is the idea: if you are confident in $X$, then you are
The idea is to view confidence as a property of an update, not as a property of a point of view.
In this telling
This allows us to distinguish between ...

\begin{example}
Let $X$ be a binary random variable, and suppose you have a prior probability $p$ that $X=1$.
\begin{enumerate}
    \item Suppose $p=1$.
    This is a poor choice of prior.
    If you then recieve information that $X=0$, something has gone very wrong, and the Bayesian update is undefined.

    If you rceive information that $X = 1$, your internal state should not change.
    % If we characterize

    \item One might argue that only positive probabilities are relevant.
    Suppose $p = 1 - \epsilon$.
\end{enumerate}
\end{example}


\subsection*{Issues To Address}
\begin{enumerate}
    \item The difference between having confidence in \emph{a source} and confidence in a particular \emph{fact}.  Is one more general than the other?
\end{enumerate}

Some intuitive features we might want to capture:
\begin{enumerate}

    \item \textbf{Trust Dynamics.} If a trusted source tells you something you already believe, your confidence in them goes up (maybe). Certainly if a source tells you something you know to be wrong, your confidence in the source goes down.  Whether or not you adopt the

    Also, if you ultimately end up adopting the belief, and later end upwith a more coherent picture of the world, your trust in the source should go way up.
    This is because in the end we want to place the most trust in sources that tell you the truth, not what you expect (or want) to hear.

    A source that tells you only exactly what you already believe ``artificially'' increases your confidence but does not actually provide you any information, unless they came to hold the same views independently.

    \item
    \item
\end{enumerate}

\section{Update Rules}
\def\X{\mathcal X}
Let $\X = (X, \mathcal A)$ be a measurable space, so that $X$ is a set and $\mathcal A$ is a $\sigma$-algebra over $X$, and let $\Delta \X$ denote the set of probability measures over $\X$.

Suppose further that we have a ``divergence'' function $d : X \times X \to \mathbb R^+$ on $X$, with the property that $d(x,y) = 0$ iff $x = y$.
For sets $A \subset X$, let $d(x, A) := \inf_{a \in A} d(x,a)$ be the smallest possible divergence between $x$ and any member of $A$; symmetrically, define $d(A, x) := \inf_{a \in A} d(a,x)$.

\begin{defn}
For $A,B,Z \subset X$,
we say that $A$ and $B$ are $d$-independent given $Z$ if
% $d(A, b) = d(a, B)$ for all $a\in A$ and $b \in B$.
for all $z \in Z$, we have that
$d(z, A \cap B) = d(z, A) + d(z,B)$.
\end{defn}

We are interested in updating rules
\[
    F: (\text{Confidence} : \mathbb R) \times (\text{Event} : \mathcal A) \to (\Delta\X  \to \Delta \X)
\]
which describe how to update beliefs about $X$, with the new information, at a certain level of trust. We write $F^\beta_A : \Delta\X \to \Delta X$ for the update function for a given piece of information $A$ at confidence $\beta$.

We would like update rules to satisfy the following axioms:

For all $A \in \mathcal A$, and all $\beta,\beta_1, \beta_2 \in \mathbb R$
\begin{enumerate}[label=UR\arabic{*}.,nosep]
    % \item \textbf{(zero)} $F^{0}_A(\Pr) = (\Pr)$
    \item  $F^{0}_A  =  \mathrm{Id}_{\Delta\X}$. (That is, $F^{0}_A(\Pr) = (\Pr)$ for all $\Pr \in \Delta\X$.)
        \hfill \textbf{(zero)}
    \item $F^{\beta_1}_A \circ F^{\beta_2}_A = F^{\beta_1 + \beta_2}_A$
        \hfill \textbf{(additivity)}
    \item $\displaystyle \lim_{\beta\to\infty} F^\beta_A (\Pr) = \Pr|A$
        \hfill \textbf{(absolute certainty)}
\end{enumerate}

\begin{enumerate}[resume,label=UR\arabic{*}.]
    \item If $A$ and $B$ are independent, then $F^{\beta}_A \circ F^{\beta}_B = F^{\beta}_{A \cap B}$.
        \hfill \textbf{(decomposition)}
\end{enumerate}

We can also consider the weaker variant
\begin{enumerate}
    % \item[U3$'$.]  \textbf{(effectiveness)~} $\supp F^\infty_A (\Pr) \subset A $
    \item[U3$'$.]  \textbf{(effectiveness)~} $F^\infty_A (\Pr)(A) = 1$

\end{enumerate}


\subsection{The Cannoncial Update Rules}
The Canonical Update Rule, for a measurable space $\mathcal X = (X, \mathcal A)$ and divergence measure $d: X \times X \to \mathbb R^+$ is given by
\begin{align*}
    F^\beta_A(\Pr)(x) &\propto \Pr(x) \cdot \exp(-\beta d(x, A)) \\
    F^\beta_A(\Pr) &= B \mapsto \frac{1}{\Ex_{\Pr}[\exp(-\beta d(x,A))]}\int
        1_B(x) \; \exp(-\beta d(x, A)) \,\mathrm d\Pr(x)
    %
\end{align*}
\begin{prop}
    The cannonical update rule satisfies UR1-4.
\end{prop}



\subsection{Flow, instead of Decay}
\def\vgrad{\boldsymbol\nabla}
% \def\vgrad{\vec\nabla}
The update rule above only ``moves'' distributions by decay. If we imagine a distribution representing a population of individuals, the distribution has shifted by selection: an application of the update rule $F_A^\beta$ corresponds to culling the population, in proportion to their distance from $A$, over a time period of length $\beta$.
It is worth noting that such a population is then smaller, and must be re-normalized later.

A distribution could also shift by genuine motion of the underlying individuals.
In this case, let's model this as a partial differential equation in terms of $\Pr(x,t)$.
Now, we have some kind of conservation of mass, so from the continuity equation, we get
$\frac{\partial }{\partial t}Pr(x,t) = - \vgrad \cdot \mat J$,
where $\mat J(x,t)$ is the flux of individuals at point $x$ and time $t$.
Further assuming that the flux is generated by a combination of diffusion and a potential proportional to distance to $A$, we obtain

\[
    \frac{\partial \Pr(x, t)}{\partial t} = \vgrad_{\!x} \cdot \Big( k \vgrad_{\!x} \Pr (x) - \beta \vgrad_{\!x} d(x, A) \Big)
\]
where $k$ is a diffusion coefficient, and $\beta$

% To draw a different analogy, we can think of




\section{Examples}
\subsection{Gaussians}
Consider the case where $\X$ is the set of real numbers with the Borell $\sigma$-algebra, and $d(x,y) := (x-y)^2$.
Then the cannoncial update rule corresponds to multiplying by a Gaussian density, whose variance is $\nicefrac1\beta$.

Furthermore, applying it to a distribution which is already Gaussian with mean $\mu$ and variance $\sigma^2$, we find that
\begin{align*}
    F^{\beta}_{X=b}(\mathcal N(\mu, \sigma^2))(x) &\propto
        \exp\left\{ - \frac12 \frac{(x-\mu)^2}{ \sigma^2 } - \beta(x-b)^2\right\}
    \\&\propto \mathcal N(x| \mu', \sigma'^2)
\end{align*}
which is itself a Gaussian with mean equal to the weighted average of $\mu$ and $\beta$.

The upshot is that


\subsection{Higher Order Probability Measures}
Now, suppose $\X$ is itself a set or probability measures.

Consider the following ways of extracting an (ordinary) probability measure $\mu(\Omega)$
from a higher order probability $\Pr(\mu(\Omega))$.

\begin{enumerate}
    \item \textbf{Centroid.}
        % $\mu^* := \textit{centroid}(\Pr)$.\\
        On a measurable set $A \in \mathcal A$, the centroid of $\Pr$ is defined as
        \[
            \mu^{\text{avg}}(A) := \Ex_{\mu \sim \Pr} \mu(A) =
                \iint \Pr(\mu) \mu(x) 1_{A}(x)\, \mathrm d x\,\mathrm d\mu,
        \]
        the centroid is the usual ``flattening'' of a higher-order probability distribution, and corresponds to
        of the Giry Monad.

    \item \textbf{Highest Likelihood.}
        $\textit{MLE}(\Pr) := A \mapsto (\arg\max_\mu \Pr(\mu))(A)$

        This is the way that people often do inference.

    \item \textbf{MAP Inference.}
        Suppose we have a prior $\Pr_0$ over  $\Pr$

\end{enumerate}


\section{PDGs}

Given a PDG $\dg M$, let $\X := \Delta\V(\dg M) = \Delta(\prod_{X\in\N}\V(X))$. For each edge we have
\[
    F_L^{\beta}(\Pr) (\mu) \propto \Pr(\mu) \exp(-\beta \kldiv\mu \bp)
\]
This extends to
\[
    F_{\dg M}^k(\Pr)(\mu) \propto \Pr(\mu) \exp(- k \bbr{\dg M}_0(\mu))
\]
Note that
\begin{align*}
    \lim_{k \to \infty} F_{\dg M}^k(\Pr)(\mu) &\propto
        % \mathbbm1[\mu \in {\displaystyle\SD{\dg M}}]
        \Pr(\mu) \mathbbm1[\mu \in {\bbr{\dg M}^*_0}] \\
        &= \Pr \,|\, \bbr{\dg M}^*_0 \\
        &= \Pr \,|\, \SD{\dg M} \text{~if $\dg M$ is consistent.}
        % &= \mathrm{Unif}_{\{\!\!\{{\dg M}\}\!\!\}} \text{~if $\dg M$ is consistent.}
\end{align*}

On the other hand,
\begin{align*}
    \lim_{k \to 0} F_{\dg M}^k(\Pr)(\mu) &= \Pr \,|\, \Inc(\mu) < \infty \\
        &= \Pr \,|\, \{\mu : \forall L.~\mu(Y|X) \ll \bp \},
\end{align*}
while $F_{\dg M}^0(\Pr) = \Pr$. So $F$ can only be continuous at $k=0$ for $\Pr$ if, for all edges $L \in \Ed$, we have that $\Pr(\mu(Y|X) \ll \bp) = 1$---that is, if $\mu$ is absolutely continuous with respect to $\bp$.




\subsection{The Qualitative Half}
We need a different distance function for the qualitative loss function of a PDG.


Why measure distance differently in this case?


\section{Epistemic Entrenchment and Confidence}
\section{Weighted Probability Measures}



\end{document}
