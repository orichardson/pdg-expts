\documentclass{article}

\usepackage{mathtools,amssymb}
\usepackage{enumitem}
\usepackage{tikz-cd}

\usepackage{tcolorbox}
    \tcbuselibrary{most}
    \newtcolorbox{phaseout}{%
        colback={gray!02!white},
        coltext={gray!35!white},
        % colframe={red!02!white},
        % coltitle={red!35!white},
        % title={~\hfill(depricated)},
        enhanced jigsaw,
        % enhancd,
        fontupper=\small,
        parbox=false,
        boxrule=0pt,
        frame hidden,
        borderline west={4pt}{0pt}{blue!20!black!10!white},
        borderline east={4pt}{0pt}{blue!20!black!10!white},
        sharp corners,
        breakable
        % frametitlerule=false,
        % innertopmargin=3pt, frametitlebelowskip=1ex, frametitleaboveskip=1ex,
        % colframe=brownish!40!white,%
        % skipabove=1em,skipbelow=1em,
        % frametitlefont={\scshape},leftmargin=-10pt, rightmargin=-10pt
        }

\DeclareMathAlphabet{\mathdcal}{U}{dutchcal}{m}{n}
\DeclareMathAlphabet{\mathbdcal}{U}{dutchcal}{b}{n}

% \newcommand\confdom{\mathbdcal{X}}
\newcommand\confdom{\mathbdcal{C}}

\relax % Categorical Macros
    \DeclareMathOperator{\ob}{\mathrm{ob}}
    \DeclareMathOperator{\id}{id}
    \DeclareMathOperator{\Hom}{\mathrm{Hom}}
    
    % Caligraphic Symbols for Categories
    \newcommand\X{\mathcal{X}}
    \newcommand\A{\mathcal{A}}
    \newcommand\B{\mathcal{B}}
    \newcommand\Ob{\mathcal{O}}

    % Predefined Categories
    \newcommand\Set{\mathbb{S}\mathsf{et}}
    \newcommand\Meas{\mathbb{M}\mathsf{eas}}
    \newcommand\DiffMan{\mathbb{D}\mathsf{iff}\mathbb{M}\mathsf{an}}
\relax % PDG macros
	\newcommand{\bp}[1][L]{\mat{p}\ssub{#1}}
	\newcommand{\bP}[1][L]{\mat{P}\ssub{#1}}
	\newcommand{\V}{\mathcal V}
	\newcommand{\N}{\mathcal N}
	\newcommand{\Ed}{\mathcal E}
    \newcommand{\balpha}{\boldsymbol\alpha}
    \newcommand{\bbeta}{\boldsymbol\beta}

    \newcommand{\dg}[1]{\mathbdcal{#1}}

    % where $\B^{[0,1]}$ is the category 

\usepackage{amsthm,thmtools} % Theorem Macros
	\usepackage[noabbrev,nameinlink,capitalize]{cleveref}
    \theoremstyle{plain}
    \newtheorem{theorem}{Theorem}
	\newtheorem{coro}{Corollary}[theorem]
    \newtheorem{prop}[theorem]{Proposition}
    \newtheorem{conj}[theorem]{Conjecture}
    \newtheorem{claim}{Claim}
    \newtheorem{remark}{Remark}
    \newtheorem{lemma}[theorem]{Lemma}
    \theoremstyle{definition}
    % \newtheorem{defn}{Definition}
    % \declaretheorem[name=Definition]{defn}
    \declaretheorem[name=Definition, qed=$\square$]{defn}
    \declaretheorem[name=Example, qed=$\triangle$]{example}

\usepackage{parskip}
\usepackage[margin=0.9in]{geometry}

\begin{document}
    \begin{center}
        \Huge
        % \Large
        A Categorical Account of Updating
        \bigskip
    \end{center}
    
    In each case, the general ingredients are:
    \begin{enumerate}
        % [nosep]
        \item A category $\Ob$, whose morphisms correspond to observations.
        Composition in this category corresponds to sequential observation, and identities correspond to trivial observations.
        Associativity of composition means we're ruling out the possibility that observations can ``bind together'' in a meaningful way.
        Concretely observing ``($A$ followed by $B$), followed by $C$'' is no different from observing ``$A$ followed by ($B$ followed by $C$)''. 
        Put another way, only the identities and order of the observations matters. 
        This assumption is certainly appropriate if we imagine that observations are integrated one-by-one into one's beliefs.
         
        \item A concrete category $\B$ of beliefs.
        The objects $B \in \ob \B$ correspond to different ``types'' of beliefs (which may or may not be important; one might have beliefs of only one type),
        and each $b \in B$ corresponds to a belief state. 
        In all of our examples, $\B$ is a subcategory of $\DiffMan$, the category of differentiable maps between manifolds. 
        
        \item A functor $F^\top : \Ob \to \B$, which tells you how each observation results in an update. 
        
    \end{enumerate}
    % Now, for partial confidence, we also 
    % In the case of partial confidence, 
    We now turn to the case of partial incorporation, for which we will need a domain of variable confidence.
    \begin{defn}
        A \emph{confidence domain} $\confdom$ is a tuple 
        $(\confdom, \cdot, \bot, \top)$
        where
        \begin{itemize}[nosep]
            \item $\confdom$ is a carrier set of intermediate confidence values, and moreover is a differentiable manifold;
            \item $\cdot : \confdom \times \confdom \to \confdom$ is a binary operation that describes how to combine independent confidences in the same proposition
            \item $\bot$ is a neutral element for $\cdot$, indicating no confidence
             (intuitively, combining any evidence with independent untrusted information leaves you with the same confidence you had before)
            \item $\top$ is an absorbing element for $\cdot$, indicating full confidence 
             (intuitively, definitive evidence, together with other evidecne of any kind, is still definitive evidence). 
        \end{itemize}
    \end{defn}
    
    Instead of item 3, we will instead need:
    
    \begin{enumerate}
        \item[3'.] A functor $F : \mathcal O \to \B^\confdom$. 
    \end{enumerate}
    where $\B^\confdom$ is the category 
    \begin{align*}
        \ob \B^\confdom &:= \ob \B \\
        \Hom_{\B^\confdom}(A,B) &:= \Big\{ \text{ maps } f : A \times \confdom \to B \Big\} \\
        % &
        % \hspace{-3em}\begin{aligned}
        (B\xrightarrow{g}C) \circ (A \xrightarrow{f} B) 
            &:= (a, \chi) \mapsto g(f(a,\chi),\chi)
        % \end{aligned}         
    \end{align*}
        
    % \twocolumn
    
    \section{Example 1: Probabilities and Events}
    
    \subsection{First Pass}
    First, let's consider the motivating example, in which there's some set $W$ of possible worlds, together with a $\sigma$-algebra $\mathcal A \subset 2^W$ of measurable sets, and observations are events $A \in \mathcal A$. 
    Here is one categorical encoding of these observations. 
    
    \begin{align*}
        \ob \Ob &:= \text{ a singleton } \{\star\}; \\
        \Hom_{\Ob}(\star,\star) &:= 
            \{ \text{ events $A \in \A$ } \} = \A; \\
        \id_{\star} &:= W; \\
        &\begin{aligned}
        \circ : \Hom_{\Ob} \times \Hom_{\Ob} &\to \Hom_{\Ob} \\
            := (A, B) &\mapsto A \cap B
        \end{aligned}
    \end{align*}
    
    \def\DX?{\Delta^{\mathclap{\!\!?\!}}\X}
    
    Now, suppose we have probabilistic beliefs, so that $\B$ is also a category with a single object
    $\ob \B := \text{ the singleton set } \{ \DX? := \Delta\X \cup \{ \bot \} \}$
    whose unique element is the union of the set $\Delta\X$ of distributions over $\X$, and the element ``$\bot$'' representing a contradiction. The morphisms of this category are 
    \begin{align*}
        \Hom_{\Ob}(\DX?, \DX?) &:= 
            \{ \text{ differentiable maps from $\Delta\X$ to $\Delta\X$ } \} 
        % \id_{\star} &:= \id_{\DX?}; \\
    \end{align*}
    and the identities and composition are the usual ones inherrited from $\Set$. 
    
    
    An update rule is then a function
    $F^\top : \Ob \to \B$. Since both categories have only one object (i.e., they're just monoids), then this amounts to a monoid homomorphism $F^\top : (\mathcal A, \cap) \to \mathrm{Aut}(\DX?)$.     
    Here is one concrete such functor.
    
    \begin{align*}
        F^\top(A
            % \in \A
            % : \star \to \star
        ) := 
            \begin{cases}
                \mu \mapsto  \mu | A & \text{ if } \mu  \in \Delta\X \text{ and }\mu(A) > 0 \\
                \bot &\text{otherwise}                
        \end{cases}
    \end{align*}
    
    This map is indeed differentiable under a suitable stratification of the boundary. 
    Below is a straightforward validation of functoriality. 
    \begin{align*}
        F(A) \circ F(B) 
            &=  \mu \mapsto 
                \begin{cases}
                    \mu \mapsto (\mu \mid A) \mid B & \text{if $\mu(A) > 0$ and $\mu(B|A) > 0$} \\
                    \bot &\text{else}
                \end{cases} \\
            &= \begin{cases}
                    \mu \mapsto \mu \mid (A \cap B) & \text{if $\mu(A) > 0$ and $\mu(B|A) > 0$} \\
                    \bot &\text{else}
                \end{cases} \\
            &= F(A \circ B).
    \end{align*}
    % while both $F(A)$
    
    Another, less useful functor is one that maps all events to the identity:
    $
        F^\bot(A) := \id_{\DX?},
    $ 
    which corresponds to ignoring all information. 
    
    \smallskip
    \begin{phaseout}
    \textbf{Natural Transformations between them?~}
    A natural transformation $\alpha : F^\top \Rightarrow F^{\bot}$ consists of a map $\alpha_\star \in \mathrm{Aut}(\DX?)$ such that, for all $A \in \mathcal A$, we have
    \[
        \begin{tikzcd}
            \DX? \ar[r,"F^\top(A)"] \ar[d,"\alpha_\star"]
                & \DX? \ar[d, "\alpha_\star"] \\
            \DX? \ar[r,"F^{\bot}(A)"] & \DX?
        \end{tikzcd}
        \quad
        \text{ which is to say, }
        \quad
        \begin{tikzcd}[column sep=4]
            \DX? \ar[rr,"F^\top(A)"] \ar[dr,"\alpha_\star"']
                && \DX? \ar[dl, "\alpha_\star"] \\
            & \DX?
        \end{tikzcd};
    \]
    obviously, any constant function is such a natural transormation. But what about a natural transformation $\eta : F^\bot \Rightarrow F^\top$? Such a map must satisfy
    \[
        \begin{tikzcd}[column sep=4]
            & \DX? \ar[dr,"\eta"]\ar[dl,"\eta"'] & \\
            \DX? \ar[rr,"F^\top(A)"] && \DX? 
        \end{tikzcd}
    \]
    and the only such map is the constant map $\eta(\textunderscore) = \bot$.
    \end{phaseout}
    \smallskip
    
    \subsubsection*{How Intermediates fail to be Functorial}
    What about intermediates?  For any $c \in [0,1]$, we can define a map
    \begin{align*}
        F^c(A
            % \in \A
            % : \star \to \star
        ) := 
            \begin{cases}
                \mu \mapsto  (1-c)\, \mu + (c)\, \mu| A & \text{ if } \mu  \in \Delta\X \text{ and }\mu(A) > 0 \\
                \bot &\text{otherwise}                
        \end{cases}
    \end{align*}
    but it is not a functor. Even supposing that $\mu(A \cap B) > 0$, 
    % Now,
    \begin{align*}
        (F^c(A) \circ F^c(B))(\mu) 
            &= (1-c)\, F^c B(\mu) + (c)\,(F^c B(\mu))|A \\
            &= (1-c)\, \Big( (1-c)\, \mu + (c)\, \mu| B \Big) + (c)\, 
                \Big((1-c)\, \mu + (c)\, \mu| B\Big)|A \\
            &= (1-c)^2\,\mu + c(1-c) (\mu|A + \mu|B) + (c^2)\, \mu|(A \cap B)
    \end{align*}
    cannot be written as a convex combination of $\mu$ and $\mu|(A\cap B)$ unless the two are equal to each other. 
    
    \subsection{A Second Pass}
    Really the problem is the syntactic category; composition of observations is not as simple as intersection. 
    \begin{align*}
        \ob \Ob &:=  \{\text{ finite sets } W~\}; \\
        \Hom_{\Ob}(W,W) &:= 
            \Big\{ \text{ finite sequences $\mathbf s =
             % \big\langle(A_i,\beta_i)\big\rangle_{i=1}^n$ where $A_i \subset W$,} \Big\} \\
             \big( A_i \big)_{i=1}^n$ where $A_i \subset W$,} \Big\} \\
        \id_{W} &:= \langle\rangle; \\
        &\begin{aligned}
        \circ : \Hom_{\Ob} \times \Hom_{\Ob} &\to \Hom_{\Ob} \\
            := ({\bf s, t}) &\mapsto {\bf s :: t}
        \end{aligned}
    \end{align*}
    
    
    \subsection*{A Horizontal Categorification}
    In this example, we haven't used the full power of the categorical representation; since $\mathcal O$ only has one object, it's really just a monoid. 
    To begin, we could glue all of the monoids together for all choices $\mathcal X$ of measurable space, effectively doing the above construction for all measurable spaces at once. This results in a category of observations that only has morphisms from each object to itself.     
    Taking this a step futher, we get something more interesting.
    
    \begin{align*}
        \ob \mathcal O &:= 
            \{ \text{ measurable spaces $\mathcal X = (W, \mathcal A)$ } \}\\
        \Hom_{\mathcal O}(\mathcal X, \mathcal X') &:= 
            \Big\{ \text{ subsets } \Big\}
    \end{align*}
    
    
    \section{Example 2: PDGs}
    
    % Suppose $\dg M$
    Suppose $\N$ is a finite set of nodes, and $\Ed$ is a finite set of edges between them. Let $\A$ denote the free category generated by the graph $(\N, \Ed)$. 
        
    Consider an unweighted PDG, i.e., a functor $\mathbf p : \A \to \Meas_\Delta$.
    
    % What are its automaton semantics?
    % To answer this question, we first need some definitions.     
    % Suppose that 
    
    
    Define 
    
\end{document}
