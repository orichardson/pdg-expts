\documentclass{article}


\usepackage[margin=1.2in]{geometry}
\usepackage{parskip}

% \input{confidence-preamble}
% \addbibresource{conf.bib}

\def\cofunc{commitment function}
\def\confdom{\mathdcal C}

\begin{document}

\begin{center}
	\Huge 
	Updating with Confidence
\end{center}

\section{Introduction}
\def\stmt{$A$}
% \def\stmt{$\phi$}

% The ability articulate a ``degree of confidence'' is
% an important aspect of knowledge representation.
% Subpoint: it helps avoid a brittleness of always beliving things
% Subpoint: Protects against overconfidence.
There are many well-studied ways of representing and measuring uncertainty,
probability chief among them. 
Indeed, an informal poll of our colleagues suggests that computer scientists view ``confidence'' as a synonym for probability. 
The enormous success of probability seems to have shadowed a closely related, but fundementally different, meaning of the word. 
% , such as belief functions, and most importantly, probabilities.
% In each case, one adopts a more complex belief state, 
% The enormous success of probability in particular has had an enormous impact on the way computer scientists talk about 
% The success of probability has shadowed other concepts that might well also be called measures of confidence.
% Indeed, it 

% There are actually two related, but slightly different notions here. 
% % The first is a measurement of
% There is a second setting in which one might interpret the word confidence, as well, in an updating context: it's how seriously you take an input 
% Note the difference: 


Probability is a numerical scale that ranges from untenable (0) to undeniable (1). 
No number on this scale is truly neutral.
%  probability of $\frac12$ .
$\frac12$ may split the difference between the extremes, but is by no means always a neutral assessment: learning that your favored candidate is likely to win with probabitliy $\frac12$ is a big deal, if a win was previously thought to be inevitable. 
This shortcoming has perhaps been the primary selling point of many alternatives to probabiltiy, such as Dempster-Shafer Belief functions. 


Confidence is also a scale between two extremes. It is a measure of trust on a scale of completley untrustworthy $(\bot)$ to fully trusted $\top$. 
High confidence is quite like high probability: if we really trust a statement \stmt, we should fully incorporate it into our beliefs, and thereby come to believe it with high probability. 
Similarly, it only makes sense to be extremely confident in a \stmt\ if you believe that \stmt\ is extremely likely to be true. 
Low confidence, on the other hand, is quite different from low probability. 
If we have little trust in \stmt, we should \emph{ignore} \stmt, rather than coming to believe that \stmt\ is unlikely.
% To say \stmt\ has low probabilty is  high confidence in the $\lnot$\stmt. 
For example, if an adversary tells you something that you happen to already believe, you may have low confidence in their statement, but nevertheless ascribe it high probability. 
Importantly, zero confidence represents a truly neutral stance; a statement with zero confidence has no effect.  

We will analyze this notion of confidence in the context of updating.
In this setting, one has some belief state, and recieves inputs, which one might have some degree of confidence in, which is used to modify one's belief state. 
Confidence measures how seriously to take an input in updating beliefs. 
% The approach we take in this paper is very general, 
% Our approach is quite general, and applies any time a belief state is modified in resonse to an input. 
% one wants to measure (gradual) incorporation of new information into one's beliefs, and may be thought of as a measure of how much of the new information makes it into one's beliefs. 


% For example, suppose we recieve an input \stmt.  If we completely trust \stmt, we should integrate it fullly into our beliefs.  Thus, the effect of having high confidence in \stmt\ is that we come to believe \stmt\ with high probability.
% If we know that the input comes from an untrustworthy source, for instance, the best thing to do is to ignore it entirely, even if 


\section{Formalism}

\end{document}
