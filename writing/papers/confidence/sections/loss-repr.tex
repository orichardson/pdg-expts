Suppose we have:
\begin{enumerate}[nosep]
	\item A differentiable loss function $\mathcal L : \Theta \times \Phi  \to \mathbb R$, which intuitively measures the ``incompatibility'' between a belief state $\theta$ and an assertion $\varphi$, and
	\item
		% A way of taking gradients of $U$ with respect to $\theta$,
		% % $\nabla : ()$
		% such as an inner product $g_p : T_p\Theta \times T_p\Theta \to \mathbb R$, making $(\Theta, g)$ a Riemannian manifold.
		A way of taking the gradient of ${\cal L}$ with respect to $\theta$,%
			\footnote{
			such as a tangent-cotangent isomorphism $(-)^\sharp : T^*_p\Theta \to T_p \Theta$, perhaps coming from an affine connection, in turn perhaps coming from a Riemmannian metric.}
        so as to obtain a vector field on $\Theta$ which optimizes $\mathcal L.$
\end{enumerate}
% \def\GD#1{(\mathtt{GD}\;#1)}
\def\GD#1{\mathtt{GF}[#1]}
\def\NGD#1{\mathtt{NGF}[#1]}

Then we can define an update rule $\GD {\cal L}$ that reduces inconsistency by gradient flow (the continuous limit of gradient descent). Concretely, such an update rule has a vector field:
% \def\GD#1{(\mathtt{GD}\; #1)}
\[
	\GD {\cal L}'_\phi(\theta) = - \nabla_\theta {\cal L}(\theta,\phi).
\]


\begin{prop}
	An update rule $F$ on a Riemannian manifold $\Theta$ is optimizing update rule if and only if $(F')^\flat$ is a conservative co-vector field.
	\cite[Prop 11.40]{lee2013smooth}
\end{prop}




\begin{example}[Weighted Average]
	\begin{align*}
		\Theta = \mathbb R^n \times (\mathbb R_{> 0} \cup \{\infty\});
		\qquad
		\Phi = \mathbb R^n %; \qquad
	\end{align*}
	A belief state $(\mat x,w) \in \Theta$ consists a current estimate $\mat x$ of the quantity of interest, and a weight $w$ of the total internal confidence in the estimate.

	Updating proceeds by taking a weighted average of the previous estimate and the new input, weighted by their respective confidences, which is captured by:
	\begin{align*}
		F^\beta_{\mat y}(\mat x, w) &=  \left( \frac{ w \mat x + \beta \mat y}{w + \beta} , w + \beta \right)
		~~\text{and}~~
		F^{\beta}_{\mat y}(\mat x, \infty) = (\mat x, \infty)
	\end{align*}
	It is additive, since
	\begin{align*}
		&F^{\beta_2}_{\mat y} \circ F^{\beta_1}_{\mat y}(\mat x, w) \\
		&= \left( \frac{ (w + \beta_1) \frac{ w \mat x + \beta_1 \mat y}{w + \beta_1} + \beta_2 \mat y}{(w + \beta_1) + \beta_2}, (w  + \beta_1) + \beta_2 \right) \\
		&= \left( \frac{  w \mat x + (\beta_1 + \beta_2) \mat y}{w + (\beta_1 + \beta_2)}, w  + (\beta_1 + \beta_2)\right)
		= F^{\beta_1 + \beta_2}_{\mat y}(\mat x, w).
	\end{align*}
	And it is clearly differentiable, with a simple calculation revealing that
	$ F'_{\mat y}(\mat x, w) = \left( \frac{\mat y - \mat x}{w}, 1\right) $.

	Observations:
	\begin{itemize}
		\item The update rule cannot be extended differentiably to states $\theta = (\mat x, w)$ with $w = 0$.
			Intuitively, we need to have some estimate with positive confidence to update beliefs in a differentiable way.
			This is related to the fact that plain emperical risk minimization (ERM) is unstable, but stable with even a small amount of regularization.
			% It is also similar to the fact that one needs a prior in order to do
			% In this case, the prior may be arbitrarily weak.
		\item The certainties are given by
		\[
			\lim_{\beta \to \infty} F^{\beta}_{\mat y}(\mat x, w) = (\mat y, \infty)
		\]
		% \item In this case,
		\item $F$ is commutative, invertible, and symmetric with respect to permutation of the dimensions, but it is not conservative: if we had $U(\mat x, w, \mat y)$ twice differentiable such that $\nabla_{\mat x, w} U = F'$, then we would have
        \begin{align*}
        \frac{\partial^2}{\partial w \partial x_i} U &=
			\frac\partial{\partial w} \frac{y_i - x_i}{w} = \frac{x_i - y_i}{w^2},\quad\text{but}
            \\
		\frac{\partial^2}{\partial x_1 \partial w} U
			&= \frac\partial{\partial x_i} 1 = 0    
        \end{align*}
		violating Clairaut's theorem on equality of mixed partials.
		Therefore, $F$ is not an optimizing update rule.
	\end{itemize}
\end{example}

\textbf{Natural Gradients for Probability Distributions.}
When $\Theta$ parameterizes a family of probability distributions, via some $\Pr : \Theta \to \Delta \X$, there is a particularly natural metric on $\Theta$, called the Fisher information metric.
This metric is the unique one on $\Theta$ that is
independent of the representation of $\X$ \parencite{chentsov}, in the following sense.
If there are cpds $p(Y|X)$ and $q(X|Y)$ such that, for all $\theta \in \Theta$,
% $\Pr_\theta = qp\Pr(\theta)$,
% for all $\theta$, sampling $x\sim \Pr_\theta$
% is the same as the distribution over $x'$  $y \sim p(Y|x)$, and $x'\sim q(X|y)$, is equialent to samkp
the distribution $\Pr_{\theta}(X)$ is unchanged after converting to $Y$ and back again $X$ (via $p$ and $q$ respectively), as depicted by the following commutative diagram,
% $q \circ p \circ \Pr_\theta(X) = \Pr_\theta(X)$
\[
	% g [ \Theta \xrightarrow{\Pr} X ]  = g [ \Theta \xrightarrow{\Pr} X \xrightarrow{p}
	%  \Theta \xrightarrow{\Pr} X
	% \quad = \quad {} \xrightarrow{\theta} \Theta \xrightarrow{\Pr} X \overset p\to Y \overset q\to X,
	\begin{tikzcd}
		\Theta \ar[r,"\Pr"]\ar[d,"\Pr"']
			% \ar[rd,dashed,"\Pr^{(Y)}"description]
			& X \\
		X \ar[r,"p"'] & Y \ar[u, "q"']
	\end{tikzcd}
\]
then clearly the family $\Pr(Y|\Theta) := p\,\circ\,\Pr_{\theta}$ carries the same information about the parameters (and in particular how best to update them) as $\Pr_\theta$.
% since we can (losslessly) convert between the two.
Chentsov's theorem says, that, up to a multaplicative constant, the Fisher information metric is the only metric on $\Theta$, as a function of the parameterization $\Pr$, which gives identical geometry in both cases.
% $g[\Pr] = g[\Pr^{Y}]$
% \footnote{at least when $X$ and $Y$ take values in a finite set, although there have since been numerous extensions of it.}


% This allows us to compute the gradient with respect to this metric as
At each point $\Theta$, the components of the Riemannian metric form a matrix---in this case, the Fisher information matrix $\mathcal I(\theta)$---which allow us to now compute the gradient in the natural geometry from the coordinate derivatives as
\[
	\NGD {\cal L}'_\phi (\theta) = - \hat \nabla_\theta \mathcal L(\theta,\phi)
		% = \mathcal I(\theta)^{-1} \frac{\partial}{\partial \theta_i} U(\theta, \phi)
		= \mathcal I(\theta)^{\dagger}  \nabla \mathcal L(\theta, \phi)
\]
where $ \mathcal I(\theta)^{\dagger} $ denotes the Moore-Penrose psuedoinverse of the matrix $ \mathcal I(\theta)$,
and $\nabla \mathcal L$ is the gradient for the euclidean metric i.e., the vector of partials $[\frac{\partial \mathcal L}{\partial \theta_1}, % \ldots, \frac{\partial U}{\partial \theta_i},
 \ldots, \frac{\partial \mathcal L}{\partial \theta_n}]^{\mathsf T}$.
% which is the transpose of the derivative.



\subsubsection{Expected Utility Maximization Update Rules}
% \subsection{Boltzmann Update Rules}


% Again, suppose we have a differentiable function $U : \Theta \times \Phi  \to \mathbb R$.
% Suppose further that we have a recovering a parameter that gives rise to a given probability distribution, i.e., a section of $\Pr$, or concretely,
% a function $\Pr^{-1} : \Delta\X \to \Theta$ such that $\Pr(\Pr^{-1}(\mu)) = \mu$ for all $\mu \in \Delta\X$.
%
% This time, define an update rule directly, by
%
% \begin{align*}
%     (\mathtt{Boltz} U)_\varphi^\beta(\theta)
%         &:\propto \Pr\nolimits_\theta \exp(-\beta U(\theta,\varphi))\\
%         &:= \Pr\nolimits^{-1}\bigg\{
%             A \mapsto \Pr\nolimits_\theta(A)  \, \frac
%             % {1}
%             { \exp(-\beta U(\theta,\varphi)) }
%             {\Ex_{\Pr_\theta}[\exp(-\beta U(\theta,\varphi))]}
%             % \int_A  \exp(-\beta U(\theta,\varphi)) \mathrm d\,\Pr\nolimits_\theta
%         \bigg\}
% \end{align*}
% Now suppose we have a potential function $U : X \times \Phi  \to \mathbb R$.

\def\Bolz#1{\mathrm{Bolz}[#1]}
% Suppose, for each $\phi \in \Phi$, we have a potential function $U_\phi : X \to \mathbb R$ on the underlying set $X$.

% What in the special case where $\Theta$ parameterizes a family of probability distributions, $\Thet$

Suppose, for each $\phi \in \Phi$, we have a utility function $U_\phi : X \to \mathbb R$ on the underlying set $X$.
We can use this to define an update rule via exponential decay:

\begin{align*}
	\Bolz U &: (\mathbb R \times \Phi) \to \Delta\X \to \Delta\X \\
	\Bolz U^\beta_\varphi(\mu)
		&:\propto
			\mu \exp(-\beta U_\phi) \\
		&= A \mapsto \frac
			1{\Ex_{\mu} [ \exp(-\beta U_\phi)]}
			{\int \exp(-\beta U_\phi) \mathbbm 1_A \mathrm d\mu }
\end{align*}

\begin{prop}
	Bolzmann Update Rules are additive, zero, differentiable, invertable, and commutative.
\end{prop}

\begin{lproof}
	\textbf{Commutativity.}
	For some normalization factors $Z, Z', Z''$, we have:
	\begin{align*}
		 F^\beta_\phi( F^{\beta'}_{\phi'}(\mu))
		 &= F^\beta_\phi \Big( \frac{1}{Z} \,\mu\, \exp(- \beta' c_{\phi'}) \Big) \\
		 &= \frac{1}{Z'} \frac{1}{Z} \,\mu\, \exp(- \beta' c_{\phi'}) \exp(- \beta c_{\phi}) \\
		 &= \frac{1}{Z''} \,\mu\, \exp(-\beta' c_{\phi'} - \beta c_\phi)
	\end{align*}
	which is the same expression when we exchange $(\phi, \beta)$ and $(\phi', \beta')$.
\end{lproof}

Note that this is true even for costs generated by asymmetric distances $c_{\{y\}}(x) = d(y, x) \ne d(x,y) = c_{\{x\}}(a)$.


\begin{remark}
	% For fixed, $\varphi$,
	 % and in particular, if $\Phi = \{\varphi\}$ is a singleton,
	Regarding $U_\varphi : \X \to \mathbb R$ as a potential energy over $X$,
	$\Bolz U^\beta_\varphi(\Unif)$ is the Boltzmann distribution at inverse temperature (thermodynamic coldness) $\beta$
	% and base measure $\mu$.
	In the thermodynamic analogy, as temperature decreases, one becomes more certain that particles are in their most favorable states.
\end{remark}


The certainties of $\Bolz U$ are the minimizers of $U$.
% $(\mu(X),\varphi) \mapsto \mu(X ~|~ \arg\min_x U(x,\varphi))$.




\begin{wip}
	% Suppose $\Theta$
	As a reminder, we have $\Theta = \Delta \X$, and suppose we have $c : X \times \Phi \to \mathbb R$.
	Suppose $U(\theta, \varphi) = \Ex_\theta[ c_\varphi ]$ (linearity, \cref{ax:linear}).
	Then
	$% \begin{align*}
		(\mathrm{Boltz}\;U)'_\varphi \theta = \theta(\Ex_\theta[c_\varphi] - c_\varphi),
	$% \end{align*}
	while
	\begin{align*}
		 (\mathtt{GD}\; U)'_\varphi \theta &=
			 - \nabla_{\theta} \Ex\nolimits_\theta [ c_\varphi ]
			 \\ &= - c_\varphi
			 % {\color{gray}~+ \Ex\nolimits_{\theta}[c_\varphi]}
			 .
	\end{align*}
	This second expression, though, doesn't seem quite right --- it isn't even a tangent vector to the probability simplex, since its components don't sum to zero.
	This issue is in our naive computation of the gradient.
	We have computed the collection of partial derivatives $\frac{\partial}{\partial \theta_i} ( \theta \cdot  c_\varphi) = (c_\varphi)_i$,
	which is technically  co-vector field, not a vector field.%
		\footnote{it acts on a vector field $V$ by $ -c_\varphi ( V ) = - V(c_\varphi)$.}

	For simplicity, suppose that $X = \{1, \ldots, n\}$, in the discussion that follows.
	If our parameter space were all of $\mathbb R^n$, we could simply collect these terms and take a transpose, to get
	\[
		\nabla_\theta \Ex_{\theta}[c_\varphi]
	\]
	There are two ways to proceed from here.
	The first makes use of manifold theory: for each point $p \in \Delta X$, begin by identifing a neighborhood $U \ni p$ with an open subset of $\mathbb R^{(n-1)}$, and define an inner product (a metric tensor) $g_p(\cdot, \cdot)$ on tangent vectors $v \in T_p\Delta X$, making $\Delta\X$ into a Riemannian Manifold, and then compute the gradient in the standard way, using the inverse of the metric tensor $g$ in order to convert covectors to vectors in a natural way.

	% The second, which is simpler but more constrained than the first, is to simply
	The second, which is computationally simpler, is to take the metric induced by an embedding in Euclidean space.
	This approach is equally general, because Nash's Theorem \parencite{nash1956imbedding} tells us that any $n$-dimensional Riemannian manifold may be isometrically embeded in $\mathbb R^{2n+1}$.



	% we have
	% \[ \frac{\partial}{\partial \theta^i} ( \theta \cdot  c_\varphi) \]
\end{wip}

\begin{prop}
	% Fix $\varphi$.
	% Let $f(X) := \exp(-\beta U(X,\varphi))$, and $g(X) := U(X,\varphi)$.
	% With the metric on $\Delta\X$ induced by its embedding as a simplex in $\mathbb R^{|X|}$, we have that
	% When parameterized as a simplex,
	The associated vector field is given by
	%
	% \begin{align*}
	$
		% (\mathrm{Boltz}\,U'_\varphi p)_{x} = p(x) (\Ex_p[U(X,\varphi)] - U(x,\varphi) )
		(\mathrm{Boltz}\,U)'_\varphi p = p (\Ex_p[U_\varphi] - U_\varphi )
	$.
	% \end{align*}
\end{prop}
\begin{lproof}
	Let $f(X) := \exp(-\beta U(X,\varphi))$, and $g(X) := U(X,\varphi)$.
	\begin{align*}
		\mathrm{Boltz}'_\varphi\theta &= \frac{\partial}{\partial \beta} \mathrm{Boltz}^\beta_\varphi(p) \Big|_{\beta=0} \\
	\intertext{\TODO[TODO: finish typesetting algebra]}
		&= x \mapsto
			p(x) \frac{f(x)}{\Ex_p[f]}
				\left(\Ex_p\left[ \frac{f}{\Ex_{p}[f]} g\right] - g(x) \right)
				% {\exp(-\beta U(x,\varphi))}
				% {\Ex_{}}
				\Big|_{\beta=0}
				\\
		&= \frac{pf}{\Ex_p[f]^2}
			\left(\Ex\nolimits_p\left[ f g\right] - g \Ex\nolimits_{p}[f] \right)
			\Big|_{\beta=0} \\
		&= x \mapsto p(x) (\Ex\nolimits_p[g] - g(x)) &
			\text{since $f(X) = 1$ when $\beta=0$}
	\end{align*}
	As a sanity check, note that the sum over all components is
	\[ \sum_{x \in X} ((\mathrm{Boltz}\,U)'_\varphi\, \theta)_x
		 = \sum_{x \in X} p(x) (\Ex\nolimits_p[g] - g(x))
		 = \Ex\nolimits_p[ \Ex\nolimits_p [ g ]] - \Ex\nolimits_p [g] = 0,
	 \]
	 so indeed it lies within the tangent space.
\end{lproof}


\begin{prop}
	The optimizing update rules for $\Theta = \Delta X$ whose loss represntation is linear (i.e., an expected utility), are precisely the Bolzmann update rules.
	In particular, the Bolzmann update rule with potential $U(X)$ is the natural gradient flow update rule for expected value of $U$, i.e.,
	\(
		\Bolz U =
		\NGD{\mu\mapsto\Ex_\mu U} .
	\)
\end{prop}


\begin{example}[Gaussian NGD]\label{gauss-ngd}
	Consider the case where $\Theta  = \{ (\mu, \sigma^2) \in \mathbb R \times \mathbb R_+ \}$ is the half-space of parameters to a Gaussian over some real variable $X$, and $\Phi \cong \mathbb R$ consists of possible observations of $X$.

	One natural loss function is negative log likelihood (differential surprisal) of the observation $x$ according to your belief state $\theta = (\mu, \sigma^2)$:
	\begin{align*}
		&\mathcal L(x, \mu, \sigma^2) = - \log \mathcal N(x \mid \mu, \sigma^2) 
        = \frac12 \log 2\pi\sigma^2  + \frac12 \left(\frac{x-\mu}{\sigma}\right)^2 \\
        &\quad=
		\aar*{%
		\begin{tikzpicture}[center base]
			\node[dpad0](X) at (0,0){$X$};
			% \node[dpad
			\draw[arr2, <<-] (X) to node[above,pos=0.7]{$x$} ++(1.0,0);
			% \draw[arr2, <-] (X) to node[above]{$\mathcal N(X|\mu,\sigma^2)$} ++(-1.7,0);
			\node[dpad0](m) at (-1.5, 0.3) {$\mu$};
			\node[dpad0](s2) at (-1.5, -0.3) {$\sigma^2$};
			\mergearr{m}{s2}{X};
			\node[above=2pt of center-ms2X](N){$\mathcal N$};
			\draw[arr1, <<-] (m) to node[above, pos=0.7]{$\mu$} +(-0.9,0);
			\draw[arr1, <<-] (s2) to node[above, pos=0.7]{$\sigma^2$} +(-0.9,0);
		\end{tikzpicture}}.
	\end{align*}

	The Fisher information for a normal distribution is given by
	\[
	\mathcal I(\mu, \sigma) =
	% \begin{bmatrix}
	%     \Ex_{\mathcal N(X|\mu,\sigma^2}[ \frac{\partial mathcal N(X|\mu,\sigma^2}{\partial \mu}) ]
	% \end{bmatrix}
	% =
	\begin{bmatrix}
		\frac1{\sigma^2} & 0 \\
		0 & \frac{1}{2 \sigma^4}
	\end{bmatrix}
	\]



	The natural gradient update rule is given by
	\begin{align*}
		F'_{x}(\mu, \sigma^2)
        &= - \hat\nabla_{\mu, \sigma^2} \mathcal L(x,\mu,\sigma^2)
        \\ &= \mathcal I(\mu, \sigma^2)^{-1}
		\begin{bmatrix}
			\frac{x-\mu}{\sigma} \\[1ex] \frac {-\sigma^2 + (x-\mu)^2}{2 \sigma^4}
		\end{bmatrix}
		=
		\begin{bmatrix}
			x-\mu \\ (x-\mu)^2 - \sigma^2
		\end{bmatrix}.
	\end{align*}

	Note that:
	\begin{itemize}
		\item  $\Ex_{x \sim \nu} [ F'_x(\mu,\sigma^2) ] = \mat 0$ if and only if $\nu$ has mean $\mu$ and variance $\sigma^2$.
		% We can interpret this in two ways:
		% This means that if observations are drawn from a fixed distribution $\nu(X)$,
		Moreover, this point is the unique global attractor.
		This means that,
		\begin{enumerate}
			\item If observations are drawn from a fixed distribution $\nu(X)$, and we repeatedly use $F$ to update $\theta = (\nu, \sigma)$ with small confidence $\epsilon$,
			then $\mu$ will approach the mean $\Ex_{\nu}[X]$ of $\nu$
			and $\sigma^2$ will approach the variance $\Ex_{\nu}[X^2] - \Ex_{\nu}[X]^2$.
			% If we make f observations are drawn from a fixed distribution $\nu(X)$,
			% If we update our belief parameters $\theta = (\mu, \sigma)$ according to $F$ with

			\item If we perform a single high-confidence update on the extended observation $\varphi \propto \nu$, in which each $x$ has relative confidence $\nu(x)$, the result will be a Gaussian with the mean and variance of $\nu$, i.e.,
			\[
				\forall \theta.\quad
				\lim_{c \to \infty} \Pr\nolimits_{F_{\nu}^c(\theta)} = \mathcal N(\Ex\nolimits_{\nu}[X], {\mathrm{Var}}_{\nu}[X])
			\]
		\end{enumerate}
		In this sense, relative confidence acts like probability.
		% So,

		\item
		If we update with the observation $x = \mu$ of our estimate with confidence $c$,
		the mean is unchanged, and our estimate of the variance becomes the harmonic mean of our previous variance $\sigma_0^2$ and the inverse confidence $\frac1c$.
		That is,
		\[
			F^c_\mu(\mu, \sigma_0^2) =
			\left(\mu, \frac{1}{c + \frac{1}{\sigma_0^2}} \right).
		\]
		Equivalently, the precision of the resulting distribution is the average of the confidence $c$ and the previous precision $\nicefrac{1}{\sigma_0^2}$, which suggests that confidence is of the same type as precision.
		%
		Note that if $\sigma_0^2$ is very large, so that our initial beliefs are very uncertain, updating with confidence $c$ results in variance $\frac1t$.
		In this sense, the magnitude of confidence acts as the inverse of variance.
	\end{itemize}
\end{example}
