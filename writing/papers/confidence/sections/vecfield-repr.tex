
% \textbf{Differentiability}.
% A primary goal of this paper is to study how updates are made in low-confidence settings.
Confidence is meant to interpolate between fully incorporating information and ignoring it.
Such an interpolation becomes more useful if it is continuous, and more useful still if it is differentiable.
% After all, that was one of our motivations for focusing on confidence domains that can be represented as a real number.
Next, we present two variants of a differentiability axiom, depending on the structure one has in hand. 

\begin{CFaxioms}
	\item \label{ax:diffble}
	\begin{enumerate}
	\item $\Theta$ has a manifold structures, and
		for all $\theta$ and $\phi$, the function $\beta \mapsto F^{\beta}_\phi(\theta) : \confdom \to \Theta$
		is continuously differentiable at $\beta = \bot$. %\label{ax:diffble}
	\item 
		$\Theta$ parameterizes a family of probabilities over $(\X, \mathcal A)$,
		via $\{ \Pr_\theta \}_{\theta \in \Theta}$.
		%  we can avoid talking abot a differentiable structure on $\Theta$ by simply requiring that the update rule be differentiable from the perspective of every event $A \in \mathcal A$.
		%  is a family of probability distributions.
		for all $\theta \in \Theta$, $\phi \in \Phi$, and  $A \in \mathcal A$,
		the function $\beta \mapsto \Pr_{F^{\beta}_\phi(\theta)}(A)
		: \confdom \to \mathbb [0,1]$ is
		continuously differentiable at $\beta=\bot$. 
		%(in pairs $(\beta,\Pr)$).
			% \hfill \textbf{(differentiability)}
			\label{ax:diffble2}
	\end{enumerate}
	\hfill \textbf{(differentiability)}
\end{CFaxioms}

% Suppose that the space $\Theta$ is actually a differentiable manifold.
% In this case, we might want want $F$ to be compatible with the differentiable structure.
% \begin{CFaxioms}
%     \item $\Theta$ is a differentiable manifold.
%         For fixed $\theta$ and $\phi$, the function $\beta \mapsto F^{\beta}_\phi(\theta)$
%         is continuously differentiable.
%             \hfill \textbf{(differentiability)} \label{ax:diffble2}
% \end{CFaxioms}

% If $\Theta$ is a differentiable manifold and 
If $\Theta$ is a differentiable manifold and $\Pr: \Theta \to \Delta\X$ is a differentiable map, then the second follows from the first. 
% For simplicity, from this point forwards, we will assume that $\Theta$ itself carries a differentiable structure.
It's simpler to assume that $\Theta$ carries a differentiable structure, so we will assume this when possible.
% In the following result, we will begin to see what makes $\Rplus$ such a natural confidence domain for differentiable update rules.

When we have $\confdom := \Rplus$,

\TODO
% When we combine two independent updates 

\begin{CFaxioms}
	\item For all $\beta_1, \beta_2 \in \Rplus$,~
		% $F^{c_1}_\phi \circ F^{c_2}_\phi = F^{c_1 \oplus c_2}_\phi$
		$F^{\beta_1}_\phi \circ F^{\beta_2}_\phi = F^{\beta_1 + \beta_2}_\phi$
		% \hfill \textbf{(additivity)} \label{ax:additivity}
		\hfill \textbf{(additivity)} \label{ax:additivity}
\end{CFaxioms}


% Even restricting to , additivity is a particularly natural. 

\begin{prop}
	If $F$ is a differentiable \cofunc\ with confidence domain $\Rplus$,then there is a unique update rule $G$ with the same confidence domain, that behaves approximately like $F$ for small increments of confidence, and is also additive (\cref{ax:additivity}).
\end{prop}





\label{sec:vecrep}
For a smooth manifold $M$
(such as the space $\Delta \X$ of distributions over $\X$),
and a point $p \in M$, we follow convention by writing $T_p M$ for the tangent space to $M$ at point $p$ \parencite{lee2013smooth}, and % $TM := \sqcup_{p \in M} (p, T_p M)$
$TM := \sum_{p \in M} T_p M$ for the full tangent bundle over $M$.
%
A vector field over $M$ is a smooth map $\mat v : M \to T M$ assigning a tangent vector $\mat v(p) \in T_p M$, to every point $p \in M$.
% A vector field is called \emph{complete} if it generates a global flow.
% , or equivalently, a smooth section of the projection map $\pi : T M \to M$, where $\pi((p, v)) = p$.

\begin{theorem}
Every update rule $F : \Phi \times \mathbb R \to (\Theta  \to \Theta)$
satisfying \cref{ax:zero,ax:additivity,ax:diffble} corresponds to a unique
$\Phi$-indexed collection of vector fields
    $F' : \Phi \times \Theta \to T\Theta$
\end{theorem}
% \begin{coro}\label{thm:vecrep}
%     There is a natural bijection between
%     % update rules $F : \Phi \times \mathbb R \to \Delta \X \to \Delta \X$
%     update rules $F : \Phi \times \mathbb R \to (\Theta  \to \Theta)$
%         satisfying \cref{ax:zero,ax:additivity,ax:diffble},
%     and $\Phi$-indexed collections of complete vector fields
%         % $\{ F'_\phi : \Delta X \to T \Delta X \}_{\phi \in \Phi}$%
%         % $\{ F'_\phi : \Theta \to T \Theta \}_{\phi \in \Phi}$%
%         $ F' :  \Phi \times \Theta \to T \Theta$%
%         % $F' : \Phi \to \Delta\X \to T\Delta \X$%
%     .
% \end{coro}

In the language of 

\begin{coro}\label{thm:vecrep}
There is a bijective correspondence between udpate rules satisfying \cref{ax:zero,ax:additivity,ax:diffble} and $\Phi$-indexed collections of \textbf{complete} vector fields.
\end{coro}


We call $F'$ the \emph{vector field representation} of a differentiable update rule $F$.

This vector field representation of a commitment function captures everything except the confidence itself. 

One defining feature of vector fields is closure under linear
% (and, in particular, convex)
combination.  Because they are in natural correspondance with differentiable additive update rules, update rules also inherit this structure.

In particular, given \cofunc s $F, G : \mathbb R \to \Theta$, we can define
$F \oplus G$ via the vector field $(F \oplus G)' = F' + G'$.

\begin{wip}
\textbf{Interaction With Certainty Axioms.}

\TODO[TODO: prove impossible to individually they satisfy certainty, but not together.]
\end{wip}

\begin{defn}
For an assertion language $\Phi$, let $\ext\Phi$ denote
the space of weighted formal sums of elements of $\Phi$.
% the space $\mathbb R^{\Phi}_{\mathrm{fin}}$ of finitely supported vectors over $\Phi$,
\end{defn}

% If $\Phi$
\begin{prop}
Every  update rule $F$ on $\Phi$, can be naturally extended to an update rule
$\bar F$ on $\ext\Phi$
% $\mathbb R^{\Phi}$,
via the total vector field
\[
    % \bar F'_{\vec{x}}( \theta ) := \sum_{\phi} F'_\phi(\theta) x_\phi.
    \bar F'_{\textstyle\sum_i a_i \phi_i} ( \theta ) := \sum_{i} a_i F'_{\phi_i}(\theta).
\]
% \end{prop}
%
\end{prop}

If $\Phi$ is itself a measurable space, we can extend this further:
Every  update rule $F$ on $\Phi$, can be naturally extended to an update rule $\bar F$ on the space $\mathcal M(\Phi)$ of measures over $\Phi$, via
\[
% \bar F'(\theta) := \sum_{} \beta_x \
\bar F'_{\beta(\Phi)}( \theta ) := \int_{\Phi} F'_\phi(\theta) \mathrm d\beta.
\]



% \subsection{Commutative Update Rules}
%
% All differentiable update rules are ``locally'' commutative, in the sense that the difference between
% $F_{\phi_1}^\epsilon \circ F_{\phi_2}^\epsilon$ and
% $F_{\phi_2}^\epsilon \circ F_{\phi_1}^\epsilon$ goes to zero as $\epsilon \to 0$.
% This is an immediate consequence of differentiability and the fact that they share a limit point (the identity function).
%
% If we fix a commutative and differentiable update rule $F$, and an initial point $\theta_0$, then the space $\mathbb R^\Phi$ of real-valued vectors over $\Phi$,
% serves as a coordinate system for $\Theta$.

%
% Not all update rules of interest are commutative, even if otherwise well-behaved.
%
% \begin{example}
%     The inconsistency-reduction update rule, $\tau$, is not commutative, but it is differentiable, additive, invertable, and even conservative.
% \end{example}
\subsection{Linear Update Rules}

% In some sense, ALL update rules are linear in $\bar\Phi$ by definition.

There are many definitions of linear update rules:
\begin{defn}\label{ax:linear}
Let $F$ be a differentiable update rule on $\Theta$. We say that $F$ is \textellipsis
\begin{itemize}
\item \emph{linear} if $\Theta$ is a vector space over $\mathbb R$, and the
vector field $F'_\phi$ is a linear operator, i.e., for all $a, b \in \mathbb R$, we have that
\[ F'_\phi(a \theta_1 + b \theta_2) = a F'_\phi(\theta_1) + b F'_\phi(\theta_2). \]

\item \emph{cvx-linear} if $\Theta \subset \mathbb R^n$ is a convex set, and, for all $a \in [0,1]$, we have that
\[ F'_\phi(a \theta_1 + (1-a) \theta_2) = a F'_\phi(\theta_1) + (1-a) F'_\phi(\theta_2). \]

\item \emph{$\mathcal L$-cvx-linear} if $\Theta \subset \mathbb R^n$ and $F$ is an optimizing update rule with a loss representation $\mathcal L$ linear in its first argument, i.e.,
\[
    \mathcal L(a \theta_1 + (1-a) \theta_2, \varphi) = a \mathcal L(\theta_1, \varphi) + (1-a) \mathcal L(\theta_2, \theta).
\]
\end{itemize}
% $F'_\phi(\theta) = \mathrm{V}_\phi \theta$ for some linear operator $V_\phi \in \mathbb R^{n \times n}$.
% $F'_\phi(\theta) = \mathrm{V}_\phi \theta$ for some linear operator $V_\phi$.
\end{defn}

\begin{prop}
If $F$ is a $\mathcal L$-cvx-linear, then it is also cvx-linear.
\end{prop}

In fact, the first condition is much stronger;
\begin{prop}
if $F$ is a nontrivial $\mathcal L$-cvx-linear optimizing UR, then $\Theta$ equals cone generated by  the rays $\{ F'_\varphi\theta : \theta \in \Theta, \varphi \in \Phi \}$. In particular, if there is some $\theta$ such that $0$ is in the interior of the convex hull $\mathrm{conv}(\{F'_\phi\theta\}_{\phi \in \Phi})$, then $\Theta = \mathbb R^n$.
\end{prop}

% Implicit in this definition is the supposition that the integral curves generated by the differential equations, started at any point $\theta \in \Theta$, are

\begin{prop}
% If $F$ is a  differ
Every linear update rule is of the form
$
    F^{\beta}_\phi(\theta) =  \theta^{T} \exp(\beta V)
$,
where $\exp(\beta V)$ is the matrix exponential.%
    \footnote{Concretely, if $V = U^T \mathrm{Diag}(\lambda_1, \ldots \lambda_n) U$ is an eigendecomposition of $V$, then $\exp(V) = U^T \mathrm{Diag}(e^{\beta\lambda_1}, \ldots e^{\beta\lambda_n}) U$.}
\end{prop}

\begin{prop}
A linear update rule $F$ is commutative iff, for every pair of statements  $\phi, \phi' \in \Phi$, the
matrices $V_\phi$ and $V_{\phi'}$ commute.
\end{prop}
