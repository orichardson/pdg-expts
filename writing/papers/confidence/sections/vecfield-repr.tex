

% Suppose that the space $\Theta$ is actually a differentiable manifold.
% In this case, we might want want $F$ to be compatible with the differentiable structure.
% \begin{CFaxioms}
%     \item $\Theta$ is a differentiable manifold.
%         For fixed $\theta$ and $\phi$, the function $\beta \mapsto F^{\beta}_\phi(\theta)$
%         is continuously differentiable.
%             \hfill \textbf{(differentiability)} \label{ax:diffble2}
% \end{CFaxioms}
% If $\Theta$ is a differentiable manifold and 


Recall that the number of training iterations $n$ in \cref{ex:classifier}
and Shafer's weight of evidence $w$ in \cref{ex:shafer}
are measurements of confidence that do not lie in $[0,1]$, but
 rather in $[0,\infty]$. 
% In this case, 
In such cases,
% we must instead use an analogous function of the form 
the appropriate analogue is a function
\begin{equation}
	F : \Phi \times [0, \infty] \times \Theta \to \Theta
	\label{eq:zero-inf-form}
\end{equation}
satisfying modified versions of
\cref{ax:zero,ax:idemp,ax:cont,ax:diffble,ax:seq-for-more,ax:nopause}
% identical except with $\infty$ in place of 1. 
that use $\infty$ in place of 1 for the upper limit of confidence. 
% We abuse terminology by stating that $F$ satisfies these axioms.
One reason to prefer this scale is that it 
allows us to represent degree of confidence in a way that combines additively. 
Nearly all measurable quantities used in science 
and everyday life can be described additively:
if you have six (feet/meters/galons/joules/people/votes/dollars), 
and then you find seven additional (distinct) ones, then you have thirteen altogether. 
We would like a measure of confidence that also works this way.

\begin{CFaxioms}
	\item For all
		% $\beta_1, \beta_2 \in \Rplus$,~
		% $\beta_1, \beta_2 \in [0,\infty]$,~
		$\chi_1, \chi_2 \in [0,\infty]$,~
		% $F^{c_1}_\phi \circ F^{c_2}_\phi = F^{c_1 \oplus c_2}_\phi$
		% $F^{\beta_1}_\phi \circ F^{\beta_2}_\phi = F^{\beta_1 + \beta_2}_\phi$.
		$F^{\chi_1}_\phi \circ F^{\chi_2}_\phi = F^{\chi_1 + \chi_2}_\phi$.
		% \hfill \textbf{(additivity)} \label{ax:additivity}
		% \hfill \textbf{(additivity)}
		\label{ax:additivity}
\end{CFaxioms}

% Recall that \cref{ax:seq-for-more} implies the behavior of updates is generated by low-confidence updates.
Recall how \cref{ax:seq-for-more} allows us to decompose high-confidence updates into sequences of low-confidence ones;
\cref{ax:additivity},
which implies \cref{ax:seq-for-more}, describes a particularly convenient
way that the decomposition might work. 

% In fact, there is a unique additive 
\begin{defn}
	% A function satisfying
	A \emph{flow update function}
	is a function
	$F : \Phi \times[0,\infty] \times \Theta\to \Theta$
	satisfying the appropriate analogues of
	% \cref{ax:zero,ax:idemp,ax:cont,ax:seq-for-more,ax:diffble}
	\cref{ax:zero,ax:idemp,ax:cont,ax:diffble}
	and \cref{ax:additivity}.
\end{defn}

To some,
% \cref{ax:additivity} might be pallatable already,
\cref{ax:additivity} might already be pallatable,
% but it looks like it might be an assumption that significantly restricts the expressiveness of our framework.
% but it also might look like it might significantly restrict the expressiveness .
but it is clearly a nontrivial assumption, and looks like it might severely restrict the expressiveness of our update formalism.
% This is not the case.
Fortunately, this is not the case.
While \cref{ax:additivity} does significantly pin down how confidence 
can be measured, it has no effect on what confidence can express.  
% More concretely, for every confidence function that does not satsify \cref{ax:additivity}, we can 


\begin{linked}{theorem}{add-reparam}
	If $F$ satisfies \cref{%
		ax:funcform,%
		ax:zero,ax:idemp,ax:cont,ax:seq-for-more,ax:diffble,ax:nopause},
	then there is a unique 
	flow update rule
	 $^+\!F$
	that behaves like $F$ for low confience updates
	(and is also additive: \cref{ax:additivity}). 
	%
	% Moreover,
	Furthermore, there exists a function 
	% $g$, increasing in $\chi$ such that,
	$g$ such that,
	for all $\theta,\phi,$ and $\chi$,
	\[
		F( \phi, 
		% g(\phi,\beta,\theta),
			\chi,
		 \theta )
		 =
		{^+}\!F(\phi, 
		% \beta,
		g(\phi,\chi,\theta),
		 \theta).
	\]
\end{linked}
Thus, updates performed with $F$ are equivalent
to updates performed with ${^+}\!F$, except that
the degree of confidence needs to be translated appropriately (via $g$).


% Moreover, 

% Recall that \cref{ax:seq-for-more} impilies that the behavior of updates
% is generated by low-confidence updates; we saw a particularly nice
% way of doing that in \cref{ax:additivity},
% which has the feature that confidence behaves the same way no matter what your initial beliefs are. 


% Even restricting to , additivity is a particularly natural. 
% 
% \begin{prop}
% 	If $F$ is a differentiable \cofunc\ with confidence domain $\Rplus$,then there is a unique update rule $G$ with the same confidence domain, that behaves approximately like $F$ for small increments of confidence, and is also additive (\cref{ax:additivity}).
% \end{prop}


\subsection{Vector Field Representations}
\label{sec:vecrep}

We now turn to another representation of additive 
update functions, as vector fields. 
This representation allows us to extend the set $\Phi$ of possible observations to 
% to a set $\ext\Phi \supseteq \Phi$ with some algebraic operations.  
to a larger set $\ext\Phi \supseteq \Phi$ with some algebraic operations.  

In \cref{ax:diffble}, we assumed that $\Theta$ has a differentiable 
structure; thus, it makes sense to talk about its tangent space
$T\Theta$, which consists of pairs $(\theta, \mat v)$, where
$\theta \in \Theta$, and $\mat v$,
% is intuitively an infinitessial direction rooted at $\theta$.
intuitively, is a direction that one can travel in $\Theta$ beginning at $\theta$
% tangent to $\Theta$
% rooted at $\theta$
\parencite[\S3]{lee2013smooth}.
% structure; for ease of presentation, suppose that it is an $n$-dimensional manifold \parencite{lee2013smooth}. 
% For a smooth manifold $M$ (such as the space $\Delta \X$ of distributions over $\X$),
% and a point $p \in M$, we follow convention by writing $T_p M$ for the tangent space to $M$ at point $p$ \parencite{lee2013smooth}, and % $TM := \sqcup_{p \in M} (p, T_p M)$
% $TM := \sum_{p \in M} T_p M$ for the full tangent bundle over $M$.
%
% A vector field over $M$ is a smooth map $\mat v : M \to T M$ assigning a tangent vector $\mat v(p) \in T_p M$, to every point $p \in M
%
A vector field $X$ over $\Theta$ is then
a smooth map $X : \Theta \to T \Theta$ 
assigning a tangent vector $X(\theta) = (\theta, \mat v) \in TM$ 
to every $\theta \in \Theta$; the set of all 
vector fields over $\Theta$ is denoted $\mathfrak X(\Theta)$ and is closed
under linear combination
\parencite[\S8]{lee2013smooth}
\unskip.
There is a close relationship between additive confidence and such vector fields.
% A vector field is called \emph{complete} if it generates a global flow.
% , or equivalently, a smooth section of the projection map $\pi : T M \to M$, where $\pi((p, v)) = p$.
Given a flow update function $F$, and observation $\phi$, the
differential of $F$ is a vector field
\begin{equation}
	F'_\phi 
	:= \frac{\partial}{\partial \chi} F_{\theta}^{\chi} \Big|_{\chi=0}
	\in  \mathfrak X(\Theta).
	\label{eq:f-field}
\end{equation}
Moreover, we can recover $F_\phi$ as the integral curves of $F'_\phi$.
% In other words, if we knew only the vector field $F'_\phi$, 
% we could 
% because $F_\phi$ is the unique function satsifying \eqref{eq:f-field}.

\begin{prop}
	% Suppose $X_\phi \in \mathfrak X(\Theta)$ be a vector field.
	% Let $F$ be a flow-update rule. 
	% Suppose $X_\phi \in \mathfrak X(\Theta)$ be a vector field.
	% Then there is at most one flow-update function satisfying \eqref{eq:f-field}.
	% Fix the vector field $X := F'_{\phi}$.
	% $F$ is the only flow-update rule satisfying \eqref{eq:f-field}.
	Let $F$ be a flow update function, and fix the vector field $F'_{\phi}$.
	$F_\phi$ is the only flow update satisfying \eqref{eq:f-field}.
	% If $F$ is the uni
\end{prop}

% Therefore, \cref{theorem:add-reparam}
Thus, every flow update function can be equivalently represented
by its differential.
% \begin{prop}
% 	% Let $F$ be a flow-update rule.
% 	% Then, there is a bijective correspondence between 
% 	There is a biective correspondence between
% 	flow-update rules 
% 	% $F : \Phi \times[0,\infty] \times \Theta \to \Theta$.
% 	and
% 	$\Phi$-indexed families of vector fields $X : \Phi \to \mathfrak X(\Theta)$.
% % Every update rule $F : \Phi \times \mathbb R \to (\Theta  \to \Theta)$
% % satisfying \cref{ax:zero,ax:additivity,ax:diffble} corresponds to a unique
% % $\Phi$-indexed collection of vector fields
% %     $F' : \Phi \times \Theta \to T\Theta$
% \[
% 	X()
% \]
% \end{prop}
% \begin{coro}\label{thm:vecrep}
%     There is a natural bijection between
%     % update rules $F : \Phi \times \mathbb R \to \Delta \X \to \Delta \X$
%     update rules $F : \Phi \times \mathbb R \to (\Theta  \to \Theta)$
%         satisfying \cref{ax:zero,ax:additivity,ax:diffble},
%     and $\Phi$-indexed collections of complete vector fields
%         % $\{ F'_\phi : \Delta X \to T \Delta X \}_{\phi \in \Phi}$%
%         % $\{ F'_\phi : \Theta \to T \Theta \}_{\phi \in \Phi}$%
%         $ F' :  \Phi \times \Theta \to T \Theta$%
%         % $F' : \Phi \to \Delta\X \to T\Delta \X$%
%     .
% \end{coro}
% In the language of 
% 
% Not all vector fields can be integrated to get an update function
% 
% \begin{coro}\label{thm:vecrep}
% There is a bijective correspondence between udpate rules satisfying \cref{ax:zero,ax:additivity,ax:diffble} and $\Phi$-indexed collections of \textbf{complete} vector fields.
% \end{coro}
% We call $F'$ the \emph{vector field representation} of an update function $F$. 
% This vector field representation of an update function 
It may seem counter-intuitive that $F'_\phi$,
which no longer mentions confidence at all,
can capture confidence. In a sense, it does so by specifying 
everything about the update \emph{except} for the degree of confidence.
Having separated the confidence from the mechanics of the update,
this vector field representation allows us to 

\subsection{Orderless Combination of Observations}

One defining feature of vector fields is that they
can be linearly combined to form new vector fields.
Therefore flow update rules, which are in natural correspondance with differentiable additive update rules, also inherrit this structure.

The first way of combining propositions is to rescale them.
% \begin{prop}
	% Suppose $F$ is a flow udpate rule.
For $\phi \in \Phi$ and a scalar $k \in [0,\infty)$, 
we can extend $F$ a new input $k \cdot \phi \in \ext\Phi$ by
\[
	F^{\chi}_{k\cdot\phi}(\theta) := F^{k\chi}_{\phi}(\theta).
\]
% \end{prop}
In this way, the set $\Phi$ inherits 
the additivity of the update rule in the form of scalar multiplication.
It turns out more is possible: updates inherit the entire vector space structure.


The second way of combining propositions is to ``run them concurrently''.
% In particular, given \cofunc s $F, G : \mathbb R \to \Theta$, we can define
% $F \oplus G$ via the vector field $(F \oplus G)' = F' + G'$.
% \begin{defn}
% 	For $\phi_1, \phi_2 \in \Phi$, we extend $F$ to 
% 	$\phi_1 \oplus \phi_2$
% \end{defn}
Given $\phi_1, \phi_2 \in \Phi$, we can form a new input 
$\phi_1 \oplus \phi_2 \in \ext \Phi$, and we extend $F$ on it by
adding the vector fields
$F'_{\phi_1 \oplus \phi_2} := F'_{\phi_1} + F'_{\phi_2}$.
Note that by definition, $\phi_1 \oplus \phi_2 = \phi_2 \oplus \phi_1$,
so this is a way of combining observations orderlessly, even in cases
where $\phi_1$ and $\phi_2$ do not commute. And when $\phi_1$ and $\phi_2$
already do not depend on order, $\phi_1\oplus \phi_2$ has the same effect
as $\phi_1$ followed by $\phi_2$.

\begin{prop}
	% If $\phi_1$ and $\phi_2$ commute
	% (i.e., $F^{\chi}_{\phi_1} \circ F^{\chi}_{\phi_2} =
	%  	F^{\chi}_{\phi_2} \circ F^{\chi}_{\phi_1}$ for all $\chi$)
	% \unskip, then both are equal to $F^{\chi}_{\phi_1 \oplus \phi_2}$ 
	% for all $\chi
	%  % \in [0,\infty]
	%  $.
	If $F^{\chi}_{\phi_1} \circ F^{\chi}_{\phi_2} =
	 	F^{\chi}_{\phi_2} \circ F^{\chi}_{\phi_1}$,
	then both updates are equal to $F^{\chi}_{\phi_1 \oplus \phi_2}$.
	\commentout{That is,
	\[
		F^{\chi}_{\phi_1}( F^{\chi}_{\phi_2}(\theta))
		=
		F^{\chi}_{\phi_2 \oplus \phi_1} (\theta)
		=
		F^{\chi}_{\phi_1 \oplus \phi_2} (\theta)
		=
		F^{\chi}_{\phi_1}( F^{\chi}_{\phi_2}(\theta)) 
		.
	\]}
\end{prop}

Intuitively, $\phi_1 \oplus \phi_2$ is a ``mixture observation'' containing
one part $\phi_1$ and one part $\phi_2$. This intuition is made
precise by the following proposition, which 

\begin{prop}
	
\end{prop}

\begin{example}
	ML example: dataset = orderless combination.
	Rescaling = changing learning rate.

	\TODO
	
\end{example}


\begin{defn}
For an assertion language $\Phi$, let $\ext\Phi$ denote
the space of weighted formal sums of elements of $\Phi$.
% the space $\mathbb R^{\Phi}_{\mathrm{fin}}$ of finitely supported vectors over $\Phi$,
\end{defn}

% If $\Phi$
\begin{prop}
Every  update rule $F$ on $\Phi$, can be naturally extended to an update rule
$\bar F$ on $\ext\Phi$
% $\mathbb R^{\Phi}$,
via the total vector field
\[
    % \bar F'_{\vec{x}}( \theta ) := \sum_{\phi} F'_\phi(\theta) x_\phi.
    \bar F'_{\textstyle\sum_i a_i \phi_i} ( \theta ) := \sum_{i} a_i F'_{\phi_i}(\theta).
\]
% \end{prop}
%
\end{prop}

If $\Phi$ is itself a measurable space, we can extend this further:
Every  update rule $F$ on $\Phi$, can be naturally extended to an update rule $\bar F$ on the space $\mathcal M(\Phi)$ of measures over $\Phi$, via
\[
% \bar F'(\theta) := \sum_{} \beta_x \
\bar F'_{\beta(\Phi)}( \theta ) := \int_{\Phi} F'_\phi(\theta) \mathrm d\beta.
\]



% \subsection{Commutative Update Rules}
%
% All differentiable update rules are ``locally'' commutative, in the sense that the difference between
% $F_{\phi_1}^\epsilon \circ F_{\phi_2}^\epsilon$ and
% $F_{\phi_2}^\epsilon \circ F_{\phi_1}^\epsilon$ goes to zero as $\epsilon \to 0$.
% This is an immediate consequence of differentiability and the fact that they share a limit point (the identity function).
%
% If we fix a commutative and differentiable update rule $F$, and an initial point $\theta_0$, then the space $\mathbb R^\Phi$ of real-valued vectors over $\Phi$,
% serves as a coordinate system for $\Theta$.

%
% Not all update rules of interest are commutative, even if otherwise well-behaved.
%
% \begin{example}
%     The inconsistency-reduction update rule, $\tau$, is not commutative, but it is differentiable, additive, invertable, and even conservative.
% \end{example}
\subsection{Linear Update Rules}

% In some sense, ALL update rules are linear in $\bar\Phi$ by definition.

There are many definitions of linear update rules:
\begin{defn}\label{ax:linear}
Let $F$ be a differentiable update rule on $\Theta$. We say that $F$ is \textellipsis
\begin{itemize}
\item \emph{linear} if $\Theta$ is a vector space over $\mathbb R$, and the
vector field $F'_\phi$ is a linear operator, i.e., for all $a, b \in \mathbb R$, we have that
\[ F'_\phi(a \theta_1 + b \theta_2) = a F'_\phi(\theta_1) + b F'_\phi(\theta_2). \]

\item \emph{cvx-linear} if $\Theta \subset \mathbb R^n$ is a convex set, and, for all $a \in [0,1]$, we have that
\[ F'_\phi(a \theta_1 + (1-a) \theta_2) = a F'_\phi(\theta_1) + (1-a) F'_\phi(\theta_2). \]

\item \emph{$\mathcal L$-cvx-linear} if $\Theta \subset \mathbb R^n$ and $F$ is an optimizing update rule with a loss representation $\mathcal L$ linear in its first argument, i.e.,
\[
    \mathcal L(a \theta_1 + (1-a) \theta_2, \varphi) = a \mathcal L(\theta_1, \varphi) + (1-a) \mathcal L(\theta_2, \theta).
\]
\end{itemize}
% $F'_\phi(\theta) = \mathrm{V}_\phi \theta$ for some linear operator $V_\phi \in \mathbb R^{n \times n}$.
% $F'_\phi(\theta) = \mathrm{V}_\phi \theta$ for some linear operator $V_\phi$.
\end{defn}

\begin{prop}
If $F$ is a $\mathcal L$-cvx-linear, then it is also cvx-linear.
\end{prop}

In fact, the first condition is much stronger;
\begin{prop}
if $F$ is a nontrivial $\mathcal L$-cvx-linear optimizing UR, then $\Theta$ equals cone generated by  the rays $\{ F'_\varphi\theta : \theta \in \Theta, \varphi \in \Phi \}$. In particular, if there is some $\theta$ such that $0$ is in the interior of the convex hull $\mathrm{conv}(\{F'_\phi\theta\}_{\phi \in \Phi})$, then $\Theta = \mathbb R^n$.
\end{prop}

% Implicit in this definition is the supposition that the integral curves generated by the differential equations, started at any point $\theta \in \Theta$, are

\begin{prop}
% If $F$ is a  differ
Every linear update rule is of the form
$
    F^{\beta}_\phi(\theta) =  \theta^{T} \exp(\beta V)
$,
where $\exp(\beta V)$ is the matrix exponential.%
    \footnote{Concretely, if $V = U^T \mathrm{Diag}(\lambda_1, \ldots \lambda_n) U$ is an eigendecomposition of $V$, then $\exp(V) = U^T \mathrm{Diag}(e^{\beta\lambda_1}, \ldots e^{\beta\lambda_n}) U$.}
\end{prop}

\begin{prop}
A linear update rule $F$ is commutative iff, for every pair of statements  $\phi, \phi' \in \Phi$, the
matrices $V_\phi$ and $V_{\phi'}$ commute.
\end{prop}
