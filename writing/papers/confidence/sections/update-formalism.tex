
Throughout, we use $\Theta$ to refer to some space of possible belief states,
and $\Phi$ for a set of possible inputs. 
\commentout{% have now already given these examples more clearly; now cut to the chase!
	For example, a belief state $\theta \in \Theta$
	might be probability distribution,
	a Dempster-Shafer belief function, or
	weights for a neural network,
	while an input $\phi \in \Phi$ might be an event, or a sample from a dataset. 
}
%
% $\Theta$ might be \textellipsis
% \begin{itemize}[nosep]
%     \item $\Delta W$, the set of probability distributions over measurable space $W$ of outcomes;
%     \item the set of parameters to some parametric family of distributions;
%     \item the set of belief functions over a common space of outcomes;
%     \item the set of all possible PDGs;
%     \item a set of worlds an agent considers possible;
%     \item
% % \end{itemize}
Now, suppose we have some belief state $\theta$, and observe $\phi$ with confidence $\alpha \in [0,1]$. 
% How can we describe the processing of updating $\theta$, so as to obtain some $\theta'$ which takes $\phi$ into account?

% The first assumption we need to make to study this mathematically is that the process can be captured in functional form.
To do so mathematically, we have to assume that it be captured in functional form.
\unskip\footnote{we can just as easily handle randomized updates;
	the point is simply that the update can be prescribed by an algorithm.}
% The primary assumption that we have to make is:
\begin{CFaxioms}
	\item[\textbf{F}]
		% The updating process can be described as a function
		% There exists a function
		% % $F : \Theta \times \Phi \to \Theta$,
		% $F : \Phi \to ( \Theta \to \Theta)$,
		% which, given the old belief state $\theta$ and new information $\phi$, produces the belief state $F_\phi(\theta)$ that corresponds to the result of observing $\phi$ in state $\theta$.
		% which produces a new belief state, as a function of the old belief state and the new information.
		There exists a function $F : \Phi \times \Theta \to \Theta$ that,
		given prior beliefs $\theta$ and new information $\phi$, produces 
		posterior beliefs $\theta' = F(\phi, \theta)$.
		% which incorporates $\phi$ into $\theta$.
\end{CFaxioms}

% In some ways this seems very reasonable:
Given such a function $F$ and a statement $\phi$, we call $F_\phi = F(\phi, -) : \Theta \to \Theta$ an \emph{update}.
% At this point, since we have no way of articulating how confident we are in the new information, the only reasonable thing to
% At face value, \textbf{F} seems very reasonable: what more could your next belief state depend on, aside from your previous one and the new information?
At face value, \textbf{F} seems quite reasonable:
aside from your beliefs and the new information, what more could your next belief state depend on?
% We will soon argue that that there is another aspect worth separating out: the confidence you have in the new information.
% We argue that the confidence you have in the new information is a third aspect, worthy of separate treatment.
We argue that one's \emph{confidence} in the new information $\phi$ is a third consideration,
% (distinct from $\phi$ itself),
 worthy of special treatment.
% We argue that there is a specific 
% Perhaps this is can be
% Nevertheless, we will soon argue that the resulting belief state $\theta'$ should also depend on something else: one's confidence in the statement.
% We argue: one's confidence in the statement.
% Perhaps your confidence in it.
 % after which we will weaken \textbf{F} appropriately.

 

% Suppose for now that our confidence in $\phi$ is not 
Suppose
Since the purpose of $F$ is to \emph{fully} incorporate the new information into our beliefs
(and without a notion of confidence, we can't very well specify how much to incorporate),
then
% it should be the case that a second update with the same information should not affect the belief state.
% a second update with the same information should not affect the belief state.
two successive updates with the same information ought to have the same effect as a single one.
Intuitively, this is because if we have just updated our beliefs to be consistent with the information $\phi$, then a second observation of $\phi$ will require no further alterations of our belief state.
In this case, we call $F$ an \emph{update rule}, or more precisely, a \emph{$\Theta$-update rule on $\Phi$}, and insist that
\begin{CFaxioms}
	\item[\textbf{UR}] For all $\phi \in \Phi$, the update $F_\phi$ is
	% require that it be idempotent.
	idempotent.
    % (i.e., $F_\phi \circ F_\phi = F_\phi$).
\end{CFaxioms}

% In curried form, $F : \Phi \to (\Theta \to \Theta)$.

% We now proceed with the formal details.
% \textbf{Update Rules.}
% Consider a space $\Theta$
% of possible belief states,
% and a set $\Phi$ of statements.
% % and a set $\Phi$ of ``statements'', i.e., the things one can have confidence in.
% % An \emph{update rule} (or more precisely, a \emph{$\Theta$-updating rule on $\Phi$})
% An \emph{update rule}, or more precisely, a \emph{$\Theta$-update rule on $\Phi$},
% is a function of the form
% \[
%     % F :  (\mathbb R \times \Phi) \to \Big( \Theta \to \Theta \Big)
%     F :  \Phi \to \Big( \Theta \to \Theta \Big)
% \]
% % which describes how to update beliefs about $X$, with the new information, at a certain level of trust.
% which describes how to (fully) update beliefs $\Theta$ with new information $\Phi$.
% and for $F$ to be an update rule, we require that , meaning that updating any belief with $\phi$ twice in a row is equivalent to single update.
% Having said that, one reading of this paper is a relaxation of this requirement.
% Here are some examples.
Once $\Theta$, $\Phi$, and any implicit structure in them is specified, there is often a natural choice of update rule.
To illustrate, we now consider three different rules for different choices of $\Phi$.
In each case, the possible belief states $\Theta := \Delta W$ be the set of all probability distributions over a finie set $W = \{w_1, \ldots, w_n\}$ of ``possible worlds''.

\begin{enumerate}
	\item %\textbf{Conditioning.}
	\textbf{Conditioning.}
	First, consider the case where observations are events, i.e., $\Phi := 2^W$.
	% One particularly natural $\Delta W$-update rule on $2^W$ is given by conditioning,
	% Here, the appropriate update seems to be condition:
	Here, the appropriate rule seems to be conditioning:
	\[
	\begin{aligned}
		(-) \smash{\,\Big|\,} (\;\cdot\;) : \qquad 2^W &\to (\Delta W \to \Delta W) \\
		% (-) \smash{\,\Big|\,} (\;\cdot\;) : \qquad 2^W \times \Delta W &\to \Delta W \\
		% (-) \,\Big|\, (\;\cdot\;) : \Delta X \times 2^X \to \Delta X.
		% (\mu \in \Delta X, A \subset X) &\mapsto \mu | A .
		% (A \subset W) &\mapsto (  ~~\mu~~ \mapsto \mu \mid A ),
		A  &\mapsto (  ~\mu~~ \mapsto \mu \mid A ~),
		% (A, \mu) ~&\mapsto~  \mu \mid A\, ,
	\end{aligned}
	% \qquad
	% \qquad
	% \begin{tikzpicture}[center base]
	%     \node[dpad0] (W) {$W$};
	%     \node[dpad0, right=0.5 of W] (Aq) {$A?$};
	%     \draw[arr2, <-] (W) to node[above]{$\mu$} ++(-1, 0);
	%     \draw[arr2, ->>] (W) to node[above,pos=0.3]{} (Aq);
	%     \draw[arr2, <<-] (Aq) to node[above,pos=0.7]{$A!$} ++(1.2,0);
	% \end{tikzpicture}
	\]
	% where $(\mu \mid A)(x) = \frac{\mu(\{x\})}{\mu(A)}$
	% in which learning $A$ maps
	% where the action of the conditional measure $\mu\mid A$ is given by $(\mu \mid A) \{w\} = \ifrac{\mu\{w\}}{\mu(A)}$.
	% where the action of the conditional measure $\mu\mid A$ is given by $(\mu \mid A)(B) = \ifrac{\mu(B \cap A)}{\mu(A)}$, provided $\mu(A) > 0$,
	where the conditional measure $\mu\mid A$ is given by $(\mu \mid A)(B) = \ifrac{\mu(B \cap A)}{\mu(A)}$, provided $\mu(A) > 0$,
	% and may be defined arbitrarily otherwise.
	and otherwise is just equal to $\mu$.
	Observe:
	\begin{itemize}[nosep]
		\item Provided $\mu(A) > 0$, then $(\mu\mid A) \mid A = \mu \mid A$, so conditioning is an update rule.
		\item If $\mu(A \cap B) > 0$, then $(\mu \mid A) \mid B = \mu \mid (A \cap B) = (\mu \mid B) \mid A$, so the order that information is recieved does not matter, so long as that information is consistent with one's beliefs.
	\end{itemize}
	% However, it is not the only $\Delta W$-updating rule for $2^W$.
	 % $\Phi$.

	\item
	\textbf{Imaging.}
	A second example of an update rule is the ``imaging'' approach of David Lewis
	\parencite{lewis1976probabilities}.
	% , albeit in very different notation.
	% Once again, consider a finite set $W$, and belief states $\Theta := \Delta W$.
	Suppose that, for some set $\Phi$, that we already have a $W$-update rule
	$f : \Phi \to (W \to W)$, which we interpret as assigning, to each statement $\phi \in \Phi$ and $w \in W$, a unique world $f_\phi w \in W$ which is the world ``most similar to $w$, in which $\phi$ is true'' \parencite{gardenfors1979imaging}.
	In this case, idempotence of $f_\phi$ amounts to the (very reasonable) requirement that the world most similar to $f_\phi w$ in which $\phi$ is true, is $f_\phi w$ itself.
	From $f$, we can construct a $\Delta W$-update rule by
	\[
    	\begin{aligned}
    		F_\phi(\mu) &:=
    			f^{\sharp}(\mu)
    			% &= A \mapsto \mu( f^{-1}_\phi( A ))\\
    			= A \mapsto \mu(\{w : f(w) \in A\})
    	\end{aligned}
		% \qquad
		% \qquad
		% \begin{tikzpicture}[center base]
		% 	\node[dpad0] (W) {$W$};
		% 	\node[dpad0, right=1 of W] (W') {$W$};
		% 	\node[dpad0, below right=0.2 and 0.2 of W] (Phi) {$\Phi$};
		% 	\mergearr{W}{Phi}{W'}
		% 	\node[above=1pt of center-WPhiW']{$f$};
		% 	\draw[arr2, <-] (W) to node[above]{$\mu$} ++(-1, 0);
		% 	\draw[arr2, <<-] (Phi) to node[below]{$\phi$} ++(-1.3, 0);
		% 	\draw[arr2, <-, dashed, gray] (W') to node[above]{$F_\phi(\mu)$} ++(2, 0);
		% \end{tikzpicture}
	\]
	which intuitively moves the probability mass on each world $w$ to the $f_\phi w$, the closest world to $w$ in which $\phi$ is true.
	% is the pushforward measure of $\mu$ through $f_\phi$, which Lewis calls the ``image of $\mu$ on $\phi$''
	And, since $f$ is idempotent, $F$ will be as well.


	\commentout{
	\item More generally, consider a measurable space $\mathcal W = (W, \mathcal A)$, where $W$ is a set and $\mathcal A$ is a $\sigma$-algebra over $W$, and let $\mathcal F \subset \mathcal A$ be closed under supersets in $\mathcal A$.
	% Now, let $\Theta$ be the set of conditional probabili$

	\TODO[Properly Use Conditional Probability Measure, to define on all events]

	Conditioning a probability distribution $\mu \in \Delta\X$ on an event $A \in \mathcal A$ also makes sense in this more general measure-theoretic setting, at least so long as $\mu(A) > 0$, and is given by
	% the Lebesgue integral
	% \[
	$$
		% (\mu \mid A) (B) = \frac{1}{\mu(A)} \int \mathbf 1_{B}(x)  \mathrm d\mu(x)
		(\mu \mid A) (B) = \frac{\mu(B \cap A)}{\mu(A)}
	$$
	}


	\item \textbf{Jeffrey's Rule.}
	% Once more, suppose that $W$ is a finite set and $\Theta := \Delta W$.
	Next, consider a more general form of observation, in which observations themselves are probabilities.
	% Formally, suppose $\Phi$ consists of pairs $(X,\pi)$,
	% Formally, suppose $\Phi$ consists of marginal distributions $\pi(X)$
	Formally, suppose $\Phi$ consists of distributions $\pi(X)$,
	% written $\pi(X)$,
	where $X : W \to S$ is a random variable,
	% (i.e., some function of $W$),
	and $\pi \in \Delta S$ is a distribution over the possible values that $X$ can take.
	Jeffrey's rule, given by
	\begin{align*}
		% \mathrm{Jeffrey}_{(X,\pi)}
		% \mathrm{Jeffrey}_{\pi(X)}
		\mathrm{J}_{\pi(\mskip-2muX\mskip-2mu)}
		(\mu) &:= \sum_{x \in S} \pi(X{=}x) \;  \mu \big|
            X{=}x
            % \{ w : X(w) = x \}
			\\
			&= A \mapsto \sum_{x \in S} \pi(X{=}x)\, \mu( A \mid X \!= x)
	\end{align*}

	When $\pi(X) = \delta_x$ is a point mass $X=x$, then Jeffrey's Rule simply conditions on the event $X = x$.
	 % but for other choices of $\pi(X)$,
	For this reason, Jeffrey's Rule is sometimes often thought of as a generalization of conditioning that admits for less that complete certainty (i.e., ``low-confidence'' updates), but as we will see, it instead is perhaps better thought of as a high-confidence update on a more expresive class of observations.

	Note that if $\mu' := J_{\pi(X)}(\mu)$ is the result of applying Jeffrey's rule to $\pi(X)$ and $\mu$,
	% then $\pi$ will be fully incorporated (that is, $\mu'(X) = \pi(X)$),
	then $\mu'(X) = \pi(X)$, so $\pi(X)$ has been fully incorporated into $\mu'$, and the old beliefs $\mu(X)$ about $X$ have been completely destroyed by the update.
\end{enumerate}
