\def\stmt{$A$}
% \def\stmt{$\phi$}

% The ability express information with varying confidence is an important aspect of knowledge representation.
% Being able to articulate things with variable confide

% The ability articulate a ``degree of confidence'' is an important aspect of knowledge representation.
The ability articulate
a \emph{degree of confidence}
% is an important aspect of knowledge representation.
is important.
% We give a formal account of a new concept
% Subpoint: it helps avoid a brittleness of always beliving things
% Subpoint: Protects against overconfidence.
% This is helpful to avoid 

Of course, there are many well-established ways of representing uncertainty,
	probability chief among them.
	% and chief among them is probability.
% probability chief among them.
% Indeed, an informal poll of our colleagues suggests that computer scientists use ``confidence'' as a synonym for probability.
% The concept is so dominant that an informal poll of our colleagues suggests that most computer scientists view ``confidence'' as a synonym for probability.
% Indeed, an informal poll of our colleagues suggests that most computer scientists view ``confidence'' as a synonym for probability.
Indeed, many people use ``confidence'' as a synonym for probability.
% In our view, this practice
Although this use of the word is perfectly functional, it seems to have shadowed another conception of confidence---one that is fundementally different, if at first sublty so.
% which seems have been shadowed by the enormous success of probability---one that is fundementally different, if at first subtly so.
% which seems to have been shadowed by the enormous success of probability---a concept that is fundementally different, if at first subtly so.

% , such as belief functions, and most importantly, probabilities.
% In each case, one adopts a more complex belief state,
% The enormous success of probability in particular has had an enormous impact on the way computer scientists talk about
% The success of probability has shadowed other concepts that might well also be called measures of confidence.
% Indeed, it

% There are actually two related, but slightly different notions here.
% % The first is a measurement of
% There is a second setting in which one might interpret the word confidence, as well, in an updating context: it's how seriously you take an input
% Note the difference:




% For us,
Our notion of
confidence is a measure of trust, rather than likelihood,
 and it applies to incoming information, rather than to  a belief state.
Like probability, it is a scale between two extremes.
% While probability ranges from untenable (0) to undeniable (1),
While probability ranges from untenable (0) to undeniable (1),
confidence ranges from completely untrustworthy $(\bot)$ to fully trusted ($\top$).
%
% We apply this notion confidence of updating.
% This paper explores how confidence works in state-updating context.
This paper explores how confidence works in the context of learning.
In this setting, one has some belief state, and recieves inputs, which one might have some degree of confidence in, which is used to modify one's belief state.
Our use of confidence can be viewed as a measure of how seriously to take an input in updating our beliefs.

High confidence is in many ways like high probability: if we really trust a statement \stmt, we should fully incorporate it into our beliefs, and thereby come to believe it with high probability.
Similarly, it only makes sense to be extremely confident in a \stmt\ if you believe that \stmt\ is extremely likely to be true.
Low confidence, on the other hand, is quite different from low probability.
If we have little trust in \stmt, we should \emph{ignore} \stmt, rather than coming to believe that \stmt\ is unlikely.
% To say \stmt\ has low probabilty is  high confidence in the $\lnot$\stmt.
For example, if an adversary tells you something that you happen to already believe,
% is likely to be true
you might say you have low confidence in their statement, but nevertheless ascribe it high probability.

% The approach we take in this paper is very general,
% Our approach is quite general, and applies any time a belief state is modified in resonse to an input.
% one wants to measure (gradual) incorporation of new information into one's beliefs, and may be thought of as a measure of how much of the new information makes it into one's beliefs.

\subsection{Other Conceptions of Confidence.}

\textbf{Probability.}
% Probability is a numerical scale that ranges from untenable (0) to undeniable (1).
% No number on this scale is truly neutral.
% One of the biggest shortcomings of probability is its inability to represent a truly neutral attitude towards a proposition.
Some people do use ``confidence'' to mean the same thing as probability. When they say they have low confidence in $\phi$, they mean that they think $\phi$ is unlikely.

One of the biggest shortcomings of probability is its inability to represent a truly neutral attitude towards a proposition.
%  probability of $\frac12$ .
% This shortcoming has perhaps been the primary selling point of many alternatives to probabiltiy, such as Dempster-Shafer Belief functions.
A value of $\frac12$ may be equally far from zero as it is from one, but is by no means a neutral assessment in all cases: hearing that your favored candidate has a 50\% chance of winning is big news if a win was previously thought to be inevitable.
For this reason, telling someone the odds are 50/50 is quite different from saying you have no idea.
% By contrast, zero confidence represents a truly neutral stance; a statement with zero confidence has no effect.
By contrast, zero confidence represents something truly neutral:
	a statement made with zero confidence does not stake out a claim, and
	a statement recieved with zero confidence does not affect the recipient's beliefs.
Nevertheless, in some contexts, we will see that confidences correspond to to probabilities.

\textit{Opacity.} To use a graphical metaphor, think of certainty as black or white.
Probability describes shades of gray, while confidence describes opacity.
If we are painting with black and start with a white canvas, there is a precise correspondence between the opacity and the resulting shade of gray.

\textbf{Upper and Lower Probabilities.}
Upper and lower probabilities can describe a neutral attitude towards a proposition, but they are not really a specification of trust, but rather a direct specification of a belief state.
It isn't immediately clear how to use these representations of uncertainty to update, and they're a little too complex to function effectively as the primitive measure of trust that we're after.


\textbf{Shafer's Weight of Evidence.}
Shafer's ``weight of evidence'' is precisely the same concept we have in mind.
Our analysis precsely reduces to his, in the setting where belief states are Belief functions (which generalize probabilities, but not, say, neural network weights), and observations are events.
% This paper can be a generalization of Shafer's ``weight of evidence'' to a broader class of settings, where one might have very different belief states and observations.
Thus, this paper can be viewed as generalizing this concept to a broader class of settings, without requiring that one adopt Shafer's conception of a belief state or an observation.


\textbf{Variance and Entropy.}
The inverse of variance, sometimes known as precision,
	is also commonly used to measure confidence.
If a sensor is unreliable and can give a range of answers, the variance of the estimate is a very common way of quantifying this reliablility.
If measurements have zero variance, in some sense one has absolute confidence ($\top$) in the sensor. If measurements have infinite variance, then in some sense one has no confidence in the sensor, since individual samples convey no information about the true value of the quantity measured.
As with probability, inverse variance will coincide with confidence in some settings; we will see how in \cref{sec:variance}.

Entropy, like variance, is a standard way of measuring uncertainty, and in some settings, confidence coincides with entropy (see \cref{sec:entropy}).
The assumption underlying both approaches is that there's some ``true'' value of the variable, and that the randomness is epsistemic (due to sensor errors) rather than aleotoric (inherrent in the quantity being measured).

\textbf{Confidence Intervals and Error Bars.}
Another notion of the word ``confidence'' comes from the term ``confidence interval''.
This concept arises in settings involving a probability distribution $\Pr(X)$ over a metric space $X$, typically $X = \mathbb R$.
A 95\% confidence interval is the (largest) ball containing 95\% of the probability, and its size is a geometric measurement of how .
This intuition behind this reading of the word confidence is the same as
