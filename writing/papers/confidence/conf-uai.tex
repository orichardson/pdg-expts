\documentclass{uai2022} % for initial submission
% \documentclass[accepted]{uai2022} % after acceptance, for a revised
                                    % version; also before submission to
                                    % see how the non-anonymous paper
                                    % would look like
%% There is a class option to choose the math font
% \documentclass[mathfont=ptmx]{uai2022} % ptmx math instead of Computer
                                         % Modern (has noticable issues)
% \documentclass[mathfont=newtx]{uai2022} % newtx fonts (improves upon
                                          % ptmx; less tested, no support)
% NOTE: Only keep *one* line above as appropriate, as it will be replaced
%       automatically for papers to be published. Do not make any other
%       change above this note for an accepted version.

%% Choose your variant of English; be consistent
\usepackage[american]{babel}
% \usepackage[british]{babel}

%% Some suggested packages, as needed:
% \usepackage{natbib} % has a nice set of citation styles and commands
    % \bibliographystyle{plainnat}
    % \renewcommand{\bibsection}{\subsubsection*{References}}
% \usepackage{mathtools} % amsmath with fixes and additions
% \usepackage{siunitx} % for proper typesetting of numbers and units
\usepackage{booktabs} % commands to create good-looking tables
% \usepackage{tikz} % nice language for creating drawings and diagrams

\input{confidence-preamble}
\addbibresource{conf.bib}

%% Provided macros
% \smaller: Because the class footnote size is essentially LaTeX's \small,
%           redefining \footnotesize, we provide the original \footnotesize
%           using this macro.
%           (Use only sparingly, e.g., in drawings, as it is quite small.)

%% additional macros
\newcommand{\ext}[1]{\overline #1} %  measures over phi
\newcommand{\Unif}{\mathrm{Unif}}

\def\cofunc{commitment function}
\def\confdom{\mathdcal C}
\def\ZO{\mathrm{ZO}}
% \def\ZO{[0,1]}
\def\Rplus{\mathbb R_+}

% \let\oldTheta\Theta
% \renewcommand\Theta{\mathdcal{\Theta}}

% \title{Measures of Confidence}
\title{Updating with Confidence}

% The standard author block has changed for UAI 2022 to provide
% more space for long author lists and allow for complex affiliations
%
% All author information is authomatically removed by the class for the
% anonymous submission version of your paper, so you can already add your
% information below.

% Add authors
\author[1]{\href{mailto:<oer5@cornell.edu>?Subject=Confidence}{Oliver E Richardson}{}}
\author[1]{Joseph Y Halpern}
% Add affiliations after the authors
\affil[1]{%
    Computer Science Dept.\\
    Cornell University\\
    Ithaca, New York, USA
}
  
  \begin{document}
\maketitle

\begin{abstract}
 
\end{abstract}

\section{Introduction}\label{sec:intro}

\def\stmt{$A$}
% \def\stmt{$\phi$}

% The ability express information with varying confidence is an important aspect of knowledge representation.
% Being able to articulate things with variable confide

The ability articulate a ``degree of confidence'' is an important aspect of knowledge representation.
% We give a formal account of a new concept
% We are about to propose a concept . 
% Subpoint: it helps avoid a brittleness of always beliving things
% Subpoint: Protects against overconfidence.
Of course, there are many well-established ways of representing uncertainty,
	probability chief among them.
	% and chief among them is probability.
% probability chief among them. 
% Indeed, an informal poll of our colleagues suggests that computer scientists use ``confidence'' as a synonym for probability.
% The concept is so dominant that an informal poll of our colleagues suggests that most computer scientists view ``confidence'' as a synonym for probability.
% Indeed, an informal poll of our colleagues suggests that most computer scientists view ``confidence'' as a synonym for probability.
Indeed, many computer scientists use ``confidence'' as a synonym for probability.
% In our view, this practice 
Although this use of the word is perfectly reasonable, it seems to have shadowed another conception of confidence---one that is fundementally different, if at first sublty so. 
% which seems have been shadowed by the enormous success of probability---one that is fundementally different, if at first subtly so.
% which seems to have been shadowed by the enormous success of probability---a concept that is fundementally different, if at first subtly so.
 
% , such as belief functions, and most importantly, probabilities.
% In each case, one adopts a more complex belief state, 
% The enormous success of probability in particular has had an enormous impact on the way computer scientists talk about 
% The success of probability has shadowed other concepts that might well also be called measures of confidence.
% Indeed, it 

% There are actually two related, but slightly different notions here. 
% % The first is a measurement of
% There is a second setting in which one might interpret the word confidence, as well, in an updating context: it's how seriously you take an input 
% Note the difference: 




For us, confidence is a measure of trust. 
Like probability, it is a scale between two extremes. 
% While probability ranges from untenable (0) to undeniable (1),
While probability ranges from untenable (0) to undeniable (1),
confidence ranges from completely untrustworthy $(\bot)$ to fully trusted ($\top$). 
%
% We apply this notion confidence of updating.
% This paper explores how confidence works in state-updating context.
This paper explores how confidence works in the context of learning.
In this setting, one has some belief state, and recieves inputs, which one might have some degree of confidence in, which is used to modify one's belief state. 
Our use of confidence can be viewed as a measure of how seriously to take an input in updating our beliefs.

High confidence is in many ways like high probability: if we really trust a statement \stmt, we should fully incorporate it into our beliefs, and thereby come to believe it with high probability. 
Similarly, it only makes sense to be extremely confident in a \stmt\ if you believe that \stmt\ is extremely likely to be true. 
Low confidence, on the other hand, is quite different from low probability. 
If we have little trust in \stmt, we should \emph{ignore} \stmt, rather than coming to believe that \stmt\ is unlikely.
% To say \stmt\ has low probabilty is  high confidence in the $\lnot$\stmt. 
For example, if an adversary tells you something that you happen to already believe,
% is likely to be true
you might say you have low confidence in their statement, but nevertheless ascribe it high probability. 

% The approach we take in this paper is very general, 
% Our approach is quite general, and applies any time a belief state is modified in resonse to an input. 
% one wants to measure (gradual) incorporation of new information into one's beliefs, and may be thought of as a measure of how much of the new information makes it into one's beliefs. 

\section{Flow Representation}
\section{Vector Field Representation}
\section{Loss Represntation}

\section{Related Work}
\section{Discussion}


\begin{contributions} % will be removed in pdf for initial submission,
                      % so you can already fill it to test with the
                      % ‘accepted’ class option
    Briefly list author contributions.
    This is a nice way of making clear who did what and to give proper credit.

    H.~Q.~Bovik conceived the idea and wrote the paper.
    Coauthor One created the code.
    Coauthor Two created the figures.
\end{contributions}

\begin{acknowledgements} % will be removed in pdf for initial submission,
                         % so you can already fill it to test with the
                         % ‘accepted’ class option
    Briefly acknowledge people and organizations here.

    \emph{All} acknowledgements go in this section.
\end{acknowledgements}

% \bibliography{uai2022-template}
\printbibliography

\appendix
% NOTE: necessary when ptmx or no mathfont class option is given
\providecommand{\upGamma}{\Gamma}
\providecommand{\uppi}{\pi}

\end{document}
