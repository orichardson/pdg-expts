\def\year{2021}\relax

\documentclass[letterpaper]{article} %
\usepackage[margin=1in]{geometry}
\usepackage{times} %
\usepackage{helvet} %
\usepackage{courier} %
\usepackage[hyphens]{url} %
\usepackage{graphicx} %
\urlstyle{rm} %
\def\UrlFont{\rm} %
\usepackage{graphicx} %
\usepackage{natbib} %
\usepackage{caption} %
\frenchspacing %
\setlength{\pdfpagewidth}{8.5in} %
\setlength{\pdfpageheight}{11in} %

\usepackage{tikz}
	\usetikzlibrary{positioning,fit,calc, decorations, arrows, shapes, shapes.geometric}
	\usetikzlibrary{backgrounds}
	\usetikzlibrary{patterns}
	\usetikzlibrary{cd}
	
	\pgfdeclaredecoration{arrows}{draw}{
		\state{draw}[width=\pgfdecoratedinputsegmentlength]{%
			\path [every arrow subpath/.try] \pgfextra{%
				\pgfpathmoveto{\pgfpointdecoratedinputsegmentfirst}%
				\pgfpathlineto{\pgfpointdecoratedinputsegmentlast}%
			};
	}}
	\tikzset{AmpRep/.style={ampersand replacement=\&}}
	\tikzset{center base/.style={baseline={([yshift=-.8ex]current bounding box.center)}}}
	\tikzset{paperfig/.style={center base,scale=0.9, every node/.style={transform shape}}}

	\tikzset{is bn/.style={background rectangle/.style={fill=blue!35,opacity=0.3, rounded corners=5},show background rectangle}}
	\tikzset{dpadded/.style={rounded corners=2, inner sep=0.7em, draw, outer sep=0.3em, fill={black!50}, fill opacity=0.08, text opacity=1}}
	\tikzset{dpad0/.style={outer sep=0.05em, inner sep=0.3em, draw=gray!75, rounded corners=4, fill=black!08, fill opacity=1}}
	\tikzset{dpad/.style args={#1}{every matrix/.append style={nodes={dpadded, #1}}}}
	\tikzset{light pad/.style={outer sep=0.2em, inner sep=0.5em, draw=gray!50}}
		
	\tikzset{arr/.style={draw, ->, thick, shorten <=3pt, shorten >=3pt}}
	\tikzset{arr0/.style={draw, ->, thick, shorten <=0pt, shorten >=0pt}}
	\tikzset{arr1/.style={draw, ->, thick, shorten <=1pt, shorten >=1pt}}
	\tikzset{arr2/.style={draw, ->, thick, shorten <=2pt, shorten >=2pt}}
	\tikzset{archain/.style args={#1}{arr, every arrow subpath/.style={draw,arr, #1}, decoration=arrows, decorate}}


	\tikzset{fgnode/.style={dpadded,inner sep=0.6em, circle},
	factor/.style={light pad, fill=black}}	
	
	
	\newcommand\cmergearr[4]{
		\draw[arr,-] (#1) -- (#4) -- (#2);
		\draw[arr, shorten <=0] (#4) -- (#3);
	}
	\newcommand\mergearr[3]{
		\coordinate (center-#1#2#3) at (barycentric cs:#1=1,#2=1,#3=1.2);
		\cmergearr{#1}{#2}{#3}{center-#1#2#3}
	}
	\newcommand\cunmergearr[4]{
		\draw[arr,-, , shorten >=0] (#1) -- (#4);
		\draw[arr, shorten <=0] (#4) -- (#2);
		\draw[arr, shorten <=0] (#4) -- (#3);
	}
	\newcommand\unmergearr[3]{
		\coordinate (center-#1#2#3) at (barycentric cs:#1=1.2,#2=1,#3=1);
		\cunmergearr{#1}{#2}{#3}{center-#1#2#3}
	}

	
	\usetikzlibrary{matrix}
	\tikzset{toprule/.style={%
	        execute at end cell={%
	            \draw [line cap=rect,#1] 
	            (\tikzmatrixname-\the\pgfmatrixcurrentrow-\the\pgfmatrixcurrentcolumn.north west) -- (\tikzmatrixname-\the\pgfmatrixcurrentrow-\the\pgfmatrixcurrentcolumn.north east);%
	        }
	    },
	    bottomrule/.style={%
	        execute at end cell={%
	            \draw [line cap=rect,#1] (\tikzmatrixname-\the\pgfmatrixcurrentrow-\the\pgfmatrixcurrentcolumn.south west) -- (\tikzmatrixname-\the\pgfmatrixcurrentrow-\the\pgfmatrixcurrentcolumn.south east);%
	        }
	    },
	    leftrule/.style={%
	        execute at end cell={%
	            \draw [line cap=rect,#1] (\tikzmatrixname-\the\pgfmatrixcurrentrow-\the\pgfmatrixcurrentcolumn.north west) -- (\tikzmatrixname-\the\pgfmatrixcurrentrow-\the\pgfmatrixcurrentcolumn.south west);%
	        }
	    },
	    rightrule/.style={%
	        execute at end cell={%
	            \draw [line cap=rect,#1] (\tikzmatrixname-\the\pgfmatrixcurrentrow-\the\pgfmatrixcurrentcolumn.north east) -- (\tikzmatrixname-\the\pgfmatrixcurrentrow-\the\pgfmatrixcurrentcolumn.south east);%
	        }
	    },
	    table with head/.style={
		    matrix of nodes,
		    row sep=-\pgflinewidth,
		    column sep=-\pgflinewidth,
		    nodes={rectangle,minimum width=2.5em, outer sep=0pt},
		    row 1/.style={toprule=thick, bottomrule},
  	    }
	}


\usepackage{booktabs}       %
\usepackage{amsfonts}       %
\usepackage{nicefrac}       %
\usepackage{microtype}      %
\usepackage{mathtools}		%
\usepackage{amssymb, bbm}


\usepackage{relsize}
\usepackage{environ} %

\usepackage{color}

\usepackage{amsthm}
\usepackage{thmtools}

\theoremstyle{plain}
\newtheorem{theorem}{Theorem}[section]
\newtheorem{coro}{Corollary}[theorem]
\newtheorem{prop}[theorem]{Proposition}
\newtheorem{lemma}[theorem]{Lemma}
\newtheorem{fact}[theorem]{Fact}
\newtheorem{conj}[theorem]{Conjecture}

\theoremstyle{definition}

\declaretheorem[name=Definition,qed=$\square$,numberwithin=section]{defn} %
\declaretheorem[name=Construction,qed=$\square$,sibling=defn]{constr}
\declaretheorem[qed=$\square$]{example}

\theoremstyle{remark}
\newtheorem*{remark}{Remark}

\usepackage{xstring}
\usepackage{enumitem}

\usepackage{environ}
\usepackage{xstring}

% Wow this works I'm brilliant
\def\wrapwith#1[#2;#3]{
	\expandarg\IfSubStr{#1}{,}{
		\expandafter#2{\expandarg\StrBefore{#1}{,}}
		\expandarg\StrBehind{#1}{,}[\tmp]
		\xdef\tmp{\expandafter\unexpanded\expandafter{\tmp}}
		#3
		\wrapwith{\tmp}[#2;{#3}]
	}{ \expandafter#2{#1} }
}
\def\hwrapcells#1[#2]{\wrapwith#1[#2;&]}
\def\vwrapcells#1[#2]{\wrapwith#1[#2;\\]}
\NewEnviron{mymathenv}{$\BODY$}

\newcommand{\smalltext}[1]{\text{\footnotesize#1}}
\newsavebox{\idxmatsavebox}
\def\makeinvisibleidxstyle#1#2{\phantom{\hbox{#1#2}}}
\newenvironment{idxmatphant}[4][\color{gray}\smalltext]{%
	\def\idxstyle{#1}
	\def\colitems{#3}
	\def\rowitems{#2}
	\def\phantitems{#4}
	\begin{lrbox}{\idxmatsavebox}$%$\begin{mymathenv}
	\begin{matrix}  \begin{matrix} \hwrapcells{\colitems}[\idxstyle]  \end{matrix}
		% &\vphantom{\idxstyle\colitems}
		\\[-0.05em]
		\left[
		\begin{matrix}
			\hwrapcells{\phantitems}[\expandafter\makeinvisibleidxstyle\idxstyle]  \\[-1.2em]
	}{
		\end{matrix}\right]		&\hspace{-0.8em}\begin{matrix*}[l] \vwrapcells{\rowitems}[\idxstyle] \end{matrix*}\hspace{0.1em}%
	\end{matrix}%
	$%\end{mymathenv}
	\end{lrbox}%
	\raisebox{0.75em}{\usebox\idxmatsavebox}
%	\vspace{-0.5em}
}

\newenvironment{idxmat}[3][\color{gray}\smalltext]
	{\begingroup\idxmatphant[#1]{#2}{#3}{#3}}
	{\endidxmatphant\endgroup}

\newenvironment{sqidxmat}[2][\color{gray}\smalltext]
	{\begingroup\idxmat[#1]{#2}{#2}}
	{\endidxmat\endgroup}


%%%%%%%%%%%%
% better alignment for cases
\makeatletter
\renewenvironment{cases}[1][l]{\matrix@check\cases\env@cases{#1}}{\endarray\right.}
\def\env@cases#1{%
	\let\@ifnextchar\new@ifnextchar
	\left\lbrace\def\arraystretch{1.2}%
	\array{@{}#1@{\quad}l@{}}}
\makeatother

\newcommand\numberthis{\addtocounter{equation}{1}\tag{\theequation}}

\usepackage{zref-xr}
\zxrsetup{toltxlabel=true, tozreflabel=false}
\zexternaldocument*{pdg-arXiV-stripped}

\usepackage[noabbrev,nameinlink,capitalize]{cleveref}
\crefname{example}{Example}{Examples}
\crefname{defn}{Definition}{Definitions}
\crefname{prop}{Proposition}{Propositions}
\crefname{constr}{Construction}{Constructions}
\crefname{conj}{Conjecture}{Conjectures}
\crefname{fact}{Fact}{Facts}


\usepackage{float}
\newcounter{subfigure}
	\renewcommand\thesubfigure{\thefigure(\alph{subfigure})}
    
\newenvironment{old}[1]{\par\noindent{\bf \Cref{#1}.} \em \noindent}{\par\medskip}


\newcommand{\restate}[2]
	{\medskip\par\noindent{\bf \expandarg\Cref{thmt@@#1}.}%
 	\noindent\begingroup\em #2 \endgroup\par\smallskip}


\allowdisplaybreaks

\newcommand{\begthm}[3][]{\begin{#2}[{name=#1},restate=#3,label=#3]}

\newcommand{\createversion}[2][{gray}{0.75}]{
	\definecolor{v#2color}#1\relax
    \expandafter\xdef\csname v#2on\endcsname{%
	}
	\expandafter\xdef\csname v#2off\endcsname{
	}
}
\createversion{test}


\definecolor{vfullcolor}{gray}{0.7}
\newcommand\vfull[1]{{\color{vfullcolor} #1}}
\renewcommand\vfull[1]{} %

\definecolor{vleftoverscolor}{gray}{0.85}
\newcommand{\vleftovers}[1]{{\color{vleftoverscolor} #1}} 
\renewcommand{\vleftovers}[1]{} %

\definecolor{notationcolor}{rgb}{0.9,0.9,.9} 
\newcommand{\notation}[1]{{\color{notationcolor} #1}}
\renewcommand{\notation}[1]{\ignorespaces} %

\definecolor{contentiouscolor}{rgb}{0.7,0.3,.1} 
\newcommand{\commentout}[1]{\ignorespaces} 

\newcommand{\valpha}[1]{#1}





\DeclarePairedDelimiterX{\bbr}[1]{[}{]}{\mspace{-3.5mu}\delimsize[#1\delimsize]\mspace{-3.5mu}}
\DeclarePairedDelimiter{\norm}{\lVert}{\rVert}

\let\Horig\H
\let\H\relax
\DeclareMathOperator{\H}{\mathrm{H}} %
\DeclareMathOperator{\I}{\mathrm{I}} %
\DeclareMathOperator*{\Ex}{\mathbb{E}} %
\DeclareMathOperator*{\argmin}{arg\;min}
\newcommand{\CI}{\mathrel{\perp\mspace{-10mu}\perp}} %
\newcommand\mat[1]{\mathbf{#1}}
\DeclarePairedDelimiterX{\infdivx}[2]{(}{)}{%
	#1\;\delimsize\|\;#2%
}
\newcommand{\thickD}{I\mkern-8muD}
\newcommand{\kldiv}{\thickD\infdivx}


\newcommand{\todo}[1]{{\color{red}\ \!\Large\smash{\textbf{[}}{\normalsize\textsc{todo:} #1}\ \!\smash{\textbf{]}}}}
\newcommand{\note}[1]{{\color{blue}\ \!\Large\smash{\textbf{[}}{\normalsize\textsc{note:} #1}\ \!\smash{\textbf{]}}}}



\newcommand\Set{\mathbb{S}\mathrm{et}}
\newcommand\FinSet{\mathbb{F}\mathrm{in}\mathrm{S}\mathrm{et}}
\newcommand\Meas{\mathbb{M}\mathrm{eas}}
\newcommand\two{\mathbbm 2}


\DeclarePairedDelimiterXPP{\SD}[1]{}{[}{]}{_{\text{sd}}}{\mspace{-3.5mu}\delimsize[#1\delimsize]\mspace{-3.5mu}}
		
\newcommand{\none}{\bullet}

\def\sheq{\!=\!}
\DeclareMathOperator\dcap{\mathop{\dot\cap}}
\newcommand{\tto}{\rightarrow\mathrel{\mspace{-15mu}}\rightarrow}

\newcommand{\bp}[1][L]{\mat{p}_{\!_{#1}\!}}
\newcommand{\V}{\mathcal V}
\newcommand{\N}{\mathcal N}
\newcommand{\Ed}{\mathcal E}
\newcommand{\pdgvars}[1][]{(\N#1, \Ed#1, \V#1, \mat p#1, \beta#1)}


\DeclareMathAlphabet{\mathdcal}{U}{dutchcal}{m}{n}
\DeclareMathAlphabet{\mathbdcal}{U}{dutchcal}{b}{n}
\newcommand{\dg}[1]{\mathbdcal{#1}}
\newcommand{\var}[1]{\mathsf{#1}}
\newcommand\Pa{\mathbf{Pa}}

\newcommand{\IDef}[1]{\mathit{IDef}_{\!#1}}

\newcommand\Inc{\mathit{Inc}}
\newcommand{\PDGof}[1]{{\dg M}_{#1}}
\newcommand{\UPDGof}[1]{{\dg N}_{#1}}
\newcommand{\WFGof}[1]{\Psi_{{#1}}}
\newcommand{\FGof}[1]{\Phi_{{#1}}}
\newcommand{\Gr}{\mathcal G}
\newcommand\GFE{\mathit{G\mkern-4mu F\mkern-4.5mu E}}
\newcommand{\varsNV}[1][\N,\V]{(#1)}



\newcommand{\ed}[3]{#2\!%
  \overset{\smash{\mskip-5mu\raisebox{-1pt}{$\scriptscriptstyle
        #1$}}}{\rightarrow}\! #3} 
\newcommand{\alle}[1][L]{_{ \ed {#1}XY}}


\begin{document}
\appendix
\onecolumn
\section{Proofs} \label{sec:proofs}
		For brevity, we use the standard notation and write $\mu(x, y)$
	instead of $\mu(X \!=\! x, Y \!=\! y)$, $\mu(x \mid y)$ instead of
	$\mu(X \!=\! x\mid Y \!=\! y)$, and so forth.

	\commentout{
\begin{defn}[Conditional Entropy]
	If $p$ is a distribution over a set $\Omega$ of out comes, and $X$ and $Y$ are random variables on $\Omega$, then the \emph{conditional entropy}, $\H_p(X \mid Y)$, is defined as 

\end{defn}

	\begin{defn}[Sets as Variables] \label{def:set-rv}
	Sets of random variables as random variables. If $S$ is a set of random variables $X_i : \Omega \to \V(X_i)$ on the same set of outcomes $\Omega$, we consider $S$ itself to be the random variable taking values $\V(X) = \{(x_1, \ldots, x_i \ldots) \}$ for $x_i \in \V(X_i)$. Formally, we define its value on a world $\omega$ to be $S(\omega) := (X_1(\omega), \ldots, X_i(\omega), \ldots)$. 
\end{defn}

\begin{defn}[Strong Convexity] \label{def:strong-convexity}
	A real-valued function is $m$-\emph{strongly convex}, if there is a quadratic lower bound, with coefficient $m$, away from its first order approximation. More precisely, it is $m$ strongly convex if for every $x, y$ in its domain, 
	\[ f(y) \geq f(x) + \Big\langle\nabla f(x), y-x \Big\rangle + m\norm{x-y}^2_2 \]
\end{defn}

\begin{prop}\label{prop:neg-ent-convex}
  Negative entropy, restricted to a finite probability
			simplex, is 1-strongly convex. 
\end{prop}
\begin{proof}
	Let $X$ be a finite set; the function $f: \Delta(X) \to \mathbb R$ given by $\vec x \mapsto \sum x_i \log x_i$ is strongly convex, as 
	\begin{equation*}
		\partial_j f(\vec x) =  \partial_j\left[\sum_i x_i \log x_i \right] = 
			x_j \partial_j \big[\log x_j \big] + \log x_j = 1 + \log x_j
	\end{equation*}
	So
	\begin{align*}
		\Big\langle \nabla f(x) - \nabla f(y),~ x-y\Big\rangle 
			&= \sum_i \Big((\partial_i f)(\vec x) - (\partial_i f)(\vec y)\Big)(x_i - y_i) \\
			&= \sum_i \Big(\log x_i  - \log y_i \Big)(x_i - y_i) \\
		\intertext{As $\log$ is concave, we have $\log(y_i) \leq \log(x_i) + (y_i-x_i) \frac{\mathrm d}{\mathrm d x_i} [\log(x_i)]$, and so $\log x_i - \log y_i \geq (1/x) (x - y)  \geq (x-y)$, we have}
		\Big\langle \nabla f(x) - \nabla f(y),~ x-y\Big\rangle
			&= \sum_i \Big(\log x_i  - \log y_i \Big)(x_i - y_i) \\ %
			&\geq \sum_i (x_i-y_i)^2 \cdot \frac1{x_i}\\
			&\geq \sum_i (x_i-y_i)^2 \\
			&= \norm{x-y}^2_2 \numberthis\label{proofeqn:strong1}
	\end{align*}
	At the same time, the condition for convexity can be phrased in terms of gradients as the condition that for all $x,y$,
	\[  \Big\langle \nabla f(x) - \nabla f(y),~ x-y\Big\rangle \geq 0\]
	So together with \eqref{proofeqn:strong1}, we conclude that the function $f - \norm{x-y}^2_2$ is convex. Therefore, $f$ is 1-strongly convex.
\end{proof}

	}
	
\subsection{Properties of Scoring Semantics}

\vleftovers{
	\thmsetconvex*
	\begin{proof}
		Choose any two distributions $p, q \in \SD{M}$
	consistent with $M$, any mixture coefficient $\alpha \in
	[0,1]$, and any edge $(A,B) \in \Ed$. 

		By the definition of $\SD{M}$, we have $p(B = b \mid A = a) = q(B = b \mid A = a) = \bmu_{A,B}(a,b)$.  
		For brevity,j we will use little letters ($a$) in place of events ($A = a$).
		Therefore, $p(a\land b) = \bmu_{A,B}(a,b) p(a)$ and $q(ab) = \bmu_{A,B}(a,b) q(a)$. Some algebra reveals:
		\begin{align*}
			\Big( \alpha p + (1-\alpha) q \Big) (B = b \mid A = a) &= 
			\frac{\Big( \alpha p + (1-\alpha) q \Big) (b \land a)}{\Big( \alpha p + (1-\alpha) q \Big) (a)} \\
			&= \frac{ \alpha p(b \land a) + (1-\alpha) q(b \land a) }{\Big( \alpha p(a) + (1-\alpha) q (a)} \\
			&= \frac{ \alpha \bmu_{A,B}(a,b) p(a) + (1-\alpha) \bmu_{A,B}(a,b) q(a) }{\Big( \alpha p(a) + (1-\alpha) q (a)} \\
			&=\bmu_{A,B}(a,b) \left(\frac{ \alpha  p(a) + (1-\alpha) q(a) }{\Big( \alpha p(a) + (1-\alpha) q (a)}\right)\\
			&= \bmu_{A,B}(a,b)
		\end{align*}
		and so the mixture $\Big(\alpha p + (1-\alpha) q \Big)$ is also contained in $\SD{M}$.
	\end{proof}
}
In this section, we prove the properties of scoring functions that we
mentioned in the main text,
Propositions~\ref{prop:sd-is-zeroset}, \ref{prop:sem3}, and
\ref{prop:consist}.  We repeat the statements for the reader's convenience.

\restate{prop:sd-is-zeroset}{
$\SD{\dg M} \!= \{ \mu : \bbr{\dg M}_0(\mu) \!=\! 0\}$ for all $\dg M$.
}
\begin{proof}
	 By taking $\gamma = 0$, the score is just $\Inc$. By
			 definition, a distribution $\mu \in \SD{\dg M}$ satisfies
	  all the
			 constraints, so $\mu(Y = \cdot \mid X=x) =
			 \bp(x)$ for all edges $X \rightarrow Y \in \Ed^{\dg
			   M}$ and $x$ with 
			 $\mu(X=x)>0$. By Gibbs inequality
			 \cite{mackay2003information}, 
			 $\kldiv{\mu(Y|x)}{\bp(x)} = 0$. Since this is true
			 for all edges, we must have $\Inc_{\dg M}( \mu) =
			 0$. Conversely, if $\mu \notin \SD{\dg M}$, then it
			 fails to marginalize to the cpd $\bp$ on some edge
							  $L$, and so again by Gibbs inequality,
			 $\kldiv{\mu(Y|x)}{\bp(x)} > 0$. As relative entropy
			 is non-negative, the sum of these terms over all
			 edges must be positive as well, and so $\Inc_{\dg M}(
			 \mu) \neq 0$. %
\end{proof}


Before proving the remaining results, we prove a lemma that will be useful
in other contexts as well. 

\begin{lemma}
	\label{thm:inc-convex}
	$\Inc_{\dg M}( \mu)$ is a convex function of $\mu$.
\end{lemma}
\begin{proof}
    It is well known that $\thickD$ is convex \cite[Theorem
            2.7.2]{coverThomas}, in the sense that  
	\[ \kldiv{\lambda q_1 + (1-\lambda) q_2 }{ \lambda p_1
			  + (1-\lambda) p_2} \leq \lambda \kldiv {q_1}{ p_1} +
							(1-\lambda) \kldiv{q_2}{p_2}. \] 
Given an edge $L \in \Ed$ from $A$ to $B$ and $a \in \mathcal V(A)$,
and   
setting $q_1 = q_2 = \bp(a)$, we get that
	\[ \thickD(\bp(a) \ ||\ \lambda p_1 + (1-\lambda) p_2)
			\leq \lambda \thickD (\bp(a) \ ||\ p_1) + (1-\lambda)
							\thickD(\bp(a)\ ||\ p_2). \] 
	Since this is true for every $a$ and edge, we can take
		   a weighted sum of these inequalities for each $a$
		   weighted by $p(A=a)$; thus, 
	\begin{align*}
		\Ex_{a\sim p_A} \kldiv{\bp(a)}{\lambda p_1 +
			(1-\lambda) p_2} &\leq 
			 \Ex_{a\sim p_A}\lambda \kldiv {\bp(a)}{p_1} +
											(1-\lambda)
                         			 \kldiv{\bp(a)}{p_2}. \\
                        \intertext{Taking a sum over all edges, we get
                        that}
					\sum_{(A, B) \in \Ed}\mskip-10mu\Ex_{a\sim p_A} \kldiv{\bp(a) }{\lambda p_1 + (1-\lambda) p_2} 
			&\leq \sum_{(A, B) \in
							  \Ed}\mskip-10mu\Ex_{a\sim p_A}\lambda
							\kldiv{\bp(a)}{p_1} + (1-\lambda)
							\kldiv{\bp(a)}{p_2}. \\
    \shortintertext{It follows that} 
		\Inc_{\dg M}( \lambda p_1) + (1-\lambda)p_2)
					&\leq \lambda \Inc_{\dg M}(p_1) + (1-\lambda)
					\Inc_{\dg M}(p_2). 
	\end{align*}
	Therefore, $\Inc_{\dg M}( \mu)$ is a convex function of $\mu$.
\end{proof}

The next proposition gives us a useful representation of $\bbr{M}_\gamma$.
\restate{prop:nice-score}{
Letting $x^{\mat w}$ and $y^{\mat w}$ denote the values of
 $X$ and $Y$, respectively, in $\mat w \in \V(\dg M)$, 
we have 
\begin{equation*}
\begin{split}
\bbr{\dg M}(\mu) =  \Ex_{\mat w \sim \mu}\! \Bigg\{
\sum_{ X \xrightarrow{\!\!L} Y  }
\bigg[\,
	 \!\beta_L \log \frac{1}{\bp(y^{\mat w} |x^{\mat w})}
   +
(\valpha{\alpha_L}\gamma - \beta_L ) \log \frac{1}{\mu(y^{\mat w} |x^{\mat w})} 
 \bigg] - 
	\gamma \log \frac{1}{\mu(\mat w)}
   \Bigg\} .
\end{split}
\end{equation*}
}
\begin{proof}
We use the more general formulation of $\IDef{}$ given
in \Cref{sec:expfam}, in which each edge $L$'s conditional  
information is weighted by $\alpha_L$.
  \begin{align*}
	\bbr{\dg M}_\gamma(\mu) &:= \Inc_{\dg M}( \mu) + \gamma \IDef{\dg M}(\mu) \\
		&= \left[\sum\alle \beta_L \Ex_{x\sim \mu_X}\kldiv[\Big]{ \mu(Y | X \sheq x) }{\bp(x) } \right]  + \gamma \left[\sum\alle \alpha_L \H_\mu(Y\mid X) ~-\H(\mu)\right]\\
		&= \sum\alle 
			\Ex_{x \sim \mu_{\!_X}}  \left[ \beta_L\; \kldiv[\Big]{ \mu(Y \mid x) }{\bp(Y \mid x) } + \gamma \; \alpha_L \H(Y \mid X\sheq x) \right]  - \gamma \H(\mu) \\ 
		&= \sum\alle 
			\Ex_{x \sim \mu_{\!_X}}  \left[ \beta_L\; \left(\sum_{y \in \V(Y)} \mu(y \mid x) \log\frac{\mu(y\mid x)}{\bp(y\mid x)}\right) + \alpha_L\gamma \; \left(\sum_{y \in \V(Y)} \mu(y\mid x) \log \frac{1}{\mu(y\mid x)} \right) \right]  - \gamma  \H(\mu) \\ 
		&= \sum\alle 
			\Ex_{x \sim \mu_{\!_X}}  \left[ \sum_{y \in \V(Y)} \mu(y \mid x) \left(  \beta_L\; \log\frac{\mu(y\mid x)}{\bp(y\mid x)} + \alpha_L \gamma \; \log \frac{1}{\mu(y\mid x)} \right) \right]  - \gamma  \H(\mu) \\
		&= \sum\alle 
			\Ex_{x \sim \mu_{\!_X}}  \left[ \Ex_{y \sim \mu(Y \mid X=x)} \left(  \beta_L\; \log\frac{\mu(y\mid x)}{\bp(y\mid x)} + \alpha_L \gamma \; \log \frac{1}{\mu(y\mid x)} \right) \right]  - \gamma \sum_{\mat w \in \V(\dg M)} \mu(\mat w) \log \frac{1}{\mu(\mat w)} \\  
		&= \sum\alle 
			\Ex_{x,y \sim \mu_{\!_{XY}}}  \left[ \beta_L\; \log\frac{\mu(y\mid x)}{\bp(y\mid x)} + \alpha_L\gamma \; \log \frac{1}{\mu(y\mid x)}  \right]  - \gamma  \Ex_{\mat w \sim \mu} \left[ \log \frac{1}{\mu(\mat w)}\right] \\
		&= \Ex_{\mat w \sim \mu} \Bigg\{   \sum_{ X \xrightarrow{\!\!L} Y  } \left[
			\beta_L \log \frac{1}{\bp(y\mid x)}   - \beta_L  \log \frac{1}{\mu(y \mid x)}+ \alpha_L\gamma \log \frac{1}{\mu(y \mid x)} \right]\Bigg\}  -  \gamma  \Ex_{\mat w \sim \mu} \left[\log \frac{1}{\mu(\mat w)}\right] \\
		&=  \Ex_{\mat w \sim \mu} \Bigg\{ \sum_{ X \xrightarrow{\!\!L} Y  } \left[
			\beta_L \log \frac{1}{\bp(y\mid x)} +
	                        (\alpha_L\gamma - \beta_L ) \log
	                        \frac{1}{\mu(y \mid x)} \right] -
	                        \gamma \log \frac{1}{\mu(\mat w)}  \Bigg\}.  
	\end{align*}
\end{proof}

We can now prove         Proposition~\ref{prop:sem3}.
\restate{prop:sem3}{
If $\dg M$ is a PDG and $0 < \gamma \leq \min_L \nicefrac{\beta_L^{\dg M}}{\alpha_L^{\dg M}}$, then
$\bbr{\dg M}_\gamma^*$ is a singleton. 
}
\begin{proof}
It suffices to show that $\bbr{\dg
			  M}_\gamma$ is a strictly convex function of $\mu$,
since every strictly convex function has a unique minimum.
Note that
\begin{align*}
\bbr{M}_\gamma(\mu) 
	&= \Ex_{\mat w \sim \mu} \Bigg\{   \sum_{ X \xrightarrow{\!\!L} Y  } \left[
		\beta_L \log \frac{1}{\bp(y\mid x)} + (\valpha{\alpha_L}\gamma - \beta_L ) \log \frac{1}{\mu(y \mid x)} \right] - \gamma \log \frac{1}{\mu(\mat w)} \Bigg\} \\
	&= \Ex_{\mat w \sim \mu} \Bigg\{   \sum_{ X \xrightarrow{\!\!L} Y  } \left[ \gamma \valpha{\alpha_L} \log \frac{1}{\bp(y\mid x)} + 
		(\beta_L - \valpha{\alpha_L} \gamma) \log \frac{1}{\bp(y\mid x)} - (\beta_L - \valpha{\alpha_L} \gamma) \log \frac{1}{\mu(y \mid x)} \right] - \gamma \log \frac{1}{\mu(\mat w)} \Bigg\}  \\
	&= \Ex_{\mat w \sim \mu} \Bigg\{   \sum_{ X \xrightarrow{\!\!L} Y  } \left[ \gamma \valpha{\alpha_L} \log \frac{1}{\bp(y\mid x)} + 
		(\beta_L - \valpha{\alpha_L} \gamma) \log \frac{\mu(y\mid x)}{\bp(y\mid x)} \right] - \gamma \log \frac{1}{\mu(\mat w)} \Bigg\} \\
	&=  \sum_{ X \xrightarrow{\!\!L} Y  } \left[ \gamma \valpha{\alpha_L} \Ex_{x,y \sim \mu_{\!_{XY}}} \left[ \log \frac{1}{\bp(y\mid x)} \right] + 
		(\beta_L - \valpha{\alpha_L} \gamma) \Ex_{x\sim\mu_X}
          \kldiv[\Big]{\mu(Y\mid x)}{\bp( x)} \right] - \gamma \H(\mu). 
\end{align*}
	The first term, 
	\( \Ex_{x,y \sim \mu_{\!_{XY}}} \left[-\log {\bp(y\mid x)}\right] \) 
	is linear in $\mu$, as $\bp(y\mid x)$ does not depend on $\mu$. %
As for the second term, it is well-known that KL divergence is convex, in the sense that 
	\[ \kldiv{\lambda q_1 + (1-\lambda) q_2 }{ \lambda p_1 +
          (1-\lambda) p_2} \leq \lambda \kldiv {q_1}{ p_1} +
                (1-\lambda) \kldiv{q_2}{p_2}. \] 
	Therefore, for a distribution on $Y$, setting $p_1 =
 p_2 = \bp(x)$, for all conditional marginals $\mu_1(Y \mid X=x)$ and
			$\mu_2(Y\mid X=x)$,
	\[ \kldiv{\lambda \mu_1(Y\mid x) + (1-\lambda)
			  \mu_2(Y\mid x) }{ \bp(x) } \leq \lambda \kldiv
			   {\mu_1(Y\mid x)}{\bp(x)} + (1-\lambda)
								  \kldiv{\mu_2(Y\mid x)}{\bp(x)}. \] 
	So $\kldiv*{\mu(Y\mid x)}{\bp( x)}$ is convex. As
			convex combinations of convex functions are convex,
			the second term, $\Ex_{x\sim\mu_X}\kldiv*{\mu(Y\mid
			  x)}{\bp( x)}$, is convex.
Finally, negative entropy is well known to be strictly convex.                

			Any non-negative linear combinations of the three
			terms is convex, and if this combination applies a
			positive coefficient to the (strictly convex) negative entropy,
			it must be strictly convex. Therefore, as
			long as $\beta_L \geq \gamma$ for all edges $L \in
			\Ed^{\dg M}$, $\bbr{\dg M}_\gamma$ is
strictly convex.  The result follows.
\end{proof}


We next prove \Cref{prop:limit-uniq}.  The first step is provided by the
following lemma.
\begin{lemma}\label{lem:gamma2zero}
 $\lim\limits_{\gamma\to0}\bbr{\dg M}_\gamma^* \subseteq \bbr{\dg M}_0^*$. 
\end{lemma}
\begin{proof}
\def\lb{k}
\def\ub{K}  

Since $\IDef{\dg M}$ is a finite weighted sum of entropies
and conditional entropies over the variables $\N^{\dg M}$, which have
finite support%
, it is bounded.
Thus, there exist bounds $k$ and $K$ depending only on $\N^{\dg M}$ and
$\V^{\dg M}$, such that $\lb \leq \IDef{\dg M}(\mu) \leq \ub$ for all $\mu$.
Since $\bbr{\dg M}_\gamma = \Inc_{\dg M} + \gamma \IDef{\dg M}$,
it follows that, for all $\mu \in \V(\dg M)$, we have
\[ \Inc_{\dg M}( \mu) + \gamma\lb \leq~ \bbr{\dg M }_\gamma(\mu) 
\leq~  \Inc_{\dg M}( \mu) + \gamma\ub. \]
For a fixed $\gamma$, since this inequality holds for all $\mu$, and
both $\Inc$ and $\IDef{}$ are bounded below, it must be the case that  
\[
\min_{\mu \in \Delta\V(\dg M)} \Big[ \Inc_{\dg M}( \mu) + \gamma\lb \Big]
~\leq~ \min_{\mu \in \Delta\V(\dg M)}\bbr{\dg M }_\gamma(\mu) ~\leq~
\min_{\mu \in \Delta\V(\dg M)} \Big[ \Inc_{\dg M}( \mu) + \gamma\ub
    \Big], \] 
even though the distributions that minimize each expression will in general be different.
Let $\Inc(\dg M) = \min_{\mu} \Inc_{\dg M}(\mu)$.
Since $\Delta\V(\dg M)$ is compact, the minimum of the middle term is
achieved.  
Therefore, for $\mu_\gamma \in \bbr{\dg M}^*_\gamma(\mu)$ that
minimizes it, we have 
$$\Inc(\dg M) +\gamma \lb \le \bbr{\dg M }_\gamma(\mu_\gamma) \le
		 \Inc(\dg M) +\gamma \ub$$ for all $\gamma \ge 0.$
Now taking the limit as $\gamma\rightarrow 0$ from above, we get that
$\Inc(\dg M) = \bbr{\dg M }_0(\mu^*)$.
Thus, $\mu^* \in \bbr{\dg M}_0^*$, as desired.
\commentout{

		\begin{alignat*}{4}\relax
			&\forall\gamma,\mu.~&\gamma\lb &~\leq~& \gamma\IDef{\dg M}(\mu)  &~\leq~&  \gamma\ub \\
			&\forall\gamma,\mu.~&
			\Inc_{\dg M}( \mu) + \gamma\lb &~\leq~& \Inc_{\dg M}( \mu) +& \gamma\IDef{\dg M}(\mu)  &~\leq~&  \Inc_{\dg M}( \mu) + \gamma\ub \\
			&\forall\gamma,\mu.~&
			\Inc_{\dg M}( \mu) + \gamma\lb &~\leq~& \bbr{\dg M }_\gamma&(\mu)  &~\leq~&  \Inc_{\dg M}( \mu) + \gamma\ub \\


\intertext{Since this holds for every $\mu$,
 it in particular must hold for the minimum
						 across all $\mu$, which must be achiveved as
						 $\Inc$ and $\IDef{}$ are bounded below and
						 continuous, and $\Delta\V(\dg M)$ is
						 compact.}




  \implies
		&\forall\gamma.~& 
			\min_{\mu \in \Delta\V(\dg M)} \Big[ \Inc_{\dg M}( \mu) + \gamma\lb \Big]&~\leq~& 
				\min_{\mu \in \Delta\V(\dg M)}& \bbr{\dg M }_\gamma(\mu)  &~\leq~&  
				\min_{\mu \in \Delta\V(\dg M)} \Big[ \Inc_{\dg M}( \mu) + \gamma\ub \Big]\\
		&\forall\gamma.~&
			\min_{\mu \in \Delta\V(\dg M)} \Big[ \Inc_{\dg M}( \mu)\Big] + \gamma\lb &~\leq~& 
				\min_{\mu \in \Delta\V(\dg M)}& \bbr{\dg M }_\gamma(\mu)  &~\leq~&  
				\min_{\mu \in \Delta\V(\dg M)} \Big[ \Inc_{\dg M}( \mu) \Big] + \gamma\ub\\
		&\forall\gamma.~&
			\Inc(\dg M) + \gamma\lb &~\leq~& 
				\min_{\mu \in \Delta\V(\dg M)}& \bbr{\dg M }_\gamma(\mu)  &~\leq~&  
				\Inc(\dg M) + \gamma\ub\\
		\intertext{Since this holds for all $\gamma$, it must
				  hold in the limit as $\gamma \to 0$ from above.}
		&&
			\Inc(\dg M) + \lim_{\gamma\to 0} [\gamma\lb ]&~\leq~& 
				\lim_{\gamma\to 0}\min_{\mu } &\bbr{\dg M }_\gamma(\mu)  &~\leq~&  
				\Inc(\dg M) + \lim_{\gamma\to 0} [\gamma\ub] \\
		&&
			\Inc(\dg M) &~\leq~& 
				\lim_{\gamma\to 0}\min_\mu & \bbr{\dg M }_\gamma(\mu)  &~\leq~&  
				 \Inc(\ M)\\
	\end{alignat*}
		Therefore, we must have
		\[\lim_{\gamma\to 0}\min_\mu \bbr{\dg M }_\gamma(\mu) = \Inc(\dg M) \]
		and in particular, $\lim_{\gamma\to 0}\min_\mu
				\bbr{\dg M }_\gamma(\mu) = 0$ when
$\dg M$ is consistent, by \Cref{prop:sd-is-zeroset}. Therefore all distributions
$\mu_* \in \lim_{\gamma \to 0}\argmin_\mu \bbr{\dg M}_\gamma(\mu)$
must satisfy $\bbr{\dg M}_0(\mu_*) = 0$, and thus $\mu_* \in \SD{\dg
M}$. 
}
\end{proof}

We now apply Lemma~\ref{lem:gamma2zero} to show that the limit as
$\gamma \to 
0$ is unique, as stated in \Cref{prop:limit-uniq}. 
\restate{prop:limit-uniq}{
	For all $\dg M$, $\lim_{\gamma\to0}\bbr{\dg M}_\gamma^*$ is a singleton.
}
\begin{proof}
First we show that $\lim_{\gamma \to 0}\bbr{\dg M}_\gamma^*$ cannot be empty.
Let $(\gamma_n) = \gamma_1, \gamma_2, \ldots$ be a sequence of
positive reals 
converging to zero.  For all $n$, choose some $\mu_n \in \bbr{\dg
M}_{\gamma_n}^*$. Because $\Delta\V(\dg M)$ is a compact metric
space, it is sequentially compact, and so, by the
Bolzano–Weierstrass Theorem, the sequence $(\mu_n)$ has at least one
accumulation point, say $\nu$. By our definition of the limit, $\nu \in
\lim_{\gamma\to0}\bbr{\dg M}_\gamma^*$, as witnessed by the sequence
$(\gamma_n, \mu_n)_n$.  It follows that $\lim_{\gamma\to0}\bbr{\dg
  M}_\gamma^* \ne \emptyset$.

Now, choose $\nu_1, \nu_2  \in  \lim_{\gamma\to0}\bbr{\dg
  M}_\gamma^*$. 
Thus, there are subsequences $(\mu_{i})$ and $(\mu_{j})$ of
$(\mu_n)$ converging
to $\nu_1$ and $\nu_2$, respectively.
By \Cref{lem:gamma2zero}, $\nu_1, \nu_2 \in \bbr{\dg M}_0^*$, so
$\Inc_{\dg M}(\nu_1) = \Inc_{\dg M}(\nu_2)$.  
Because  $(\mu_{j_n}) \to \nu_1$, $(\mu_{k_n}) \to \nu_2$, and
$\IDef{\dg M}$ is
continuous on $\Delta\V(\dg M)$,
we conclude that  
$(\IDef{\dg M}(\mu_{i}))\to \IDef{\dg M}(\nu_1)$ and
$(\IDef{\dg M}(\mu_{j}))\to \IDef{\dg M}(\nu_2)$.

Suppose that $\IDef{\dg
M}(\nu_1) \neq \IDef{\dg M}(\nu_2)$. Without loss of generality,
suppose that $\IDef{\dg M}(\nu_1) > \IDef{\dg M}(\nu_2)$. 
Since $(\IDef{\dg M}(\mu_{i})) \to \IDef{\dg M}(\nu_1)$, there exists some $i^*
\in \mathbb N$ such that for all $i > i^*$,  
$ \IDef{\dg M}(\mu_{i}) >  \IDef{\dg M}(\nu_2) $.
But then for all $\gamma$ and $i > i^*$, we have 
\[ \bbr{\dg M}_\gamma(\mu_i) = \Inc(\mu_i) + \gamma\IDef{\dg M}(\mu_i)
> \Inc(\nu_2)  
+ \gamma \IDef{\dg M}(\nu_2) = \bbr{\dg M}_\gamma(\nu_2),\]
contradicting the assumption that $\mu_{i}$ minimizes
$\bbr{\dg M}_{\gamma_{i}}$. We thus conclude that we
cannot have $\IDef{\dg M}(\nu_1) > \IDef{\dg M}(\nu_2)$.  By the same
argument, we also cannot have $\IDef{\dg M}(\nu_1) < \IDef{\dg
  M}(\nu_2)$, so $\IDef{\dg M}(\nu_1) =\IDef{\dg M}(\nu_2)$.  
  
Now, suppose that $\nu_1$ and $\nu_2$ distinct. Since $\bbr{\dg M}_\gamma$
is strictly convex for $\gamma > 0$, among the possible convex
combinations of $\nu_1$ and $\nu_2$, the distribution $\nu_3 = \lambda
\nu_1 + (1-\lambda) \nu_2$ that minimizes $\bbr{\dg M}_\gamma$ must
lie strictly between $\nu_1$ and $\nu_2$. 
Because $\Inc$ itself is convex and $\Inc_{\dg M}(\nu_1) = \Inc_{\dg
  M}(\nu_2) =: v$, we must have $\Inc_{\dg M}(\nu_3) \le v$. 
But since
$\nu_1,\nu_2 \in \bbr{\dg M}_0^*$ minimize $\Inc$,
we must have $\Inc_{\dg M}(\nu_3) \ge v$.
Thus, $\Inc_{\dg M}(\nu_3) = v$. 
Now, because, for all  $\gamma > 0$,
\[ \bbr{\dg M}_\gamma(\nu_3) = v + \gamma \IDef{\dg M}(\nu_3) 
 	< v + \gamma \IDef{\dg M}(\nu_1) = \bbr{\dg M}_\gamma(\nu_1), \] 
it must be the case that $\IDef{\dg M}(\nu_3) < \IDef{\dg M}(\nu_1)$. 
        
We can now get a contradiction by applying the same argument as that used to show
that $\IDef{\dg M}(\nu_1) =\IDef{\dg M}(\nu_2)$.  
    Because $(\mu_{i}) \to \nu_1$, there exists some
    $i^*$ such that for all $i > i^*$, we have $\IDef{\dg M}(\mu_{i}) >
    \IDef{\dg M}(\nu_3)$. Thus, for all $i > i^*$ and all
    $\gamma > 0$, 
    \[ \bbr{\dg M}_\gamma(\mu_{i}) = \Inc(\mu_{i}) + \gamma\IDef{\dg M}(\mu_{i}) > \Inc(\nu_3) 
    + \gamma \IDef{\dg M}(\nu_3) = \bbr{\dg M}_\gamma(\nu_3),\]
again contradicting the assumption that $\mu_{i}$ minimizes
$\bbr{\dg M}_{\gamma_{i}}$.
Thus, our supposition that $\nu_1$ was distinct from $\nu_2$ cannot hold, and so
$\lim_{\gamma \to 0}\bbr{\dg M}_\gamma^*$ must be a singleton, as desired.
\end{proof}

Finally, \Cref{prop:consist} is a simple corollary of \Cref{lem:gamma2zero} and \Cref{prop:limit-uniq}, as we now show. 
\restate{prop:consist}{
$\bbr{\dg M}^* \in \bbr{\dg M}_0^*$, so if $\dg M$ is consistent,
then $\bbr{\dg M}^* \in \SD{\dg  M}$.
}

\begin{proof}
By \Cref{prop:limit-uniq}, $\lim_{\gamma \to 0}\bbr{\dg M}_\gamma^*$
is a singleton. As in the body of the paper, we refer to its unique element by $\bbr{\dg M}^*$
\Cref{lem:gamma2zero} therefore immediately gives us $\bbr{\dg M}^* \in \bbr{\dg M}_0^*$.  

If $\dg M$ is consistent, then by \Cref{prop:sd-is-zeroset},
$\Inc({\dg M}) = 0$, so $\bbr{\dg M}_0(\bbr{\dg M}^*) = 0$, and thus
$\bbr{\dg M}^* 
\in \SD{\dg M}$. 
\end{proof}


	\subsection{PDGs as Bayesian Networks}
In this section, we prove Theorem~\ref{thm:bns-are-pdgs}.  
We start by recounting some standard results and notation, all of
which can be found in a standard introduction to information
theory (e.g., \cite[Chapter 1]{mackay2003information}).  

First, note that just as we introduced new variables to model joint dependence
in PDGs, we can view a finite collection $\mathcal X=X_1, \ldots, X_n$ of random
variables, where each $X_i$ has the same sample space, as itself a random
variable%
, taking the value $(x_1, \ldots, x_n)$ iff each $X_i$ takes the value $x_i$.
Doing so allows us to avoid cumbersome and ultimately irrelevant notation which treats sets of raomd variables differently, and requires lots of unnecessary braces, bold face, and uniqueness issues. 
Note the notational convention that the joint variable $X,Y$ may be indicated by a comma.

\begin{defn}[Conditional Independence]\label{defn:cond-indep}
    If $X$, $Y$, and $Z$ are random variables,
    and $\mu$ is a distribution over them, 
    then ${X}$ is \emph{conditionally independent of ${Z}$ given ${Y}$}, 
       (according to $\mu$),  denoted `${ X} \CI_\mu { Z}
        \mid { Y}$, iff for all ${ x}, { y}, { z} \in
        \V({X}, { Y},{ Z})$, we
        have $\mu({ x} \mid { y}) \mu({ z} \mid { y}) =
        \mu({ x,z} \mid { y})$.
\end{defn}

\begin{fact}[Entropy Chain Rule]\label{fact:entropy-chain-rule}
    If $X$ and $Y$ are random variables, then the entropy of the joint
   variable $(X,Y)$ can be written as $\H_\mu(X,Y) = 
\H_\mu( Y \mid X) + \H_\mu(X)$.
It follows that if $\mu$ is a
       distribution over the $n$ variables $X_1, \ldots, X_n$,  then
	\[ \H(\mu) = \sum_{i = 1}^n \H_\mu(X_i \mid X_1, \ldots X_{i-1}). \]
\end{fact}
\begin{defn}[Conditional Mutual Information]\label{defn:cmi}
   The \emph{conditional mutual information} between two (sets of) random
    variables is defined as  
    \[ \I_\mu(X ; Y \mid Z) := \sum_{x,y,z \in \V(X,Y,Z)} \mu(x,y,z)
        \log\frac{\mu(z) \mu(x,y,z)}{\mu(x,z)\mu(y,z)}. \] 
\end{defn}


\begin{fact}[Properties of Conditional Mutual Information]\label{fact:cmi}
For random variables $X,Y$, and $Z$ over a common set of outcomes,
distributed according to a distribution $\mu$,
the following properties hold:
\begin{enumerate}
    \item \textbf{(difference identity)} $\I_\mu(X ; Y \mid Z) =
                  \H_\mu(X \mid Y) - \H_\mu(X \mid Y, Z)$; 
   \item \textbf{(non-negativity)} $\I_\mu({ X }; { Y} \mid {Z}) \ge 0$;
    \item \textbf{(relation to independence)} $\I_\mu({ X }; { Y}
         \mid { Z}) = 0$ iff $X \CI_\mu Z \mid Y$.
\end{enumerate}
\end{fact}

We now provide the formal details of
the transformation of a BN 
into
a PDG.

	\begin{defn}[Transformation of a BN to a PDG]\label{def:bn2PDG}
Recall that a (quantitative) Bayesian Network $(G, f)$ consists of two
parts: its qualitative graphical structure $G$, 
described by a dag,
and its quantitative data $f$, an assignment of 
a cpd $p_i(X_i \mid \Pa(X_i))$ to each variable $X_i$.
If $\cal B$ is a Bayesian network on random variables
$X_1, \ldots, X_n$, we construct the corresponding PDG
$\PDGof{{\mathcal B}}$
			as follows: we take $\N := \{X_1, \ldots, X_n \} \cup
			\{ \Pa(X_1), \ldots, \Pa(X_n)\}$.  
That is, the variables of 
	  $\PDGof{{\mathcal B}}$
consist of all the variables in
${\cal B}$ together with a variable corresponding to the parents
of $X_i$%
.  (This will be used to deal with the hyperedges.) 
			The values $\V(X_i)$ for a random variable
			$X_i$ are unchanged, 
(i.e., $\V^{\PDGof{{\mathcal B}}}(\{X_i\}) := \V(X_i)$)
and $\V^{\PDGof{{\mathcal B}}}(\Pa(X_i)) := \prod_{Y \in \Pa(X_i)} \V(Y)$
(if $\Pa(X_i) = \emptyset$, so that $X_i$ has no parents, then we 
then we identify $\Pa(X_i)$ with $\var 1$ and
take $\V(\Pa(X_i)) = \{\star\}$). 
We take the set of edges $\Ed^{\PDGof{{\mathcal B}}} := \{ (\Pa(X_i), X_i) : 
i = 1, \ldots, n \} \cup \{ (\Pa_i, Y) : Y \in
			\Pa(X_i)\}$ to be the set of edges to a variable $X_i$
	  from its parents, together with an edge from
	  from $\Pa(X_i)$ to each of the elements of $\Pa(X_i)$, for
	  $i = 1, \ldots, n$.  
	Finally, we set $\mat p^{\PDGof{{\mathcal
				B}}}_{(\Pa(X_i), X_i)}$ to be the cpd associated
			with $X_i$ in $\cal B$, and for each node $X_j \in \Pa(X_i)$,
			we define
	\[ \mat p^{\PDGof{\mathcal B}}_{(\Pa(X_i),
			  X_j)}(\ldots, x_j, \ldots) = \delta_{x_j};\]
that is,
$\mat p_{(\Pa(X_i), X_j)}^{\PDGof{\mathcal B, \beta}}$ is the the cpd 
on $X_j$ that, given a setting $(\ldots, x_j, \ldots)$ of $\Pa(X_i)$, yields the distribution that puts all mass on $x_j$. 
\end{defn}


\commentout{
The following lemma does most of the work in the proof of 
Theorem~\ref{thm:bns-are-pdgs}. 
\begin{lemma} \label{lem:bnmaxent-component}
If $\mu$ is a probability distribution on some set $W$ and 
and $X$, $Y$, $Z$ are random variables on $W$, 
	then  
\[ \tilde H_\mu(X \mid Y, Z) := \E_{y \sim \mu_{_{Y}}} \Big[
	\H_\mu(X \mid Y \!=\!y) \Big]  - \H_\mu( X \mid Y, Z)\] 
is (a) non-negative, and (b) equal to zero if and only if $X$ and $Z$ are independent given $Y$.
\end{lemma}
\begin{proof}
\begin{align*}
	\tilde H_\mu(X \mid Y, Z) &= \E_{y \sim \mu_{_{Y}}}  \Big[ \H_\mu(X \mid Y \!=\!y)\Big] - \H_\mu( X \mid Y, Z)  \\
	&=  \left[\sum_{y} \mu(y) \sum_x  \mu(x\mid y) \log \frac{1}{\mu(x \mid y)} \right]+ \left[\sum_{x,y, z} \mu(x, y, z) \log \frac{\mu(x,y,z)}{\mu(y, z)}\right] \\[0.5em]
	&= \left[\sum_{x,y} \mu(x,y) \log \frac{\mu(y)}{\mu(x,y)}
	\right] + {\left[\sum_{x,y, z} \mu(x, y, z) \log \frac{\mu(x,y,z)}{\mu(y, z)} \right]} \\
	&= \left[\sum_{x,y, z} \mu(x,y ,z) \log \frac{\mu(y)}{\mu(x,y)}
	\right] + {\left[\sum_{x,y, z} \mu(x, y, z) \log \frac{\mu(x,y,z)}{\mu(y, z)} \right]} \\
	&= \sum_{x,y, z} \mu(x,y ,z) \left[ \log \frac{\mu(y)}{\mu(x,y)} + \log \frac{\mu(x,y,z)}{\mu(y, z)} \right] \\
	&= \sum_{x,y, z}  \mu(x,y ,z) \log
			\left[\frac{\mu(y)\ \mu(x,y,z)}{\mu(x,y)\ \mu(y,z).}
			\right]  \\ 
\end{align*}
Define $q(x,z,y) := {\mu(x,y)\ \mu(y,z) }/{\mu(y)}$, wherever
			$\mu(y)\neq 0$, and $q(x,y,z) = 0$ otherwise. $q$ is in fact
	a distribution over the values of $X$, $Y$, and $Z$, since it  
is clearly non-negative, and sums to 1, as we now show:
\[
\sum_{x,y,z} q(x,y, z) = \sum_{x,y,z} \frac{\mu(x,y)\ \mu(y,z)}{\mu(y)}
= \sum_{x,y,z} \mu(x \mid y) \mu(y,z)
= \sum_{y,z} \left(\sum_x \mu(x \mid y)\right) \mu(y,z)
= \sum_{y,z}  \mu(y,z)
		= 1.
\]	
With this definition, we return to our computation of $\tilde H_\mu(X \mid Y, Z)$:
\begin{align*}
	\tilde H_\mu(X \mid Y, Z) &= \sum_{x,y, z}  \mu(x,y ,z) \log \left[\frac{\mu(y)\ \mu(x,y,z)}{\mu(x,y)\ \mu(y,z)} \right]  \\ %
	&= \sum_{x,y, z}  \mu(x,y ,z) \log \frac{\mu(x,y,z)}{q(x,y,z)}  \\
					&= \kldiv{\mu_{_{XYZ}}}{q},
\end{align*}
where $\mu_{_{XYZ}}$ is the marginal of $\mu$ on the settings of $XYZ$, and $\kldiv{\mu_{_{XYZ}}}{q}$ is the relative entropy to $\mu_{_{XYZ}}$ from $q$. By Gibbs' inequality (non-negativity of relative entropy), $\tilde H$ is  (1) non-negative, and (2) equal to zero if and only if $\mu_{_{XYZ}} = q$, meaning that 
\[  \mu(x,y,z) =\begin{cases} \frac{\mu(x,y)\ \mu(y,z)}{\mu(y)} & \text{if }\mu(y) > 0\\ 0 & \text{otherwise} \end{cases} \qquad \implies \qquad \mu(x,y,z) \mu(y) = \mu(x,y) \mu(y, z) \] 
and so $\tilde H_\mu(X \mid Y, Z)$ is (1) non-negative, and
	(2) equal to zero if and only if $X$ and $Z$ are independent
			given $Y$ according to $\mu$. 
\end{proof}
}

Let $\mathcal X$ be the variables of some BN $\mathcal B$, and
$\mathcal M = \pdgvars$ 
be the PDG $\PDGof{\mathcal B}$.
\commentout{
We admit that in general, the set of variables
$\mathcal X$ is a strict subset of $\N$, and so a reader would be justifiably
suspicious of any claim (such as the one in \Cref{thm:bns-are-pdgs})
in which a distribution over $\mathcal X$ is in a set of distributions
over $\N$---the types do not work out.  

However, there is a natural injection $\iota: \Delta \V(\mathcal X) \to \Delta
\V(\mathcal Y)$, taking a joint distribution on the variables $\mathcal X$ and
returning the unique distribution on $\N$ for which the value of a node labeled
$X_1 \times \ldots \times X_n$ is always equal to the tuple of values on $X_1,
\ldots, X_n$. Technically, the statement of theorem should read
\[ \bbr{\PDGof{\mathcal B, \beta}}^*_\gamma
= \{ \iota \Pr\nolimits_{\mathcal B} \} . \]
Moreover, a distribution $\mu \in \Delta(\V(\N))$ that is not in the
image of $\iota$, will have $\bbr{\dg M}_\gamma(\mu) = \infty$ (for
all gamma), and so there is in fact a 1-1 correspondence  
\[ \Big\{ \nu \in \Delta\V(\mathcal X)~\Big|~ \bbr{\dg M}_\gamma(\iota\nu) < \infty \Big \} \quad\leftrightsquigarrow\quad 
\Big\{ \mu \in \Delta\V(\N)~\Big|~ \bbr{\dg M}_\gamma(\mu) < \infty \Big \}.
\]
Therefore, from the perspective of scoring functions (and by
extension, all PDG semantics), the two spaces are equivalent. So long
as we refer only to the scores given by $\bbr{\PDGof{\mathcal B}}$, we
may therefore conflate distributions from the two spaces,  which
justifies the statement of \Cref{thm:bns-are-pdgs}, which we now
restate and prove.
}
Because the set  $\mathcal N$ of variables in $\PDGof{{\mathcal
    B},\beta}$ includes  
variables of the form $\Pa(X_i)$, it is a strict superset of
$\mathcal X = \{X_1,\ldots, X_n\}$, the set of variables of $\mathcal B$.
For the purposes of this theorem, we identify a distribution
$\mu_{\mathcal X}$ over $\mathcal X$ 
with the unique distribution $\Pr_{\cal B}$ whose marginal on the
variables in $\mathcal X$ is $\mu_{\mathcal X}$ such that if $X_j \in
\Pa(X_i)$, then 
$\mu_{\mathcal N}(X_j = x_j' \mid \Pa(X_i) = (\ldots, x_j,\ldots)) =
1$ iff $x_j = x_j'$.  In the argument below, we abuse notation,
dropping the the subscripts $\mathcal X$ and $\mathcal N$ on a
distribution $\mu$.

\restate{thm:bns-are-pdgs}{
If $\cal B$ is a Bayesian network
and $\Pr_{\cal B}$ is the distribution it specifies, then
  for all $\gamma > 0$ and all vectors $\beta$ such
  that $\beta_L > 0$ for all edges $L$,
  $\bbr{\PDGof{\mathcal B, \beta}}_\gamma^* = \{ \Pr_{\cal B}\}$, 
and thus $\bbr{\PDGof{\mathcal B, \beta}}^* = \Pr_{\cal B}$.    
}
\begin{proof}
  For the cpd $p(X_i \mid \Pa(X_i))$ associated to a node $X_i$ in 
$\cal B$, we have that $\Pr_{\cal B}(X_i
\mid \Pa(X_i)) = p(X_i \mid \Pa(X_i))$.  
For all nodes $X_i$ in $\mathcal B$ and $X_j \in \Pa(X_i)$, 
by construcction, $\Pr_{\cal B}$, when viewed as a distribution on
$\mathcal N$, is also with the cpd on the edge from $\Pa(X_i)$ to
$X_j$.
Thus, $\Pr_{\cal B}$ is consistent with all the cpds in
$\PDGof{\mathcal B, \beta}$;
so$\Inc_{\PDGof{\mathcal B,\beta}}(\Pr_{\cal B}) = 0$.

We next want to show  that $\IDef{\PDGof{\mathcal B,\beta}}(\mu) \ge 0$ for all
distributions $\mu$.  To do this, we first need some definitions.
Let $\rho$ be a permutation of $1, \ldots,  n$.  Define an order
$\prec_{\rho}$ by taking $j \prec_{\rho} i$ if $j$ precedes $i$ in the
permutation; that is, if 
$\rho^{-1}(j)$ < $\rho^{-1}(i)$. Say that a permutation is \emph{compatible with
  $\mathcal B$} if $X_j \in \Pa(X_i)$ implies $j \prec_{\rho} i$.   There
is at least one permutation compatible with $\mathcal B$, since 
the graph underlying $\mathcal B$ is acyclic.
  
Consider an arbitrary distribution $\mu$ over the variables in
$\mathcal X$ (which we also view as a distribution over the variables
in $\mathcal N$, as discussed above).
Recall from \Cref{def:bn2PDG}
that the cpd on the edge in $\PDGof{{\cal B},\beta}$ from $\Pa(X_i)$ to $X_i$
is just the cpd associated with $X_i$ in ${\cal B}$, while the cpd on
the edge in $\PDGof{{\cal B},\beta}$ from $\Pa(X_i)$ to $X_j \in \Pa(X_i)$
consists only of deterministic distributions (i.e., ones that put
probability 1 on one element), which all have entropy 0.  
Thus,
\begin{equation}\label{eq:fact2}
\sum_{\ed LXY \in \Ed^{\PDGof{\mathcal B}}} \H_\mu(Y\mid
X)=\sum_{i=1}^n \H_\mu(X_i \mid \Pa(X_i)). 
\end{equation}

Given a permutation $\rho$, let ${\bf X}_{\prec_\rho i} = \{X_j: j
\prec_\rho i\}$.  Observe that 
\begin{align*}
    \IDef{\PDGof{\mathcal B,\beta}}(\mu)
 	&= \left[\sum_{\ed LXY \in \Ed^{\PDGof{\mathcal B}}} \H_\mu(Y\mid X) \right] - \H(\mu) \\
	&= \sum_{i=1}^n \H_\mu(X_i \mid \Pa(X_i)) - \sum_{i = 1}^n
\H_\mu(X_i \mid {\bf X}_{\prec_\rho i}) & \text{[by
    \Cref{fact:entropy-chain-rule} and \eqref{eq:fact2}]}\\ 
	&= \sum_{i=1}^n \Big[\H_\mu(X_i \mid \Pa(X_i)) - \H_\mu(X_i
  \mid {\bf X}_{\prec_\rho i} )\Big] \\ 
      &= \sum_{i=1}^n \I_\mu \Big( X_i ~;~ {\bf X}_{\prec_\rho i}
    \setminus \Pa(X_i) ~\Big|~ \Pa(X_i) \Big). & \text{[by
        \Cref{fact:cmi}]} 
\end{align*}

Using \Cref{fact:cmi}, it now follows that,
for all distributions $\mu$,
$\IDef{\PDGof{\mathcal B}}(\mu) \ge 0$.
Furthermore, for all $\mu$ and permutations $\rho$,
\begin{equation}\label{eq:key}
  \IDef{\PDGof{\mathcal B}}(\mu) = 0 \quad\mbox{ iff }\quad 
    \forall i.~X_i \CI_\mu {\bf X}_{\prec_\rho i}.
\end{equation}

Since the left-hand side of (\ref{eq:key}) is independent of $\rho$,
it follows that $X_i$ is independent of 
${\bf X}_{\prec_\rho i}$ for some permutation $\rho$ iff $X_i$ is independent of
  ${\bf X}_{\prec_\rho i}$ for every permutation $\rho$.  Since there
is a permutation compatible with $\mathcal B$, we get that 
$\IDef{\PDGof{\mathcal B,\beta}}(\Pr_{\cal B}) = 0$.
We have now shown that that $\IDef{\PDGof{\mathcal B, \beta}}$ and $\Inc$ are 
non-negative functions of $\mu$, and both are zero at $\Pr_{0\cal B}$. 
Thus, for all $\gamma \geq 0$ and all vectors $\beta$, we
have that   $\bbr{\PDGof{\mathcal B, \beta}}_\gamma( \Pr_{\cal
  B}) \le \bbr{\PDGof{\mathcal B, \beta}}_\gamma( \mu)$ for all
distributions $\mu$.  We complete the proof by showing that if
$\mu \ne \Pr_{\cal B}$, then 
$\bbr{\PDGof{\mathcal B, \beta}}_\gamma(\mu) > 0$
for $\gamma > 0$.

\commentout{
	

We have shown that, for all topological orderings of
the variables of $\cal B$, $\IDef{\PDGof{\cal B}}(\mu) =
 0$ if and only if  each $X_i \CI_\mu
X_j \mid \Pa(X_i)$ for $j  < i$; we will refer to this
as $(\star)$. 
	
	Now, suppose $X_j$ were a non-descendent of $X_i$, with $j > i$. Because $X_j$ is not a descendent of $X_i$, we can construct a second toplogoical sort of the variables in $\cal B$, in which $\#(X_j) < \#(X_i)$, where $\#(X)$ is the index of $X$ in the new ordering. 
	We can obtain $\#$, for instance, by topologically sorting $X_j$ and its ancestors, and then adding the rest of the variables (which we call $\bf R$) in their original order. The concatination of these two is a valid topological sort because the ancestors of $X_j$ are topologicaly ordered, and the parents of each $X \in \bf R$ occur no later than before.
	
	
	With this new order, suppose that 
	$\IDef{\PDGof{\cal B}}(\mu) = 0$.
	By $(\star)$, since $\#(X_j) < \#(X_i)$, we know that $X_i \CI X_j \mid \Pa(X_i)$ according to $\mu$. Since this is true for an aribitrary $i$ and $j$ without changing the distribution $\mu$, we conclude that if
	$\IDef{\PDGof{\cal B}}(\mu) = 0$, 
	then $\mu$ makes \emph{every} variable $X_i$ independent of its non-descendents $X_j$, given its parents.
	Conversely, if every variable is independent of its non-descendents given its parents, then $\mu$ is the unique distribution determined by $\cal B$, and since each variable of $\cal B$ is independent of previous variables given the values of its parents,  we know by $(\star)$ that
	$\IDef{\PDGof{\cal B}}(\mu) = 0$. 
	Therefore, if $\mathit{NonDesc}(X)$ is the set of non-descendents of $X$ according to $\mathcal B$, we have
\begin{equation}\label{eq:idef-bn-indeps}
 	\IDef{\PDGof{\mathcal B,\beta}}(\mu) = 0 \quad\iff\quad X_i \CI_\mu X_j \mid \Pa(X_i) \quad\text{for all $X_i$ and $X_j \in \mathit{NonDesc}(X_i)$} 
\end{equation}

These independencies are exactly the ones prescribed by $\cal B$.
Because $\Pr_{\mathcal B}$ in particular satisfies them,
we have $\IDef{\PDGof{\mathcal B,\beta}}(\Pr_{\cal B}) = 0$.
We also know that that $\Pr_{\cal B} \in \SD{\PDGof{\mathcal B,\beta}}$, for
every vector of weights $\beta$. By \Cref{prop:sd-is-zeroset},
$\Inc_{\PDGof{\mathcal B,\beta}}(\Pr_{\mathcal B}) = 0$. Therefore, for all
$\gamma \geq 0$, we have
\[ \bbr{\PDGof{\mathcal B, \beta}}_\gamma(\Pr\nolimits_{\cal B})
	= \Inc_{\PDGof{\mathcal B,\beta}}(\Pr\nolimits_{\mathcal B}) + \gamma \cdot
	\IDef{\PDGof{\mathcal B, \beta}}(\Pr\nolimits_{\cal B}) = 0
\]
Both $\Inc_{\PDGof{\mathcal B,\beta}}$ and $\IDef{\PDGof{\mathcal B, \beta}}$
are non-negative for every $\mu$, which is sufficient to show $\Pr_{\mathcal B}$
minimizes $\bbr{\PDGof{\mathcal B, \beta}}_\gamma$ for all $\gamma \geq 0$. 

If $\gamma > 0$, we can ensure that $\Pr_{\cal B}$ is its \emph{unique} minimizer. For $\gamma > 0$, if $\bbr{\PDGof{\mathcal B, \beta}}_\gamma(\mu) = 0$, then $\mu$ must have the came cpds as $\mathcal B$ (since $\Inc(\mu) = 0$) and also of the conditional independencies of $\mathcal B$ (by \eqref{eq:idef-bn-indeps} and the fact that $\IDef{}(\mu) = 0$).
We therefore conclude that for all $\gamma\geq0$ and vectors $\beta$ of
weights,  
\[ \{ \Pr\nolimits_{\cal B} \} = \bbr{\PDGof{\mathcal B, \beta}}_\gamma^* .\]

	
}
So suppose that $\mu \ne \Pr_{\cal B}$. 
Then $\mu$ must also match each cpd of $\cal B$,
for otherwise $\Inc_{\PDGof{\mathcal B,
\beta}}(\mu) > 0$, and we are done.  
Because $\Pr_{\cal B}$ is the \emph{unique} distribution that matches the 
both the cpds and independencies of $\cal B$, $\mu$ must not have all of the 
independencies of $\cal B$. 
Thus,
some variable $X_i$, $X_i$ is not independent of some nondescendant $X_j$ in
$\mathcal B$ with respect to $\mu$.  There must be some permutation
$\rho$ of the variables in $\mathcal X$ compatible with ${\mathcal B}$
such that $X_j \prec_{\rho} X_i$ (e.g., we can start with $X_j$ and
its ancestors, and then add the remaining variables appropriately).
Thus, it is not the case that $X_i$ is independent of $X_{\prec \rho,
  i}$, so by (\ref{eq:key}), $\IDef{\PDGof{\mathcal B}}(\mu) > 0$.
This completes the proof.
\end{proof}

\subsection{Factor Graph Proofs}
\cref{thm:fg-is-pdg,thm:pdg-is-fg} are immediate corolaries of their
more general counterparts, \cref{thm:pdg-is-wfg,thm:wfg-is-pdg}, which
we now prove. 


\restate{thm:pdg-is-wfg}{}
\begin{proof}
	Let $\dg M := (\dg N, \mat v, \gamma \mat v)$ be the PDG in question.
	Explicitly, $\alpha^{\dg M}_L = v_L$ and $\beta_L^{\dg M} =  \gamma v_L$.
	By \Cref{prop:nice-score},
	\[ \bbr{\dg M}_\gamma(\mu)= \Ex_{\mat w \sim \mu} \Bigg\{   \sum_{ X \xrightarrow{\!\!L} Y  } \left[
		\beta_L \log \frac{1}{\bp(y\mid x)} + (
			\alpha_L
		\gamma - \beta_L ) \log \frac{1}{\mu(y \mid x)}
					\right] - \gamma \log \frac{1}{\mu(\mat w)}
			\Bigg\}.  \]
	Let $\{\phi_L\}_{L \in \Ed} := \Phi_{\dg N}$ denote the
			factors of the factor graph associated with $\dg M$. 
	Because we have $\alpha_L\gamma  = \beta_L$, the middle term cancels, leaving us with
	\begin{align*}
	\bbr{\dg M}_\gamma(\mu) &= \Ex_{\mat w \sim \mu} \Bigg\{   \sum_{ X \xrightarrow{\!\!L} Y  } \left[
		\beta_L \log \frac{1}{\bp(y\mid x)} \right] - \gamma \log \frac{1}{\mu(\mat w)} \Bigg\} \\
		&= \Ex_{\mat w \sim \mu} \Bigg\{   \sum_{ X \xrightarrow{\!\!L} Y  } \left[
	\gamma v_L \log \frac{1}{\phi(x,y)}  \right] - \gamma \log \frac{1}{\mu(\mat w)} \Bigg\} 
					&\text{[as $\beta_L = v_L \gamma$]}\\
		&= \gamma \Ex_{\mat w \sim \mu} \Bigg\{   \sum_{ X \xrightarrow{\!\!L} Y  } \left[
v_L \log \frac{1}{\phi(x,y)}
			 \right] -\log \frac{1}{\mu(\mat w)} \Bigg\} \\
			&= \gamma \GFE_{(\FGof{\dg N}, \mat v)}. 
	\end{align*}
	It immediately follows that the associated factor graph has 
	$\bbr{\dg M}^*_\gamma
 	= \{\Pr_{\Phi(\dg M)}\}$, because the free energy is clearly a constant plus the KL divergence from its associated probability distribution.
\end{proof}

\restate{thm:fg-is-pdg}{
	For all WFGs $\Psi = (\Phi,\theta)$ and all $\gamma > 0$,
	we have that
	$\GFE_\Psi
	= \nicefrac1{\gamma} \bbr{{\dg M}_{\Psi,\gamma}}_{\gamma} 
	+ C$   
	for some constant $C$, so
	$\Pr_{\Psi}$ is the unique element of
	$\bbr{{\dg M}_{\Psi,\gamma}}_{\gamma}^*$.
}
\begin{proof}
  In $\PDGof{\Psi,\gamma}$,  there is an edge $1 \to X_J$ for every $J
  \in \mathcal J$, and also edges 
  $X_J \tto X_j$ for each $X_j
    \in X_J$. Because the latter edges are deterministic, a
distribution $\mu$ that is not  consistent
with one of the edges, say $X_J \tto X_j$, has $\Inc_{\dg M}(\mu)
= \infty$.  This is a 
property of relative entropy: if there exist $j^* \in \V(X_j)$ and 
$\mat z^* \in \V(J)$ such that $\mat z^*_J \ne j^*$ and $\mu$ places positive
probability on their co-occurance (i.e., $\mu(j^*, \mat z^*) > 0$),
then we would have
\[ \Ex_{\mat z \sim \mu_{J}}\kldiv[\Big]{\mu(X_j \mid X_J = \mat z)}
	{\mathbbm1[X_j = \mat z_{j}]}
 	= \sum_{\substack{\mat z \in \V(X_J),\\ \iota \in \V(X_j)}} \mu(\mat z, \iota) \log \frac{\mu(\iota \mid \mat z)}{\mathbbm1[\mat z_j = \iota]}
	\geq \mu(\mat z^*, j^*) \log \frac{\mu(j^* \mid \mat z)}{\mathbbm1[\mat z^*_j = j_*]}
	= \infty. \]
Consequently, a distribution $\mu$ that does not satisfy the the projections has
$\bbr{\dg M_{\Psi,\gamma}}_\gamma(\mu) = \infty$ for every $\gamma$.
          Thus, a distribution that 
        has a finite score must match the constraints,
so we can identify such a distribution with its restriction to 
the original  
variables of $\Phi$.
Moreover, for all distributions $\mu$ with finite score and
projections $X_J \tto 
X_j$, the conditional entropy 
$\H(X_j \mid X_J) = -\Ex_\mu\log(\mu(x_j \mid x_J))$ and divergence from
the constraints are both zero. 
Therefore the per-edge terms for both $\IDef{\dg M}$
and $\Inc_{\dg M}$ can be safely ignored for the projections.
Let $\bp[J]$ be
the normalized distribution $\frac{1}{Z_J}\phi_J$ over $X_J$,
where $Z_J = \sum_{x_J} \phi_J(x_J)$ is the appropriate normalization constant.
By
\cref{def:wfg2pdg}, we have $\PDGof{\Psi,\gamma} = (\UPDGof{\Phi}, \theta, \gamma\theta)$,
so by \cref{prop:nice-score},
	\begin{align*}
\bbr{\PDGof{\Psi,\gamma}}_\gamma(\mu) 
	&= \Ex_{\mat x \sim \mu} \Bigg\{   \sum_{ J \in \mathcal J } \left[
		\beta_J \log \frac{1}{ \bp[J](x_J) } + 
			(\alpha_J \gamma -\beta_J)
		 \log \frac{1}{\mu(x_J)} \right] - \gamma \log \frac{1}{\mu(\mat x)} \Bigg\} \\
		 &= \Ex_{ \mat x \sim \mu} \Bigg\{   \sum_{ J \in \mathcal J } \left[
	 		(\gamma\theta_J) \log \frac{1}{ \bp[J](x_J) } + 
	 			(\theta_J \gamma - \gamma\theta_J)
	 		 \log \frac{1}{\mu(x_J)} \right] - \gamma \log \frac{1}{\mu(\mat x)} \Bigg\} \\
		&= \Ex_{ \mat x \sim \mu} \Bigg\{  \sum_{ J \in \mathcal J }\left[
			\gamma\theta_J \log \frac{1}{\bp[J](x_J)}  \right] - \gamma \log \frac{1}{\mu(\mat x)} \Bigg\} 
			\\
		&= \gamma \cdot \Ex_{\mat x \sim \mu} \Bigg\{  \sum_{ J \in \mathcal J } \theta_J
			\log \frac{Z_J}{\phi_J(x_J)}   -\log \frac{1}{\mu(\mat x)} \Bigg\} \\
		&= \gamma \cdot \Ex_{\mat x \sim \mu} \Bigg\{  \sum_{ J \in \mathcal J } \theta_J \left[
			\log \frac{1}{\phi_J(x_J)} + \log Z_J \right]  - \log \frac{1}{\mu(\mat x)} \Bigg\} \\
		&= \gamma \cdot \Ex_{\mat x \sim \mu} \Bigg\{  \sum_{ J \in \mathcal J } \theta_J 
			\log \frac{1}{\phi_J(x_J)}  - \log \frac{1}{\mu(\mat x)} \Bigg\}
			 +  \sum_{J \in \mathcal J} \theta_J \log Z_J  \\
        	&= \gamma\, \GFE_{\Psi} + k \log \prod_{J} Z_J,
	\end{align*}
which differs from $\GFE_{\Psi}$ by the value $\sum_J \theta_J \log Z_J$, which 
is constant in $\mu$.

\end{proof}

\commentout{
\begin{proof}
	Because each local normalization results in a local joint
			distribution $\bp[J] = \frac{1}{Z_J}
			\phi_J$ on the variables associated with $J$, and these distributions differ from the original factors $\phi_J$ by only a multiplicative 
		   constant, the product of these locally normalized factors differs from the product of the factors by only a constant, and so 
	\[ \Pr\nolimits_F(\vec x) \propto \prod_{J \in \cal J} \phi_J(\vec x) \propto \prod_{J \in \cal J} \left(\frac{\phi_J(\vec x)}{Z_J}\right) \propto \Pr_{\Phi(\PDGof{F})}(\vec x) \]
	and since the two distributions are normalized, they must be equal.
\end{proof}
}

\commentout{
\subsection{Dependency Networks}
Finally, we prove

\restate{thm:dns-are-pdgs}{}
\begin{proof}
	Sketch:  By Theorem 2 of their paper, the stationary point of this procedure is unique, and equal to the only distribution $p$ which both is consistent with the independece assumptions, and also the cpds.  A feq examples strongly suggest thte independece assumptions associated with this structure, are correctly encoded in $\IDef{\PDGof{\mathcal D}}$. This needs to be done more rigorously but I don't expect it to be too bad.
\end{proof}
}

\vfull{
\section{Further Details on the Information Deficit}

    The examples here are in reference to \Cref{fig:info-diagram}.
    \ref{subfig:justX-0}, \ref{subfig:justX-1}, and \ref{subfig:justX-2} show that adding edges makes distriutions more deterministic. 
    As each edge $\ed LXY$ corresponds to an assertion about the ability to determine $Y$ from $X$, this should make some sense.
    In particular, \ref{subfig:justX-2} can be justified by the fact that if you can determine X from two different random draws, the draws probably did not have much randomness in them. Thus we can qualitatively encode a double-headed arrow as two arrows, further justifying the notation.
    Without any edges (e.g., \ref{subfig:justX-0},\ref{subfig:justXY}), the $G$-information rewards distributions with the most uncertainty. Each additional edge adds a penalty for a crescent, as when we move from \ref{subfig:justXY} to \ref{subfig:XtoY} to \ref{subfig:XY-cycle}.
    Some graphs (\Cref{subfig:justX-1, subfig:1XY}) are \emph{universal}, in that every distribution gets the same score (so that score must be zero, beause this is the score a degenerate distribution gets). Such a graph has a structure such that \emph{any} distribution can be precisely encoded by the process in (b). 
    The $G$-information can also indicate independencies and conditional independencies, illustrated respectively in \ref{subfig:XYindep} and \ref{subfig:1XYZ}.

    So far all of the behaviors we have seen have been instances of entropy maximization / minimization, or independencies, but $G$-information captres more: for instance, if $G$ has cycles, as in \ref{subfig:XY-cycle} or \ref{subfig:XYZ-cycle}, the $G$-information prioritizes shared information between all variables. 

    In more complicated examples, where both penalties and rewards exist, we argue that the $G$-information still implicitly captures the qualitative structure. In \ref{subfig:XYZ-bichain}, $X$ and $Y$ determine one another, and $Z$ and $Y$ determine one another. It is clear that $X$ and $Z$ should be indpenedent given $Y$; it can also be argued that $Y$ should not have any randomness of its own (otherwise the draws from $X$ or $Z$ would likey not match one another) and that this structure suggests co-variation of all three variables.


    \definecolor{subfiglabelcolor}{RGB}{0,0,0}
    \begin{figure}
    	\centering
    	\def\vsize{0.4}
    	\def\spacerlength{0.5em}
    \scalebox{0.85}{
    \stepcounter{figure}

    	\begin{tikzpicture}[center base]\refstepcounter{subfigure}\label{subfig:justX-0}
    		\node[dpad0] (X) at (0,1){$X$};
    		\draw[fill=green!50!black]  (0,0) circle (\vsize)  ++(-90:.22) node[label=below:\tiny$X$]{};
    		\useasboundingbox (current bounding box);
    		\node at (-0.5, 0.6){\slshape\color{subfiglabelcolor}\thesubfigure};
    	\end{tikzpicture}\!
    \begin{tabular}{c}
    	\begin{tikzpicture}[is bn]\refstepcounter{subfigure}\label{subfig:justX-1}
    		\node[dpad0] (1) at (-0.4,.85){$\var 1$};
    		\node[dpad0] (X) at (0.4,.85){$X$};
    		\draw[arr1] (1)  -- (X);
    		\draw[fill=white!70!black]  (0,0) circle (\vsize) ++(-90:.22) node[label=below:\tiny$X$]{};
    		\node at (-0.6,0.35){};
    		\useasboundingbox (current bounding box);
    		\node at (-0.7, 0.35){\slshape\color{subfiglabelcolor}\thesubfigure};
    	\end{tikzpicture} \\[0.5em]
    	\begin{tikzpicture}\refstepcounter{subfigure}\label{subfig:justX-2}
    		\node[dpad0] (1) at  (-0.45,.85){$\var 1$};
    		\node[dpad0] (X) at  (0.45,.85){$X$};
    		\draw[arr1] (1) to[bend left=20] (X);
    		\draw[arr1] (1) to[bend right=20] (X);
    		\draw[fill=red!50!black] (0,0) circle (\vsize) ++(-90:.22) node[label=below:\tiny$X$]{};
    		\useasboundingbox (current bounding box);
    		\node at (-0.7, 0.35){\slshape\color{subfiglabelcolor}\thesubfigure};
    	\end{tikzpicture}
    \end{tabular}%
    \hspace{\spacerlength}\vrule\hspace{\spacerlength}
    	\begin{tabular}{c}
    	\begin{tikzpicture}[]  \refstepcounter{subfigure}\label{subfig:justXY}
    		\node[dpad0] (X) at (-0.45,.85){$X$};
    		\node[dpad0] (Y) at (0.45,.85){$Y$};
    		\path[fill=green!50!black] (-0.2,0) circle (\vsize) ++(-110:.23) node[label=below:\tiny$X$]{};
    		\path[fill=green!50!black] (0.2,0) circle (\vsize) ++(-70:.23) node[label=below:\tiny$Y$]{};
    		\begin{scope}
    			\clip (-0.2,0) circle (\vsize);
    			\clip (0.2,0) circle (\vsize);
    			\fill[green!50!black] (-1,-1) rectangle (3,3);
    		\end{scope}
    		\draw (-0.2,0) circle (\vsize);
    		\draw (0.2,0) circle (\vsize);
    		\useasboundingbox (current bounding box);
    		\node at (-0.8, 0.4){\slshape\color{subfiglabelcolor}\thesubfigure};
    	\end{tikzpicture}\\[0.5em]
    	\begin{tikzpicture}[]\refstepcounter{subfigure}\label{subfig:XtoY}
    		\node[dpad0] (X) at (-0.45,0.85){$X$};
    		\node[dpad0] (Y) at (0.45,0.85){$Y$};
    		\draw[arr1] (X) to[] (Y);
    		\path[fill=green!50!black] (-0.2,0) circle (\vsize) ++(-110:.23) node[label=below:\tiny$X$]{};
    		\path[fill=white!70!black] (0.2,0) circle (\vsize) ++(-70:.23) node[label=below:\tiny$Y$]{};
    		\begin{scope}
    			\clip (-0.2,0) circle (\vsize);
    			\clip (0.2,0) circle (\vsize);
    			\fill[green!50!black] (-1,-1) rectangle (3,3);
    		\end{scope}
    		\draw (-0.2,0) circle (\vsize);
    		\draw (0.2,0) circle (\vsize);
    		\useasboundingbox (current bounding box);
    		\node at (-0.8, 0.4){\slshape\color{subfiglabelcolor}\thesubfigure};
    	\end{tikzpicture}
    \end{tabular}%
    \begin{tabular}{c}
    	\begin{tikzpicture}[center base]\refstepcounter{subfigure}\label{subfig:XY-cycle}
    		\node[dpad0] (X) at (-0.45,0.85){$X$};
    		\node[dpad0] (Y) at (0.45,0.85){$Y$};
    		\draw[arr1] (X) to[bend left] (Y);
    		\draw[arr1] (Y) to[bend left] (X);
    		\draw[fill=white!70!black] (-0.2,0) circle (\vsize) ++(-110:.25) node[label=below:\tiny$X$]{};
    		\draw[fill=white!70!black] (0.2,0) circle (\vsize) ++(-70:.25) node[label=below:\tiny$Y$]{};
    		\begin{scope}
    			\clip (-0.2,0) circle (\vsize);
    			\clip (0.2,0) circle (\vsize);
    			\fill[green!50!black] (-1,-1) rectangle (3,3);
    		\end{scope}
    		\draw (-0.2,0) circle (\vsize);
    		\draw (0.2,0) circle (\vsize);
    		\useasboundingbox (current bounding box.south west) rectangle (current bounding box.north east);
    		\node at (-0.85, 0.4){\slshape\color{subfiglabelcolor}\thesubfigure};
    	\end{tikzpicture}\\[2.5em]
    	\begin{tikzpicture}[center base, is bn] \refstepcounter{subfigure}\label{subfig:XYindep}
    		\node[dpad0] (1) at (0,0.75){$\var 1$};
    		\node[dpad0] (X) at (-0.7,0.95){$X$};
    		\node[dpad0] (Y) at (0.7,0.95){$Y$};
    		\draw[arr0] (1) to[] (X);
    		\draw[arr0] (1) to[] (Y);
    		\draw[fill=white!70!black] (-0.2,0) circle (\vsize) ++(-110:.23) node[label=below:\tiny$X$]{};
    		\draw[fill=white!70!black] (0.2,0) circle (\vsize) ++(-70:.23) node[label=below:\tiny$Y$]{};
    		\begin{scope}
    			\clip (-0.2,0) circle (\vsize);
    			\clip (0.2,0) circle (\vsize);
    			\fill[red!50!black] (-1,-1) rectangle (3,3);
    		\end{scope}
    		\draw (-0.2,0) circle (\vsize);
    		\draw (0.2,0) circle (\vsize);
    		\useasboundingbox (current bounding box.south west) rectangle (current bounding box.north east);
    		\node at (-0.88, 0.4){\slshape\color{subfiglabelcolor}\thesubfigure};
    	\end{tikzpicture}
    \end{tabular}
    \hspace{\spacerlength}
    	\begin{tikzpicture}[center base, is bn]\refstepcounter{subfigure}\label{subfig:1XY}
    		\node[dpad0] (1) at (0.15,2){$\var 1$};
    		\node[dpad0] (X) at (-0.45,1.4){$X$};
    		\node[dpad0] (Y) at (0.35,1){$Y$};
    		\draw[arr0] (1) to[] (X);
    		\draw[arr1] (X) to[] (Y);
    		\path[fill=white!70!black] (-0.2,0) circle (\vsize) ++(-110:.23) node[label=below:\tiny$X$]{};
    		\path[fill=white!70!black] (0.2,0) circle (\vsize) ++(-70:.23) node[label=below:\tiny$Y$]{};
    		\begin{scope}
    			\clip (-0.2,0) circle (\vsize);
    			\clip (0.2,0) circle (\vsize);
    		\end{scope}
    		\draw (-0.2,0) circle (\vsize);
    		\draw (0.2,0) circle (\vsize);
    		\useasboundingbox (current bounding box);
    		\node at (-0.7, 0.6){\slshape\color{subfiglabelcolor}\thesubfigure};
    	\end{tikzpicture}
    \hspace{\spacerlength}\hspace{2.5pt}\vrule\hspace{2.5pt}\hspace{\spacerlength}
    	\begin{tikzpicture}[center base,is bn] \refstepcounter{subfigure}\label{subfig:1XYZ}
    		\node[dpad0] (1) at (-0.5,2.3){$\var1$};
    		\node[dpad0] (X) at (-0.5,1.5){$X$};
    		\node[dpad0] (Y) at (0.35,1.25){$Y$};
    		\node[dpad0] (Z) at (0.25,2.25){$Z$};subfiglabelcolor
    		\draw[arr1] (1) to (X);
    		\draw[arr1] (X) to[] (Y);
    		\draw[arr2] (Y) to[] (Z);
    		\path[fill=white!70!black] (210:0.22) circle (\vsize) ++(-130:.25) node[label=below:\tiny$X$]{};
    		\path[fill=white!70!black] (-30:0.22) circle (\vsize) ++(-50:.25) node[label=below:\tiny$Y$]{};
    		\path[fill=white!70!black] (90:0.22) circle (\vsize) ++(40:.29) node[label=above:\tiny$Z$]{};
    		\begin{scope}
    			\clip (90:0.22) circle (\vsize);
    			\clip (210:0.22) circle (\vsize);
    			\fill[red!50!black] (-1,-1) rectangle (3,3);
    			\clip (-30:0.22) circle (\vsize);
    			\fill[white!70!black] (-1,-1) rectangle (3,3);
    		\end{scope}
    		\begin{scope}
    			\draw[] (-30:0.22) circle (\vsize);
    			\draw[] (210:0.22) circle (\vsize);		
    			\draw[] (90:0.22) circle (\vsize);
    		\end{scope}
    		\useasboundingbox (current bounding box);
    		\node at (-0.7, 0.7){\slshape\color{subfiglabelcolor}\thesubfigure};
    	\end{tikzpicture}
    	\hspace{3pt}
    \hspace{\spacerlength}%
    	\begin{tikzpicture}[center base] \refstepcounter{subfigure}\label{subfig:XYZ-cycle}
    		\node[dpad0] (X) at (-0.5,1.75){$X$};
    		\node[dpad0] (Y) at (0.35,1.25){$Y$};
    		\node[dpad0] (Z) at (0.25,2.25){$Z$};
    		\draw[arr1] (X) to[bend right=25] (Y);
    		\draw[arr1] (Y) to[bend right=25] (Z);
    		\draw[arr1] (Z) to[bend right=25] (X);

    		\draw[fill=white!70!black] (210:0.22) circle (\vsize) ++(-130:.27) node[label=below:\tiny$X$]{};
    		\draw[fill=white!70!black] (-30:0.22) circle (\vsize) ++(-50:.27) node[label=below:\tiny$Y$]{};
    		\draw[fill=white!70!black] (90:0.22) circle (\vsize) ++(40:.31) node[label=above:\tiny$Z$]{};

    		\begin{scope}
    			\clip (-30:0.22) circle (\vsize);
    			\clip (210:0.22) circle (\vsize);
    			\clip (90:0.22) circle (\vsize);
    			\fill[green!50!black] (-1,-1) rectangle (3,3);
    		\end{scope}
    		\begin{scope}
    			\draw[] (-30:0.22) circle (\vsize);
    			\draw[] (210:0.22) circle (\vsize);		
    			\draw[] (90:0.22) circle (\vsize);
    		\end{scope}
    		\useasboundingbox (current bounding box);
    		\node at (-0.7, 0.7){\slshape\color{subfiglabelcolor}\thesubfigure};
    	\end{tikzpicture}
    \hspace{3pt}
    \hspace{\spacerlength}%
    	\begin{tikzpicture}[center base] \refstepcounter{subfigure}\label{subfig:XZtoY}
    		\node[dpad0] (X) at (-0.45,1.9){$X$};
    		\node[dpad0] (Y) at (0.3,1.25){$Y$};
    		\node[dpad0] (Z) at (0.4,2.15){$Z$};
    		\draw[arr0] (X) to[] (Y);
    		\draw[arr1] (Z) to[] (Y);
    		\path[fill=green!50!black] (210:0.22) circle (\vsize) ++(-130:.25) node[label=below:\tiny$X$]{};
    		\path[fill=red!50!black] (-30:0.22) circle (\vsize) ++(-50:.25) node[label=below:\tiny$Y$]{};
    		\path[fill=green!50!black] (90:0.22) circle (\vsize) ++(40:.29) node[label=above:\tiny$Z$]{};
    		\begin{scope}
    			\clip (-30:0.22) circle (\vsize);
    			\clip (90:0.22) circle (\vsize);
    			\fill[white!70!black] (-1,-1) rectangle (3,3);
    		\end{scope}
    		\begin{scope}
    			\clip (-30:0.22) circle (\vsize);
    			\clip (210:0.22) circle (\vsize);
    			\fill[white!70!black] (-1,-1) rectangle (3,3);

    			\clip (90:0.22) circle (\vsize);
    			\fill[green!50!black] (-1,-1) rectangle (3,3);
    		\end{scope}
    		\draw[] (-30:0.22) circle (\vsize);
    		\draw[] (210:0.22) circle (\vsize);		
    		\draw[] (90:0.22) circle (\vsize);
    		\useasboundingbox (current bounding box);
    		\node at (-0.7, 0.7){\slshape\color{subfiglabelcolor}\thesubfigure};
    	\end{tikzpicture}~
    	\hspace{\spacerlength}%
    		\begin{tikzpicture}[center base] \refstepcounter{subfigure}\label{subfig:XYZ-bichain}
    			\node[dpad0] (X) at (-0.3,1.2){$X$};
    			\node[dpad0] (Y) at (0.3,1.9){$Y$};
    			\node[dpad0] (Z) at (-0.35,2.5){$Z$};
    			\draw[arr1] (X) to[bend right=15] (Y);
    			\draw[arr1] (Y) to[bend right=15] (X);
    			\draw[arr1] (Y) to[bend right=15] (Z);
    			\draw[arr1] (Z) to[bend right=15] (Y);
    			\path[fill=white!70!black] (210:0.22) circle (\vsize) ++(-130:.25) node[label=below:\tiny$X$]{};
    			\path[fill=red!50!black] (-30:0.22) circle (\vsize) ++(-50:.25) node[label=below:\tiny$Y$]{};
    			\path[fill=white!70!black] (90:0.22) circle (\vsize) ++(40:.29) node[label=above:\tiny$Z$]{};
    			\begin{scope}
    				\clip (-30:0.22) circle (\vsize);
    				\clip (90:0.22) circle (\vsize);
    				\fill[white!70!black] (-1,-1) rectangle (3,3);
    			\end{scope}
    			\begin{scope}
    				\clip (90:0.22) circle (\vsize);
    				\clip (210:0.22) circle (\vsize);
    				\fill[red!50!black] (-1,-1) rectangle (3,3);
    			\end{scope}
    			\begin{scope}
    				\clip (-30:0.22) circle (\vsize);
    				\clip (210:0.22) circle (\vsize);
    				\fill[white!70!black] (-1,-1) rectangle (3,3);

    				\clip (90:0.22) circle (\vsize);
    				\fill[green!50!black] (-1,-1) rectangle (3,3);
    			\end{scope}
    			\draw[] (-30:0.22) circle (\vsize);
    			\draw[] (210:0.22) circle (\vsize);		
    			\draw[] (90:0.22) circle (\vsize);
    			\useasboundingbox (current bounding box);
    			\node at (-0.7, 0.7){\slshape\color{subfiglabelcolor}\thesubfigure};
    		\end{tikzpicture}
    }
    \addtocounter{figure}{-1} %
    \caption{
	Illustrations of example graph information
    	  functions $\{ \IDef{G_i} \}$, drawn underneath their
    	  associated multigraphs $\{ G_i\}$. Each circle represents a
    	  variable; an area in the intersection of circles $\{C_j\}$
    	  but outside of circles $\{D_k\}$ corresponds to information
    	  that is shared between all $C_j$'s, but not in any
    	  $D_k$. Variation of a candidate distribution $\mu$ in a
    	  green area makes its qualitative fit better (according to
    	  $\IDef{}$), while variation in a red area makes its
    	  qualitative fit worse; grey is neutral. Only the boxed
    	  structures in blue, whose graph information functions can be
    	  seen as assertions of (conditional) independence, are
    	  expressible as BNs.
		  } 

    \label{fig:info-diagram}
    \end{figure}
}%
\bibliographystyle{alpha}
\bibliography{allrefs,z,joe}        
\end{document}
