\documentclass{article}


\relax % Controls
    \newif\ifmarginprooflinks
    	\marginprooflinkstrue
    	% \marginprooflinksfalse

\relax % Bibliography, etc
	\usepackage[american]{babel}
	\usepackage{csquotes}
	\usepackage[backend=biber, style=authoryear]{biblatex}
	\DeclareLanguageMapping{american}{american-apa}
	% \usepackage[backend=biber,style=authoryear,hyperref=true]{biblatex}
	% \addbibresource{refs.bib}
	% \addbibresource{conf.bib}

	\DeclareFieldFormat{citehyperref}{%
	  \DeclareFieldAlias{bibhyperref}{noformat}% Avoid nested links
	  \bibhyperref{#1}}

	\DeclareFieldFormat{textcitehyperref}{%
	  \DeclareFieldAlias{bibhyperref}{noformat}% Avoid nested links
	  \bibhyperref{%
	    #1%
	    \ifbool{cbx:parens}
	      {\bibcloseparen\global\boolfalse{cbx:parens}}
	      {}}}

	\savebibmacro{cite}
	\savebibmacro{textcite}

	\renewbibmacro*{cite}{%
	  \printtext[citehyperref]{%
	    \restorebibmacro{cite}%
	    \usebibmacro{cite}}}

	\renewbibmacro*{textcite}{%
	  \ifboolexpr{
	    ( not test {\iffieldundef{prenote}} and
	      test {\ifnumequal{\value{citecount}}{1}} )
	    or
	    ( not test {\iffieldundef{postnote}} and
	      test {\ifnumequal{\value{citecount}}{\value{citetotal}}} )
	  }
	    {\DeclareFieldAlias{textcitehyperref}{noformat}}
	    {}%
	  \printtext[textcitehyperref]{%
	    \restorebibmacro{textcite}%
	    \usebibmacro{textcite}}}

	\DeclareCiteCommand{\brakcite}
	  {\usebibmacro{prenote}}
	  {\usebibmacro{citeindex}%
	   \printtext[bibhyperref]{[\usebibmacro{cite}]}}
	  {\multicitedelim}
	  {\usebibmacro{postnote}}

\relax % Standard Packages
    \usepackage[dvipsnames]{xcolor}
    % \usepackage[utf8]{inputenc}
    \usepackage{mathtools}
    \usepackage{amssymb}
		\DeclareMathSymbol{\shortminus}{\mathbin}{AMSa}{"39}
    % \usepackage{parskip}
    % \usepackage{algorithm}
    \usepackage{bbm}
	\usepackage{lmodern}
	% \usepackage{times}
    \usepackage{faktor}
    % \usepackage{booktabs}
	% \usepackage[margin=1in]{geometry}
    \usepackage{graphicx}
    \usepackage{scalerel}
    \usepackage{enumitem}
    \usepackage{nicefrac}\let\nf\nicefrac

    % \usepackage{color}
    %\usepackage{stmaryrd}
    \usepackage{hyperref} % Load before theorems...
        \hypersetup{colorlinks=true, linkcolor=blue!75!black, urlcolor=magenta, citecolor=green!50!black}

\usepackage{tikz}
	\usetikzlibrary{positioning,fit,calc, decorations, arrows, shapes, shapes.geometric}
	\usetikzlibrary{cd}

	%%%%%%%%%%%%
	\tikzset{AmpRep/.style={ampersand replacement=\&}}
	\tikzset{center base/.style={baseline={([yshift=-.8ex]current bounding box.center)}}}
	\tikzset{paperfig/.style={center base,scale=0.9, every node/.style={transform shape}}}

	% Node Stylings
	\tikzset{dpadded/.style={rounded corners=2, inner sep=0.7em, draw, outer sep=0.3em, fill={black!50}, fill opacity=0.08, text opacity=1}}
	\tikzset{dpad0/.style={outer sep=0.05em, inner sep=0.3em, draw=gray!75, rounded corners=4, fill=black!08, fill opacity=1, align=center}}
	\tikzset{dpadinline/.style={outer sep=0.05em, inner sep=2.5pt, rounded corners=2.5pt, draw=gray!75, fill=black!08, fill opacity=1, align=center, font=\small}}

 	\tikzset{dpad/.style args={#1}{every matrix/.append style={nodes={dpadded, #1}}}}
	\tikzset{light pad/.style={outer sep=0.2em, inner sep=0.5em, draw=gray!50}}

	\tikzset{arr/.style={draw, ->, thick, shorten <=3pt, shorten >=3pt}}
	\tikzset{arr0/.style={draw, ->, thick, shorten <=0pt, shorten >=0pt}}
	\tikzset{arr1/.style={draw, ->, thick, shorten <=1pt, shorten >=1pt}}
	\tikzset{arr2/.style={draw, ->, thick, shorten <=2pt, shorten >=2pt}}

	\newcommand\cmergearr[5][]{
		\draw[arr, #1, -] (#2) -- (#5) -- (#3);
		\draw[arr, #1, shorten <=0] (#5) -- (#4);
		}
	\newcommand\mergearr[4][]{
		\coordinate (center-#2#3#4) at (barycentric cs:#2=1,#3=1,#4=1.2);
		\cmergearr[#1]{#2}{#3}{#4}{center-#2#3#4}
		}
	\newcommand\cunmergearr[5][]{
		\draw[arr, #1, -, shorten >=0] (#2) -- (#5);
		\draw[arr, #1, shorten <=0] (#5) -- (#3);
		\draw[arr, #1, shorten <=0] (#5) -- (#4);
		}
	\newcommand\unmergearr[4][]{
		\coordinate (center-#2#3#4) at (barycentric cs:#2=1.2,#3=1,#4=1);
		\cunmergearr[#1]{#2}{#3}{#4}{center-#2#3#4}
		}

\usepackage{amsthm,thmtools} % Theorem Macros
	\usepackage[noabbrev,nameinlink,capitalize]{cleveref}
    \theoremstyle{plain}
    \newtheorem{theorem}{Theorem}
	\newtheorem{coro}{Corollary}[theorem]
    \newtheorem{prop}[theorem]{Proposition}
    \newtheorem{conj}[theorem]{Conjecture}
    \newtheorem{claim}{Claim}
    \newtheorem{remark}{Remark}
    \newtheorem{lemma}[theorem]{Lemma}
    \theoremstyle{definition}
    % \newtheorem{defn}{Definition}
    % \declaretheorem[name=Definition]{defn}
    \declaretheorem[name=Definition, qed=$\square$]{defn}
    \declaretheorem[name=Example, qed=$\triangle$]{example}

	\crefname{defn}{Definition}{Definitions}
	\crefname{prop}{Proposition}{Propositions}
    \crefname{issue}{Issue}{Issues}

\relax %%%%%%%%% GENERAL MACROS %%%%%%%%
    \let\Horig\H
	\let\H\relax
	\DeclareMathOperator{\H}{\mathrm{H}} % Entropy
	\DeclareMathOperator{\I}{\mathrm{I}} % Information
	\DeclareMathOperator*{\Ex}{\mathbb{E}} % Expectation
	\DeclareMathOperator*{\EX}{\scalebox{1.5}{$\mathbb{E}$}}

    \newcommand{\mat}[1]{\mathbf{#1}}
    \DeclarePairedDelimiterX{\infdivx}[2]{(}{)}{%
		#1\;\delimsize\|\;#2%
	}
	\newcommand{\thickD}{I\mkern-8muD}
	\newcommand{\kldiv}{\thickD\infdivx}
	\newcommand{\tto}{\rightarrow\mathrel{\mspace{-15mu}}\rightarrow}

	\newcommand{\datadist}[1]{\Pr\nolimits_{#1}}
	% \newcommand{\datadist}[1]{p_\text{data}}

	\makeatletter
	\newcommand{\subalign}[1]{%
	  \vcenter{%
	    \Let@ \restore@math@cr \default@tag
	    \baselineskip\fontdimen10 \scriptfont\tw@
	    \advance\baselineskip\fontdimen12 \scriptfont\tw@
	    \lineskip\thr@@\fontdimen8 \scriptfont\thr@@
	    \lineskiplimit\lineskip
	    \ialign{\hfil$\m@th\scriptstyle##$&$\m@th\scriptstyle{}##$\hfil\crcr
	      #1\crcr
	    }%
	  }%
	}
	\makeatother
	\newcommand\numberthis{\addtocounter{equation}{1}\tag{\theequation}}

\relax %%%%%%%%%   PDG  MACROS   %%%%%%%%
	\newcommand{\ssub}[1]{_{\!_{#1}\!}}
	% \newcommand{\bp}[1][L]{\mat{p}_{\!_{#1}\!}}
	% \newcommand{\bP}[1][L]{\mat{P}_{\!_{#1}\!}}
	\newcommand{\bp}[1][L]{\mat{p}\ssub{#1}}
	\newcommand{\bP}[1][L]{\mat{P}\ssub{#1}}
	\newcommand{\V}{\mathcal V}
	\newcommand{\N}{\mathcal N}
	\newcommand{\Ed}{\mathcal E}

    \newcommand{\balpha}{\boldsymbol\alpha}
    \newcommand{\bbeta}{\boldsymbol\beta}

	\DeclareMathAlphabet{\mathdcal}{U}{dutchcal}{m}{n}
	\DeclareMathAlphabet{\mathbdcal}{U}{dutchcal}{b}{n}
	\newcommand{\dg}[1]{\mathbdcal{#1}}
	\newcommand{\PDGof}[1]{{\dg M}_{#1}}
	\newcommand{\UPDGof}[1]{{\dg N}_{#1}}
	\newcommand\VFE{\mathit{V\mkern-4mu F\mkern-4.5mu E}}

	\newcommand\Inc{\mathit{Inc}}
	\newcommand{\IDef}[1]{\mathit{IDef}_{\!#1}}
	% \newcommand{\ed}[3]{%
	% 	\mathchoice%
	% 	{#2\overset{\smash{\mskip-5mu\raisebox{-3pt}{${#1}$}}}{\xrightarrow{\hphantom{\scriptstyle {#1}}}} #3} %display style
	% 	{#2\overset{\smash{\mskip-5mu\raisebox{-3pt}{$\scriptstyle {#1}$}}}{\xrightarrow{\hphantom{\scriptstyle {#1}}}} #3}% text style
	% 	{#2\overset{\smash{\mskip-5mu\raisebox{-3pt}{$\scriptscriptstyle {#1}$}}}{\xrightarrow{\hphantom{\scriptscriptstyle {#1}}}} #3} %script style
	% 	{#2\overset{\smash{\mskip-5mu\raisebox{-3pt}{$\scriptscriptstyle {#1}$}}}{\xrightarrow{\hphantom{\scriptscriptstyle {#1}}}} #3}} %scriptscriptstyle
	\newcommand{\ed}[3]{#2%
	  \overset{\smash{\mskip-5mu\raisebox{-1pt}{$\scriptscriptstyle
	        #1$}}}{\rightarrow} #3}

    \newcommand{\nhphantom}[2]{\sbox0{\kern-2%
		\nulldelimiterspace$\left.\delimsize#1\vphantom{#2}\right.$}\hspace{-.97\wd0}}
		% \nulldelimiterspace$\left.\delimsize#1%
		% \vrule depth\dp#2 height \ht#2 width0pt\right.$}\hspace{-.97\wd0}}
	\makeatletter
	\newsavebox{\abcmycontentbox}
	\newcommand\DeclareDoubleDelim[5]{
	    \DeclarePairedDelimiterXPP{#1}[1]%
			{% box must be saved in this pre code
				\sbox{\abcmycontentbox}{\ensuremath{##1}}%
			}{#2}{#5}{}%
		    %%% Correct spacing, but doesn't work with externalize.
			% {\nhphantom{#3}{##1}\hspace{1.2pt}\delimsize#3\mathopen{}##1\mathclose{}\delimsize#4\hspace{1.2pt}\nhphantom{#4}{##1}}
			%%% Fast, but wrong spacing.
			% {\nhphantom{#3}{~}\hspace{1.2pt}\delimsize#3\mathopen{}##1\mathclose{}\delimsize#4\hspace{1.2pt}\nhphantom{#4}{~}}
			%%% with savebox.
		    {%
				\nhphantom{#3}{\usebox\abcmycontentbox}%
				\hspace{1.2pt} \delimsize#3%
				\mathopen{}\usebox{\abcmycontentbox}\mathclose{}%
				\delimsize#4\hspace{1.2pt}%
				\nhphantom{#4}{\usebox\abcmycontentbox}%
			}%
	}
	\makeatother
	\DeclareDoubleDelim
		\SD\{\{\}\}
	\DeclareDoubleDelim
		\bbr[[]]
	% \DeclareDoubleDelim
	% 	\aar\langle\langle\rangle\rangle
	\makeatletter
	\newsavebox{\aar@content}
	\newcommand\aar{\@ifstar\aar@one@star\aar@plain}
	\newcommand\aar@one@star{\@ifstar\aar@resize{\aar@plain*}}
	\newcommand\aar@resize[1]{\sbox{\aar@content}{#1}\scaleleftright[3.8ex]
		{\Biggl\langle\!\!\!\!\Biggl\langle}{\usebox{\aar@content}}
		{\Biggr\rangle\!\!\!\!\Biggr\rangle}}
	\DeclareDoubleDelim
		\aar@plain\langle\langle\rangle\rangle
	\makeatother


	% \DeclarePairedDelimiterX{\aar}[1]{\langle}{\rangle}
	% 	{\nhphantom{\langle}{#1}\hspace{1.2pt}\delimsize\langle\mathopen{}#1\mathclose{}\delimsize\rangle\hspace{1.2pt}\nhphantom{\rangle}{#1}}

\relax %%%%% restatables and links
	% \usepackage{xstring} % for expandarg
	\usepackage{xpatch}
	\makeatletter
	\xpatchcmd{\thmt@restatable}% Edit \thmt@restatable
	   {\csname #2\@xa\endcsname\ifx\@nx#1\@nx\else[{#1}]\fi}% Replace this code
	   % {\ifthmt@thisistheone\csname #2\@xa\endcsname\typeout{oiii[#1;#2\@xa;#3;\csname thmt@stored@#3\endcsname]}\ifx\@nx#1\@nx\else[#1]\fi\else\csname #2\@xa\endcsname\fi}% with this code
	   {\ifthmt@thisistheone\csname #2\@xa\endcsname\ifx\@nx#1\@nx\else[{#1}]\fi
	   \else\fi}
	   {}{\typeout{FIRST PATCH TO THM RESTATE FAILED}} % execute on success/failure
	\xpatchcmd{\thmt@restatable}% A second edit to \thmt@restatable
	   {\csname end#2\endcsname}
	   {\ifthmt@thisistheone\csname end#2\endcsname\else\fi}
	   {}{\typeout{FAILED SECOND THMT RESTATE PATCH}}

	% \def\onlyaftercolon#1:#2{#2}
	\newcommand{\recall}[1]{\medskip\par\noindent{\bf \Cref{thmt@@#1}.} \begingroup\em \noindent
	   \expandafter\csname#1\endcsname* \endgroup\par\smallskip}

   	\setlength\marginparwidth{1.55cm}
	\newenvironment{linked}[3][]{%
		\def\linkedproof{#3}%
		\def\linkedtype{#2}%
		% \reversemarginpar
		% \marginpar{%
		% \vspace{1.1em}
		% % \hspace{2em}
		% 	% \raggedleft
		% 	\raggedright
		% 	\hyperref[proof:\linkedproof]{%
		% 	\color{blue!50!white}
		% 	\scaleleftright{$\Big[$}{\,{\small\raggedleft\tt\begin{tabular}{@{}c@{}} proof of \\\linkedtype~\ref*{\linkedtype:\linkedproof}\end{tabular}}\,}{$\Big]$}}
		% 	}%
        % \restatable[#1]{#2}{#2:#3}\label{#2:#3}%
		\ifmarginprooflinks
		\marginpar{%
			% \vspace{-3em}% %% for bottom
			\vspace{1.5em}
			\centering%
			\hyperref[proof:\linkedproof]{%
            % \hyperref[proof:#3]{
			\color{blue!30!white}%
			\scaleleftright{$\Big[$}{\,\mbox{\footnotesize\centering\tt\begin{tabular}{@{}c@{}}
				% proof of \\\,\linkedtype~\ref*{\linkedtype:\linkedproof}
				link to\\[-0.15em]
				proof
			\end{tabular}}\,}{$\Big]$}}~
			}%
		\fi
        \restatable[#1]{#2}{#2:#3}\label{#2:#3}%
        }%
		{\endrestatable%
		}
	\makeatother
		\newcounter{proofcntr}
		\newenvironment{lproof}{\begin{proof}\refstepcounter{proofcntr}}{\end{proof}}

		\usepackage{cancel}
		\newcommand{\Cancel}[2][black]{{\color{#1}\cancel{\color{black}#2}}}

		\usepackage{tcolorbox}
		\tcbuselibrary{most}
		\tcolorboxenvironment{lproof}{
			% fonttitle=\bfseries,
			% top=0.5em,
			enhanced,
			parbox=false,
			boxrule=0pt,
			frame hidden,
			borderline west={4pt}{0pt}{blue!20!black!40!white},
			% coltext={blue!20!black!60!white},
			colback={blue!20!black!05!white},
			sharp corners,
			breakable,
			% bottomsep at break=4cm,
			% enlarge bottom at break by=-4cm,
			% topsep at break=3cm,
			% enlarge top at break by=-3cm
		}
		% \usepackage[framemethod=TikZ]{mdframed}
		% \surroundwithmdframed[ % lproof
		% 	   topline=false,
		% 	   linewidth=3pt,
		% 	   linecolor=gray!20!white,
		% 	   rightline=false,
		% 	   bottomline=false,
		% 	   leftmargin=0pt,
		% 	   % innerleftmargin=5pt,
		% 	   skipabove=\medskipamount,
		% 	   skipbelow=\medskipamount
		% 	]{lproof}
	%oli16: The extra space was because there was extra space in the paragraph, not
	%because this length was too big. By breaking arrays, everything will be better.
	\newcommand{\begthm}[3][]{\begin{#2}[{name=#1},restate=#3,label=#3]}

\relax %TODOs and footnotes
    \newcommand{\TODO}[1][INCOMPLETE]{{\centering\Large\color{red}$\langle$~\texttt{#1}~$\rangle$\par}}
    \newcommand{\dfootnote}[1]{%
        \let\oldthefootnote=\thefootnote%
        % \addtocounter{footnote}{-1}%
		\setcounter{footnote}{999}
        \renewcommand{\thefootnote}{\textdagger}%
        \footnote{#1}%
        \let\thefootnote=\oldthefootnote%
    }
	\newcommand{\dfootnotemark}{
		\footnotemark[999]
	}


% \usepackage[framemethod=TikZ]{mdframed}
% \colorlet{color1}{Emerald}
% \colorlet{color3}{color1>wheel,2,3}
% \colorlet{pinkish}{color3!25!magenta}
\definecolor{brownish}{rgb}{0.5, 0.2, 0.1}
% \newmdenv[roundcorner=5pt, 
%     backgroundcolor=brownish!20!white,
%     % frametitle={$\langle$under construction$\rangle$},
%     frametitle={$\langle$incomplete or buggy$\rangle$},
%     frametitlerule=false,
%     innertopmargin=3pt, frametitlebelowskip=1ex, frametitleaboveskip=1ex,
%     frametitlebackgroundcolor=brownish!40!white,
%     skipabove=1em,skipbelow=1em,
%     frametitlefont={\scshape},leftmargin=-10pt, rightmargin=-10pt]
% 		{wip}
\newtcolorbox{wip}{%
    colback=brownish!20!white,%
    % frametitle={$\langle$under construction$\rangle$},
    title={$\langle$under construction$\rangle$},%
    enhanced jigsaw,
    breakable,
    % frametitlerule=false,
    % innertopmargin=3pt, frametitlebelowskip=1ex, frametitleaboveskip=1ex,
    colframe=brownish!40!white,%
    % skipabove=1em,skipbelow=1em,
    % frametitlefont={\scshape},leftmargin=-10pt, rightmargin=-10pt
}
        
% \tcolorboxenvironment{example}{
\newtcolorbox[use counter=example]{examplex}{
        fonttitle=\bfseries,
        % empty,
        enhanced jigsaw,
        title={Example \thetcbcounter.},
        before=\par\medskip\noindent,
        % top=0.5em,
        % enhanced,
        parbox=false,
        % boxrule=0pt,
        % frame hidden,
        % borderline west={4pt}{0pt}{green!20!black!40!white},
        % coltext={blue!20!black!60!white},
        % colback={blue!20!black!05!white},
        colback=white,
        sharp corners,
        breakable,
        % bottomsep at break=4cm,
        % enlarge bottom at break by=-4cm,
        % topsep at break=3cm,
        % enlarge top at break by=-3cm
    }

\usepackage{enumitem}
\setlength\parskip{1ex}

\newcommand\pts{\mathop{\mathit{pts}}}

\begin{document}
    \section{Introduction}
    
    We define a commitment machines
    
    \section{Preliminaries}
    
    
    \subsection*{Convex Relations}
    Let $\mat A = \{ A_1, \ldots, A_n \}$ be a set of sets, which we will call attributes.
    A convex relation $R(\mat A)$, or simply $R$,  is a function
        $R : \prod_{i=1}^n A_i \to [0,1]$.
    There are two two extremal relations, the zero relation and the one relation.
    Suppose $R$ and $S$ are convex relations over the same attributes $\mat A$. We can form new relations in a number of ways.
    % First, here are some 
    \begin{itemize}
        \item \textbf{convex combination.} For $\alpha \in [0,1]$, we get a convex relation $(R : \alpha : S)(\mat A) := (1-\alpha) R(\mat A) + \alpha S(\mat A)$.
        % \item \textbf{.} More generally, if $\cal R = \{ R_1, \ldots, R_n\}$ is a set of convex relations with each $R_i \in \square A$ over the same attributes $A$, and $M : \mathcal R \to [0,1]$ is a convex relation on that set of relations, then 
        % \[
        %     \mathcal R ^{M} (A) := \sum_{i=1}^n R_i(A) M(R_i)
        % \]  
        \item \textbf{warping.} For $\gamma \ge 0$,  $R^\gamma(\mat A)$ is also a relation on $\mat A$.
        \item \textbf{multiplication.} $(R \cdot S)(\mat A) := R(\mat A) \cdot S(\mat A)$.
        \item \textbf{max.} $(R \land S)(\mat A) := \max \{ R(\mat A),  S(\mat A) \}$.
        \item \textbf{axis normalization.} for a subset $\mat B \subseteq \mat A$ of attributes, we can define ...
        
        \TODO
        
        % \begin{enumerate}
        %     \item \textit{(sum)} 
        %     \[
        %         R[\mat B] := 
        %     \]
        % \end{enumerate}
    \end{itemize}
    Let $\square \mat A$ denote the set of all convex relations on attributes $\mat A$. 
    
    % Here are some examples of convex relations.
    Lots of things can be represented as convex relations; here are a few. 
    \begin{enumerate}
        \item an ``ordinary'' relation $R(\mat A)$
        % is a subset of 
            % $A_1 \times \cdots\times A_n$, or a function 
            % $R : \prod_{i=1}^n A_i \to \{0,1\}$.
            can be viewed as the special case of a convex relation whose output is always zero or one, and not anything in between.
        \item a probability distribution $p(X)$ can be viewed as a convex relation on one attribute: the values $\V X$ that $X$ can take, with the special property that 
        $\sum_{x \in \V X}p(X) = 1$.
    \end{enumerate}
    
    \begin{remark}
        % Some facts.
        \begin{enumerate}
            \item ordinary relations are invariant under warping. 
            \item the convex relations that are both ordinary relations and probability distributions are deterministic distributions.
            % \item 
        \end{enumerate}
    \end{remark}
    
    \textbf{Operations on convex relations}
    
    \section{Formalism}
    The core idea is to provide a model in terms of two parts: a manifold $\Theta$
    of configurations and a language $\Phi$. 
    % a manifold $\Theta$ and $\Phi$. 
    The elements of $\Theta$ are states: think parameters, probability distributions, data, assignements to program variables, and so forth.
    Meanwhile, the elements of $\Phi$ are elements of elements

    \begin{defn}
        A \emph{commitment function} is a function
        \[
            F: \Theta \times \Phi \times [0,1] \to \Theta,
        \]
        % whose application we will compress by writing $F^c_\phi\theta$ instead of $F(\theta, \phi, c)$,
        satisfying the following properties:
        \begin{enumerate}[nosep,label={CF\arabic*.}]
            \item $F$ is differentiable;
            % \item for all $\phi,\theta$, we have $F^0_\phi\theta = \theta$;
            \item for all $\phi$, the function $F^0_\phi : \Theta \to \Theta$ is the identity;
            \item for all $\phi$, the function $F^1_\phi : \Theta \to \Theta$ is idempotent.
        \end{enumerate}
    \end{defn}
    
    
    
    \begin{defn}
        A \emph{commitment system}
         % $(\Theta, \Phi, \chi)$.
        is a set of runs, each of which has state $\theta(t)$ that could evolve in time,
        and a commitment $\chi(\phi, t)$ in each 
        
        Formally, a commitment system is a tuple 
        % $(\chi, \theta)$
        $(\mathcal R, \Theta, \Phi, \chi, \theta, f)$,
        where
        \begin{itemize}[]
            \item $\mathcal R$ is a set of possible runs.
            % to each $r \in \mathcal R$ of which is a 1-dimensional differentiable manifold $T_r$ of times associated to $r$ (that is, ). 
            % We write $(r,t) \in \mathcal R$ to refer to a run $r$ and a time $t \in T_r$.
            Let $\pts \mathcal R$ refer to the set of points $(r,t)$ such that $r \in \mathcal R$ and $t \in \mathbb R_{\ge 0}$.
            % and $t \in T_r$.            
            % In most cases, we are interested in 
            % Unless specified otherwise, we assume $T_r = \mathbb R_{\ge 0}$.
            \item $\Theta$ is a differentiable manifold of possible configurations.
            \item $\vartheta : \pts \mathcal R \to \Theta$ is a function that                describes the configuration $\vartheta(r,t)$ of the system at time $t$ in run $r$.
            \item
            % $\chi: (r: \mathcal R) \times T_r \times \Phi \to \mathbb R$
            $\chi: \pts \mathcal R \times \Phi \to [0,1]$
            is a convex relation between points and language; 
            the number $\chi(r, t, \phi) \in [0,1]$ represents the ``commitment'' of the system to the statement $\phi$ at time $t$ and run $r$.
            
            % \item
            %  $f: \Phi \to  \mathfrak X(\Theta)$ describes the dynamics of the system, 
            %  by assigning a vector field $f_\phi : \Theta \to T\Theta$ to each sentence of the language.
            \item $F : \Theta \time \Phi \times [0,1] \to \Theta$ is a commitment function.
        \end{itemize}
        Furthermore, we require that the configurations and commitments be compatible with the dynmamics.
        This amounts to requiring that,
        for each $(r,t) \in \pts \mathcal R$, 
            if $\chi(r,t,\phi) < 1$ for all $\phi \in \Phi$, then
        the configuration trajectory $t' \mapsto \vartheta(r,t')$ is continuously differentiable in $t'$ at $t'=t$, and
        \[
            \dot\vartheta(r,t) = \frac{\partial}{\partial t} \vartheta(r, t) = 
                % \sum_{\phi \in \Phi} \chi(r,t,\phi) f(\phi)(\vartheta(r,t))
                % \sum_{\phi \in \Phi} -\log (1 - \chi(r,t,\phi)) f_\phi(\vartheta(r,t)).
                \sum_{\phi \in \Phi} -\log (1 - \chi(r,t,\phi)) F'_\phi(\vartheta(r,t)).
                % \int_{\Phi} f(\chi()) \mathrm d\phi
        \]
        % for each $(r,t) \in \pts \mathcal R$, and 
        % \[
        %     \vartheta(r, t + \tau) =     
        %     % \dot\vartheta(r,t) = \frac{\partial}{\partial t} \vartheta(r, t) = 
        %     %     \sum_{\phi \in \Phi} -\log (1 - \chi(r,t,\phi)) f_\phi(\vartheta(r,t)).
        % \]
        Otherwise, if $\chi(r,t,\phi) = 1$, we require that
        $\displaystyle
            \lim_{\epsilon \to 0} \vartheta(r,t+\epsilon) = 
                F^1_\phi(\vartheta(r,t))
        $.
        % the limit $\lim_{\epsilon \to 0} \vartheta(r,t+\epsilon)$ exists and is equal to
        % the 
        % \[
        %     \lim_{\epsilon \to 0} \vartheta(r,t+\epsilon) = 
        %     % \vartheta(r,t) +   
        %     \lim_{\tau \to \infty}   \int_{}
        % \]
        
        % \begin{align*}
        %     \Big(
        %         \Omega,\quad
        %         \chi : \Omega \times T \times \Phi \to \mathbb R,\quad
        %         \theta : \Omega \times T \to \Theta, \quad
        %         F : \Phi \to \mathfrak X(\Theta)
        %     \Big) 
        % \end{align*}
        Since $\Theta$, $\mathcal R,$ and $\Phi$ are implicit in the type signatures of $\theta$ and $\chi$, we can refer to the system without loss of precision as via the triple $(\theta, \chi, F)$. 
    \end{defn}
    
    \begin{remark}
        \begin{enumerate}
            \item 
            $(\Theta, F: \Theta \to \Theta^{})$ is a coalgebra for the signature $G_\Phi(-) := (-)^{\Phi \times [0,1]}$ of partial updates.
            \item $(\Phi, \iota : \mathrm{Exp} \Phi \to \Phi)$ is a algebra for the signature of expressions in the language $\Phi$. 
        \end{enumerate}
    \end{remark}
    
    
    We now present two generic ways to describe a system: with an external supply of information, and by autonomous evolution. 
    
    \begin{defn}
        The \emph{controlled commitment system} 
        $\mathit{ControlSys}(F)$
        generated by a commitment function $F$ is the set  of all possible trajectories of configurations and commitments that are compatible with $F$. 
        Explicitly, $\mathcal R$ is given by
        \[
            \Big\{ (\theta_0, h) ~\Big|~  \theta_0 \in \Theta,~
                h : \Phi \times \mathbb R_{\ge 0} \to [0,1],~
                 \Big\}
        \]
        so that $\pts \mathcal R = \mathcal R \times \mathbb R_{\ge 0}$,
        and then define
        $\chi((\theta_0,h,t),\phi) := h(\phi, t)$,
        and 
        $\vartheta(\theta,h,-)$ to be the unique solution $\theta(t)$ to the ODE
        \[
            \theta(0) = \theta_0,
            \qquad \frac{\mathrm d\theta}{\mathrm d t} (t) = 
                \sum_{\phi \in \Phi} h(\phi, t) F'_\phi(\theta(t)).
        \]
        % \begin{align*}
            % \varphi(\phi,t)
        % \end{align*}
    \end{defn}
    
    If we know something about the control signal $\chi = h$, we can also restrict to subsystems generated by only a subset of possible 
    control signals $h$. For instance, we can look at the system 
    
    \begin{defn}
        The \emph{discretely controlled commitment system} generated by the commitment function $F$ is the subsystem of $\mathit{ControlSys}(F)$ in which the control signal $\chi$ either commits fully or not at all---that is,
        % 
        the restriction of it to runs $r = (\theta_0, h)$ such that $h(\phi, t) \in \{0,1\}$
                for all $t \ge 0$ and $\phi \in \Phi$.
        % there exists some $\phi^* \in \Phi$ such that
        % the commitment $\chi(r,t,\phi) = \mathbbm 1[ \phi = \phi^*]$.
    \end{defn}


    % 
    % \begin{remark}
    %     As a result
    % \end{remark}
    
    
    \begin{example}[probabilistic conditioning]
        Consider a measurable space $\mathcal X = (\Omega, \mathcal A)$ of outcomes, with a base measure $\lambda(X)$.
        Updating probabilistic by conditioning in response to new information may be described by a commitment system whose states $\Theta := \Delta \mathcal X$ are probability distributions, and whose language $\Phi = \mathcal A$ is the $\sigma$-algebra of events. 
        
        Now, fix any commitment function such that $F(A,\mu,1) = \mu | A$, such as the linear one 
        \[ 
            F(A, \mu, c) := 
             \begin{cases}
                (1-c) \mu + (c)\, \mu | A & \text{~if~} \mu(A) > 0 \\
                \mu & \text{~if~} \mu(A)=0
            \end{cases}
        \]
        Now, the discretely controlled commitment system generated by $F$ consists of all runs that begin with a distribution, and recieve events  
    \end{example}
    


    
    
    \section{A Compilation Target}
    
    \begin{center}
    \tikzset{lang/.style={align=center,draw,outer sep=3pt}}
    \begin{tikzcd}
        \node[lang] (pdg) {PDGs};
        \node[lang,below left=0 and 1 of pdg] (bn) {BNs};
        \node[lang,above left=0 and 1 of pdg] (fg) {FGs};
        \node[lang,below right=-0.1 and 1 of pdg] (opt) {Optimization\\Problems};
        \node[lang,below=1 of pdg] (ml) {Simple\\ML Systems};
        %
        \node[lang,below=0.5 of opt] (scs) {Stationary\\ Commitment\\ Systems};
        \node[lang,above right=-0.2 and 1 of scs] (acs) {Autonomous\\ Commitment\\ Systems};
        \node[lang,below=1 of acs] (ccs) {Control\\ Commitment\\ Systems};
        %
        \draw[right hook->]
            (bn) edge (fg)
            (fg) edge (pdg)
            (bn) edge (pdg)
            (pdg) edge (opt)
            (pdg) edge (ml)
            (ml) edge (opt)
            (scs) edge (acs)
            (scs) edge (ccs);
        \draw[double] (opt) edge (scs);            
        %    
    \end{tikzcd}
    \end{center}
    
    \subsection{ML Systems}
    \subsection{PDGs to }
    
        
    \begin{example}[factor graphs as commitment systems]
        Consider a factor graph 
        % with factors $f_1(X_1), \ldots, f_n(X_n)$. 
        $\{ f_j(\mat X_j) \}_{j \in J}$ over variables $\mathcal X = \cup_{j \in J} \mat X_j$. 
        The corresponding system has a run for all positive probability distributions $\mu_0 \in \Delta^{\!\!{+}\!} \mathcal X$, in which $\mu$ approaches the optimal proability, which is the solution to the ODE
        \begin{align*}
            \theta(0) = \mu_0;
                \qquad
            \dot\theta(t) = 
        \end{align*}
        So, in all, we have
        % \[
        %     \mathccal R
        % \]
    \end{example}
    
    
    

    
\end{document}
