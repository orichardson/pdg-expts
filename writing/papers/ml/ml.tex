\documentclass{article}

% \usepackage[margin=1in]{geometry}
\usepackage{mathtools} %also loads amsmath
\usepackage{amssymb, bbm}

\usepackage[backend=biber,
	style=alphabetic,
	%	citestyle=authoryear,
	natbib=true,
	url=true, 
	doi=true]{biblatex}

%\usepackage{blkarray} % for matrices with labels
\usepackage{microtype}
\usepackage{relsize}
\usepackage{environ}% http://ctan.org/pkg/environ; for capturing body as a parameter for idxmats
\usepackage{tikz}
	\usetikzlibrary{positioning,fit,calc, decorations, arrows, shapes, shapes.geometric}
	\usetikzlibrary{cd}
	
	\pgfdeclaredecoration{arrows}{draw}{
		\state{draw}[width=\pgfdecoratedinputsegmentlength]{%
			\path [every arrow subpath/.try] \pgfextra{%
				\pgfpathmoveto{\pgfpointdecoratedinputsegmentfirst}%
				\pgfpathlineto{\pgfpointdecoratedinputsegmentlast}%
			};
	}}
	%%%%%%%%%%%%
	\tikzset{center base/.style={baseline={([yshift=-.8ex]current bounding box.center)}}}
	
	\tikzset{dpadded/.style={rounded corners=2, inner sep=0.9em, draw, outer sep=0.4em, fill=gray, fill opacity=0.08, text opacity=1}}
	\tikzset{active/.style={fill=blue, fill opacity=0.1}}
	\tikzset{square/.style={regular polygon,regular polygon sides=4, rounded corners = 0}}
	\tikzset{octagon/.style={regular polygon,regular polygon sides=8, rounded corners = 0}}
	
	
	\tikzset{alternative/.style args={#1|#2|#3}{name=#1, circle, fill, inner sep=1pt,label={[name={lab-#1},gray!30!black]#3:\scriptsize #2}} }
	
	
	\tikzset{bpt/.style args={#1|#2}{alternative={#1|#2|above}} }
	\tikzset{tpt/.style args={#1|#2}{alternative={#1|#2|below}} }
	\tikzset{lpt/.style args={#1|#2}{alternative={#1|#2|left}} }
	\tikzset{rpt/.style args={#1|#2}{alternative={#1|#2|right}} }
	\tikzset{pt/.style args={#1}{alternative={#1|#1|above}} }
	

	\tikzset{mpt/.style args={#1|#2}{name=#1, circle, fill, inner sep=1pt,label={[name={lab-#1},gray]\scriptsize #2}} }
	\tikzset{pt/.style args={#1}{name=#1, circle, fill, inner sep=1pt,label={[name={lab-#1},gray]\scriptsize #1}} }
	
		
		 %\foreach \x in {#1}{(\x) (lab-\x) } 
		 
	\tikzset{Dom/.style args={#1 (#2) around #3}{dpadded, name=#2, label={[name={lab-#2},align=center] #1}, fit={ #3 } }}
	\tikzset{bDom/.style args={#1 (#2) around #3}{dpadded, name=#2, label={[name={lab-#2},align=center]below:#1}, fit={ #3 } }}
	\tikzset{arr/.style={draw, ->, thick, shorten <=3pt, shorten >=3pt}}
	\tikzset{archain/.style args={#1}{arr, every arrow subpath/.style={draw,arr, #1}, decoration=arrows, decorate}}
	%\tikzset{every label/.append style={text=red, font=\scriptsize}}
	\tikzset{dpad/.style args={#1}{every matrix/.append style={nodes={dpadded, #1}}}}
	\tikzset{light pad/.style={outer sep=0.2em, inner sep=0.5em, draw=gray!50}}
	
	
	\newcommand\cmergearr[4]{
		\draw[arr,-] (#1) -- (#4) -- (#2);
		\draw[arr, shorten <=0] (#4) -- (#3);
	}
	\newcommand\mergearr[3]{
		\coordinate (center-#1#2#3) at (barycentric cs:#1=1,#2=1,#3=1.2);
		\cmergearr{#1}{#2}{#3}{center-#1#2#3}
	}

\NewEnviron{ctikzpicture}{\begin{center}\expandafter\begin{tikzpicture}\BODY\end{tikzpicture}\end{center}}
%\newenvironment{ctikzpicture}
%	{\begin{center}\begin{tikzpicture}}
%	{\end{tikzpicture}\end{center}}

\usepackage{color}
\definecolor{deepgreen}{rgb}{0,0.5,0}

\usepackage[colorlinks=true, citecolor=deepgreen]{hyperref}


\setlength{\skip\footins}{1cm}
\setlength{\footnotesep}{0.4cm}


\usepackage{stmaryrd}
\usepackage{trimclip}

\makeatletter
\DeclareRobustCommand{\shortto}{%
	\mathrel{\mathpalette\short@to\relax}%
}

\newcommand{\short@to}[2]{%
	\mkern2mu
	\clipbox{{.5\width} 0 0 0}{$\m@th#1\vphantom{+}{\shortrightarrow}$}%
}
\makeatother

\usepackage{parskip}
\usepackage{amsthm, thmtools}
\usepackage{
	nameref,%\nameref
	hyperref,%\autoref
	% n.b. \Autoref is defined by thmtools
	cleveref,% \cref
	% n.b. cleveref after! hyperref
}

\hypersetup{colorlinks=true, linkcolor=blue, urlcolor=magenta}

\begingroup
\makeatletter
\@for\theoremstyle:=definition,remark,plain\do{%
	\expandafter\g@addto@macro\csname th@\theoremstyle\endcsname{%
		\addtolength\thm@preskip\parskip
	}%
}
\endgroup
\makeatother

\theoremstyle{plain}
\newtheorem{theorem}{Theorem}[section]
\newtheorem{coro}{Corollary}[theorem]
\newtheorem{prop}[theorem]{Proposition}
\newtheorem{lemma}[theorem]{Lemma}
\newtheorem{fact}[theorem]{Fact}
\newtheorem{conj}[theorem]{Conjecture}

\theoremstyle{definition}
\newtheorem{defn}{Definition}[section]
\newtheorem*{defn*}{Definition}
\newtheorem{examplex}{Example}[section]
\newenvironment{example}
	{\pushQED{\qed}\renewcommand{\qedsymbol}{$\triangle$}\examplex}
	{\popQED\endexamplex\vspace{-1em}\rule{1cm}{0.7pt}\vspace{0.5em}}

\theoremstyle{remark}
\newtheorem*{remark}{Remark}

\usepackage{xstring}
\usepackage{enumitem}

\DeclarePairedDelimiterX{\infdivx}[2]{(}{)}{%
	#1\;\delimsize\|\;#2%
}
\newcommand{\kldiv}{D_\mathrm{KL}\infdivx}
%\DeclarePairedDelimiter{\norm}{\lVert}{\rVert}

%\newcommand\duplicat[1]{\gdef\mylist{}\foreach \x in {#1}{\xdef\mylist{\mylist (\x) (lab-\x) }}\mylist} %% this doesn't work :((
\newcommand\lab[1]{(#1)(lab-#1)}


\newcommand{\CI}{\mathrel{\perp\mspace{-10mu}\perp}}
\newcommand\E{{\mathbb E}}


\newcommand{\todo}[1]{{\color{red}\large\textbf{[}{\normalsize\texttt{todo:} \itshape#1}\textbf{]}}}
\newcommand{\note}[1]{{\color{blue}\large\textbf{[}{\normalsize\texttt{note:} \itshape#1}\textbf{]}}}

\newcommand\geqc{\succcurlyeq}
\newcommand\leqc{\preccurlyeq}
\newcommand\mat[1]{\mathbf #1}
\newcommand{\indi}[1]{\mathbbm{1}_{\left[\vphantom{\big[}#1 \vphantom{\big]}\right]}}
\newcommand\m[1]{\mathbf m_{\mathsf #1}}
\def\cpm#1(#2|#3){\mathbf #1 \left[ #2 \middle|#3\right]}


\newcommand\recall[1]{\expandarg\cref{#1}:\vspace{-1em} \begingroup\small\color{gray!80!black}\begin{quotation} \expandafter\csname #1\endcsname* \end{quotation}\endgroup }

%OMG THIS WORKS

\def\wrapwith#1[#2;#3]{
	\expandafter#2{\expandarg\StrBefore{#1}{,}}
	\expandarg\StrBehind{#1}{,}[\tmp] 
	\xdef\tmp{\expandafter\unexpanded\expandafter{\tmp}}
	#3
	\expandarg\IfSubStr{\tmp}{,}{\wrapwith{\tmp}[#2;{#3}]}{ \expandafter#2{\tmp} }
}
\def\hwrapcells#1[#2]{\wrapwith#1[#2;&]}
\def\vwrapcells#1[#2]{\wrapwith#1[#2;\\]}



\newsavebox{\idxmatsavebox}
\def\makeinvisibleidxstyle#1#2{\phantom{\hbox{#1#2}}}
\newenvironment{idxmatphant}[4][\footnotesize\color{gray}\text]{%
	\def\idxstyle{#1}
	\def\colitems{#3}
	\def\rowitems{#2}
	\def\phantitems{#4}
	\begin{lrbox}{\idxmatsavebox}$
	\begin{matrix}  \begin{matrix} \hwrapcells{\colitems}[\idxstyle]  \end{matrix} \\[0.1em]
		\left[ 
		\begin{matrix}
			\hwrapcells{\phantitems}[\expandafter\makeinvisibleidxstyle\idxstyle]  \\[-1em]
	}{
		\end{matrix}\right]		&\hspace{-0.5em}\begin{matrix*}[l] \vwrapcells{\rowitems}[\idxstyle] \end{matrix*}
	\end{matrix}
	$\end{lrbox}
	\raisebox{0.75em}{\usebox\idxmatsavebox}
%	\vspace{-0.5em}
}

\newenvironment{idxmat}[3][\footnotesize\color{gray}\text]
	{\begingroup\idxmatphant[#1]{#2}{#3}{#3}}
	{\endidxmatphant\endgroup}

\newenvironment{sqidxmat}[2][\footnotesize\color{gray}\text]
	{\begingroup\idxmat[#1]{#2}{#2}}
	{\endidxmat\endgroup}
	
	
%%%%%%%%%%%%
% better alignment for cases
\makeatletter
\renewenvironment{cases}[1][l]{\matrix@check\cases\env@cases{#1}}{\endarray\right.}
\def\env@cases#1{%
	\let\@ifnextchar\new@ifnextchar
	\left\lbrace\def\arraystretch{1.2}%
	\array{@{}#1@{\quad}l@{}}}
\makeatother
\usepackage[margin=1in, inner margin=0.7in, outer margin=1.3in]{geometry}
\usepackage{mathtools,amssymb,amsfonts}
\usepackage{parskip}

\usepackage{tikz}
	\usetikzlibrary{positioning,fit,calc, decorations, arrows, shapes, shapes.geometric}
	\usetikzlibrary{cd}
	\pgfdeclaredecoration{arrows}{draw}{
		\state{draw}[width=\pgfdecoratedinputsegmentlength]{%
			\path [every arrow subpath/.try] \pgfextra{%
				\pgfpathmoveto{\pgfpointdecoratedinputsegmentfirst}%
				\pgfpathlineto{\pgfpointdecoratedinputsegmentlast}%
			};
	}}
	%%%%%%%%%%%%
	\tikzset{AmpRep/.style={ampersand replacement=\&}}
	\tikzset{center base/.style={baseline={([yshift=-.8ex]current bounding box.center)}}}
	\tikzset{paperfig/.style={center base,scale=0.9, every node/.style={transform shape}}}

	% Node Stylings
	\tikzset{dpadded/.style={rounded corners=2, inner sep=0.7em, draw, outer sep=0.3em, fill={black!50}, fill opacity=0.08, text opacity=1}}
	\tikzset{dpad0/.style={outer sep=0.05em, inner sep=0.3em, draw=gray!75, rounded corners=4, fill=black!08, fill opacity=1}}
	\tikzset{dpad/.style args={#1}{every matrix/.append style={nodes={dpadded, #1}}}}
	\tikzset{light pad/.style={outer sep=0.2em, inner sep=0.5em, draw=gray!50}}
		
	\tikzset{arr/.style={draw, ->, thick, shorten <=3pt, shorten >=3pt}}
	\tikzset{arr0/.style={draw, ->, thick, shorten <=0pt, shorten >=0pt}}
	\tikzset{arr1/.style={draw, ->, thick, shorten <=1pt, shorten >=1pt}}
	\tikzset{arr2/.style={draw, ->, thick, shorten <=2pt, shorten >=2pt}}
	\tikzset{archain/.style args={#1}{arr, every arrow subpath/.style={draw,arr, #1}, decoration=arrows, decorate}}


	\tikzset{fgnode/.style={dpadded,inner sep=0.6em, circle},
	factor/.style={light pad, fill=black}}	
	
	\newcommand\cmergearr[4]{
		\draw[arr,-] (#1) -- (#4) -- (#2);
		\draw[arr, shorten <=0] (#4) -- (#3);
	}
	\newcommand\mergearr[3]{
		\coordinate (center-#1#2#3) at (barycentric cs:#1=1,#2=1,#3=1.2);
		\cmergearr{#1}{#2}{#3}{center-#1#2#3}
	}
	\newcommand\cunmergearr[4]{
		\draw[arr,-, , shorten >=0] (#1) -- (#4);
		\draw[arr, shorten <=0] (#4) -- (#2);
		\draw[arr, shorten <=0] (#4) -- (#3);
	}
	\newcommand\unmergearr[3]{
		\coordinate (center-#1#2#3) at (barycentric cs:#1=1.2,#2=1,#3=1);
		\cunmergearr{#1}{#2}{#3}{center-#1#2#3}
	}
	
	\usetikzlibrary{matrix}
	\tikzset{toprule/.style={%
	        execute at end cell={%
	            \draw [line cap=rect,#1] 
	            (\tikzmatrixname-\the\pgfmatrixcurrentrow-\the\pgfmatrixcurrentcolumn.north west) -- (\tikzmatrixname-\the\pgfmatrixcurrentrow-\the\pgfmatrixcurrentcolumn.north east);%
	        }
	    },
	    bottomrule/.style={%
	        execute at end cell={%
	            \draw [line cap=rect,#1] (\tikzmatrixname-\the\pgfmatrixcurrentrow-\the\pgfmatrixcurrentcolumn.south west) -- (\tikzmatrixname-\the\pgfmatrixcurrentrow-\the\pgfmatrixcurrentcolumn.south east);%
	        }
	    },
	    leftrule/.style={%
	        execute at end cell={%
	            \draw [line cap=rect,#1] (\tikzmatrixname-\the\pgfmatrixcurrentrow-\the\pgfmatrixcurrentcolumn.north west) -- (\tikzmatrixname-\the\pgfmatrixcurrentrow-\the\pgfmatrixcurrentcolumn.south west);%
	        }
	    },
	    rightrule/.style={%
	        execute at end cell={%
	            \draw [line cap=rect,#1] (\tikzmatrixname-\the\pgfmatrixcurrentrow-\the\pgfmatrixcurrentcolumn.north east) -- (\tikzmatrixname-\the\pgfmatrixcurrentrow-\the\pgfmatrixcurrentcolumn.south east);%
	        }
	    },
	    table with head/.style={
		    matrix of nodes,
		    row sep=-\pgflinewidth,
		    column sep=-\pgflinewidth,
		    nodes={rectangle,minimum width=2.5em, outer sep=0pt},
		    row 1/.style={toprule=thick, bottomrule},
  	    }
	}
\usepackage{booktabs}       % professional-quality tables
\usepackage{amsfonts}       % blackboard math symbols
\usepackage{nicefrac}       % compact symbols for 1/2, etc.
\usepackage{microtype}      % microtypography
\usepackage{mathtools}		%also loads amsmath
\usepackage{amssymb, bbm}
\usepackage{relsize}
\usepackage{color}
%\usepackage{stmaryrd}
\usepackage{hyperref} % Load before theorems...
\hypersetup{colorlinks=true, linkcolor=blue!75!black, urlcolor=magenta, citecolor=deepgreen}	
\usepackage{amsthm,thmtools}

\theoremstyle{plain}
\newtheorem{theorem}{Theorem}[section]
\newtheorem{coro}{Corollary}[theorem]
\newtheorem{prop}[theorem]{Prop}
\newtheorem{claim}[theorem]{Claim}% maybe don't use this.
\newtheorem{lemma}[theorem]{Lemma}
\newtheorem{fact}[theorem]{fact}
\newtheorem{conj}[theorem]{Conjecture}
\newtheorem{question}[theorem]{Question}

% \newcommand{\regugitatethmname}{}
% \newtheorem{regurgitatethm}[theorem]{\regugitatethmname}% to be defined later
% \newtheorem*{regurgitatethm*}{\regugitatethmname}% to be defined later

\declaretheorem[name={}]%[numbered=unless unique,name=#1]
	{namedtheoremINNER}
\newenvironment{namedthm}[1]
  {\renewcommand{\thenamedtheoremINNER}{#1}\namedtheoremINNER}
  {\endnamedtheoremINNER}

\newcounter{proofcntr}
\newenvironment{lproof}{\begin{proof}\refstepcounter{proofcntr}}{\end{proof}}
\declaretheorem[numberwithin=theorem,name=Claim]{iclaim}

\theoremstyle{definition}
\declaretheorem[name=Definition,qed=$\square$,numberwithin=section]{defn} %
\declaretheorem[name=Definition,qed=$\square$,numberwithin=theorem]{idefn} %
\declaretheorem[name=Construction,qed=$\square$,sibling=defn]{constr}
\declaretheorem[qed=$\square$]{example}

\theoremstyle{remark}
\newtheorem*{remark}{Remark}

  
\usepackage{xstring}
\usepackage{enumitem}

\usepackage[noabbrev,nameinlink,capitalize]{cleveref}
\crefname{example}{Example}{Examples}
\crefname{defn}{Definition}{Definitions}
\crefname{prop}{Prop}{Propositions}
\crefname{claim}{Claim}{Claims}
\crefname{iclaim}{Claim}{Claims}
\crefname{constr}{Construction}{Constructions}
\crefname{conj}{Conjecture}{Conjectures}
\crefname{fact}{Fact}{Facts}


\newcommand\commentout[1]{}
%%%%%%%%%%%%%%%%%%%%%%%%%%%%%%%%%
\let\Horig\H
\let\H\relax
\DeclareMathOperator{\H}{\mathrm{H}} % Entropy
\DeclareMathOperator{\I}{\mathrm{I}} % Information
\DeclareMathOperator*{\Ex}{\mathbb{E}} % Expectation
\DeclareMathOperator{\supp}{\mathrm{supp}} % support
\DeclareMathOperator*{\argmin}{arg\,min}
\newcommand\mat[1]{\mathbf{#1}}
\DeclarePairedDelimiterX{\infdivx}[2]{(}{)}{%
	#1\;\delimsize\|\;#2%
}
\newcommand{\thickD}{I\mkern-8muD}
\newcommand{\kldiv}{\thickD\infdivx}

\newcommand{\tto}{\rightarrow\mathrel{\mspace{-15mu}}\rightarrow}
\newcommand{\bp}[1][L]{\mat{p}_{\!_{#1}\!}}
\newcommand{\V}{\mathcal V}
\newcommand{\N}{\mathcal N}
\newcommand{\Ed}{\mathcal E}

\DeclareMathAlphabet{\mathdcal}{U}{dutchcal}{m}{n}
\DeclareMathAlphabet{\mathbdcal}{U}{dutchcal}{b}{n}
\newcommand{\dg}[1]{\mathbdcal{#1}}


\newcommand\smid{\!\mid\!}
\newcommand\MAP{\mathrm{MAP}}
\DeclareMathOperator{\bundle}{\sqcup}
\newcommand\ado{\mathrm{do}}

\relax % Short arrows. 
    \newcommand{\veryshortarrow}[1][3pt]{\mathrel{%
       \vcenter{\hbox{\rule[-.5\fontdimen8\textfont3]{#1}{\fontdimen8\textfont3}}}%
       \mkern-4mu\hbox{\usefont{U}{lasy}{m}{n}\symbol{41}}}}
    \makeatletter
    \setbox0\hbox{$\xdef\scriptratio{\strip@pt\dimexpr
        \numexpr(\sf@size*65536)/\f@size sp}$}
    \newcommand{\scriptveryshortarrow}[1][3pt]{\mathrel{%
        \vcenter{\hbox{\rule[-.5\fontdimen8\scriptfont3]
                   {\scriptratio\dimexpr#1\relax}{\fontdimen8\scriptfont3}}}%
       \mkern-4mu\hbox{\let\f@size\sf@size\usefont{U}{lasy}{m}{n}\symbol{41}}}}
    \newcommand{\sto}{\scriptveryshortarrow}
    \makeatother

    
\newcommand{\grad}{\vec\nabla}
\DeclareMathOperator{\QC}{\mathrm{QC}}
\DeclareMathOperator\src{\mathbf{src}}
\DeclareMathOperator\tgt{\mathbf{tgt}}

\newcommand{\datadist}[1]{\Pr\nolimits_{#1}}

%%%%%%%%%%%%%%%%%%%%

\newcommand{\pdgunit}{\mathrlap{\mathit 1} \mspace{2.3mu}\mathit 1}

\newcommand\Pa{\mathbf{Pa}}
\newcommand{\IDef}[1]{\mathit{IDef}_{\!#1}}
\newcommand\Inc{\mathit{Inc}}
\newcommand{\PDGof}[1]{{\dg M}_{#1}}
\newcommand{\UPDGof}[1]{{\dg N}_{#1}}
\newcommand{\WFGof}[1]{\Psi_{{#1}}}
\newcommand{\FGof}[1]{\Phi_{{#1}}}
\newcommand{\Gr}{\mathcal G}
\newcommand\GFE{\mathit{G\mkern-4mu F\mkern-4.5mu E}}
% \newcommand\aar[1]{\langle\mskip}

\newcommand{\ed}[3]{%
	\mathchoice%
	{#2\overset{\smash{\mskip-5mu\raisebox{-3pt}{${#1}$}}}{\xrightarrow{\hphantom{\scriptstyle {#1}}}} #3} %display style
	{#2\overset{\smash{\mskip-5mu\raisebox{-3pt}{$\scriptstyle {#1}$}}}{\xrightarrow{\hphantom{\scriptstyle {#1}}}} #3}% text style
	{#2\overset{\smash{\mskip-5mu\raisebox{-3pt}{$\scriptscriptstyle {#1}$}}}{\xrightarrow{\hphantom{\scriptscriptstyle {#1}}}} #3} %script style
	{#2\overset{\smash{\mskip-5mu\raisebox{-3pt}{$\scriptscriptstyle {#1}$}}}{\xrightarrow{\hphantom{\scriptscriptstyle {#1}}}} #3}} %scriptscriptstyle


\newcommand{\alle}[1][L]{_{\ed {#1}XY}}


\DeclarePairedDelimiterX{\SD}[1]{\{}{\}}{\,\llap{\delimsize\{}#1\rlap{\delimsize\}}\,}
%better version.
\DeclarePairedDelimiterX{\bbr}[1]{[}{]}
	{\mspace{3mu}\mathllap{\delimsize[}#1\mathrlap{\delimsize]}\mspace{3mu}}
\DeclarePairedDelimiterX{\aar}[1]{\langle}{\rangle}
	{\mspace{3mu}\mathllap{\delimsize\langle}#1\mathrlap{\delimsize\rangle}\mspace{3mu}}
\DeclarePairedDelimiterXPP{\aarz}[1]{}{\langle}{\rangle}{_{\!_0}} % \downarrow
	{\mspace{3mu}\mathllap{\delimsize\langle}#1\mathrlap{\delimsize\rangle}\mspace{3mu}}


%% Theorem Restatement
\usepackage{xpatch}
\makeatletter
\xpatchcmd{\thmt@restatable}% Edit \thmt@restatable
   {\csname #2\@xa\endcsname\ifx\@nx#1\@nx\else[{#1}]\fi}% Replace this code
   % {\ifthmt@thisistheone\csname #2\@xa\endcsname\typeout{oiii[#1;#2\@xa;#3;\csname thmt@stored@#3\endcsname]}\ifx\@nx#1\@nx\else[#1]\fi\else\csname #2\@xa\endcsname\fi}% with this code
   {\ifthmt@thisistheone\csname #2\@xa\endcsname\ifx\@nx#1\@nx\else[{#1}]\fi
   \else\fi}
   {}{\typeout{FIRST PATCH TO THM RESTATE FAILED}} % execute on success/failure 
\xpatchcmd{\thmt@restatable}% A second edit to \thmt@restatable
   {\csname end#2\endcsname}
   {\ifthmt@thisistheone\csname end#2\endcsname\else\fi}
   {}{\typeout{FAILED SECOND THMT RESTATE PATCH}}

% \def\onlyaftercolon#1:#2{#2}
\newcommand{\recall}[1]{\medskip\par\noindent{\bf \expandarg\Cref{thmt@@#1}.} \begingroup\em \noindent
   \expandafter\csname#1\endcsname* \endgroup\par\smallskip}
\newenvironment{linked}[3][]{%
	\def\linkedproof{#3}%
	\def\linkedtype{#2}%
	\restatable[#1]{#2}{#2:#3}\label{#2:#3}}
	{\endrestatable%
	\marginpar{%
	% \vspace{-0.5em}
	% \hspace{2em}
		\raggedright
		\hyperref[proof:\linkedproof]{%
		\color{blue!50!white}
		$\Big[$\,{\small\tt\begin{tabular}{@{}l@{}} proof of \\~\cref*{\linkedtype:\linkedproof}\end{tabular}}\,$\Big]$}
		}%
	}
\makeatother
%oli16: The extra space was because there was extra space in the paragraph, not
%because this length was too big. By breaking arrays, everything will be better.
\allowdisplaybreaks


\newcommand{\begthm}[3][]{\begin{#2}[{name=#1},restate=#3,label=#3]}



\usepackage[framemethod=TikZ]{mdframed}
\allowdisplaybreaks

\surroundwithmdframed[
   topline=false,
   linewidth=3pt,
   linecolor=gray!20!white,
   rightline=false,
   bottomline=false,
   leftmargin=0pt,
   % innerleftmargin=5pt,
   skipabove=\medskipamount,
   skipbelow=\medskipamount
]{lproof}
\newmdenv[roundcorner=5pt, backgroundcolor=gray!20!white, frametitle={$\langle$under construction $\rangle$},frametitlerule=false,
innertopmargin=2pt, frametitlebelowskip=3pt, frametitleaboveskip=2pt, frametitlebackgroundcolor=gray!70!white, skipabove=1em,skipbelow=1em, frametitlefont={\normalfont\itshape},leftmargin=-10pt, rightmargin=-10pt]
		{wip}


\newcommand{\TODO}[1][INCOMPLETE]{{\centering\Large\color{red}$\langle$~\texttt{#1}~$\rangle$\par}}
\usepackage{subfiles}

\begin{document}
\begin{center}
	{\bfseries\Large Inconsistency, The Universal Loss Function}
\end{center}

\section{Introduction}


One very important component of a ML system is the loss function.

A loss function, like preference or utility function, is a .
Also like a preference, there is 

There are many standard loss functions.


\section{Classification}

\[
		-\log p(x) = 
		\I_p(x) = 
	\aar[\Big] {
	\begin{tikzpicture}[center base]
		\node[dpad0] (X) {$X$};
		\coordinate (A) at ($(X) + (-0.9,0)$);
		\draw[arr1] (A) -- node[above]{$\scriptstyle p$}  (X);
	%
		\draw[arr2, <<-] (X) --  node[above,pos=0.8]{$\scriptstyle x$} ++(0.9, 0);
	\end{tikzpicture}
	}.
\]



\section{R\'enyi Entropies}

% Let $p$ and $q$ be probability distributions over $X$, and let $\dg N = p \bundle q$ be the corresponding unweighted PDG. Then,
% \begin{enumerate}
% 	\item The inconsistency of the PDG $\Inc(\dg N, [\frac12,\frac12])$
% \end{enumerate}	

\begin{theorem}\label{thm:bhattacharyya-dist}
	The inconsistency of the PDG consisting of $p$ and $q$, each with confidence $\frac12$ is equal to the Bhattacharyya distance $\thickD_B(p,q)$ between $p$ and $q$. 
\end{theorem}

\section{Variational Inference}
Broadly speaking, ``variational inference'' is the practice of optimizing the parameters of an auxilary model to get bounds on a quantity of interest, thereby leveraging optimization to do inference.  
In modern machine learning, the term usually refers to a specific motivating instance, which we now review in more detail. 
As we do so, we justify some of the arguably unnatural (but standard) choices made without argument in modern tutorials \cite{}.

\subsection{The Motivation for Variational Bounds}
	\label{sec:intro-to-variational-inference}
\def\xsamp{{\underline{\mat X}}}

% \subsubsection{Setup.}
Suppose we would like to construct a distribution (generative model) $p(X)$ for variable(s) $X$, given samples $\xsamp = \{ x^{(i)} \}_{i=1}^m$. The usual approach is to optimize $p$ so as to minimize some loss function $\ell(p; \xsamp)$, and by far the most common loss function is the negative (log)-likelihood of generating the data $\xsamp$ with independent draws from $p$, or equivalently, the total information content of the samples (according to $p$), which is given by 
\begin{equation}
 	\ell(p; \xsamp) = \sum_{i=1}^m \I_p[ X = x^{(i)}]  % \underbrace{}_{\mathclap{\text{information in the samples}}} 
	% = - \log p(\xsamp) 
	= -\sum_{i=1}^m \log p(x^{(i)}). \label{eq:negloglike} \end{equation}
One intuition is that we want to maximize the probability of seeing our data; another is that we'd like to minimize the model's surprise at seeing these samples.%
The logarithm may seem extraneous, since the relative ordering on distributions $p$ is the same without it---but the expression as given is strictly convex, factors product distributions into sums, solves numerical issues, and meshes nicely with information theory / statistical mechanics.
The optimal distribution according to $\ell$ the data distribution 
$\datadist\xsamp(x) = \frac1m \#\{i : x = x^{(i)}\}$  itself, since
\[ 
	\ell(p; \xsamp) =~ m \cdot\!\! \underbrace{\Ex_{x \sim \datadist\xsamp} \Big[ \log \frac1{p(x)}\Big] }_{\text{cross entroy from $\datadist\xsamp$ to $p$}} =\quad \kldiv{\datadist\xsamp}{p} + \underbrace{\H(\datadist\xsamp)}_{\text{constant in $p$}}. 
\] 
Given that we already know its optimum ($\datadist\xsamp$) in closed form, it would be reasonable to question the choice of $\ell(p;\xsamp)$ as an optimization objective. Worse still, $\datadist\xsamp$ does not generalize at all to unseen samples, and so we would be quite unhappy if we were to recover it from the optimization process. 
Still, ``maximizing the probability of your data'' is generally considered the standard way to fit your model; this issue of overfitting is either implicitly dealt with by  restricting to possible classifiers $p$ with nice properties (i.e., not expressive enough to represent $\datadist\xsamp$), or by adding additional terms to $\ell$ to disincentivize complex, ad-hoc models (``regularizers'').
In recent years, the ML community has realized that adding such terms is not only useful (i.e., for preventing overfitting \cite{} and providing adversairal protection \cite{}), but also natural, as they correspond to prior beliefs \cite{}.
We will return to this point later, but for now we will continue with our un-regularized log likelihood objective $\ell$.

When working with latent variable models---that is, models involving unobserved variables $Z$---computing the likelihood function is more difficult, as now we only have observations for \emph{part} of our variable. Typically, we have a joint model $p(X,Z)$, but samples only for $X$. Sure, since we have a joint distribution $p(X,Z)$ we implicitly also have the marginal distribution $p(X)$, but computing it could be expensive. To compute the marginal likelihood directly, we would have to sum (or integrate) over all possible values of $z$, as in \eqref{eq:marginal-likelihood}.
\begin{equation}
	\ell(p ; \xsamp) = - \log p(\xsamp) = - \sum_{i} \log \left(\sum_z p(x^{(i)}, z) \right) \label{eq:marginal-likelihood}
\end{equation}
This may not sound so bad, but in fact it is prohibitively expensive, because the number of terms scales exponentially with the dimension of $Z$ and number of latent variables. 

When confronted with intractable integrals such as this one, a standard trick is to draw samples and average, which is known as a \emph{Monte Carlo} estimate.
To justify taking samples like this, we have to rewrite the quantity of interest with an expectation of something. To start, let $q(Z)$ be an arbitrary distribution from which we draw samples:  
\begin{equation*}
	-\log p(x) = - \log \sum_z p(x, z) 
		= -\log \sum_z \left[ q(z) \cdot\frac{p(x, z)}{q(z)} \right] 
		= -\log \Ex_{z\sim q(Z)} \left[\frac{p(x, z)}{q(z)} \right] 
		% &= -\log \Ex_{z\sim p(Z)} \Big[p(x \mid z) \Big] \\
		% &\approx - \sum_{i} \log \left(\Ex_{\mat z^{(1)}, \ldots \mat z^{(m)} \sim q(Z)} p(x^{(i)}, z) \right) 
\end{equation*}
Note that in order to introduce the expectation, we had to correct for the density of $q$ which we artificially introduced, resulting in a factor of ${1}/{q(z)}$, which can be recognized as \emph{inverse propensity weighting}. 
Had we not introduced a logarithm at the beginning, this would have given us an unbiased estimator of the desired quantity, but alas $\log \Ex [X] \ne \Ex [\log X]$. Still, since $\log$ is a concave function, Jenson's inequality tells us that 
% \[ \ell(p; \xsamp) = -\sum_i \log \Ex_{z\sim q(Z)} \left[\frac{p(x^{(i)}, z)}{q(z)} \right]
%  	\ge - \sum_i \Ex_{z\sim q(Z)}\left[  \log \frac{p(x^{(i)}, z)}{q(z)} \right]. \]
\[ \log p(x) = \log \Ex_{z\sim q(Z)} \left[\frac{p(x^{(i)}, z)}{q(z)} \right]
 	\ge \Ex_{z\sim q(Z)}\left[  \log \frac{p(x, z)}{q(z)} \right] =: \mathrm{ELBO}_{p,q}(x). \]
This final quantity is known as the ``evidence lower bound'', since it lower bounds the ``evidence'' ($\log p(x)$). As a result,
\[ \ell(\xsamp; p) = - \sum_{i=1}^m \log p(x^{(i)}) \le - \sum_i \mathrm{ELBO}_{p,q}(x^{(i)}), \] 
so one approach to minimizing the loss $\ell(p; \xsamp)$, is to minimize this upper bound, simultaneously varying the parameters of both $p$ and $q$, getting successively tighter approximations and smaller losses. 
At convergence we will have distributions $p^*(X,Z)$ and $q^*(Z)$; the former is the posterior distribution that we wanted (i.e., $p$ conditioned on the $\xsamp$), and $q^*$ is known ``variational posterior''.

\subsection{PDG Inference from Variational Techniques}
	\label{sec:inference-from-variation}

    The relationship between a pdg $\dg M$ and a test distribution $\mu$ is quite similar. 
    In our case, we would like to alter some of our beliefs (some cpds of $\dg M$) to reduce the global inconsistency $\aar{\dg M}$, so $\dg M$ is analogous to the model $p$, while the test distribution $\mu$ corresponds to the variational posterior $q$. 
    Just as any $q$ will give an upper bound $\mathrm{ELBO}_{p,q}(x) \ge \ell(x)$ in the previous section, so too will any $\mu$ give an upper bound 
    \[ \bbr{\dg M}_\gamma(\mu) \ge \aar{\dg M}_\gamma \]
    on the inconsistency of $\dg M$. So by choosing a nice class of test distributions $\mathcal M \subset \Delta(\dg M)$, we can pull the same trick---simultaneously optimize $\mu \in \mathcal M$ and (some parameters of) $\dg M$, so that we modify $\dg M$ to minimize inconsistency via this variational bound.

    Moreover, since $\bbr{\dg M}_\gamma(\mu)$ is strictly convex in $\dg M$ and $\gamma$-strongly convex in $\mu$, this procedure gives a family of inference algorithms for PDGs in line with standard algorithms for training neural networks.

    \subsection{Variational Inference via PDGs}
    In \Cref{sec:inference-from-variation}, we saw that PDG semantics can be described as a variational inference procedure. 
    Perhaps surprisingly, the converse is also true: variational inference is captured by the internal workings of a PDG. 
    Moreover, PDGs give a concise graphical language for the kind of reasoning in \cref{sec:intro-to-variational-inference}, which can be notoriously difficult to track symbolically. We now illustrate this with examples.

    Let's start with the first setting in \cref{sec:intro-to-variational-inference}, where we have observations of all data, and we can compute $\ell$ directly. 

    \begin{prop} \label{prop:many-equal-simple}
    	Consider a distribution $p(X)$.
    	The suprise $\I_p(x)$ at seeing a sample $x$ (or equivalently, the negative log likelihood of $x$) is given by the inconsistency of the pdg containing a point mass on $x$ and $p$, i.e.,
    	\[
            -\log p(x) = 
            \I_p(x) = 
    	\aar[\Big] {
    	\begin{tikzpicture}[center base]
    		\node[dpad0] (X) {$X$};
    		\coordinate (A) at ($(X) + (-0.9,0)$);
    		\draw[arr1] (A) -- node[above]{$\scriptstyle p$}  (X);
    %
    		\draw[arr2, <<-] (X) --  node[above,pos=0.8]{$\scriptstyle x$} ++(0.9, 0);
    	\end{tikzpicture}
    	}.
    	\]
    \end{prop}
    
    
    \begin{prop}
    	Given samples $\xsamp$ determining an emperical distribution $D := \datadist\xsamp$,  the following are equal, for all $\gamma \ge 0$:
    	\begin{enumerate}
    	\item The negative log likelihood $\ell(p; \xsamp)$
    	\item The cross entropy of $p$ relative to $D$
    	\item $\bbr{\,p\,}_\gamma(D) + (1+\gamma)\H(D)$
    	% FALSE! \item $\bbr{\,p^{\{\alpha=1\}} \,}_1 (\datadist\xsamp^{\{\alpha=1\}})$
    	
    	\item \(\aar[\Big] {
    		\begin{tikzpicture}[center base]
    			\node[dpad0] (X) {$X$};
    			\coordinate (A) at ($(X) + (-0.9,0)$);
    			\draw[arr1] (A) -- node[above]{$p$}  (X);
    			\draw[arr2, <-] (X) --  node[above,pos=0.6]{$\overset{\{\beta = \infty\}}D$} ++(1.2, 0);
    		\end{tikzpicture}
    		}_\gamma + (1+\gamma) \H(D)
    		\)
    	\item 
    	\(\aar[\Bigg] {
    		\begin{tikzpicture}[center base]
    			\node[dpad0] (X) {$X$};
    			\coordinate (A) at ($(X) + (-1.2,0)$);
    			\draw[arr1] (A) -- node[above,pos=0.4]{$ \overset {\{\alpha = 1\}} p$}  (X);
    	%
    			\draw[arr2, <-] (X) --  node[above,pos=0.6]{$ \overset{\{\beta = \infty,\alpha=\gamma\}}{D}$} ++(1.5, 0);
    		\end{tikzpicture}
    		}\!\bigg._1 \)
    \end{enumerate}
    \end{prop} 

    \begin{remark}
    Note that the entropy of the data distribution $\H(D)$ is constant in $p$, and so this additional term in (3, 4) as well as the the qualitative components of the PDGs, are irrelevant if we intend to optimize this objective by varying $p$.	
    \end{remark}
       
    We argue that the pdgs in \cref{prop:many-equal-simple} are natural. Aside from the weights, this should be uncontrovertial --- we've simply translated the entire problem into a set of probability distributions and included them into the pdg. 
    In the current setting, we argue that the weights are also the ones we would expect. The cross entropy measures the expected code length per sample, when a (possibly incorrect) distribution $p$ is used to design a code, in place of the true one $D$.  So in our corresponding PDG, we should be much more certain of $D$ than of $p$, justifying the choice of $\beta$. 
    Other choices of $\beta$ are more natural in other contexts, and we will see later that they correspond to other losses. (For instance when $\beta_p = \beta_D = \frac12$, the result is the Bhattacharyya distance, rather than the cross entropy; see \cref{thm:bhattacharyya-dist}.)
    Furthermore, $p$ is the qualitative model we have in mind, and so we set $\alpha_p = 1$
    Although the results of \cref{prop:many-equal-simple} are not particularly useful, they do hint at a strong tie between the standard metrics used to train probabilistic models, and the inconsistency of a corresponding PDG. 
    We will now see that this tie extends to latent variable models in a nice way.


    \begin{linked}{prop}{pdg-elbo-x}%[Capturing the ELBO]
    		% \label{prop:pdg-elbo-x}
    	% If $p(X,Z)$ is a joint probabilistic model of observed variables $X$ and hidden variables $Z$, and $q(Z)$ is any distribution 
    	% That is, 
    	The negative $\mathrm{ELBO}_{p,q}(x)$ is the inconsistency of the PDG containing $p,q$, and $x$, with very high confidence in $q$. 
    	That is,
    	\[ 
	    	-\mathrm{ELBO}_{p,q}(x) = 
	    		% \lim_{t \to \infty}
	    	 % \aarz[\Bigg]{
	    	 \Inc\left[
	    		\begin{tikzpicture}[center base]
	    			\node[dpad0] (Z) {$Z$};
	    			\node[dpad0,right=.5 of Z] (X) {$X$};
	    			\coordinate (A) at ($ (X)!.5!(Z) + (0,0.7)$);
	    			\draw[arr1] (A) -- node[right]{$\scriptstyle p$} ++(0,-0.25) -- (X);
	    			\draw[arr1] (A) -- ++(0,-0.25) -- (Z);
				    %
	    			\draw[arr2, <<-] (X) --  node[above,pos=0.8]{$\scriptstyle x$} ++(0.9, 0);
	    			\draw[arr2, <-] (Z) -- node[above,pos=0.6]{$\scriptstyle q!$} ++(-0.9, 0);
	                % ^{\{\beta =\infty\}}
	    			% \ar[r,"p"] \& Z \ar[r,"p", bend left] \& X \ar[l,"q", bend left] \& \ar[l, two heads, "x"'] 
	    		\end{tikzpicture}
	    		\right] 
	            % }%_{\!\!0}
	    		.
	    	\]
    \end{linked}
    
    We now have an intuitive visual proof of the variational bound. 
    \[
        - \log p(x) = 
        \Inc\left[
           \begin{tikzpicture}[center base]
               \node[dpad0] (Z) {$Z$};
               \node[dpad0,right=.5 of Z] (X) {$X$};
               \coordinate (A) at ($ (X)!.5!(Z) + (0,0.7)$);
               \draw[arr1] (A) -- node[right]{$\scriptstyle p$} ++(0,-0.25) -- (X);
               \draw[arr1] (A) -- ++(0,-0.25) -- (Z);
               %
               \draw[arr2, <<-] (X) --  node[above,pos=0.8]{$\scriptstyle x$} ++(0.9, 0);
               % ^{\{\beta =\infty\}}
               % \ar[r,"p"] \& Z \ar[r,"p", bend left] \& X \ar[l,"q", bend left] \& \ar[l, two heads, "x"'] 
           \end{tikzpicture}
           \right]
          \le 
    	 \Inc\left[
    		\begin{tikzpicture}[center base]
    			\node[dpad0] (Z) {$Z$};
    			\node[dpad0,right=.5 of Z] (X) {$X$};
    			\coordinate (A) at ($ (X)!.5!(Z) + (0,0.7)$);
    			\draw[arr1] (A) -- node[right]{$\scriptstyle p$} ++(0,-0.25) -- (X);
    			\draw[arr1] (A) -- ++(0,-0.25) -- (Z);
                %
    			\draw[arr2, <<-] (X) --  node[above,pos=0.8]{$\scriptstyle x$} ++(0.9, 0);
    			\draw[arr2, <-] (Z) -- node[above,pos=0.6]{$\scriptstyle q!$} ++(-0.9, 0);
                % ^{\{\beta =\infty\}}
    			% \ar[r,"p"] \& Z \ar[r,"p", bend left] \& X \ar[l,"q", bend left] \& \ar[l, two heads, "x"'] 
    		\end{tikzpicture}
    		\right] 
    \]

    The proof of \Cref{prop:pdg-elbo-x} hinges critically on the fact that we force a single sample $x$; its PDG does not capture the whole context $\xsamp$. 
    Nevertheless, we get a somewhat analogous result when 

    \begin{wip}
    \begin{prop}%{prop:pdg-elbo-X}
    	Consider a model $p(X,Z)$, auxiliary distribution $q(Z)$, and samples $\xsamp = \{x^{(i)}\}$ defining a data distribution $d = \datadist\xsamp$. 
    	The following are equal:
    	\begin{enumerate}[label=(\arabic*)]
    		\item $- \Ex_{d} \mathrm{ELBO}_{p,q}(X)$
    		\item $\displaystyle\bbr{\,p\,}(q \otimes d)$
    		\item \(\displaystyle%\lim_{\beta_q, \beta_{\datadist\xsamp} \to \infty}
    			\lim_{t \to \infty}
    		  \aar[\Bigg]{
    		  \begin{tikzpicture}[center base]
    	  		\node[dpad0] (Z) {$Z$};
    	  		\node[dpad0,right=.5 of Z] (X) {$X$};
    	  		\coordinate (A) at ($ (X)!.5!(Z) + (0,0.7)$);
    	  		\draw[arr1] (A) -- node[right]{$\scriptstyle p$} ++(0,-0.25) -- (X);
    	  		\draw[arr1] (A) -- ++(0,-0.25) -- (Z);
    	  %
    	  		\draw[arr1, <-] (X) --  node[above,pos=0.2,anchor=south west]{$\scriptstyle d^{\{\beta= t\}}$} ++(0.9, 0); 
    	  			% ^{\{\beta=\beta_0\}} 
    	  		\draw[arr1, <-] (Z) -- node[above]{$\scriptstyle q^{\{\beta= t\}} $} ++(-0.9, 0);
    	  	\end{tikzpicture} } + \H(\datadist\xsamp)\)
    	\end{enumerate}

    \end{prop}
    \begin{proof}\label{proof:pdg-elbo-X}
    	
    \end{proof}

    \begin{prop}
    		\label{prop:pdg-loglikelihood}
    	The loss $\ell(p,\xsamp)$ is the inconsistency of the PDG containing $p$ and the data distribution $\datadist\xsamp$. 
    	That is,
    	\[ 
    	\ell(p;\xsamp) = 
    	 \aar[\Bigg]{
    	 % \Inc\left(
    		\begin{tikzpicture}[center base]
    			\node[dpad0] (Z) {$Z$};
    			\node[dpad0,right=.5 of Z] (X) {$X$};
    			\coordinate (A) at ($ (X)!.5!(Z) + (0,0.7)$);
    			\draw[arr1] (A) -- node[right]{$\scriptstyle p$} ++(0,-0.25) -- (X);
    			\draw[arr1] (A) -- ++(0,-0.25) -- (Z);
    %
    			\draw[arr1, <-] (X) --  node[above,pos=0.8]{$\scriptstyle \datadist\xsamp$} ++(0.9, 0);
    			% \draw[arr1, <-] (Z) -- node[above]{$\scriptstyle q$} ++(-0.9, 0);
    		\end{tikzpicture}
    		}%_{\!\!0}
    		% \right) 
    	\]
    \end{prop}
    \end{wip}


\subsection{Variational Auto-Encoders and PDGs}

    An autoencoder is a probabilistic model intended to compress a variable $X$ (e.g., an image) to a compact latent representation $Z$ (a compact vector), and is specified by two conditional distributions: 
    an encoder $e(Z \mid X)$, and a decoder $d(X \mid Z)$.  
    Of course, not all pairs of cpds fill this role equally well. 
    Perhaps most importantly, we would to have low \emph{reconstruction error} \eqref{eq:rec}---when we decode an encoded image, we would like it to be reasonably similar to the original.

    \begin{equation}
     \mathrm{Rec}(x) = \Ex_{z \sim e \mid x} \underbrace{\mathrm I_{d\mid z}[x]}_{\substack{\text{Additional bits to}\\\text{specify $x$ from $d|z$}}} 
    	= \sum_z e(z \mid x) \log \frac1{d(x \mid z)}\label{eq:rec}
    \end{equation}

    % Note the similarity to the cross entropy objective from before, except in place of our 

    There are other desiderata as well. It would be nice if the distribution on $Z$ had a simple form---perhaps factoring into independent features, which we might use to describe $X$. To deal with this, we can introduce a new distribution $p(Z)$

    In this setting, $e$ is our variational approximation to the latent variables, and differs from $q$ in the previous section, in that it can now depend on $X$. The evidence lower bound now becomes
    \begin{align*}
    	\mathrm{ELBO}_{p,e,d}(x) &= \Ex_{z \sim e|x} \left[\log \frac{p(z) d(x\mid z)}{q(z\mid x)} \right] \\
    		&= \Ex_{z \sim e|x}\left[ \log \frac{p(z)}{q(z\mid x)}  \right] - \Ex_{z \sim e|x} \left[\log \frac1{d(x\mid z)} \right] \\
    		&= \kldiv{e(Z|x)}{p} - \mathrm{Rec}(x)
    \end{align*}
    \[ \] 

    \begin{prop}
    	\[ 
    	-\mathrm{ELBO}_{p,e,d}(x) = 
    	 \aarz[\Bigg]{
    		\begin{tikzcd}[AmpRep,row sep=1em,column sep=1.5em]
    			\ar[r,"p"] \& Z \ar[r,"d", bend left] \& X \ar[l,"q", bend left] \& \ar[l, two heads, "x"'] 
    		\end{tikzcd}} 
    	\]
    \end{prop}
	
	
	\subsection{The ``Reparameterization Trick''}

\appendix
\section{Proofs}
\subsection{Variational Inference Proofs}
\recall{prop:pdg-elbo-x}
\begin{lproof}
	\label{proof:pdg-elbo-x}
	Let $\dg M$ be the PDG in the picture. We compute its quantitative semantics by
	\[ \Inc_{\dg M}(\mu) = \Ex_{x',z \sim \mu} \left[ \infty \log \frac{\mu(z)}{q(z)} + \log \frac{\mu(x',z)}{p(x',z)} + \log \frac{\mu(x')}{\mathbbm 1[x = x']]}\right]. \]
	If $\mu$ places $\epsilon > 0$ mass on some $x' \ne x$, then the final term will be infinite, since the final term in the expectation will exceed $\epsilon \log [\epsilon / 0]$. 
	Similarly, this value will also be infinite if $\mu(z) \ne q$, due to the first term. Taken together, this means that the only distribution with a finite score is $\mu(X,Z) = q(Z) \delta_x(X)$, which has an inconsistency of
	\begin{align*}
		\Inc_{\dg M}(q \delta_x) &= \Ex_{z \sim q} \left[ \log \frac{q(z)}{p(x,z)}\right] \\
		 	&= - \Ex_{z \sim q} \left[\log \frac{p(x,z)}{q(z)} \right] 
			= - \mathrm{ELBO}_{p,q}(x),
	\end{align*}
	and so $\Inc(\dg M) = -\mathrm{ELBO}(x)$. 
	% Moreover, since $\alpha^{\dg M} = \mat 0$, we know $\IDef{\dg M}(\mu) = \H(\mu)$
	% \begin{align*}
	% 	- \mathrm{ELBO}(x) &= \Ex_{z \sim q} \log \frac{p(x,z)}{q(z)} \\
	% 		&= \Ex_{z \sim q} \log \frac{p(z)}{q(z)} + \Ex_{z\sim q} \log \frac{1}{p(x\mid z)}\\
	% 		&= \kldiv q p + \Ex_{}
	% \end{align*}
\end{lproof}
\end{document}
