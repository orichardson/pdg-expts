
%% This is false. 
\textbf{Strict Convexity.}
Choose any $y_1, y_2 \in Y$. 
Becuase $X$ is compact, the infema $\inf_x f(x, y_1)$ and $\inf_x f(x, y_2)$ are achieved, and uniquely so because $f$ is strictly convex in $X$. Let $x_1 = x^*_{y_1}$ and $x_2 = x^*_{y_2}$ be the respective values of $X$ that acheive these minima, so that
\[ \inf_x f(x,y_1) = f(x_1, y_1) \qquad\text{and}\qquad \inf_x f(x,y_2) = f(x_2, y_2) .\]
Taking a linear combination of them, we get, for every $\alpha \in [0,1]$, that
\begin{align*}
    \inf_x f(x,y_1)+ (1-\alpha) \inf_x f(x,y_2) &= \alpha f(x_1,y_1) + (1-\alpha) f(x_1, x_2) & \\
    &> f\Big(\alpha  x_1 + (1-\alpha) x_2 , \alpha  y_1 + (1-\alpha) y_2 \Big)  & \text{[by strict convexity of $f$]}\\
    &\ge \inf_x f(x, \alpha  y_1 + (1-\alpha) y_2). & \text{[as $\inf$ is a lower bound]}
\end{align*}
Since this holds for all $y_1, y_2 \in Y$, this shows that $y \mapsto \inf_x f_x,y)$ is strictly convex, as desired.





%% EVENTS
We can regard an event $E \subset \Omega$ on a space $\Omega$ of out comes as a PDG
\begin{tikzpicture}[scale=1.3, center base]
	\node[dpad0] (O) at (0,0) {$\Omega$};
	\node[dpad0] (E) at (1, 0) {$E?$};
	\draw[arr2,->>] (O) -- (E);
\end{tikzpicture}\, where \tikz[center base] \node[dpad0] {$E?$}; is an indicator variable for the event $E$. Furthermore, observing $E$ corresponds to adding an unconditional distribution on  \tikz[center base] \node[dpad0] {$E?$}; placing all mass on $E? = 1$. 






%% This made it into another paper.
\subsection{Inconsistency and the Partition Function}
\begin{storynote}
	The story in this section clearly doesn't fit. Still, the partition function $Z_\Psi$ is a quintessential component of inference in factor graphs. See, for instance:
	\begin{itemize}[nosep]
		\item \url{http://proceedings.mlr.press/v31/ma13a.pdf}
		\item {\small\url{https://papers.nips.cc/paper/2017/file/8d420fa35754d1f1c19969c88780314d-Paper.pdf}}
		\item \url{http://users.cms.caltech.edu/~venkatc/csh_compinf_preprint10.pdf}
		\begin{quotation} \small\it\noindent
			It is well-known that the complexity of computing the marginal distribution at an arbitrary vertex is comparable to that of computing the partition function. A polynomial-time procedure to solve one of these problems can be used to construct a polynomial-time algorithm for the other.
		\end{quotation}
	\end{itemize}
\end{storynote}

The factors of a factor graph, which in isolation indicate relative probabilities.
Factored exponential families form the mathematical backbone of statistical
mechanics, in which the normalization constant
$Z_{\Psi}$ for the WFG $\Psi = (\phi_j, \theta_j)_{j \in \cal J}$,
given by
$$
	Z_{\Psi} := \sum_{\mat w} \prod_{j \in \mathcal J} \phi_j(\mat w_j)^{\theta_j}
	,
$$
is known as the \emph{partition function}. In this setting, a factor graph

Perhaps surprisingly, many thermodynamic quantities
(e.g., total energy, free energy, pressure, and entropy) can
be obtained by taking various partial derivatives of $Z_\Psi$.
Because both $Z_{\Psi}$ for a factor graph $\Phi$, and $\aar{\dg M}$ are
both , one might wonder if there is some connection between them. Indeed, there is.

\begin{linked}{prop}{fg-inconsistency-is-partition-function}
	For any weighted factor graph $\Psi$, we have $Z_{\Psi} = \exp\; \aar{\PDGof{\Psi}}_1$.
\end{linked}
