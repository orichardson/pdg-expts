\relax % Standard Packages
    \usepackage[dvipsnames]{xcolor}
    % \usepackage[utf8]{inputenc}
    \usepackage{mathtools}
    \usepackage{amssymb}
		\DeclareMathSymbol{\shortminus}{\mathbin}{AMSa}{"39}
    % \usepackage{parskip}
    % \usepackage{algorithm}
    \usepackage{bbm}
	\usepackage{lmodern}
	% \usepackage{times}
    \usepackage{faktor}
    % \usepackage{booktabs}
	% \usepackage[margin=1in]{geometry}
    \usepackage{graphicx}
    \usepackage{scalerel}
    \usepackage{enumitem}
    \usepackage{nicefrac}\let\nf\nicefrac

    % \usepackage{color}
    %\usepackage{stmaryrd}
    \usepackage{hyperref} % Load before theorems...
        \hypersetup{colorlinks=true, linkcolor=blue!75!black, urlcolor=magenta, citecolor=green!50!black}

\usepackage{tikz}
	\usetikzlibrary{positioning,fit,calc, decorations, arrows, shapes, shapes.geometric}
	\usetikzlibrary{cd}

	%%%%%%%%%%%%
	\tikzset{center base/.style={baseline={([yshift=-.8ex]current bounding box.center)}}}

	% Node Stylings
	\tikzset{dpadded/.style={rounded corners=2, inner sep=0.7em, draw, outer sep=0.3em, fill={black!50}, fill opacity=0.08, text opacity=1}}
	\tikzset{dpad0/.style={outer sep=0.05em, inner sep=0.3em, draw=gray!75, rounded corners=4, fill=black!08, fill opacity=1, align=center}}
 	\tikzset{dpad/.style args={#1}{every matrix/.append style={nodes={dpadded, #1}}}}
	\tikzset{light pad/.style={outer sep=0.2em, inner sep=0.5em, draw=gray!50}}

	\tikzset{arr/.style={draw, ->, thick, shorten <=3pt, shorten >=3pt}}
	\tikzset{arr0/.style={draw, ->, thick, shorten <=0pt, shorten >=0pt}}
	\tikzset{arr1/.style={draw, ->, thick, shorten <=1pt, shorten >=1pt}}
	\tikzset{arr2/.style={draw, ->, thick, shorten <=2pt, shorten >=2pt}}

	\newcommand\cmergearr[5][]{
		\draw[arr, #1, -] (#2) -- (#5) -- (#3);
		\draw[arr, #1, shorten <=0] (#5) -- (#4);
		}
	\newcommand\mergearr[4][]{
		\coordinate (center-#2#3#4) at (barycentric cs:#2=1,#3=1,#4=1.2);
		\cmergearr[#1]{#2}{#3}{#4}{center-#2#3#4}
		}
	\newcommand\cunmergearr[5][]{
		\draw[arr, #1, -, shorten >=0] (#2) -- (#5);
		\draw[arr, #1, shorten <=0] (#5) -- (#3);
		\draw[arr, #1, shorten <=0] (#5) -- (#4);
		}
	\newcommand\unmergearr[4][]{
		\coordinate (center-#2#3#4) at (barycentric cs:#2=1.2,#3=1,#4=1);
		\cunmergearr[#1]{#2}{#3}{#4}{center-#2#3#4}
		}

\usepackage{amsthm,thmtools} % Theorem Macros
	\usepackage[noabbrev,nameinlink,capitalize]{cleveref}
    \theoremstyle{plain}
    \newtheorem{theorem}{Theorem}
	\newtheorem{coro}{Corollary}[theorem]
    \newtheorem{prop}[theorem]{Proposition}
    \newtheorem{conj}[theorem]{Conjecture}
    \newtheorem{claim}{Claim}
    \newtheorem{remark}{Remark}
    \newtheorem{lemma}[theorem]{Lemma}
    \theoremstyle{definition}
    % \newtheorem{defn}{Definition}
    % \declaretheorem[name=Definition]{defn}
    \declaretheorem[name=Definition, qed=$\square$]{defn}
    \declaretheorem[name=Example, qed=$\triangle$]{example}

	\crefname{defn}{Definition}{Definitions}
	\crefname{prop}{Proposition}{Propositions}
    \crefname{issue}{Issue}{Issues}

\relax %%%%%%%%% GENERAL MACROS %%%%%%%%
    \let\Horig\H
	\let\H\relax
	\DeclareMathOperator{\H}{\mathrm{H}} % Entropy
	\DeclareMathOperator{\I}{\mathrm{I}} % Information
	\DeclareMathOperator*{\Ex}{\mathbb{E}} % Expectation
	\DeclareMathOperator*{\EX}{\scalebox{1.5}{$\mathbb{E}$}}

    \newcommand{\mat}[1]{\mathbf{#1}}
    \DeclarePairedDelimiterX{\infdivx}[2]{(}{)}{%
		#1\;\delimsize\|\;#2%
	}
	\newcommand{\thickD}{I\mkern-8muD}
	\newcommand{\kldiv}{\thickD\infdivx}
	\newcommand{\tto}{\rightarrow\mathrel{\mspace{-15mu}}\rightarrow}

	\newcommand{\datadist}[1]{\Pr\nolimits_{#1}}
	% \newcommand{\datadist}[1]{p_\text{data}}

	\makeatletter
	\newcommand{\subalign}[1]{%
	  \vcenter{%
	    \Let@ \restore@math@cr \default@tag
	    \baselineskip\fontdimen10 \scriptfont\tw@
	    \advance\baselineskip\fontdimen12 \scriptfont\tw@
	    \lineskip\thr@@\fontdimen8 \scriptfont\thr@@
	    \lineskiplimit\lineskip
	    \ialign{\hfil$\m@th\scriptstyle##$&$\m@th\scriptstyle{}##$\hfil\crcr
	      #1\crcr
	    }%
	  }%
	}
	\makeatother
	\newcommand\numberthis{\addtocounter{equation}{1}\tag{\theequation}}

\relax %%%%%%%%%   PDG  MACROS   %%%%%%%%
	\newcommand{\ssub}[1]{_{\!_{#1}\!}}
	% \newcommand{\bp}[1][L]{\mat{p}_{\!_{#1}\!}}
	% \newcommand{\bP}[1][L]{\mat{P}_{\!_{#1}\!}}
	\newcommand{\bp}[1][L]{\mat{p}\ssub{#1}}
	\newcommand{\bP}[1][L]{\mat{P}\ssub{#1}}
	\newcommand{\V}{\mathcal V}
	\newcommand{\N}{\mathcal N}
	\newcommand{\Ed}{\mathcal E}

    \newcommand{\balpha}{\boldsymbol\alpha}
    \newcommand{\bbeta}{\boldsymbol\beta}

	\DeclareMathAlphabet{\mathdcal}{U}{dutchcal}{m}{n}
	\DeclareMathAlphabet{\mathbdcal}{U}{dutchcal}{b}{n}
	\newcommand{\dg}[1]{\mathbdcal{#1}}
	\newcommand{\PDGof}[1]{{\dg M}_{#1}}
	\newcommand{\UPDGof}[1]{{\dg N}_{#1}}
	\newcommand\VFE{\mathit{V\mkern-4mu F\mkern-4.5mu E}}

	\newcommand\Inc{\mathit{Inc}}
	\newcommand{\IDef}[1]{\mathit{IDef}_{\!#1}}
	% \newcommand{\ed}[3]{%
	% 	\mathchoice%
	% 	{#2\overset{\smash{\mskip-5mu\raisebox{-3pt}{${#1}$}}}{\xrightarrow{\hphantom{\scriptstyle {#1}}}} #3} %display style
	% 	{#2\overset{\smash{\mskip-5mu\raisebox{-3pt}{$\scriptstyle {#1}$}}}{\xrightarrow{\hphantom{\scriptstyle {#1}}}} #3}% text style
	% 	{#2\overset{\smash{\mskip-5mu\raisebox{-3pt}{$\scriptscriptstyle {#1}$}}}{\xrightarrow{\hphantom{\scriptscriptstyle {#1}}}} #3} %script style
	% 	{#2\overset{\smash{\mskip-5mu\raisebox{-3pt}{$\scriptscriptstyle {#1}$}}}{\xrightarrow{\hphantom{\scriptscriptstyle {#1}}}} #3}} %scriptscriptstyle
	\newcommand{\ed}[3]{#2%
	  \overset{\smash{\mskip-5mu\raisebox{-1pt}{$\scriptscriptstyle
	        #1$}}}{\rightarrow} #3}

    \newcommand{\nhphantom}[2]{\sbox0{\kern-2%
		\nulldelimiterspace$\left.\delimsize#1\vphantom{#2}\right.$}\hspace{-.97\wd0}}
		% \nulldelimiterspace$\left.\delimsize#1%
		% \vrule depth\dp#2 height \ht#2 width0pt\right.$}\hspace{-.97\wd0}}
	\makeatletter
	\newsavebox{\abcmycontentbox}
	\newcommand\DeclareDoubleDelim[5]{
	    \DeclarePairedDelimiterXPP{#1}[1]%
			{% box must be saved in this pre code
				\sbox{\abcmycontentbox}{\ensuremath{##1}}%
			}{#2}{#5}{}%
		    {%
				\nhphantom{#3}{\usebox\abcmycontentbox}%
				\hspace{1.2pt} \delimsize#3%
				\mathopen{}\usebox{\abcmycontentbox}\mathclose{}%
				\delimsize#4\hspace{1.2pt}%
				\nhphantom{#4}{\usebox\abcmycontentbox}%
			}%
	}
	\makeatother
	\DeclareDoubleDelim
		\SD\{\{\}\}
	\DeclareDoubleDelim
		\bbr[[]]
	\makeatletter
	\newsavebox{\aar@content}
	\newcommand\aar{\@ifstar\aar@one@star\aar@plain}
	\newcommand\aar@one@star{\@ifstar\aar@resize{\aar@plain*}}
	\newcommand\aar@resize[1]{\sbox{\aar@content}{#1}\scaleleftright[3.8ex]
		{\Biggl\langle\!\!\!\!\Biggl\langle}{\usebox{\aar@content}}
		{\Biggr\rangle\!\!\!\!\Biggr\rangle}}
	\DeclareDoubleDelim
		\aar@plain\langle\langle\rangle\rangle
	\makeatother

\relax %TODOs and footnotes
    \newcommand{\TODO}[1][INCOMPLETE]{{\centering\Large\color{red}$\langle$~\texttt{#1}~$\rangle$\par}}
