\documentclass{article}


%
\usepackage{mathtools} %also loads amsmath
\usepackage{amssymb, bbm}

\usepackage[backend=biber,
	style=alphabetic,
	%	citestyle=authoryear,
	natbib=true,
	url=true, 
	doi=true]{biblatex}

%\usepackage{blkarray} % for matrices with labels
\usepackage{microtype}
\usepackage{relsize}
\usepackage{environ}% http://ctan.org/pkg/environ; for capturing body as a parameter for idxmats
\usepackage{tikz}
	\usetikzlibrary{positioning,fit,calc, decorations, arrows, shapes, shapes.geometric}
	\usetikzlibrary{cd}
	
	\pgfdeclaredecoration{arrows}{draw}{
		\state{draw}[width=\pgfdecoratedinputsegmentlength]{%
			\path [every arrow subpath/.try] \pgfextra{%
				\pgfpathmoveto{\pgfpointdecoratedinputsegmentfirst}%
				\pgfpathlineto{\pgfpointdecoratedinputsegmentlast}%
			};
	}}
	%%%%%%%%%%%%
	\tikzset{center base/.style={baseline={([yshift=-.8ex]current bounding box.center)}}}
	
	\tikzset{dpadded/.style={rounded corners=2, inner sep=0.7em, draw, outer sep=0.3em, fill={black!50}, fill opacity=0.08, text opacity=1}}
	\tikzset{active/.style={fill=blue, fill opacity=0.1}}
	\tikzset{square/.style={regular polygon,regular polygon sides=4, rounded corners = 0}}
	\tikzset{octagon/.style={regular polygon,regular polygon sides=8, rounded corners = 0}}
	
	
	\tikzset{alternative/.style args={#1|#2|#3}{name=#1, circle, fill, inner sep=1pt,label={[name={lab-#1},gray!30!black]#3:\scriptsize #2}} }
	
	
	\tikzset{bpt/.style args={#1|#2}{alternative={#1|#2|above}} }
	\tikzset{tpt/.style args={#1|#2}{alternative={#1|#2|below}} }
	\tikzset{lpt/.style args={#1|#2}{alternative={#1|#2|left}} }
	\tikzset{rpt/.style args={#1|#2}{alternative={#1|#2|right}} }
	\tikzset{pt/.style args={#1}{alternative={#1|#1|above}} }
	

	\tikzset{mpt/.style args={#1|#2}{name=#1, circle, fill, inner sep=1pt,label={[name={lab-#1},gray]\scriptsize #2}} }
	\tikzset{pt/.style args={#1}{name=#1, circle, fill, inner sep=1pt,label={[name={lab-#1},gray]\scriptsize #1}} }
	
		
		 %\foreach \x in {#1}{(\x) (lab-\x) } 
		 
	\tikzset{Dom/.style args={#1 (#2) around #3}{dpadded, name=#2, label={[name={lab-#2},align=center] #1}, fit={ #3 } }}
	\tikzset{bDom/.style args={#1 (#2) around #3}{dpadded, name=#2, label={[name={lab-#2},align=center]below:#1}, fit={ #3 } }}
	\tikzset{arr/.style={draw, ->, thick, shorten <=3pt, shorten >=3pt}}
	\tikzset{archain/.style args={#1}{arr, every arrow subpath/.style={draw,arr, #1}, decoration=arrows, decorate}}
	%\tikzset{every label/.append style={text=red, font=\scriptsize}}
	\tikzset{dpad/.style args={#1}{every matrix/.append style={nodes={dpadded, #1}}}}
	\tikzset{light pad/.style={outer sep=0.2em, inner sep=0.5em, draw=gray!50}}
	
	
	\newcommand\cmergearr[4]{
		\draw[arr,-] (#1) -- (#4) -- (#2);
		\draw[arr, shorten <=0] (#4) -- (#3);
	}
	\newcommand\mergearr[3]{
		\coordinate (center-#1#2#3) at (barycentric cs:#1=1,#2=1,#3=1.2);
		\cmergearr{#1}{#2}{#3}{center-#1#2#3}
	}
	
	\usetikzlibrary{matrix}
	\tikzset{toprule/.style={%
	        execute at end cell={%
	            \draw [line cap=rect,#1] 
	            (\tikzmatrixname-\the\pgfmatrixcurrentrow-\the\pgfmatrixcurrentcolumn.north west) -- (\tikzmatrixname-\the\pgfmatrixcurrentrow-\the\pgfmatrixcurrentcolumn.north east);%
	        }
	    },
	    bottomrule/.style={%
	        execute at end cell={%
	            \draw [line cap=rect,#1] (\tikzmatrixname-\the\pgfmatrixcurrentrow-\the\pgfmatrixcurrentcolumn.south west) -- (\tikzmatrixname-\the\pgfmatrixcurrentrow-\the\pgfmatrixcurrentcolumn.south east);%
	        }
	    },
	    leftrule/.style={%
	        execute at end cell={%
	            \draw [line cap=rect,#1] (\tikzmatrixname-\the\pgfmatrixcurrentrow-\the\pgfmatrixcurrentcolumn.north west) -- (\tikzmatrixname-\the\pgfmatrixcurrentrow-\the\pgfmatrixcurrentcolumn.south west);%
	        }
	    },
	    rightrule/.style={%
	        execute at end cell={%
	            \draw [line cap=rect,#1] (\tikzmatrixname-\the\pgfmatrixcurrentrow-\the\pgfmatrixcurrentcolumn.north east) -- (\tikzmatrixname-\the\pgfmatrixcurrentrow-\the\pgfmatrixcurrentcolumn.south east);%
	        }
	    },
	    table with head/.style={
		    matrix of nodes,
		    row sep=-\pgflinewidth,
		    column sep=-\pgflinewidth,
		    nodes={rectangle,minimum width=2.5em, outer sep=0pt},
		    row 1/.style={toprule=thick, bottomrule},
  	    }
	}

	

\NewEnviron{ctikzpicture}{\begin{center}\expandafter\begin{tikzpicture}\BODY\end{tikzpicture}\end{center}}
%\newenvironment{ctikzpicture}
%	{\begin{center}\begin{tikzpicture}}
%	{\end{tikzpicture}\end{center}}

\usepackage{color}
\definecolor{deepgreen}{rgb}{0,0.5,0}

\usepackage[colorlinks=true, citecolor=deepgreen]{hyperref}


\usepackage{stmaryrd}
\usepackage{trimclip}

\makeatletter
\DeclareRobustCommand{\shortto}{%
	\mathrel{\mathpalette\short@to\relax}%
}

\newcommand{\short@to}[2]{%
	\mkern2mu
	\clipbox{{.5\width} 0 0 0}{$\m@th#1\vphantom{+}{\shortrightarrow}$}%
}
\makeatother


\usepackage{amsthm, thmtools}
\usepackage{
	nameref,%\nameref
	hyperref,%\autoref
	% n.b. \Autoref is defined by thmtools
%	cleveref,% \cref
	% n.b. cleveref after! hyperref
}
\usepackage[noabbrev]{cleveref}
\hypersetup{colorlinks=true, linkcolor=blue, urlcolor=magenta}

\begingroup
\makeatletter
\@for\theoremstyle:=definition,remark,plain\do{%
	\expandafter\g@addto@macro\csname th@\theoremstyle\endcsname{%
		\addtolength\thm@preskip\parskip
	}%
}
\endgroup
\makeatother

\theoremstyle{plain}
\newtheorem{theorem}{Theorem}[section]
\newtheorem{coro}{Corollary}[theorem]
\newtheorem{prop}[theorem]{Proposition}
\newtheorem{lemma}[theorem]{Lemma}
\newtheorem{fact}[theorem]{Fact}
\newtheorem{conj}[theorem]{Conjecture}

\theoremstyle{definition}
\newtheorem{defn}{Definition}[section]
\newtheorem*{defn*}{Definition}
\newtheorem{examplex}{Example}
\newenvironment{example}
	{\pushQED{\qed}\renewcommand{\qedsymbol}{$\triangle$}\examplex}
	{\popQED\endexamplex\vspace{-1em}\rule{1cm}{0.7pt}\vspace{0.5em}}

\theoremstyle{remark}
\newtheorem*{remark}{Remark}

%\crefname{lemma}{lemma}{lemmas}
\crefname{example}{example}{examples}
\crefname{defn}{definition}{definitions}


\usepackage{xstring}
\usepackage{enumitem}

\DeclarePairedDelimiterX{\infdivx}[2]{(}{)}{%
	#1\;\delimsize\|\;#2%
}
\newcommand{\kldiv}{D_\mathrm{KL}\infdivx}
\DeclarePairedDelimiter{\bbr}{\llbracket}{\rrbracket}
\DeclarePairedDelimiter{\ppr}{\llparenthesis}{\rrparenthesis}
%\DeclarePairedDelimiter{\norm}{\lVert}{\rVert}

%\newcommand\duplicat[1]{\gdef\mylist{}\foreach \x in {#1}{\xdef\mylist{\mylist (\x) (lab-\x) }}\mylist} %% this doesn't work :((
\newcommand\lab[1]{(#1)(lab-#1)}


\newcommand{\CI}{\mathrel{\perp\mspace{-10mu}\perp}}
\newcommand\E{\mathop{\mathbb E}}


\newcommand{\todo}[1]{{\color{red}\large\textbf{[}{\normalsize\texttt{todo:} \itshape#1}\textbf{]}}}
\newcommand{\note}[1]{{\color{blue}\textbf{[}{\normalsize\texttt{note:} \itshape#1}\textbf{]}}}

\newcommand\geqc{\succcurlyeq}
\newcommand\leqc{\preccurlyeq}
\newcommand\mat[1]{\mathbf #1}
\newcommand{\indi}[1]{\mathbbm{1}_{\left[\vphantom{\big[}#1 \vphantom{\big]}\right]}}
\newcommand\m[1]{\mathbf m_{\mathsf #1}}
\def\cpm#1(#2|#3){\mathbf #1 \left[ #2 \middle|#3\right]}


\newcommand\recall[1]{\expandarg\cref{#1}:\vspace{-1em} \begingroup\small\color{gray!80!black}\begin{quotation} \expandafter\csname #1\endcsname* \end{quotation}\endgroup }

%OMG THIS WORKS

\def\wrapwith#1[#2;#3]{
	\expandarg\IfSubStr{#1}{,}{
		\expandafter#2{\expandarg\StrBefore{#1}{,}}
		\expandarg\StrBehind{#1}{,}[\tmp] 
		\xdef\tmp{\expandafter\unexpanded\expandafter{\tmp}}
		#3
		\wrapwith{\tmp}[#2;{#3}]
	}{ \expandafter#2{#1} }
}
\def\hwrapcells#1[#2]{\wrapwith#1[#2;&]}
\def\vwrapcells#1[#2]{\wrapwith#1[#2;\\]}



\newsavebox{\idxmatsavebox}
\def\makeinvisibleidxstyle#1#2{\phantom{\hbox{#1#2}}}
\newenvironment{idxmatphant}[4][\footnotesize\color{gray}\text]{%
	\def\idxstyle{#1}
	\def\colitems{#3}
	\def\rowitems{#2}
	\def\phantitems{#4}
	\begin{lrbox}{\idxmatsavebox}$
	\begin{matrix}  \begin{matrix} \hwrapcells{\colitems}[\idxstyle]  \end{matrix} \\[0.1em]
		\left[ 
		\begin{matrix}
			\hwrapcells{\phantitems}[\expandafter\makeinvisibleidxstyle\idxstyle]  \\[-1em]
	}{
		\end{matrix}\right]		&\hspace{-0.5em}\begin{matrix*}[l] \vwrapcells{\rowitems}[\idxstyle] \end{matrix*}
	\end{matrix}
	$\end{lrbox}
	\raisebox{0.75em}{\usebox\idxmatsavebox}
%	\vspace{-0.5em}
}

\newenvironment{idxmat}[3][\footnotesize\color{gray}\text]
	{\begingroup\idxmatphant[#1]{#2}{#3}{#3}}
	{\endidxmatphant\endgroup}

\newenvironment{sqidxmat}[2][\footnotesize\color{gray}\text]
	{\begingroup\idxmat[#1]{#2}{#2}}
	{\endidxmat\endgroup}
	
	
%%%%%%%%%%%%
% better alignment for cases
\makeatletter
\renewenvironment{cases}[1][l]{\matrix@check\cases\env@cases{#1}}{\endarray\right.}
\def\env@cases#1{%
	\let\@ifnextchar\new@ifnextchar
	\left\lbrace\def\arraystretch{1.2}%
	\array{@{}#1@{\quad}l@{}}}
\makeatother

\usepackage{amsmath,amssymb}
\usepackage{mathtools}
\usepackage[margin=1in]{geometry}
\usepackage{parskip}
\usepackage{bbm}
\usepackage{enumitem}

\usepackage{xcolor}

\renewcommand{\H}{\mathop{\mathrm H}}
\newcommand{\E}{\mathop{\mathbb E}}
\newcommand{\bp}[1][L]{\mathbf{p}_{\!_#1\!}}
\newcommand{\V}{\mathcal V}
\newcommand{\N}{\mathcal N}
\newcommand{\Ed}{\mathcal A}

\newcommand{\dg}[1]{\mathsf #1}
%\def\mnvars[#1]{(\N#1, \Ed#1, \V#1, \bp#1)}
\newcommand\Pa{\mathbf{Pa}}
\newcommand\mat[1]{\mathbf #1}
%\newcommand\SD{_{\text{sd}}}

\usepackage{amsthm,thmtools}
\begingroup
\makeatletter
\@for\theoremstyle:=definition,remark,plain\do{%
	\expandafter\g@addto@macro\csname th@\theoremstyle\endcsname{%
		\addtolength\thm@preskip\parskip
	}%
}
\endgroup
\makeatother

\theoremstyle{plain}
\newtheorem{theorem}{Theorem}
\newtheorem{coro}{Corollary}[theorem]
\newtheorem{prop}[theorem]{Proposition}
\newtheorem{lemma}[theorem]{Lemma}
\newtheorem{fact}[theorem]{Fact}
\newtheorem{claim}[theorem]{Claim}
\newtheorem{conj}[theorem]{Conjecture}
\newtheorem{constr}[theorem]{Construction}

\theoremstyle{definition}
\newtheorem{defn}{Definition}
\newtheorem*{defn*}{Definition}
\newtheorem{examplex}{Example}
\newenvironment{example}
	{\pushQED{\qed}\renewcommand{\qedsymbol}{$\triangle$}\examplex}
	{\popQED\endexamplex%\vspace{-1.6em}\rule{2cm}{0.7pt}\vspace{0.5em}}
}

\theoremstyle{remark}
\newtheorem*{remark}{Remark}

\newcommand{\alle}[1][L]{_{ X \xrightedge{\!\!#1} Y }}
\newcommand{\cdo}{\mathop{\mathrm{do}}}
\newcommand{\evt}[2]{#1\!\!=\!#2}

\begin{document}


\section{Preliminaries}
\subsection{Causal Models}

Throughout this document, when I say ``probabilistic causal model'', I am referring to a probabilistic model that can answer queries about interventions and conditioning. Causal BNs are the prototypical example. I will also refer extensively to Structural Equation Models (SEMs) in this analysis. The relation between them will be useful in understanding $\alpha$.
% when I say ``causal model'', I am generally referring to a model such as a variant of a SEM, which a structural equation for each variable. 

First, observe that these two notions are related:

\begin{enumerate}
\item A structural equation model (SEM) $\mathcal M = (\mathcal U, \mathcal V, \mathcal R, \mathcal I), \mathcal F$, together with a distribution $\mu$ over the exogenous variables, is effectively a probabilistic causal model on the endogenous variables, where if $Y$ is dependent on the set of variables $\mathbf X$
\[ Pr(Y = y \mid \mathbf X = \vec x) := \sum_{\vec u \in \mathcal R(\mathcal U)} \mu(\vec u) f_Y(\vec x, \vec u) \]
are the tables that ``hold causally'', and more generally we answer queries by the conditional expectation
\begin{constr}
	\[ \Pr(Y\!=\!y \mid \cdo(X\!=\!x), Z\!=\!z) = \E_{\vec u \sim \mu}\bigg[ \mathbbm1\Big[ \mathcal M_{A \gets a}, \vec u\models (B\!=\!b) \Big]~\bigg|~Z\!=\!z\bigg] \]
\end{constr}

\item Conversely, a causal Bayesian Network $\mathcal B$, over the variables $\mathcal X$ is effectively an SEM where $\mathcal X$ are the exogenous variables, and there is an endogenous variable $U_Y$ which determines the randomness for each variable $Y$ --- together with a distribution on each $U$. For instance, if we cheat a little by letting $U$ take values in the interval $[0,1]$, the structural equations could then given by 
\[ f_Y(\Pa(Y) = \mathbf x, U_Y = u) = \begin{cases}
	y_1 & \text{if }0 \leq u < \Pr(y_1\mid \mathbf x) \\
	&\vdots \\
	y_i & \text{if }\sum_{j < i} \Pr(y_j\mid \mathbf x) \leq u < \sum_{j \leq i} \Pr (y_j\mid \mathbf x)\\
	&\vdots
\end{cases} 
\]
where $\mu$ independently distributes each $U_Y$ uniformly.
% This is the prototypical example of what we will call a local structural equation model, or LSEM.
\end{enumerate}

\begin{claim}
	The two conversions above result in equivalent causal models, in that after the conversion, they give the same answers to any query that makes sense models.
\end{claim}


\begin{defn}
	We will call a SEM $\mathcal M = ((\mathcal U, \mathcal V, \mathcal R, \mathcal I), \mathcal F)$, together with a 	distribution $\mu$ over its exogenous variables $\mathcal U$, a probabilistic SEM, or pSEM.
\end{defn}
\begin{remark}
Via construction 1, a pSEM is also a probabilistic causal model.
\end{remark}
\begin{defn}
	A \emph{multi}-SEM, is an SEM where $\mathcal F$ may contain more than one equation per variable, all of which must hold. More precisely, $f_X$ is now a \emph{set} of structural equations that produce values of $X$, all of which should hold --- that is, 
	\[ f_X^{(1)} (\vec u, \vec v) = f_X^{(2)}(\vec u, \vec v) \]
	for $f_X^{(1)}, f_X^{(2)} \in f_X$, whenever it is possible for $(\vec u, \vec v)$ to be a state of the model.
\end{defn}

\begin{defn}
	 A pSEM is a \emph{local}, or an lpSEM, if for any two distinct endogenous variables $X \neq Y \in \mathcal V$, the sets of variables $\mathcal U_X$ and $\mathcal U_Y$, that $f_X$ and $f_Y$ are respectively dependent upon, are distributed independently according to $\mu$.
	% claim: equivalent to a PDG.
\end{defn}

\begin{remark}
	LSEMs, are clearly a strict subclass of $\{$SEMs + distributions on $\mathcal U\}$, as they assume no confounding / shared information between endogenous variables (i.e., that each $U_i$ is independent). 
	We will see later that in a PDG representing such a causal model, $\alpha < 1$ can be thought of a relaxation of this requirement, so that confounding is possible, while $\alpha > 1$ can be thought of a strengthening of it, so that the value of $Y$ does not even depend on an implicit noisy variable $U_Y$.
\end{remark}

\begin{conj}
	Bayesian Networks, Conditional Bayesian Networks, Dependency Networks, MRFs with interventions, and pure probabilistic programs, can all be viewed as LSEMs by construction 2 above; in doing so, any probabilistic causal structure is unchanged.
\end{conj}
A strict PDG can be thought of as a mechanism for simultaneously (1) factoring a distribution over possible LSEMs into the same local components as used to factor the distribution itself, (2) allowing a user to express uncertanity by backing off of the locality observation, and 


\subsection{PDGs}
	\def\mnvars[#1]{(\N#1, \Ed#1, \V#1, \mat p#1, \alpha#1,\beta#1)}
	\begin{defn}[PDG]\label{def:model}
	A strict PDG is a tuple $\mnvars[]$ where
	\begin{description}[nosep]
		\item[$\N$]~is a finite collection of nodes, corresponding to variables
		\item[$\Ed$]~is a collection of directed edges (arrows), each with a source, target, and a (possibly empty) label.
		\item[$\V$]~associates each node $N \in \N$ with a set $\V(N)$,
		representing the values that the variable $N$ can take. 
		\item[$\mathbf p$] associates, for each edge $L = (X,Y, \ell) \in \Ed$ and $x \in \V(X)$ a distribution $\bp(x)$ on $Y$, whenever $\beta_L > 0$.
		\item[$\beta$]~associates to each edge $L$, a number in $[0,\infty]$, indicating certainty in the conditional distribution $\bp(Y \mid X)$ 
		\item[$\alpha$]~associates to each edge $L$, a number in $[0,1]$, indicating degree of belief that $L$ holds causally.
\end{description}
\end{defn}

The definition of a general PDG is the same, except $\mathcal V(X)$ may include a special ``\texttt{null}'' value, on which a cpd $\bp$ for an edge $L$ whose source is $X$ does not need to give a distribution (i.e., $\bp(\texttt{null})$ may be undefined). Non-strict PDGs are strictly more expressive in several important ways, but from this point forward we consider only strict PDGs, and drop the word `strict'.


\begin{defn}
	We define the following additional subclasses of PDGs, based on their parameters $\alpha, \beta$. A PDG $\dg M$ is:
	\begin{itemize}[nosep]
		\item \emph{qualitative}, if every $\beta_L = 0$, and $\alpha_L > 0$. Such PDGs consist effectively of only the data $(\N, \Ed, \alpha)$, and we may refer to them QDGs.
		\item \emph{exact} if every $\alpha_L$ is either $0$ or $1$.
		\item \emph{over-constrained} (resp. under-constrained, perfectly constrained) at a node $Y$ if the sum $\sum_{\overset{L}{\to}Y} \alpha_L > 1$ (resp. $\sum_{\overset{L}{\to}Y} \alpha_L < 1$, $\sum_{\overset{L}{\to}Y} \alpha_L =1$), and globally so if this is true for every variable $Y$.
	\end{itemize}
\end{defn}


Many models, both primarily probabilistic and primarily causal including BNs, DNs, and SEMs, can be represented as PDGs in which each node has only one incoming (non-projection) link, each with $\alpha=1$ (making them exact).%
	% \footnote{which might suggest that the term `PDG' emphasizes the probabilistic bit too strongly; we might want another name}
In each case, we can encode every cpd or structural equation for a variable $Y$ in the new PDG as the data associated to a new edge $L_Y$, from joint settings of all variables%
		\footnote{Recall that we also have to expand joint settings as variables themselves and add projections, so that PDGs can be formalized this way; the function $\Gamma$, introduced for BNs, will still work more generally even when there are cycles.}
that the structural equation / cpd depends on, to $Y$, and associating uniform high weights $\beta \simeq \infty$ and $\alpha = 1$.%

Every PDG produced this way is exact.

From the perspective of PDGs, the difference between the two ways of embedding a causal model (as a BN, or SEM) comes down to whether exogenous variables are explicitly identified with the appropriate connections, with every edge deterministic, or implicit and independent. Making use of the latter allows for a a better separation between implicit noise, and a known conditional dependence, with the qualitative structure alone.


\begin{figure}.
	[PDG modeling like an SEM] [PDG modeling like a BN] 
	\caption{TODO: figure}
\end{figure}

%A member of the more general class of \emph{exact} (but possibly not perfectly constrained) PDGs, $\dg Q = (\N, \Ed)$ should be thought of as a collection $\Ed$ of causal equations with targets in $\N$, that may contain multiple equations that have the same target.
%\begin{claim}
%%	If $\Pr_1$ and $\Pr_2$ are causal models for a set of variables $\mathcal X$, then 
%	If $\dg Q$ is a QDG, that has edges $X_1 \to Y$ and $X_2 \to Y$, then $G$ must have 
%\end{claim}
%


% {\color{gray}
% \begin{defn}
% 	A noisy channel from $X$ to $Y$ is
% \end{defn}
% }

\clearpage
\section{Alpha}



Let $L$ be an edge $X \to Y$. We now analyze the relationships between several alternate definitions of $\alpha_L$.

\begin{itemize}
	\item[\textbf{A1.}] $\alpha_L$ is an agent's subjective degree of belief that the function $f_Y$ which generates $Y$, depends only on $X$ and an endogenous variable $U_L$ that is independent of everything else.
	\item[\textbf{A2.}] $\alpha_L$ is an agent's subjective degree of belief in the proposition that the edge $L$ could be associated with a cpd that holds causally.
\end{itemize}




% (1) intervening on $X$ causes a change in $Y$, and also (2) having fixed $X$, further interventions $\cdo(Z=z)$ on nearby variables do not affect the value of $Y$. 

\begin{defn}
	A cpd $p(Y \mid X)$ holds $\epsilon$-causally in a probabilistic causal model $\Pr$, iff 
	\begin{enumerate}[nosep]
		\item there exists some 
		 $x \in X$ such that $p(Y \mid X=x) \neq \Pr(Y)$
		 \item for every $x$, $p(Y \mid X=x) = \Pr(Y \mid \cdo(X=x,Z=z))$
	 	for any intervention $Z=z$ with $\Pr(Z = z) > \epsilon$.
	\end{enumerate}
	To say that a cpd holds causally is to say that it holds $0$-causally.	
\end{defn} 

In the context of an SEM with signature ($\mathcal U, \N, \mathcal R, \mathcal I$), if  $X,Y \in \mathcal V$, $I \subseteq \mathcal I$, and $x,y \in \mathcal R(X,Y)$ then let
\[ \mathcal U(Y\!=\!y \mid X \!=\!x)_{I} := \bigg\{ \vec u \in \mathcal R(\mathcal U) :
		(\mathcal M, \vec u) \models [Z \gets z](X \!=\! x \Rightarrow Y \!=\! y)
	~\text{for all } (Z=z) \in I
	% ~\bigg|~
		% (\mathcal M, \vec u) \models 
	\bigg\}% = p_i(Y_i \!=\! y_i \mid X_i \!=\! x_i)
 \]
be the set of settings of exogenous vaiables which ensure that  $Y\!=y$ whenever $X\!=x$, regardless of any interventions.

% {\color{gray}
% \begin{defn}
% 	A collection of cpds $\{ p_i (Y_i \mid X_i) \}$ holds causally in a pSEM $\cal M = (U,  V,  R,  I), F, \mu$ with each $X_i,Y_i \in \mathcal V$, iff 
% 	for all $x_i \in \mathcal R(X_i)$, $y_i \in \mathcal R(Y_i)$, and $(Z=z) \in \mathcal I$,
% 	$\mu \mathcal U(Y = y \mid X = x)$
% \end{defn}
% 
% \begin{prop}
% 	If a cpd holds causally in $(\mathcal M,\mu)$ iff it holds causally in $\Pr_{\mathcal M, \mu}$ 
% \end{prop}
% }


%\begin{defn}
%	A \emph{causal graph} for a causal model $\Pr$ is a directed (multi-)graph whose nodes correspond to random variables, and whose edges hold causally.
%\end{defn}

\begin{defn}
	% Equality of two random things. 
	An exact QDG $\dg Q = (\N,\Ed,\alpha)$ is \emph{causally consistent} with an SEM $\mathcal M = (\mathcal U, \N, \mathcal R, \mathcal I), \mathcal F$
	%, and distribution $\mu$ over $\mathcal V$,
	if there exists an assignment $\mat p$ of cpds to every edge in $\dg Q$, such that  $\bp$ holds causally in $\Pr_{\mathcal M, \mu}$ for each $L \in \Ed_M$.
	$\dg Q$ is \emph{causally consistent} if there exists such an SEM $\mathcal M$ and distribution $\mu$.
\end{defn}
%\begin{defn}
%	% Equality of two random things. 
%	An exact QDG $\dg Q = (\N,\Ed,\alpha)$ is \emph{causally consistent} with an SEM $\mathcal M = (\mathcal U, \N, \mathcal R, \mathcal I), \mathcal F$,% and distribution $\mu$ over $\mathcal U$
%	if for every $X \overset{L}{\to}Y \in \Ed$, 
%	% Want to say: distribution doesn't change given any interventions
%	% and moreover that each link always gives same information
%	% More simply: anything true in SEM
%	\[ f_Y(x, \vec u) \]
%	%
%	% for any $\vec u \in \mathcal U$, we have $f$ 
%	%
%	$\dg Q$ is \emph{causally consistent} if there exists such an SEM $\mathcal M$.% and distribution $\mu$.
%\end{defn}



\begin{claim}
	If an exact PDG $\dg Q$ is causally consistent with an LSEM $(\mathcal M, \mu)$, and contains two edges $X_1 \overset{L_1}\longrightarrow Y$ and $X_1 \overset{L_2}\longrightarrow Y$, then $\Pr_{\mathcal M, \mu}(Y \mid X_1, X_2)$ is deterministic.
\end{claim}
\begin{proof}
	Consider the structural equation $f_Y$ in $\mathcal M$.
	Because $\mathcal M, \mu$ is causally consistent with the first edge $X_1 \overset{L_1}\longrightarrow Y$, 
	we know that 
	
	Consider the two structural equations on Y: $f_1$ and $f_2$. $f_1$ depends only on the values of $X_1$ and $U_1$, while $f_2$ depends only on $X_1$ and $U_2$. We also know $f_1(x_1, u_1) = f_2(x_2, u_2)$ for any values of $(x_1, x_2, u_1, u_2)$ with $\Pr(x_1, x_2, u_1, u_2) > \epsilon$.
	
	
	and $u_1$ and $u_2$ are independent 
\end{proof}

\begin{remark}
	It is, however, possible to have 
\end{remark}


\begin{claim}
%	If $Y$ is some variable, and an agent consistently writes $\alpha$ to denote their subjective degree of belief that the associated edge $\sum_{X \overset{L}{\to}Y} > 1$, then any causal model $\Pr$ they consider possible must have strictly positive probability 
	
\end{claim}




\section*{Scratch}

%\begin{claim}
%	Any deterministic edge must hold causally.
%\end{claim}
	\[ \E_{z \sim \Pr(Z)}\Pr(Y \mid \cdo(X=x,Z=z)) = \Pr(Y \mid X) \]


%$\hat Y_L$ from $p(x)$, with probability $\alpha_L$, such that 
%
%$\Pr(Y | \cdo{\evt Xx,\evt Zz})
%
%"may as well have been drawn from $p(x)$", independent of the context. 


For each edge $L_i : X_i \to Y$ coming into $Y$, draw $x_i \sim X_i$ and $\hat y_i \sim p(Y|x_i)$. 
If there exists distribution $q$ on $Y$, such that $q(Y=\hat y_i \mid x_i) \geq \alpha_i$, and $q(Y | x_i) = \Pr(Y | \cdo(\evt{X_i}{x_i}, \evt Zz) $
%The vector $\vec \alpha_{\shortto Y}$ consisting of $\alpha_L$ for each edge into $L$ is an agent's subjective belief that each of the 0
%\[ p(\evt Yy \mid \evt Xx) = \Pr(Y \]


\[  \alpha_L :=  \frac{\Pr_{ z \sim Z} \Big[ p( Y | X=x) = \Pr(Y | \cdo(Z\!=\!z,X\!=\!x)) \Big]}{\Pr_{ z \sim Z} \Big[ p( Y | X=x) = \Pr(Y | \cdo(Z\!=\!z)) \Big]} \] 


The primary example we want to have in mind is the following:


%Why is it not reasonable to think $\alpha$ is a primitive quantity?

\end{document}0
